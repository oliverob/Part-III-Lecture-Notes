\documentclass{article}
\usepackage[utf8]{inputenc}
\usepackage{bm}
\usepackage{amssymb}
\usepackage{amsmath}
\usepackage{braket}
\usepackage{cancel}
\title{Quantum Information Theory}
\author{oliverobrien111 }
\date{July 2021}

\begin{document}

\maketitle

\section{Lecture 1}
\subsection{Path Integrals in Quantum Mechanics}
Goal is to reformulate Schrodingers equation as a path integral.
$$
\hat H(\hat x, \hat p) \text{ with } [\hat x , \hat p] = i \hbar
$$
Assuming $\hat H = \frac{\hat p^2}{2m} + V(\hat x)$. The schrodinger picture is:
$$
i \hbar \frac{d}{dt} \ket{\psi(t)} = \hat H \ket{\psi(t)}
$$
so
$$
\ket{\psi(t)} = e^{-i \hat H t/\hbar} \ket{\psi(t)}
$$
Wavefunction: $\Psi(x,t) = \bra{x} \ket{\psi(t)}$. We want to solve schordingers equation for this wavefucniton in a way that introduces the path integral:
$$
\Psi(x,t) = \bra{x} \ket{\psi(t)} = \bra{x} e^{- \hat H t/\hbar} \ket{\psi(0)}
$$
$$
\Psi(x,t) = \int_{-\infty}^{\infty} K(x, x_0;t) \Psi(x_0, 0)
$$
where
$$
K(x, x_0;t) = \ket{x} e^{-i \hat Ht/ \hbar} \ket{x_0}
$$
is what we want a path integral expression for\\\\
Lets consider $n$ intermediate times/positions. Let $0 =t_0 <t_1 < t_2 <\ldots < t_n < t_{n+1} = T$:
$$
e^{-i \hat H T/\hbar} = e^{-i \hat H(t_{n+1}-t_n)/\hbar}\cdotse^{-i \hat H(t_{1}-t_0)/\hbar}
$$
Insert identity: $I = \int dx_r \ket{x_r}\bra{x_r}$:
$$
K(x, x_0;t) = \int_{-\infty}^{\infty} \left( \prod_{r=1}^n dx_r \bra{x_{r+1}}e^{-i\hat H(t_{r+1}-t_r)/\hbar} \ket{x_r}\right) \bra{x_1} e^{-i\hat H(t_1 - t_0)/\hbar} \ket{x_0}
$$
Consider fixed $V(\hat x) =0$. $K_0(x, x'; t) = \bra{x} e^{-i \frac{ \hat p^2}{2m \hbar} t} \ket{x'}$. Insert the identity: $I = \int \frac{dp}{2\pi \hbar} \ket{p}\bra{p}$.
$$
K_0(x,x';t) = e^{\frac{im(x-x')^2}{2 \hbar t}} \sqrt{\frac{m}{2\pi i \hbar t}}
$$
For $V(\hat x) \neq 0$, we need very small time steps. Separate kinetic and potential parts (Suzuki-Trotter decomposition). Take $t_{r+1} - t_r = \delta t$ to be small and $n$ large so $n\delta t = T$ (constant).
$$
e^{-i \hat H \delta t / \hbar} = \exp( - \frac{i \hat p^2 \delta t}{2\pi \hbar}) \exp( - \frac{iV(\hat x) \delta t}{\hbar}) ( 1+ O((\delta t)^2))
$$
The last term here is vanishingly small under the Broker-Campbell-Henshorff thingy. Between any 2 position eigenstates:
$$
\bra{x_{r+1}} e^{-i \hat H \delta t/\hbar} \ket{x_r} = e^{-i V(x_r) \delta t/\hbar} K_0(x_{r+1}, x_r; \delta t)
$$
Putting all these pieces together:
\begin{equation}
        K(x,x_0;T) = \int [ \prod_r dx_r] (\frac{m}{2\pi i \hbar \delta t})^{\frac{n+1}{2}} \exp( i \sum_{r=0}^n [ \frac{m}{2\hbar}( \frac{x_{r+1} - x_r}{\delta t})^2  - \frac{1}{\hbar} V(x)] \delta t)
\end{equation}
In the limit $n\rightarrow a, \delta t \rightarrow 0$ the exponent becomes $\frac{1}{\hbar} \int_0^T dt [ \frac{1}{2} m \dot x^2 - V(x)] = \int_0^T dt \mathfrak{L}(x,\dot x)$. So this is classical action in the limit.\\\\
We have no found a path integral (functional integral)
$$
K(x, x_0; T) = \bra{x} e^{- i \hat H T /\hbar} \ket{x} = \int \mathcal{D}x e^{i S/\hbar}
$$
where $$\mathcal{D}x = \lim_{\delta t \rightarrow 0, n \delta t = T} \sqrt{\frac{m}{2 \pi i \hbar \delta t}} \prod_{r=1}^n ( \sqrt{\frac{m}{2 \pi i \hbar \delta t}} dx_r)$$
One way of considering the classic limit is taking $\hbar \rightarrow 0$. For $e^{iS/\hbar}$ thisincreases the phases/frequencies. The Riemann-Lebesque lemma implies tha thte smallest frequncy (i.e. the path whihc minimises $S$ dominates the integral). As smallest $S$ is the hamiltonians principle of least action so this is equivalent to the classical treatment. \\\\
Another way is for $\hbar \neq 0$ the QM amplitude is the sum of all paths each weighted by phase $e^{iS/\hbar}$. This gives the interfence patterns we see in double slit etc.\\\\
One trick we are going to play is dealing in imaginary time. You can analytically continue to imaginary time. Let $\tau = it$, then $\bra{x} e^{- \hat H t/\hbar} \ket{x_0} = \int \mathcal{D}x e^{-S/\hbar}$. Here the $\hbar \rightarrow 0$ argument is much more clear as the smallest value of $S$ will dominate in the limit. Analogy with statistical mechanics where $e^{-S /\hbar}$ is a botlzmann factor, and $\int \mathcal{D}x $ is the sum over microstates. These (with real exponentials) converge. Not all quantum questions can be answer in imaginary time. e.g. if there is a causality relationship between the initial and final space as we have convereted from mincovski to euclian. 
\end{document}
