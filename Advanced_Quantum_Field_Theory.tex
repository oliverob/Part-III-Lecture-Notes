\documentclass{article}
\usepackage[utf8]{inputenc}
\usepackage{bm}
\usepackage{amssymb}
\usepackage{amsmath}
\usepackage{braket}
\usepackage{cancel}
\title{Quantum Information Theory}
\author{oliverobrien111 }
\date{July 2021}

\begin{document}

\maketitle

\section{Lecture 1}
\subsection{Path Integrals in Quantum Mechanics}
Goal is to reformulate Schrodingers equation as a path integral.
$$
\hat H(\hat x, \hat p) \text{ with } [\hat x , \hat p] = i \hbar
$$
Assuming $\hat H = \frac{\hat p^2}{2m} + V(\hat x)$. The schrodinger picture is:
$$
i \hbar \frac{d}{dt} \ket{\psi(t)} = \hat H \ket{\psi(t)}
$$
so
$$
\ket{\psi(t)} = e^{-i \hat H t/\hbar} \ket{\psi(t)}
$$
Wavefunction: $\Psi(x,t) = \bra{x} \ket{\psi(t)}$. We want to solve schordingers equation for this wavefucniton in a way that introduces the path integral:
$$
\Psi(x,t) = \bra{x} \ket{\psi(t)} = \bra{x} e^{- \hat H t/\hbar} \ket{\psi(0)}
$$
$$
\Psi(x,t) = \int_{-\infty}^{\infty} K(x, x_0;t) \Psi(x_0, 0)
$$
where
$$
K(x, x_0;t) = \ket{x} e^{-i \hat Ht/ \hbar} \ket{x_0}
$$
is what we want a path integral expression for\\\\
Lets consider $n$ intermediate times/positions. Let $0 =t_0 <t_1 < t_2 <\ldots < t_n < t_{n+1} = T$:
$$
e^{-i \hat H T/\hbar} = e^{-i \hat H(t_{n+1}-t_n)/\hbar}\cdotse^{-i \hat H(t_{1}-t_0)/\hbar}
$$
Insert identity: $I = \int dx_r \ket{x_r}\bra{x_r}$:
$$
K(x, x_0;t) = \int_{-\infty}^{\infty} \left( \prod_{r=1}^n dx_r \bra{x_{r+1}}e^{-i\hat H(t_{r+1}-t_r)/\hbar} \ket{x_r}\right) \bra{x_1} e^{-i\hat H(t_1 - t_0)/\hbar} \ket{x_0}
$$
Consider fixed $V(\hat x) =0$. $K_0(x, x'; t) = \bra{x} e^{-i \frac{ \hat p^2}{2m \hbar} t} \ket{x'}$. Insert the identity: $I = \int \frac{dp}{2\pi \hbar} \ket{p}\bra{p}$.
$$
K_0(x,x';t) = e^{\frac{im(x-x')^2}{2 \hbar t}} \sqrt{\frac{m}{2\pi i \hbar t}}
$$
For $V(\hat x) \neq 0$, we need very small time steps. Separate kinetic and potential parts (Suzuki-Trotter decomposition). Take $t_{r+1} - t_r = \delta t$ to be small and $n$ large so $n\delta t = T$ (constant).
$$
e^{-i \hat H \delta t / \hbar} = \exp( - \frac{i \hat p^2 \delta t}{2\pi \hbar}) \exp( - \frac{iV(\hat x) \delta t}{\hbar}) ( 1+ O((\delta t)^2))
$$
The last term here is vanishingly small under the Broker-Campbell-Henshorff thingy. Between any 2 position eigenstates:
$$
\bra{x_{r+1}} e^{-i \hat H \delta t/\hbar} \ket{x_r} = e^{-i V(x_r) \delta t/\hbar} K_0(x_{r+1}, x_r; \delta t)
$$
Putting all these pieces together:
\begin{equation}
        K(x,x_0;T) = \int [ \prod_r dx_r] (\frac{m}{2\pi i \hbar \delta t})^{\frac{n+1}{2}} \exp( i \sum_{r=0}^n [ \frac{m}{2\hbar}( \frac{x_{r+1} - x_r}{\delta t})^2  - \frac{1}{\hbar} V(x)] \delta t)
\end{equation}
In the limit $n\rightarrow a, \delta t \rightarrow 0$ the exponent becomes $\frac{1}{\hbar} \int_0^T dt [ \frac{1}{2} m \dot x^2 - V(x)] = \int_0^T dt \mathfrak{L}(x,\dot x)$. So this is classical action in the limit.\\\\
We have no found a path integral (functional integral)
$$
K(x, x_0; T) = \bra{x} e^{- i \hat H T /\hbar} \ket{x} = \int \mathcal{D}x e^{i S/\hbar}
$$
where $$\mathcal{D}x = \lim_{\delta t \rightarrow 0, n \delta t = T} \sqrt{\frac{m}{2 \pi i \hbar \delta t}} \prod_{r=1}^n ( \sqrt{\frac{m}{2 \pi i \hbar \delta t}} dx_r)$$
One way of considering the classic limit is taking $\hbar \rightarrow 0$. For $e^{iS/\hbar}$ thisincreases the phases/frequencies. The Riemann-Lebesque lemma implies tha thte smallest frequncy (i.e. the path whihc minimises $S$ dominates the integral). As smallest $S$ is the hamiltonians principle of least action so this is equivalent to the classical treatment. \\\\
Another way is for $\hbar \neq 0$ the QM amplitude is the sum of all paths each weighted by phase $e^{iS/\hbar}$. This gives the interfence patterns we see in double sli, which is just represented by some classical line with no further diagrams.etc.\\\\
One trick we are going to play is dealing in imaginary time. You can analytically continue to imaginary time. Let $\tau = it$, then $\bra{x} e^{- \hat H t/\hbar} \ket{x_0} = \int \mathcal{D}x e^{-S/\hbar}$. Here the $\hbar \rightarrow 0$ argument is much more clear as the smallest value of $S$ will dominate in the limit. Analogy with statistical mechanics where $e^{-S /\hbar}$ is a botlzmann factor, and $\int \mathcal{D}x $ is the sum over microstates. These (with real exponentials) converge. Not all quantum questions can be answer in imaginary time. e.g. if there is a causality relationship between the initial and final space as we have convereted from mincovski to euclian.
\section{Lecture 2}
We showed that in quantum mechanics a path integral over positions weighted by the classical action:
$$
\int \mathcal{D}x e^{i S[x]/ \pi}
$$
this came from non-relativisitic quantum mechanics whre the position is an operator. As we saw in quantum field theory this mixed treatment of space as an operator and time as a label is not appropriate for satisfying lorentz invariance, so we demote x to be a label so that space and time are treated the same. So we work with the appropriate fields. For much of this course we will work with scalar fields and then will generalise to fermionic fields and gauge fields. QM is 0+1 dimensional field theory.
\subsection{Integrals and their diagram attic expansions}
Goal of next couple lectures is to show mathematics and show that they generate the same diagrams as in QFT (have to take it on a little bit of faith will become clear towards the end of the course). We supress the interesting relationships between space and tiem for the this chapter and just treat them as labels.\\\\
0-dimentsional field: $p: \{ \text{point} \} \rightarrow \mathbb{R}$ $q$ real variable\\\\
Path integrals as if in imaginary time (which makes the integrals better behaved as we get expoentially decaying factors rather than complex integrands):
$$
Z = \int_{\mathbb{R}} d\phi e^{- S(\phi)/\hbar}
$$
For the purposes here just assume it is well-behaved. So assume it is an even polynomial so that as $\phi\rightarrow\pm \infty$ we have $S[\phi] \rightarrow \infty$. Also look at expectation values:
$$
< f(\phi) > = \frac{1}{Z} \int d\phi f(\phi) e^{-S(\phi)/\hbar}
$$
This are sometimes refeerred to as correlation functions. Assume that $f$ does not grow so much it overwhelms the expoential so this is well behaved. \\\\
Now write down action corresponding to the free field theory which we can write down exactly and then we will do one that needs pertubation theory expansion.
\subsubsection{Free theory}
say we have $N$ real scalar fields (variables). Let $a, b \in [1,N]$:
$$
S[\phi] = \frac{1}{2} m_{ab} \phi_a \phi_b = \frac{1}{2} \phi^T m \phi
$$
with $m$ symmetric and positive definite ( $det m > 0$). If $m$ is diagonal it would obviously be a mass term, but it could also couple nearest neighbours and so could contain a discrete approximation to a derivate (so could represent difference operators on a discrte lattice but this isn't important today).\\\\
We can diagonalise $m$ with orthogonal matrices $P$ as $m$ is symmetric and positive definite:
$$
m = P \Lambda P^T
$$
$\Lambda$ is diagonal with elements $\lambda_c$ with $c \in [1,N]$. Let $\chi = P^T Q$, then:
$$
Z_0 = \int d^N \phi \exp( - \frac{1}{2\pi} \phi^T m \phi) = \prod_{c} \sqrt{ \frac{2 \pi \hbar}{\lambda_c}} = \sqrt{ \frac{ (2\pi \hbar)^N}{det M}}
$$
We will need to play some tricks when we do fermionic fields as they are antisymmetric not symmetric so you end up with the detminant in the numerator rahter than the denominator. \\\\
To go from the partition function $Z_0$ to correlation function $f^{\bm n}$ we introduce an external source $J$ (with N compononents) and replace the action $S_0(\phi)$ with $S_0(\phi) - J^T \phi$.
$$
Z_0(J) = \int d^N \phi \exp( -\frac{1}{2 \hbar} \phi^T m \phi + \frac{1}{\hbar} J^T \phi)
$$
Let $\tilde \phi = \phi - m^{-1} J$:
$$
Z_0(J) = \int d^N \phi \exp( - \frac{1}{2 \hbar} \tilde \phi^T m \tilde \phi) \exp( \frac{1}{2\hbar} J^T m^{-1} J)
$$
This is called the generating function or generating functional as it generates the correlation functions.\\\\
\textbf{Example: Generate '2-point' function}:\\
$$
< \phi_a \phi_b> = \frac{1}{Z_0(0)} \int d^N \phi \phi_a \phi_b \exp( - \frac{1}{2\hbar} \phi^T m \phi + \frac{1}{\hbar} J^T \phi)|_{J=0} = \frac{1}{Z_0(0)} \int d^N \phi ( \hbar \frac{\partial}{\partial J_a}) (\hbar \frac{\partial}{\partial J_b}) \exp ( - \frac{1}{2\hbar} \phi^T m \phi + \frac{1}{\hbar} J^T \phi)|_{J=0}
$$
Now the $\phi$ depedance is only in the exponential so can bring out derivatives:
$$
< \phi_a \phi_b> = \hbar^2\frac{\partial}{\partial J_a} \frac{\partial}{\partial J_b} Z_0(J)|_{J=0} = \hbar^2\frac{\partial}{\partial J_a} \frac{\partial}{\partial J_b} Z_0(0) \exp( \frac{1}{2\hbar} J^T m^{-1} J)|_{J=0} = \hbar (m^{-1})_{ab} 
$$
We represent this as a line between two points $a$ and $b$ also called a propagator.\\
\textbf{Example: '4-point' function}\\
$$
< \phi_b \phi_c \phi_d \phi_f > = \hbar^2[ (m^{-1})_{bc} (m^{-1})_{bc} 
 + (m^{-1})_{bd} (m^{-1})_{cf} + (m^{-1})_{bf} (m^{-1})_{cd} ]
$$
These represent the three ways of linking up 4 points with 2 lines. For $2k$ field in <> then there should be $\frac{(2k)!}{2^k k!} = \frac{ \text{permutate all 2k points}}{\text{ (permutte all points inside pairs) (permute pairs)}}$ diagrams.
\subsubsection{Interacting theory}
We just want to go beyond the action we have written down to something a bit more complicated. In cases where exact intergration is not possible. So we seek an expansion about a classical point, with small $\hbar$. Integrals don't end up being convergent they are asymptotic. e.g.
$$
\int d^N \phi f(\phi) e^{-S/\hbar}
$$
wont have a Taylor expansion about $\hbar = 0$. All is not lost we can still make progress. The expansion we are going to look at in many cases is asymptotic which means that the various terms in the series get to be better and better approximations to the full results at least up to a point.
$$
I(\hbar) \sim \sum_{n=0}^{\infty} c_n \hbar^n
$$
iff
$$
\lim_{\hbar \rightarrow 0^+} \frac{1}{\hbar^N} | I(\hbar) - \sum_{n=0}^N c_n \hbar^n | = 0
$$
Series missed out some terms $e^{-\frac{1}{\hbar^2}}$ so there are non-perturbative effects. We won't cover it in this course but these terms do contribute effects in some gauge theories. In some weakly coupled thoeries like QED this is a very good expansions, as shown by how we can very accurate measure magnetic moment of an electron to $10^{-10}$ accuracy.
\section{Lecture 3}
Last time we looked at
$$S(\phi) = \frac{1}{2} m^2 \phi^2 + \frac{\lambda}{4!} \phi^4 = S_0(\phi) + \frac{\lambda}{4!} \phi^4, m^2 > 0, \lambda >0$$
Partition $f^1$ ( generating function with respect to $J=0$):
$$
Z = \int d\phi e^{- S/ \hbar}
$$
Seperate this action into the free part and the interaction part and then expand around classical fields
$$
Z = \int d \phi e^{- S_0 / \hbar} \sum_{v=0}^{\infty} \frac{1}{v!} [ (- \frac{\lambda}{4!\hbar}) \phi^4 ]^v
$$
Truncate, swap order of sum and integral to give an asymptotic expansion.\\\\
In 0-dimensions we can integrate exactly. Let $x = \frac{1}{2\hbar} m^2 \phi^2$:
$$
Z = \frac{\sqrt{2\hbar}}{m} \sum_{v=0}^N \frac{\hbar}{V!} (- \frac{\hbar \lambda}{4! m^4})^v 2^v \int_0^{\infty} e^{-x} x^{2v + \frac{1}{2} -1} = \frac{\sqrt{2\hbar}}{m} \sum_{v=0}^N \frac{\hbar}{V!} (- \frac{\hbar \lambda}{4! m^4})^v 2^v \Gamma(2v + \frac{1}{2})
$$
$$
Z =  \frac{\sqrt{2\hbar}}{m} \sum_{v=0}^N \frac{\hbar}{V!} (- \frac{\hbar \lambda}{ m^4})^v \frac{1}{(4!)^v v!} \frac{(4v)!}{2^{2v}(2v)!}$$
Use stirlings approximation:
$$
V! \sim e^{V \log V}
$$
So this factorial growth is very fast and so will make this an asympotic expansion as it means the terms will not converge. We can see that $\frac{1}{(4!)^v v!} $ comes from expanding $\exp( - S_I / \hbar)$, whereas $\frac{(4v)!}{2^{2v}(2v)!}$ comes from the combinatoric ways of pairing the $4v$ fields.\\\\
\subsubsection{Generating function}
$$
Z(J) = \int d\phi \exp( - \frac{1}{\hbar} [ S_0(\phi) + S_I(\phi) - J \phi] )
$$
Taylor expand $e^{-S_I/\hbar}$ then replace the $\phi$ with $\hbar \frac{\partial}{\partial J}$ then pull it out and sum that infinite series
$$
Z(J) = \exp( - \frac{1}{\hbar} S_I(\hbar \frac{\partial}{\partial_J}) )\int d\phi \exp ( - \frac{1}{\hbar} [S_0(\phi) - J \phi])
$$
Drop multiplicative factor $\exp( - \frac{\lambda}{4! \hbar}( \hbar \frac{\partial }{\partial J})^4) \exp ( \frac{1}{2\hbar} J^T M^{-1} J)$. Here $J$ is just a signle variable $M = m^2$.
\begin{equation}
Z(J) = \sum_{v= 0}^N \frac{1}{V!} [ - \frac{\lambda}{4!\hbar} (\hbar \frac{\partial}{\partial J})^4 ]^v \sum_{p=0} \frac{1}{p!} ( \frac{1}{2 \hbar} J^T M^{-1} J)^p
\end{equation}
We represent this double series by diagrams of propogators and vertices. A propogrator is a line connecting to field. The propogator is $M^{-1}$ which for the moment is just a boring $m^{-2}$. At each vertex we have: $- \frac{\lambda}{\hbar} ( \hbar \frac{\partial}{\partial J})^4$.\\\\
Check $Z(0)$. In order for a term to survive to be nonzero we have to match up the derivatives with the $J$s. When $J=0$, need number of derivates to be equal to the number of sources. Generally, lets call these external sources. e.g. in the case of this field $E = 4 v - 2p$ and we want $E=0$. First nontrivial terms are $(v,p) = (1,2)$ and $(2,4)$. So:
$$
Z(0) = 1 + \text{figure eight} + ...
$$
Product rule in differentiation turns into symmetry factors or combinatorial factors associated with each diagram. Think about the "pre-diagram" with the half edges (corresponding to derivatives) and floating propogators (with ends corresponding to sources) and then we label the sources so there are $4!$ ways of assigning 4 derivates (or half edges) to 4 sources (at $a, a', b, b'$). \\\\
The numerator $A$ is cancelled by denominator $F$. 
$$
F = (v!)(4!)^v (P!)2^p $$
$F$ accounts for all the permuations of all vertices $V!$, and each vertex's legs and all propogators $P!$ and both ends of the propogators $2!$. After dividing $A$ by $F$ we get the symmetry factor. This is important to remove double counting. As some of the ways of assigning derivates to sources are equivalent. Looking for the number of distinct ways of mapping from the half edges to themselves whilst preserving the graph but creating a distigusible set of half edge assignations. Number of ways of changing the labels but keeping the same graph.
\section{Lecture 4}
Now we want to look at diagrams with external terms e.g. $E=2$ terms in the idagrammatic expansion. In the generating function:
$$
Z(J) > | + |o + | \infty + |^o_o + |oo ...
$$
We get both the vaccuum bubbles and the connected diagrams, so can factor out the vacuum bubbles:
$$
Z(J) > (| + 8 + ...)( | + |o + ...)
$$
When we go to calculate expectation values:
$$
< \phi^2 > = \frac{\hbar^2}{Z(0)} )\frac{\partial }{\partial J)^2} Z(J) |_{J=0} = \frac{1}{Z(0)} (| + |o + | 8 + ...) = (| + |0 + |oo)
$$
The generalization is straightforward
$$
< \phi^4 > = || + X + |o o| + ...
$$
\subsection{Effective actions}
In this section we show that we only need to consider the connected vaccum bubbles as the action only deals in the sums of connected bubbles rather than $Z$ which is the sum of all vaccumm bubbles.\\\\
We will show that $W = - \frac{1}{\hbar} \log Z$ Wilson effective action is sum of connected vacuum diagrams. Any diagram $D$ as products of connected diagrams $ D= \frac{1}{S_D}\sum_I(c_I)^{n_I}$
$$
\frac{Z}{Z_0} = \sum_{\{ n_I\}} D = \prod_I \sum_{n_I} (c_I)^{n_I} = \exp ( \sum_I c_I ) = e^{- (W-W_0)/ \hbar}
$$
$$
W= W_0 - \hbar \sum_I C_I
$$
Introduce external sources
$$
- \frac{1}{\hbar} W(J) = \log Z(J)
$$
$$
- \frac{1}{\hbar} \frac{\partial ^2}{\partial J^"} W |_{J=0} = \frac{1}{Z(0)} \frac{\partial^2 Z}{\partial J^2} |_{J=0} - \frac{1}{(Z(0)^2} (\frac{\partial Z}{\partial J}_{J=0} )^2 = \frac{1}{\hbar} ( < \phi^2> -< \phi>^2)
$$
In our theory for even actions the second term is zero. Above you can see we are finding the two point functions and then subtracting off the disconnected one point functions.\\\\
\textbf{Why is W "effective"}:\\
Consider a theory with 2 scalars $\phi$ and $\chi$:
$$
S(\phi, \chi) = \frac{m^2}{2} \phi^2 + \frac{M^2}{2} \chi^2 + \frac{\lambda}{4} \phi^2 \chi^2
$$
Feynmann rules:\\
Propogrators of $\frac{\hbar}{m^2}$ and $\frac{\hbar}{M^2}$, vertex rules of $- \frac{\lambda}{\hbar}$.
Therefore, $- \frac{W}{\hbar} = $ sum of connected diagrams with two dashed lines and two solid lines in coming out of each vertex.
$$
< \phi^2> = | + |0 + (|) + ...
$$
Look in typed notes to see how this works it is too difficult to latex the diagrams. If we want to remove the $\chi$ field then. Define $W(\phi)$
$$
e^{- W(\phi) /\hbar} = \int d\chi e^{- S(\phi, \chi)/\hbar}
$$
Treat $\chi^2 \phi^2$ term as a source for $\chi^2$ ( $J = - \chi^2$) so the correlation funciton of the $\phi$ fields:
$$
< f( \phi)> = \frac{1}{Z} \int d\phi d\chi f(\phi) e^{- S(\phi, \chi)} = \frac{1}{Z} \int d\phi f(\phi) e^{- W(\phi) / \hbar}
$$
As this is simple theory we can use exact integration:
$$
\int d\chi e^{- S(\phi, \chi)/\hbar} = e^{- m^2 \phi^2 /2 \hbar} \sqrt{ \frac{ 2\pi \hbar}{ M^2 + \frac{\lambda \phi^2}{2}}}
$$
$$
W(\phi) = \frac{1}{2} m^2 \phi^2 + \frac{\hbar}{2} \log (1 + \frac{\lambda}{2 M^2} \phi^2) + \frac{\hbar}{2} \log \frac{M^2}{2\pi \hbar}
$$
$$
W(\phi) = ( \frac{m^2}{2} + \frac{\hbar \lambda}{4M^2} )\phi^2 - \frac{\hbar \lambda^2}{16 m^4} \phi^4 + \frac{\hbar \lambda^3}{48 M^6} \phi^6 + ...
$$
 $$
 W(\phi) = \frac{m^2_{eff}}{2} \phi^2 + \frac{\lambda_4}{4!} \phi^4 + .. + \frac{\lambda^6}{6!} \phi^6
 $$
 \section{Lecture 5}
 Now we do it perturbatively using diagrams. Treat $\frac{\lambda}{4} \phi^2 \chi^2$ as a source term:
 $$
 Z = \int d \phi e^{- m^2 \phi^2/2\hbar} \int d\chi \exp ( - \frac{1}{\hbar}( \frac{M^2}{2} \chi^2 - J \chi^2))
 $$
 $J = - \frac{\lambda}{4} \phi^2$
 This leads to the Feynman rules with the propogator of $\frac{\hbar}{M^2}$ and we have self interaction terms with $- \frac{\lambda \phi^2{2 \hbar}}$ from the $\chi^2$ source term. Now we can find the efffective action (given by the sum of the connected diagrams):
 $$
 W(\phi) =  - \hbar(....)
 $$
 (above is all the connected diagrams with more and more vertices each with two legs so effectively looks like the increasing roots of unity)
 $$
 W(\phi) = \frac{m^2}{\phi^2}{2} - \frac{1}{2} \frac{\hbar\lambda}{2 M^2} \phi^2 - \frac{1}{4} \frac{\hbar \lambda^2}{4 M^4} \phi^4 + \frac{1}{3!} \frac{\hbar \lambda^2}{8 M^6} \phi^6 + ..= \frac{m^2_{\eff}}{2} \phi^2 + \frac{\lambda_4}{4!} \phi^4 + ...
 $$
 This is the same result as before. Now we have a theory of how the $\phi$ interacts. Now lets do the calculation with the full theory where we keep the $\phi$ explictly. Using $W(\phi)$:
 $$
 <\phi^2> = \frac{1}{Z} \int d\phi \phi^2 e^{- W(\phi) / \hbar} = | + |o + ... = \frac{\hbar}{m^2_{eff} } - \frac{\lambda_4 \hbar^2}{2 m^6_{eff}} + ...
 $$
This agrees with the full caculation from earlier. We have shown that given a theory of two fields we can intergrate out one of them to get a theory in just one field, this changes the degrees of freedom which therefore changes the coefficent.\\\\
Lets continue on with effective actions, now we want to introduce the quantum effective action\subsubsection{Quantum effective action}
Define average field in the presence of an external source $ < \phi> = \Phi = - \frac{\partial W}{\partial J} = \frac{\hbar}{Z(J)} \frac{\partial}{\partial J} \int d\phi e^{-(S-J\phi)/\hbar}$.
Legendre transform is a transformation from treating the source $J$ as the independant variable to treating the mean field $\Phi$ as the independant variable
$$
\Gamma (\Phi) = W(J_{\Phi}) + \Phi J_{\Phi}
$$
$J_{\Phi}$ is the $J$ which gives the correct expression $\frac{\partial W}{\partial J}|_{J_{\Phi}} = - \Phi$. Lets find the derivative:
$$
\frac{\partial \Gamma}{\partial \Phi} = \frac{\partial W}{\partial \Phi} + J_{\Phi} + \Phi \frac{\partial J_{\Phi}}{\partial \Phi} = - \frac{\partial W}{\partial J}_{J_{\Phi}} \frac{\partial J}{\partial \Phi} + J_{\Phi} + \Phi \frac{\parital J_{\Phi}}{\partial \Phi} 
$$
$$
\frac{\partial \Gamma}{\partial \Phi} = J_{\Phi}
$$
\subsection{$\Gamma(\Phi)$ and Feynmann diagrams}
External lines have one free end, whereas internal lines have no free ends. A bridge is any internal line of a connected graph which if cut would make the graph disconnected. A connected graph is said to be one-particle irreducible (1PI) iff it has no bridges. We are interested in these single particle irreducible graphs. Statement that we want to prove is that $\Gamma(\Phi)$ sums the 1PI graphs of the theory. We might expand about $\Phi = \Phi_0 (J = 0)$. Let $\varphi = \Phi$:
$$
\Gamma(\Phi) = \Gamma^{(0)} + \frac{1}{2} \Gamma^{(2)} \varphi^2 + ... + \frac{1}{n!} \Gamma^{(n)} \varphi^n
$$
Treat $\Phi$ as we did $\phi$ earlier. The quantum path integral for $\Phi$:
$$
e^{-W_{\Gamma}(J)/g}\int d\Phi e^{- (\Gamma(\Phi) - J \Phi)/g}
$$
$g$ is fictious planck constant.
$$
W_{\Gamma}(J) = \text{ sum of connected diagrams } =  \sum_{l=0} g^l W_{\Gamma}^{(l)}(J)
$$
In $g \rightarrrow 0$ limit $W_{\Gamma}(J) = W_{\Gamma}(J) = \Gamma(\Phi) - J \Phi$ "classical" action of the path integral ($W(J)$) which is the effective action of the original theory. $W(J) = -\hbar \log \int d \phi e^{- S(\phi)/\hbar}$
$$
W(J) = - \hbar \log (\int d\phi e^{-(S(\phi) - J\phi)/\hbar)}
$$
The sum of connected diagrams with rules derived from action $S(\phi) - J\phi$ can be obtained as the sum of tree diagrams (no loops only bridges) using $W_{\Gamma}(J)$ rules derived from action $\Gamma(\Phi) - J\Phi$. Therefore the diagrams in the original theory that don't contain any bridges have to be absorbed into the coefficents of the tree diagrams of $W_{\Gamma}(J)$.\\\\
In a theory with several scalar fields: $\phi_a$ $a = 1,..., N$:
$$
<\phi_a \phi_b>_J^{conn} = < \phi_a \phi_b>_J - <\phi_a> <\phi_b> = - \hbar \frac{\partial^2 W}{\partial J_a \partial J_b} = - \hbar \frac{\partial}{\partial J_a}( \frac{\partial W}{\partial J_b}) = \hbar \frac{\partial}{\partial J_a} \phi_B = \hbar (\frac{\partial J_a}{\partial \Phi_b})^{-1} = \hbar (\frac{\partial^2 \Gamma}{\partial \Phi_a \partial \Phi_b})^{-1}
$$
The point is that the two point function which we know in the original thepry working with action $S(\phi)$ we have to sum the diagrams with two external legs so it starts of as just a single propogator plus the loop contributions. This tells us that if we know the quantum effective action then we can just read off the two point function, which is just represented by some classical line with no further diagrams.\\\\
What is basically happening here is we can split every diagram into irreducible parts and each of them can be expressed as a single vertex with the correct them of external lines and will represent all the possible way it can be done like with loads of loops and stuff etc. so basically we can hugely simplify the number of diagrams. We have to figure out what the vertex is.
\section{Lecture 6}
Given $$\Gamma(\Phi) = \Gamma^{(0)} + \frac{1}{2} \Gamma^{(2)}( \Phi - \Phi_0)^2 + ... + \frac{1}{n!} \Gamma^{(n)}(\Phi- \Phi_0)^n +...$$ we have
$$
< \phi_a \phi_b>_J^{conn} = \hbar( \frac{\partial^2 \Gamma}{\partial \Phi_a \partial \Phi_b})^{-1} = \hbar ( \Gamma^{(0)})^{-1}
$$
in $\phi^4$ theory:
$$
< \phi_a \phi_b>_J^{conn} = -- \,+\, -o- \,+\, -o-o- \,+\, -8- \,+ \,- \theta - \,+ ..
$$
$$=  \,--\, +\, -IPI- \,+ \,-IPI-IPI- \,+ \,-IPI-IPI-IPI- =  \frac{1}{1- IPI}
$$
intuitively the last step comes from recognising the geometric series. On the example sheet we will do the same for three point function:
$$
< \phi_a \phi_b \phi_c >_J^{conn} = - \frac{1}{\hbar} ( \frac{\partial^3 \Gamma}{\partial \Phi_d \parital \Phi_e \partial_f})^{-1} <\Phi_a \Phi_d> <\Phi_b \Phi_e> <\Phi_c \Phi_f> 
$$
$$
( \frac{\partial^3 \Gamma}{\partial \Phi_d \parital \Phi_e \partial_f})^{-1}  = \Gamma^{(3)} = - <\Phi_d \Phi_e \Phi_F> ^{IPI}
$$
This is still trivial as you cant imagine having a non trivial bridge so we go to one higher level . If we go to a four point function we could have a bridge within the internal interactions e.g. it could be two 2 particle interactions rather than a single 4 particle interaction. Remember every external line has a two point interaction 1PI on it as there is a particle coming in a one coming out. This section is not in the written notes. It is slightly discussed later on page 41.
\subsection{Fermions}
Now have anticommutation relations rather than commutation relations. We introduce the abstract conscept of Grassmann numbers which are anti-commuting variables.\\
$n$ numbers $\{ \theta_a \} a = 1,..., n$ and they obey:
$$
\theta_a \theta_b = - \theta_b \theta_a \implies \theta_a^2 = 0
$$
for any scalar $\phi_b \in \mathbb{C}$
$$
\theta_a \phi_b = \phi_b \theta_a
$$
Functions can be expressed as finite sums:
$$
f(\theta) = f + \rho_a \theta_a + \frac{1}{2!} g_{ab} \theta_a \theta_b + ... + \frac{1}{n!}h_{a_1a_2...a_n} \theta_{a_1} \theta_{a_2} ... \theta_{a_n}
$$
where $g_{ab},..., h_{a_1...a_n}$ etc. are anti-symmetric in their indicies. This series is finite as adding any more $\theta$s would give a $\theta_a^2=0$ and vanish the term. Note
$$
(\theta_a \theta_b) (\theta_c \theta_d) = (\theta_c \theta_d) (\theta_a \theta_b)
$$
Differentiation is defined to anti commute:
$$
\frac{\partial}{\partial \theta_a} \theta_b + \theta_b \frac{\partial}{\partial \theta_a} = \delta_{ab}, \frac{\partial}{\partial \theta_a}( \theta_b F(\theta)) = \delta_{ab} F(\theta) - \theta_b \frac{\partial F}{\partial \theta_a}
$$
Integration: For a single grassamann $\theta$
$$
F(\theta) = f+ \rho \theta
$$
Define $\int d\theta$ and $\int \theta d\theta$. We want to require translational invariance and consider what it means to intergate over the whole range of $\theta$. So impose requirement that
$$
\int d \theta( \theta + \eta) = \int \theta d\theta 
$$
for constant grossman variable $\eta$. This implies that we want $\int d\theta = 0$ and we want $\int \theta d\theta = 1$ (this includes a choice of normalization). These are called the "Berezin rules". \\\\
Integration by parts is simplified:
$$
\int d\theta \frac{\partial}{\partial \theta} F(\theta) = 0
$$
So we have intoduced the rules for one variable now lets consider $n$ Grassmann variables $\theta_a$. In this case the only non-vanishing integral have to have one and only one $\theta$:
$$
\int d^n \theta \theta_1 \theta_2... \theta_n = 1 \iff \int d \theta_n d\theta_{n-1} ... d\theta_1 \theta_1 \theta_2... \theta_n = 1
$$
A key point is in general if we start commuting these indicies we are going to start picking up signs so
$$
\int d^n \theta \theta_{a_1} \theta_{a_2} ... \theta_{a_n} = \epsilon_{a_1 a_2... a_n}
$$
Now let use consider a change of variables $\theta' = X_{ab} \theta_b$ then:
$$
\theta_a' = X_{ab} \theta_b, X_{ab} \in \mathbb{C}
$$
$$
\int d^n \theta' \theta_{a_1} ... \theta_{a_n} = X_{a_1 b_1} ... X_{a_n b_n}  \int d^n \theta \theta_{b_1}... \theta_{b_n} = X_{a_1 b_1} ... X_{a_n b_n}   \epsilon^{b_1... b_n} = \det X \epsilon^{a_1...a_n} = \det X \int d^n \theta \theta_{a_1}... \theta_{a_n}
$$
therefore $d^n \theta = det X d^n \theta'$ which compared to scalars where we have $\phi' = Y \phi \implies d^n \phi = \frac{1}{\det Y} d^n \phi'$. So fermions give the converse relationship here to what you would expect from scalars.
\subsection{Free fermion field theory}
$d = 0$, 2 fields $\theta_1, \theta_2$. Need a scalar action, only non-constant action
$$
S(\theta) = \frac{1}{2} A \theta_1 \theta_2, A \in \mathbb{R}
$$
$$
Z_0 = \in d^2 \theta e^{-S(\theta)/\hbar} = \int d^2 \theta ( 1- \frac{A}{2\hbar} \theta_1 \theta_2) = - \frac{A}{2 \hbar}
$$
$n = 2m$ fields
$$
S= \frac{1}{2} A_{ab} \theta_a \theta_b
$$
$ A_{ab}$ is antisymmetric matrix
$$
Z_{0} = \int d\theta^{2m} e^{-S /\hbar} = \int d^{2m} \theta \sum_{j=0}^m \frac{(-1)^j}{(2\hbar)^j j!} ( A_{ab} \theta_a \theta_b)^j
$$
$$
Z_0 = \frac{(-1)^m}{(2\hbar)^m m!} \epsilon^{a_1...a_n} A_{a_1a_2} ... A_{a_{2m-1} a_{2m}} = \frac{(-1)^m}{\hbar} Pf(A) = \pm \sqrt{\frac{\det A}{\hbar^n}}
$$
$Pf(A)$ is the Pfaffian.
\subsubsection{External sources}
Need to be grassman valued $\eta$:
$$
S(\theta, \eta) = \frac{1}{2} A_{ab} \theta_a \theta_b - \eta_a \theta_a
$$
Complete the square, using translation invariance to get the following integral that we can then do:
$$
Z_0(\eta) = \exp( - \frac{1}{2\hbar} \eta^{T} A^{-1} \eta) Z_0(0)
$$
Propogator
$$
< \theta_a \theta_b> = \frac{\hbar^2}{Z_0(0)} \frac{\partial^2 Z_0(\eta)}{\partial \eta_a \partial \eta_b}|_{\eta=0} = \hbar ( A^{-1})_{ab}
$$
\section{Lecture 7}
\subsection{LSZ reduction formula}
This discussion is quite general and not very connected to the earlier section.\\\\
We are going to work through 2-2 scatting of $\phi$ particles. From last term:
$$
\phi(x) = \int \frac{d^3k}{(2\pi)^3 2 E_b} ( a(\bm k) e^{-i k \cdot  x} + a^{\dagger}(\bm k) e^{i  k \cdot x})
$$
We are using the Minkowski metric $(+, -, - ,- )$ so $k \cdot x = E_0 t - \bm k \cdot \bm x$. There is also a convention for the normalisation. We are taking the realtivistic normalisation for $a(\bm k)$. It depends on whether you want the inner product of two functions to be the delta funcitno or the dealt function times $E$.\\\\
We invert to find $a(\bm k)$:
$$
\int d^3 x e^{ik \cdot x} \phi(x) = \frac{1}{2E} a(\bm k) + \frac{1}{2E} e^{2 i Et} a^{\dagger} ( - \bm k)
$$
$$
\int d^3 x e^{ik\cdot x} \partial_a \phi(x) = \frac{-i}{2} a(\bm k) + \frac{i}{2} e^{2 i Et} a^{\dagger} (- \bm k)
$$
These imply:
$$
a(\bm k) = \int d^3 x e^{i k\cdot x} ( i \partial _a \phi(x) + E\phi(x))
$$
$$
a^{\dagger}(\bm k) = \int d^3 x e^{-i k\cdot x} ( -i \partial _a \phi(x) + E\phi(x))
$$
In free theory: the 1 particle state  is given by
$$
\ket{k } = a^{\dagger} (\bm k) \ket{0} , \bra{0} \ket{0} =1, a(\bm k) \ket{0} = 0 \forall \bm kk
$$
Norm:
$$
\bra{k}\ket{k'} = (2\pi)^3 (2E) \delta^{(3)}(\bm k - \bm k')
$$
$$
E^2= \bm k^2 + m^2
$$
We are going to assume that the interacting theory is  close to the free theory.\\\\
Introduce a gaussian wave packet:
$$
a_1^{\dagger} = \int d^3 k f_1( \bm k ) a^{\dagger}(k)
$$
with $f_1(k) \sim \exp( - \frac{(\bm k - \bm k_1)^2}{4 \sigma^2})$ and similary
$$
a_2^{\dagger} = \int d^3 k f_2( \bm k ) a^{\dagger}(k)
$$
with $\bm k_2 \neq \bm k_1$. \\\\
Now we evolve the gaussians backward in time until a time where the particles had no overlap in space and can be consider free. THere is a complication as due to the interaction $a_1^{\dagger}(t)$ and $a_2^{\dagger}(t)$ depend on t. However,  the point is that as $t \rightarrow \pm \infty$ $a_1^{\dagger}$ and $a_2^{\dagger}$ coincide with free theory expressions.\\\\
Consdiring 2-2 scattering the initial state is $\ket{i} = \lim_{t\rightarrow -\infty} a_1^{\dagger}(t) a_2^{\dagger} (t) \ket{\Omega}$ and the final state is $\ket{f} = \lim_{t\rightarrow \infty} a_1^{\dagger}(t) a_2^{\dagger}(t) \ket{\Omega}$. We also have $\bra{i} \ket{i} =  \bra{f} \ket{f} = 1$ and $\bm k_1 \neq \bm k_2, \bm k'_1 \neq \bm k'_2$.\\\\
We want to calculate the scattering amplitude $\bra{f} \ket{i}$. First note that 
$$
a_1^{\dagger}( \infty) - a_1^{\dagger}(- \infty) = \int_{\infty}^{\infty} \partial_0 a_1^{\dagger} (t) = \int d^3 f_1(\bm k) \int d^4 x \partial_0 ( e^{-k \cdot x} ( - i \partial_0 \phi + c \phi)
$$
$$
a_1^{\dagger}( \infty) - a_1^{\dagger}(- \infty) = i \int d^3 k F_1(\bm k) \int d^4 x e^{-ik \cdot x} ( \partial _0^2 + E^2 ) \phi = - i \int d^3 k f_1( \bm k) \int d^4 x e^{-ik \cdot x}( \partial^2 + m^2 ) \phi
$$
In free theory we have $(\partial^2 + m^2)\phi = 0$ (klein gordon equation) so the creation operator doesn't change.
$$
\bra{f} \ket{i} = \bra{\Omega}T  a_{1'}(\infty) a_{2'} (\infty) a_1^{\dagger}(-\infty) a_2^{\dagger}(\infty) \ket{\Omega}
$$
We can relate $$a_j^{\dagger}(-\infty) = a_j^{\dagger}(\infty} + i \int d^3 k f_j \int d^4 x e^{-ikx} (\partial^2 + m^2) \phi
$$
$$a_{j'}^{\dagger}(\infty) = a_{j'}^{\dagger}(-\infty} + i \int d^3 k f_{j'} \int d^4 x e^{ikx} (\partial^2 + m^2) \phi
$$
Time ordering moves $a_j(- \infty)$ right and $a^{\dagger}_j(\infty)$ left. Therefore the t only nonzero term is the \textbf{LSZ reduction formula}:
$$
\bra{f} \ket{i} = (i)^4 \int d^4 x_1 d^4 x_2 d^4 x'_1 d^4 x'_2 e^{-i k_1 \cdot x_1} e^{-i k_2 \cdot x_2} $e^{i k'_3 \cdot x'_3} $e^{-i k'_4 \cdot x'_4} ( \partial_1^2 + m^2 )  ( \partial_2^2 + m^2 )  
( \partial_{1'}^2 + m^2 )  
( \partial_{2'}^2 + m^2 )  \bra{\Omega} T \phi(x_1) \phi(x_2) \phi(x'_1) \phi(x'_2) \ket{\Omega}$$
having taken $\sigma \rightarrow 0$ limit of $f_j(k) \rightarrow \delta^{(3)}(\bm k - \bm k_j)$.\\\\
Generalisatoin to $m\rightarrow n$ scattering is straightforward. You performa a fourier transform to get:
$$
\phi(y) = \int \frac{d^4 q}{(2\pi)^4} \tilde \phi(q) e^{-i q \cdot y}
$$
so}}
$$
\bra{f} \ket{i} = \bra{k'_1 ... k'_n} \ket{ k_1... k_n} = (i)^{m+n} \prod_{j=1}^m ( - k_j^2 + m^2) \prod_{j=1}^n (- k'_j^2 + m^2) \bra{\Omega} T \tilde \phi( k_1) ... \tilde \phi(k_m) \tilde \phi(k'_1)... \tilde \phi(k'_n) \ket{\Omega}
$$
momentum satisfy $k^2 = m^2$ propogators $\frac{k}{k^2 +m^2}$. The $(-k^2 +m^2)$ factors cancel the poles form the external propogators.\\\\
\textbf{LSZ in momentum space}:
$$
\bra{k'_1 .. k'_n}\bra{k_1...k_n} = \bra{\Omega} T \tilde \phi( k_1)...\tilde (k'_n) \ket{\Omega}_{amputated}
$$
Usually we are interested in all momentum being unequal, so all the particles are involved in the scattering. This implies we want connected diagrams.
\section{Example sheet 1}
When calculating symmetry factors you are genuinely looking for symmetries. Rather than trying to think about automorphisms first count the number of symmetries, then consider if any of them are already acounted for by exchange of particles and ignore them. If you get confused remember how it works with a single particle we don't say how many ways of picking which two to connect up is 6 but rather say we can slip those loops or exchange those loops (think a bit like polymod though in some instances this does not hold e.g. if the diagram is symmetric under exchange of two propogators which end on different particles this is a symmetry just like if they ended on the same particle). Try question 3 on ES1 it is a very good test of if you have got it cracked. It might be worth just memorising the symmetry factors of everything less than 3.\\\\
In order to show that 
$$
- \hbar^2 \frac{\partial^3 W}{\partial J_a \partial J_b \partial J_c}|_{J=0} = < \phi_a\phi_b \phi_c> ^{conn}
$$
we need to express in terms of $Z$ and take each derivative at a time and only take J =0 at the end. This will give extra terms because the 1/Z will get differentiated as well as the $\parital Z/ \partial J$\\\\
If you are asked to find the feynmann rules, remember that the propogator is $<\phi_1 \phi_2>$ and the vertex rule you can get by adding the source term and then extracting the interaction term as derivatives and the vertex term is what connects the derivatives. You need to extract the interaction term first leaving the derivatives in the exponential outside the integral, then inside the intgeral you sometimes need to complete the square and then perform a translation.\\\\
For fermionic case remembder that $A_{ij}$ and $\lambda_{abcd}$ are antisymmetric and that for antisymmetric tensors $\lamba_{abcd} =\lambda \epsilon_{abcd}$ and $\epsilon_{abc...n}\epsilon_{abc...n} = n!$.\\\\
For grassman variables feynmann diagrams you cannot have more half edges than variables as $\theta_i^2 = 0$. Remember forumula for $Pf(A)$
\section{Lecture 8}
$$
\bra{k_1'...k_n'}\ket{k_1...k_n} = \bra{\Omega} \tilde \phi(k_1')... \tilde \phi(k'_n) \tilde \phi(- k_1) ...\tilde \phi(-k_n) \ket{\Omega}
$$
All positive signs means all particles coming out of the interaction so above we flipped the signs of the incoming particles so they are actually incoming.\\\\
In fact only weaker assumptions are needed for an LSZ formula\\
\begin{itemlist}
\item Unique ground state $\ket{\Omega}$ and the 1st excited state is a single particle\\
\item We want $\phi \ket{\Omega}$ to be a single-particle state (Generally $\phi$ could represent a composite operator) - i.e. $\bra{\Omega} \phi \ket{\Omega} = 0$ ( If $\bra{\Omega} \phi \ket{\Omega} = v \neq 0$ then let $\tilde \phi = \phi - v$.\\
\item We whant $\phi$ to be properly normalised so that when it is far away it is nroamlised like a plane wave: $\bra{k} \phi \ket{\Omega} = e^{ik \cdot x}$ as in the free case\\
\item Interactions mean we will probably need to rescale the $phi$ to some scaled field $Z_{\phi}^{\frac{1}{2}} \phi$.\\
\end{itemlist}
With these assumptions + a few more -> LSZ formula still applies. 
We may need to "renormalise" the field from
$$
\mathfrak{L} = \frac{1}{2} \partial^{\mu} \phi \partial_{\mu} \phi - \frac{1}{2} m^2 \phi^2 - \frac{\lambda}{4!} \phi^4
$$
to
$$
\mathfrak{L} = \frac{1}{2}  Z_{\phi} \partial^{\mu} \phi \partial_{\mu} \phi - \frac{Z_{\phi}}{2} m^2 \phi^2 - \frac{\lambda Z_{\phi}^2}{4!} \phi^4
$$
\subsection{Scalar Field Theory}
\subsubsection{Wick rotation}
We will would in Minkowski space-time with signature (+---)
$$
\mathfrak{L} = \frac{1}{2} \partial_{\mu} \phi \partial^{\mu} \phi - V(\phi)
$$
$$
Z = \int \mathfrak{D}\phi e^{i \int dx^n L}, L = \int d^3 x \mathfrak{L}
$$
Propogator:
$$
\frac{i}{k^2 -m^2 + i \epsilon} = \frac{i}{(k^0)^2 - |\bm k|^2 - m^2 + i\epsilon}
$$
Let $ix^0 = x_4$ metric (++++), and arrange signs s.t.:
$$
\mathfrak{L} = \frac{1}{2} \partial_{\mu}\phi \partial^{\mu} \phi + V(\phi)
$$
$$
Z = \int \mathfrak{D}\phi e^{- \int d_{x_4} L}
$$
Propogator:
$$
\frac{1}{k^2 + m^2} = \frac{1}{k^2_4 + |\bm k|^2 + m^2}
$$
This change to euclidean space time can be thought of as a rotation from the real axis to the imaginary axis and is called Wicks rotation. There are some situtaitons where we can't do this rotation but we are interested in times when we can.\\\\
\subsubsection{Feynmann Rules}
In the free case:
$$
S_0(\phi, J) = \int d^4x( \frac{1}{2} \partial_{\mu} \phi \partial^{\mu} \phi + \frac{1}{2} m^2 \phi^2 - J(x) \phi(x)
$$
Fourier transform $\phi(x) = \int \frac{d^4 k}{(2\pi)^4} e^{ik\cdot x} \tilde \phi(k)$
$$
S_0[\tilde \phi, \tilde J] = \frac{1}{2} \int \frac{d^4k}{(2\pi)^4} \left( \tilde \phi(-k) (k^2 +m^2) \tilde \phi(k) - \tilde J(-k) \tilde \phi(k) - \tilde J(k) \tilde \phi(-k) \right)
$$
$$
S_0[\tilde \phi, \tilde J] = \frac{1}{2} \int \frac{d^4k}{(2\pi)^4} \left( \tilde \chi(-k) (k^2 + m^2) \chi(x) - \frac{\tilde J(-k) \tilde J(k)}{k^2 +m^2} \right)
$$
where $\tilde \chi = \tilde \phi - \frac{\tilde J}{k^2 + m^2}$ so:
$$
Z_0[J] = Z_0[0] \exp ( \frac{1}{2} \int \frac{d^4 k}{(2\pi)^4} \frac{\tilde J(-k) \tilde J(k)}{k^2 +m^2}
$$
We can ignore the premultiplier constant.\\
The free propogator is:
$$
\tilde \Delta_0(q) = \frac{\delta^2 Z_0 [\tilde J]}{ \delta \tilde J(-qq) \delta \tilde J(q)} = \frac{1}{q^2 +m^2}
$$
Functional derivatives:
$$
\frac{\delta}{\delta f(x_1)} f(x_2) = \delta^{(4)}(x_1-x_2)
$$
$$
\frac{\delta}{\delta \tilde f(k_1)} \tilde f(k_2) = (2\pi)^4 \delta^{(4)}(k_1-k_2)
$$
Now lets fourier transform back:
$$
\Delta_0 (x-x') = \int \frac{d^4 k}{(2\pi)^4} \frac{e^{ik \cdot (x-x')}}{k^2 + m^2}
$$
So we can write the partition funciton as:
$$
Z_c[J] = \exp \left( \frac{1}{2} \int d^4 x d^4 x' J(x) \Delta(x-x') J(x')\right)
$$
Interactions come about as before:
$$
\mathfrak{L} = \mathfrak{L}_0 + \mathfrak{L}_I
$$
As in $D=0$
$$
Z[J] = \int \mathfrak{D} \phi \exp( - \int d^4 x (\mathfrak{L}_0 + \mathfrak{L}_I - J \phi)) = \exp ( - \int d^4 y \mathfrak{L}_1 ( \frac{\delta}{\delta J(y)}) )  \exp \left( \frac{1}{2} \int d^4 x d^4 x' J(x) \Delta(x-x') J(x')\right)
$$
$$
Z[J] \sim \sum_{v=0}^N \frac{1}{v!} \left( - \int d^4 y \mathfrak{L}_1 ( \frac{\delta}{\delta J(y)})\right)^v \sum_{p=0} \frac{1}{p!}  \left( \frac{1}{2} \int d^4 x d^4 x' J(x) \Delta(x-x') J(x')\right)^p
$$
For each term in $Z[J]$ there is a graph made up of a Propogator $\Delta_0(x-x')$ and verticles with $n$ lines from $\phi^n > \mathfrak{L}_I$ and at each vertex we add $- \mathfrak{L}_I (\frac{\delta}{\delta J(y)})$. Then we integrate over internal positions and apply symmetry factors.
\section{Examples Class 1}
In general in order to find the $n$-th order perturbative expression for the partition function add a source term and then remove the derivatives and then set the source to zero. Ocassionally you don't need the source tterm and can compute it exactly.\\\\
Pay attention to the notation sometiems $Z_J(\lambda)$ so $Z_0(0)$ means $Z_{J=0}(\lambda = 0)$ where $\lambda$ is the coupling constant and $J$ is the invented source term. but sometimes $Z_{\lambda}(J)$. Nice way of thinking about which terms survive from expansion is that you need terms with the same number of $\frac{\partial}{\partial J}$ and $J$.\\\\
A nice way of writing effective actions are:
$$
< \phi_a \phi_b > = \hbar^2 e^{W/\hbar} \frac{\partial}{\partial J_a} \frac{\partial}{\partial J_b} e^{-W/ \hbar} = - \hbar \frac{\partial^2 W}{\partial J_a \partial J_b} + \frac{\partial W}{\partial J_a} \frac{\partial W}{\partial J_b} = < \phi_a \phi_b>^{conn} + < \phi_a>^{conn} < \phi_b>^{conn}
$$
as $Z(J) = e^{-W/\hbar}$
$$
\frac{\partial}{\partial \phi_d} ( \frac{\partial^2 \Gamma}{\partial \phi_b \partial \phi_c})^{-1} = (\frac{\partial^2 \Gamma}{\partial \phi_b \partial \phi_e})^{-1}  \frac{\partial^3 \Gamma}{\partial \phi_d \partial \phi_e \partial \phi_f}( \frac{\partial^2 \Gamma}{\partial \phi_d \partial \phi_c})^{-1} 

$$
The reason for the action of grassmann variables always being antisymmetric prefactors is that the symmetric part of the matrix will cancel so can be ignored.\\\\
If say the vertex term is $-\lambda \epsilon_{abcd}$ what this means for the diagram is that if we switch the order of two half edges then the sign needs to change. He is going to check exactly how this works and get back to us.
\section{Lecture 9}
Let's begin by looking at some examples, starting with the 1-loop function.
$$
< \phi(x_2) \phi(x_1)> = - (\frac{\delta}{\delta J(x_2)} \frac{\delta}{\delta J(x_1)} W[J]
$$
where $W[J] = - \log Z[J]$. Take $\mathcal{L}_1 = \frac{\lambda}{3!} \phi^3$:
$$
< \phi(x_2) \phi(x_1) > = x_2 - x_1 + x_2-y_2oy_1-x_1 + ...
$$
Let's consider the first one loop diagram $D$ here in more detail.\\\\
$$
D = \frac{(- \lambda)^2}{2}\int d^{4}y_1 \int d^4 y_2 \Delta_0(x_2-y_2) \Delta_0(y_1- x_1) (\Delta_0( y_2-y_1))^2
$$
Now take fourier transform
$$
< \tilde \phi(p_2) \tild \phi (p_1) > = \int d^4 x_1 \int d^4 x_2 e^{-i(p_1 x_1 + p_2 x_2)}< \phi(x_2) \phi (x_1)> = -^{p_1} O -^{p_2}
$$
Focus on $\tilde D$:
$$
\tilde D = \frac{\lambda^2}{2} \int d^4 x_1 d^4 x_2 e^{-i(p_1 x_1 + p_2 x_2)} \int d^4 y_1 \int d^4 y_2 \int (\prod_{j=1}^4 \frac{d^4 k_j}{(2\pi)^4}) e^{ik_2(x_2-y_1)} e^{ik_1(y_1 - x_2)} e^{i(k_3 + k_4) - (y_2 - y_1)} \tilde \Delta_0(k_1) \tilde \Delta_0(k_2) \tilde \Delta_0(k_3) \tilde \Delta_0(k_4)
$$
The integrals over $x_1$ and $x_2$ give: $(2\pi)^4 \delta^{(4)} (p_1 + k_1)$ and $(2\pi)^4 \delta^{(4)} (p_2 - k_2)$
$$
\tilde D \int d^4 y_1 \int d^4 y_2 \int \frac{d^4 k_3 d^4 k_4}{(2\pi)^8} e^{-i(p_1 + k_3 + k_4) - y_1 } e^{-i( p_2 - k_3 - k_4) y_2} \tilde \Delta_0(-p_1) \tilde \Delta_0(p_2) \tilde \Delta_0(k_3) \tilde \Delta_0(k_4)
$$
$$
\tilde D \int \frac{d^4 k}{(2\pi)^4} (2 \pi)^4 \delta^{(4)}(p_1 + p_2) \tilde \Delta_0(-p_1) \Delta_0(p_2) \tilde \Delta_0 ( -(k-p_2)) \tilde \Delta_0 (k)
$$
So the loop has momentum $p$ coming in and coming out and then we have loop momentum $k$ with one side of the loop having $k-p$ momentum and the other having $k$. \\\\
Another example in $\frac{\lambda}{3!} \phi^3$:
$$
< \phi(x_1) \phi(x_2) \phi(x_3) \phi(x_4)>^{conn} = \frac{\partial^4 W}{\partial J(x_1) \partial J(x_2) \partial J(x_3) \partial J(x_4)}|_{J=0}
$$
This can be represented by three diagrams in the typed up lecture notes.\\\\\\\
Using the LSZ in Euclidean spacetime to consider 2-2 scattering:
$$
\bra{f}\ket{i} = \int d^4 x_1 d^4 x_2 d^4 x_3 d^4 x_4 e^{ik_1 x_1} e^{ik_2 x_2} e^{-ik_3 x_3} e^{-ik_4 x_4} (- \partial_1^2 + m^2)...(-\partial_4^2 + m^2) < \phi(x_1) \phi(x_2 \phi(x_3) \phi(x_4) >^{conn}
$$
Klein-Gordon $(- \partial_i^2 + m^2) \Delta_0 ( x_i -y) = \delta^{(4)}(x_i - y)$\\
Integrals over $x_i$'s collapse:
$$
\bra{f} \ket{i} = \lambda^2 \int d^4 y \int d^4 z \int \frac{d^4 q}{(2pi)^4} \frac{e^{iq (y-z)}}{q^2 + m^2} ( e^{ik_y y}e^{ik_2y} e^{-ik_3 z} e^{-ik_4z} +...)
$$
Could intergate $y$ and $z$ to get delta functions and in this case there are no loops so there shouldn't be any loops left over so we are left with the following:
$$
\bra{f} \ket{i} \lambad^2 (2 \pi)^4 \delta^{(4)} (k_1 + k_2 - k_3 - k_4) ( \frac{1}{(k_1 + k_2)(^2 + m^2} + \frac{1}{(k_1-k_3)^2 + m^2} + \frac{1}{(k_1 + k_4)^2 + m^2})
$$
\subsubsection{Momentum space Feynmann rules}
\begin{itemlist}
\item Draw external lines for each incoming/outgoing particle\\
\item Leave one end of external line free other connected to a vector from $\mathfrak{L}_I$. Invlude internal lines to do this. Include topologically distinct diagrams\\
\item Incoming lines means you draw the momentum cominginto the vertex, whereas outocmming lines have momentum away from the vertex\\
\item Conserve momentum at each vertex\\
\item External lines get a factor of $1$\\
\item Internal lines get propogation of $\frac{1}{q^2 + m^2}$\\
\item For each vertex add the interaction coupling term $- \lambda_i$ for $\mathfrak{L}_I = \frac{\lambda_n}{n!} \phi^n$\\
\item A diagram with $L$ loops will have $L$ momenta $l_i$ not fixed by momentum conservation so we integrate over these $\int \frac{d^4 l_i}{(2\pi)^4}$ (these can lead to divergences that we will come to soon.\\
\item Divide by symmetry factors to account for different ways of arranging internal propogators
\end{itemlist}
\section{Lecture 10}
\subsection{Quantum Effective Action and Vertex Functions}
Same as before but now we have functionals rather than functions:
$$
W[J] = - \log Z[J]
$$
$$
\Gamma[\Phi] = W[J] + \int d^4 x J(x) \Phi(x)
$$
The same steps as we have carried out before led us to conclude that 
$$
\frac{\delta W[J]}{\delta J(x)} = - \Phi(x), \frac{\delta \Gamma}{\delta \Phi(x)} = J(x)

$$
Again when we are working with $W$ we think of the soruce $J$ as the independant field, then wen we move to $\Gamma$ we think of $\Phi$ has the independant field. So we have:\\
$J(x)$ is independant and then correseponds to $\Phi(x)$ mean field by $<\phi(x)>_J$ whereas after the Lagrange transform we get $\Phi(x)$ is independant with a corresponding source $J(x)$ obtained by $<\phi(x)>_J = \Phi(x)$.\\
In momentum space, consider the sets of amputated 1PI diagrams with $n$ external legswhere $n\geq 2$.\\
For $n=2$ the summ of all such IP1 diagrams is called the self-energy which will be labelled as:
$$
\Pi(k^2) = -_k (1PI) -_k 
$$
lorentz invariance and momentum conservation give us that $k^2$ is the parameter.\\
For $n\geq 3$ these vertex functions are lablled as:
$$
V^{(n)} (k_1,...,k_n) = \text{mass of n legs coming into 1PI}
$$
Now lets think further in the momentum space field with $\tilde \Phi = \int d^4 x e^{-ik x} \Phi(x)$
$$
\Gamma[\tilde \Phi] = \frac{1}{2} \int \frac{d^4 k}{(2 \pi)^4} \tilde \Phi( - k) [ k^2 + m^2 - \Pi(k')]\tilde \Phi(k) - \sum_{n=3} \frac{1}{n!} [\int \hat \prod_{i,j = 1}^n \frac{d^4 k_j}{(2\pi)^j}] (2 \pi)^4 \delta(k_1 + k_2 + ... + k_n) V^{(n)} (k_1,...,k_n) \tilde \Phi(k_1) ... \tilde \Phi(k_n)
$$
This minus sign before the higher order terms is because want to think of this vertex as starting out as the classic vertex e.g. if we had $\mathfrak{L}_I = \frac{\lambda}{n!} \phi^n \implies V_0^{(n)} = - \lambda$ (corresponding to tree-level).\\\\
It is unusal that here the momentum has an impact on the interaction, as up until now we haven't been considering interaction terms with derivates that would bring in momentum dependance.\\
Note if $v^{(n)}$ has non-trivial momentum dependance then derivatives acting on $\Phi$ in coordinate space. Lets fourier tansform back
$$
\Gamma[\Phi] = \int d^4 x \{ U[\Phi(x)] + Q[ \Phi(x)] \partial_{\mu} \Phi \partial^{\mu} \Phi + ... \}
$$
The first term here $U$ is often called the effective potential.\\
\subsubsection{(Quantum) Effective Potential}
Generalising the earlier derivation we want to define:
$$
< \phi(x_1) ... \phi(x_n)>^{conn} = G^{(n)} (x_1,... x_n) = - \prod_{i=1}^n \frac{\delta}{\delta J(x_i)} W[J]|_{J=0}
$$
And we have:
$$
\Gamma^{(n)} = \prod_{i=1}^n \frac{\delta}{\delta \Phi(x_i)} \Gamma[\Phi]|_{\Phi = \Phi_0}
$$
We know that the quadratic parts are inverses of each other but now we need to integrate over :
$$
\int d^4 y G^{(2)}(x,y) \Gamma^{(2)} (y,z) = \delta^{(4)}(x-z)
$$
\subsection{Renormalisation}
We will begin by picking a particular aciton and doing our first calculations in quantum higher dimensions. Lets chose the classical action:
$$
S[\phi_0] = \int d^4 x ( \frac{1}{2} (\partial \phi_0)^2 + \frac{1}{2} m_0 \phi_0^2 + \frac{\lambda_0}{4!} \phi_0^4)
$$
this example is not great as we don't have to change the field but we do have to in the $\phi^3$ theory on the example sheet. Subscripts hint that these "original" or "bare" fields and parameters will need to be "renomarlised".\\
Consider quantum effects at 1-loop level. Take the 2-point function $$\tilde G^{(2)} (p) =  -0- = - + -1PI- + -1PI-1PI- + ... = \frac{1}{p^2 + m_0^2} + \frac{1}{p^2 + m_0^2} \Pi( p^2) \frac{1}{p^2 + m_0^2} + ... $$
$$
\tilde G^{(2)}(p) = \frac{1}{p^2+m_0^2 - \Pi(p^2)}
$$
So
$$
\tilde \Gamma^{(2)} (p) = [\tilde G^{(2)}(p)]^{-1} = p^2 + m_0^2 - \Pi(p^2)
$$
Now we need to calculate the $\Pi(p^2)$ function.\\
In $\phi^4$ theory we can calculate the one-loop contribution. In this theory there is only one diagram:
$$
\Pi_1(p^2) = -_p^o-_p
$$
we have developed the feynmann rules so this gives:
$$
\Pi_1(p^2) = - \frac{\lambda_0}{2} \int \frac{d^4k}{(2\pi)^4} \frac{1}{k^2 + m_0^2}
$$
This will diverge for large $k$ which we can show by introducing a cut off $|k| < \Lambda$. As the integrand only depends on the magnitude of $k$ so
$$
\Pi_1(p^2) = - \frac{\lambda_0}{2} \int^{\Lambda}_0 \frac{dk}{(2\pi)^4} k^3 dk d\Omega_3 \frac{1}{k^2 + m_0^2} = - \frac{\lambda_0 S_3}{2(2\pi)^4} \int^{\Lamdba}_0 \frac{k^3}{k^2_m_0^2} dk, S_3 = 2 \pi^2
$$
So let $n = \frac{k^2}{m_0^2}$
$$
\Pi_1 (p^2) = - \frac{\lambda_0}{32 \pi^2} (\Lambda^2 - m_0^2 \log ( 1 + \frac{\Lambda^2}{m_0^2}))
$$
This diverges quadaracticall as $\Lambda$ goes to infinity. \\\\
Now lets look at the four point function $f^n$ at 1-loop:
$$
V_0^{(n)} = - \lambda_0
$$
$$
V_1^{(n)} (p_1,p_2,p_3,p_4) =  >o< + =o= + 
$$
basically you need three different diagrams to allow for every combination of the external lines meeting so first is 1-2, 3-4 then 1-3, 2-4 and the 1-4 2-3 and every diagram has two internal momenta between thsee verteces. Using the feynmann ruels this gives:
$$
V_1^{(n)} = \frac{\lambda_0^2}{2} \int^{\Lamdba} \frac{dk^4}{(2\pi)^4} \frac{1}{k^2 + m_0^2} \sum_{p = \{ p_1 + p_2, p_1-p_3, p_1- p_4\}} \frac{1}{(p+k)^2 + m_0^2} 
$$
This is a mess to integrate as you have external momentum dependance. However, there is really just one source of diverengence, as $k$ gets large whatever these external momentum are they will be fixed so we only really have the one divergence which is important as we only have one parameter with which to fix the divergence. Now lets examine this divergence.\\
As $k \rightarrow \infty$, the external momenta become negligible. Examine
$$
V_1^{(4)} (0,0,0,0) = \frac{3 \lambda_0^2}{2} \int^{\Lambda} \frac{d^4k}{(2\pi)^4} \frac{1}{(k^2 + m_0^2)^2} = \frac{3}{\lambda_0^2}{32\pi^2} ( \log ( 1+ \frac{\Lambda^2}{m_0^2}) - \frac{\Lambda^2}{\Lambda^2+ m_0^2})
$$
so we have a logarithmic divergence.
\section{Lecture 11}
First we will find a shortcut for figuring out if a loop integral will diverge.\\\\
More generally in $d$ dimensions we can write down the form of the integrals we were writing before (without factors of $2\pi$ as he was being schematic?)
$$
(\int^{\Lambda} d^d k)^l ( \frac{1}{k^2 + \Delta})^I
$$
$\Delta$ may contain external parameters such as mass and external momenta. For the purposes of investigating a large $k$ we can drop the $\Delta$s are they are fixed.
$$
\int^{\Lambda} \frac{(d^d k)^l}{k^{2I}}
$$
The "superficial degree of divergence is $D= dL - 2I$. If $D>0$ then the integral diverges like $\sim \Lambda^D$. If $D=0$ then it diverges but logarithmically $\sim \log \Lambda$. If $D<0$ then it is possibly finite but we cannot guarantee that. In this course we are only looking at one loop diagrams so if $D<0$ then they will be finite.
\subsubsection{Lohmann-Kollen propogator}
$$
\tilde G^{(2)}(p) = \int d^4 x d^4 y e^{-ip(x-y)} < \phi_0 (x) \phi_0 (y)> = \sum_n \frac{|\bra{\Omega} \phi_0(0) \ket{x} |^2}{p^2 + m_n^2}
$$
with $\ket{\Omega}$ is the vaccum state of the full theory, and $\ket{n}$ eigenstate of Hamiltonian with rest mass $m_n$. Look at Peskin and Schorder. \\\\
Focus on lowest energy, single particle excitation\\
$$\tilde G^{(2)}(p) \im \frac{|\bra{\Omega} \phi_0(0) \ket{1} |^2}{p^2 + m^2_{phys}} + [\text{finte at } p^2 = - m^2_{phys}$$
So basically saying there is a pole at $p^2 = - m^2_{phys}$ with residue given above.\\\\
Last time we found for $\phi^4$ theory that:
$$
\tilde G^{(2)}(p) = \frac{1}{p^2 - m_0^2 - \Pi(p^2)}
$$
with $\Pi(p^2) = - \frac{\lambda_0}{32 \pi^2} ( \Lambda^2 - m_0^2 \log ( 1 + \frac{\Lambda}{m_0^2}))$
This is constant in $p^2$ but in general is it not. e.g. in $\phi^3$ theory the free energy depends on the momenta.\\\\
To deal with divergences we do the following:\\
We have the original Lagrangian: $\mathfrak{L}_0 = \frac{1}{2} (\partial \phi_0)^2 + \frac{1}{2} m_0^2 \phi_0^2 + \frac{\lambda_0}{4!} \phi_0^4$\\
Rescale $\phi_0 = Z_{\phi}^{\frac{1}{2}} \phi$ where $Z_p^{\frac{1}{2}}$ determined s.t. $\tilde G^{(2)} (p)$ has unit residue at pole.\\
Separate out 2 sets of terms, write $\mathfrak{L}_0$ as follows:
$$
\mathfrak{L}_0 = \mathfrak{L_{rln}} + \mathfrak{L_{ct}} = ( \frac{1}{2} (\partial phi)^2 + \frac{1}{2} m^2 \phi^2 + \frac{\lambda}{4!} \phi^4) _ ( \frac{\delta Z_{\phi}}{2} (\partial \phi)^2 + \frac{\delta m^2}{2} \phi^2 + \frac{\delta \lambda}{4!} \phi^4)
$$
equate coefficents to get: $\delta Z_{\phi} = Z_{\phi} - 1$ and $\delta m^2 = Z_{\phi} m_0^2 - m^2$ and $\delta \lambda = Z_{\phi}^2 \lambda_0 - \lambda$.\\
We now use this way of writing the lagrangian in perturbative calcualtions so the Feynmann rules need to be alterated a bit. The rules for $\mathfrak{L}_{rln}$ are the same as for $\mathfrak{L}_0$. Additionally, for $\mathfrak{L}_{ct}$ we have new rules given by two new interaction terms:
$$
-\square^{-p^2 \Omega}-, -x^{-\delta m^2}-, X^{-\delta \lamdba}
$$
The diagramatics of these aren't imporatnt but it is useful to think of these terms as additional interaction terms at the one loop level. The tree diagrams containing $\mathfrak{L}_{ct}$ vertices are the same order as 1-loop diagrams containing $\mathfrak{L}_{rlm}$ vertices. Lets revisit:
$$
\tilde \Gamma_{rlm}^{(2)}(p) = [\tilde G_{rlm}]^{-1} = p^2 + m^2 - \Pi_{rlm}(p^2)
$$
From $\mathfrak{L}_{rlm}$, $\hat \Pi_1(p^2)$ is the same equation as for $\Pi_1(p^2)$ with $m_0$,$\lambda_0$, $\phi_0 \rightarrow m, \lambda, \phi$.\\
From $\mathfrak{L}_{ct}: \Pi_{1,ct} = -x- + -\square -= - \delta m^2 - \delta Z_{\phi} p^2$ these are additoinal diagrams that contribute to the two point function. Result for $\Pi_{rlm}(p^2) = \hat \Pi_1(p^2) + \Pi_{1,ct} (p^2)$. We know there is a divergence in $\hat \Pi(p^2)$ but we can make it finite by chosing specific $\delta m^2$ and $\delta \lambda^2$:
$$\delta m^2 = - \frac{\lambda}{32 \pi^2}( \Lamdba^2 - m^2 \log ( 1 + \frac{\Lambda^2}{m^2}) + \text{finite}, \delta Z_{\phi} = 0
$$
There is freedom in how we chose the finite part which goes by the naem of the "renormaliszation scheme" or "condition". This is a choice. e.g. we are using the "on-shell" scheme by imposing the following conditions:
$$
\Pi_{rlm}(- m^2_{phys}) = m^2 - m^2_{phys}
$$
Additionally, choose $m^2 - m^2_{phys} = 0$ this sets the renoramlised mass $m = m_{phys}$. The other "on shell" condition we could use is:
$$
\frac{\partial \Pi_{rln}}{\partial p^2}|_{p^2 = - m^2_{phys}} = 0
$$
These two conditions together give the propogator: 
$$
\tilde G^{(2)}(p) = \frac{1}{p^2 + m^2 - \Pi_{rlm}(p^2)} = \frac{1}{p^2 + m^2_{phys}}
$$
To finish up we need to look at $\delta \lambda$, now lets look at corrections to the four point correlation function:
$$
V^{(4)}_{rln}(0,0,0,0) = - \lambda + \hat V^{(4)}(0,0,0,0) + V_{1,ct}^{(4)}
$$
$$
V_{1,ct}^{(4)} = - \delta \lambda
$$
Choosing $\delta \lambda = \frac{3 \lamdba^2}{32 \pi^2} ( \log \frac{\Lambda}{m^2} - 1)$
gives a finite vertex term as the divergence is cancelled by the second term
$$
V_{rln}^{(4)}(0,0,0,0) = - \lambda + \frac{32 \lambda^2}{32 \pi} ( \log ( 1+ \frac{\Lambda^2}{m^2}) + \frac{m^2}{m^2 + \Lambda^2}) = - \lambda_{eff}
$$
where $\lambda_{eff}$ is coefficient of $\frac{1}{4!}\Phi^4$  in $\Gamma[\Phi]$. Note we did make a decision about the finite piece which is that $\lambda_{eff} \rightarrow \lambda$ as $\Lamdba \rightarrow \infty$ which is another renormalisation choice or condition that we have decided to impose.
\subsection{Dimensional regularization}
This idea of momentum cutoff is fine for a scalar field theory but is no good for guage theory. In the context of perturbation theory we can regulate divergences by moving away from integer dimensions. Here $d= 4 - \epsilon$ for $0<\epsilon <<1. in the case of $\phi^3$ we are often interested in close to $6$ dimensions. Take $S = \int d^4 x ( \frac{1}{2} ( \partial \phi)^2 + \frac{m^2}{2} \phi^2 + \frac{\lambda}{4!} \phi^4)$\\\\
Dimensional analysis $[.]$ = mass dimension $\hbar = c = 1$. Given $[S] = 0$, $[\partial ] = [m] = -[x] = 1$
$$
[m^2 \phi^2] = 2[m] + 2[\phi] = d \implies [\phi] = \frac{d}{2} -1
$$
$$
[\lambda \phi^4] = d \implies [\lambda] = 4 -d = \epsilon
$$
This means that away from 4 dimensions the formerly dimensionless coupling $\lamdba$ now has dimensions $\epsilon$. This introduces a new mass scale $\mu$ s.t. $\lambda = \mu^{\epsilon} g(\mu)$ with a new dimensionless coupling $g$.
\section{Summary of renormalisation}
We can separate out the high energy modes as the integral is additive. Then if you define an effective action $S_{\Lambda}^{eff} = - \hbar \log ( \int_{Lambda}^{\Lambda_0} D \chi exp( - S_{\Lambda_0} (\phi + \chi)/\hbar))$. This will give identical correlation functions for all energy scales less than $\Lambda$ so we have:
$$
 Z_{\Lamdba}( g_i (\Lambda)) = \int^{\Lambda} D \phi e^{ - S^{eff}_{\Lambda}[\phi]/\hbar}
$$
which is the same for any $\Lambda$, so as we change the scale by integrating out modes the couplinhgs in the effective action vary exactly to account for as described by the Callan-Symanzik equation:
$$
 \Lambda \frac{d Z_{\Lambda}(g)}{d\Lambda} = ( \Lambda \frac{\partial}{\partial \Lambda}|_{g_i} + \Lambda \frac{\partial g_i ( \Lambda)}{\partial \Lambda} \frac{\partial}{\partial g_i}|_{\Lambda}) Z_{\Lambda}(g) = 0
$$
We want to think about how these couplings change as we integrate out more modes as it tells us how the validity of the perturbation method varies depending on energy scale. We define the $\beta_i$ function as $\beta_i = \Lambda \frac{\partial g_i}{\partial \Lambda}$ to describe this.\\\\
Conformal field theories are points $g^*_i$ where $\beta_i = 0$ as at this point the couplings are independant of the scale. We care about couplings that run close to these critical points but not to them. To find them we first consider irrelevant couplings which are ones with operators of dimension less than the scaling dimension $\Delta_{\phi} = (d-2) /2 + \gamma_{\phi}$, these all run into the critical point ( the surface constructed of them is called the critical surface). Relevant couplings are ones with $d > \Delta_{\phi}$ and they run away from the critical point/surface (the relevant coupling starting right at the critical point is known as the renormalized trajectory). We also have marginally relevant, or irrelevant couplings that have $\Delta_{\phi} = d$ that will have a weaker dependance but ultimately run to or away form the critical point. The whole point of this is that the renormalized trajectory will either go on forwever or meet another critical point. A generic QFT will have a selection of relevant and irrelevant couplings. As we integrate out modes the irrelevant couplings will dissapear (as they tend to the gaussian critical point with $g_i = 0$) and the relevant couplings will tend to the renormalized trajectory, which in the IR limit will flow to a second critical point with a conformal QFT. This is why renormalisation works, we can do physics in this IR limit at this conformal QFT without knowing which set of initial couplings is actually valid as long as they flow here. Theories whose RG flows all focus onto the same trajectory emanating form a given critical point are in the same universality class, and will end up looking the same at large distances/low energy scales.\\\\
Often we want to know if a low energy theory at a fixed scale $\Lambda$ is dependant on the initial cut-off $\Lamdba_0$, and if we remove the limit will this affect the predictions for low-energy. This is called taking the continuum limit. \\If the initial couplings of this theory lie on the critical surface then it is happy days as we can send $\Lambda_0$ to infinity and the irrelevant operators are suppresed by position powers of $\Lamdba/\Lamdba_0$ so the theory is driven to the critical point which is scale invariant, so the resulting scale-$\Lamdba$ effective theory will be a CFT and independant of $\Lambda$, such theories are called superrenormalizable. \\
If the theory has relevant or marginally relevant operators then consider a theory with initial conditions near but not on the critical surface. This flow will pass close to the critical point and then follow the renormlised trajectory as we send $\Lambda_0$ to infinity. Therefore, we tune the initial conditions using counterterms dependant on $\Lambda_0$ so that the scale-$\Lambda$ QFT has finite couplings as we take $\Lamdba_0$ to infinity.
\section{Lecture 12}
We analytically continue things away from 4 dimensions to $d= 4-\epsilons$, so the previously dimensionless coupling gains a dimension $\lambda = \mu^{\epsilon} g(\mu)$ with $g$ dimensionless.\\\\
Now lets return to the one-loop diagrams that we looked at before. The one-loop self energy becomes:
$$
\hat \Pi_1 = - \frac{1}{2} g(\mu) \mu^{\epsilon} \int \frac{d^d k}{(2\pi)^d} \frac{1}{k^2 + m^2}
$$
As before we can see that the integrand can be written as:
$$
\hat \Pi_1 = - \frac{1}{2} g(\mu) \mu^{\epsilon} \frac{S_{d-1}}{(2 \pi)^d} \int_0^{\infty} \frac{(k^2)^{d/2 - 1}}{2( k^2 + m^2) } dk^2
$$
This is done in more general sense in the appendix of the written notes (including Feynmann's trick with Parametrization).\\
Can show that for $d \in \mathbb{Z}^+$, we have $S_{d-1} = \frac{2 \pi^{d/2}}{\Gamma (d/2)}$. Analaytically continue this to $d \in \mathbb{R}^+$. We will also find the Schwinger trick useful which is two rewrite the denominator:
$$
\frac{1}{A^n} = \frac{1}{A^n} \frac{1}{\Gamma(n)} \int_0^{\infty} dx e^{-x} x^{n-1} 
$$
Let $s = \frac{x}{A}$ then 
$$
\frac{1}{A^n} = \frac{1}{\Gamma(n)} \int^{\infty}_0 ds e^{-As} s^{n-1}
$$
So
$$
\hat \Pi = - \frac{1}{2} g(\mu) \mu^{\epsilon} \frac{1}{(4\pi)^{d/2} \Gamma(d/2)} \int_0^{\infty} dk^2 (k^2)^{d/2-1} \int_0^{\infty} ds e^{-(k^2+m^2)s}
$$
Let $u = sk^2$:
$$
\hat \Pi_1 = - \frac{1}{2} g( \mu) \mu^{\epsilon} \frac{1}{(4\pi)^{d/2} \Gamma(d/2)} \int_0^{\infty} ds e^{-m^2 s} s^{-d/2} \int^{\infty}_0 du u^{\frac{d}{2} - 1} e^{-u}
$$
We can identify this two terms:
$$
\hat \Pi_1 = - \frac{1}{2} g(\mu) \mu^{\epsilon} \frac{1}{(4\pi)^{d/2} \Gamma(d/2)} \Gamma( 1- \frac{d}{2}) m^{d-2} \Gamma(d/2)
$$
\textbf{This is definitelly better just recognise Euler Beta fucntions B(s,t) like in notes)}:
        $$
        B(s,t) = \int^1_0 du u^{s-1} (1-u)^{t-1} = \frac{\Gamma(s) \Gamma(t)}{\Gamma(s+t)}
        $$
        Continying the maths:
        $$
        \hat \Pi_1 = - \frac{g(\mu) m^2}{2 (4 \pi)^2} ( \frac{4\pi \mu^2}{ m^2} )^{\epsilon/2}\Gamma( \frac{\epsilon}{2} - 1)
        $$
        We want to consider $\epsilon \rightarrow 0$. so need to expand about small $\epsilon$. First term is straight forawrd:
        $$
        ( \frac{4\pi \mu^2}{ m^2} )^{\epsilon/2} = 1 + \epsilon/2 \log ( \frac{4\pi \mu^2}{ m^2} ) + O(\epsilon^2)
        $$
        Expanding the gamma function needs to be analytically continued to think about what it means for negative arguments using $\alpha \Gamma( \alpha) = \Gamma( \alpha + 1)$. There must be a pole at $\alpha = 0, -1, -2$ as $0 \Gamma(0) =\Gamma(1)$ so near these poles there is a Laurent series with $\Gamma(\alpha) = \frac{1}{\alpha} - \gamma + O(\alpha)$ where $\gamma \approx 0.577216..$ so:
        $$
        \Gamma ( \frac{\epsilon}{2} - 1) = - \frac{1}{1- \frac{\epsilon}{2}} \Gamma( \epsilon/2) = - \frac{2}{\epsilon} + \gamma -1  + O(\epsilon)
        $$
        therfore:
        $$
        \hat \Pi_1 = \frac{g(\mu) m^2}{32 \pi^2} ( \frac{2}{\epsilon} - \gamma +1 + \log ( \frac{4 \pi \mu^2}{m^2}) ) + O(\epsilon)
        $$
        We have the same divergence with $\Lambda \rightarrow \infinity$ equivalent to $\frac{1}{\epsilon} = \frac{1}{4 - d}$ pole.
        Add a counterterm $\frac{1}{2} \delta m^2 \phi^2$.\\\\
        Choice of scheme (finite term):\\
        - On-shell such that $m^2 + \delta m^2 = m^2_{phys}$\\
        - Minimal subtraction (MS) where the counter term just subtracts the divergence and adds no finite term.\\
        - Modified minimal subtraction scheme ($\bar {MS}$) when you subtract the pole and also any constants that come along from the expansion e.g.
        $$
        \delta^2 m = - \frac{g(\mu) m^2}{32 \pi^2} ( \frac{2}{\epsilon} - \gamma + \log ( 4\pi))
        $$
        Then in $\bar{MS}$ we have:
        $$
        \Pi^{\bar{MS}}_{rln} = \frac{g(\mu) m^2}{32 \pi^2} ( \log \frac{\mu^2}{m^2} -1)
        $$
        \\\\
        Now lets look at the one-loop corrections to the four-point function.
        $$
        \hat V^{(4)} (0,0,0,0) = \frac{3 g^2 \mu^{2\epsilon}}{2} \int \frac{d^d k}{(2\pi)^4} \frac{1}{(k^2+m^2)^2} = 3 g^2 \mu^{\epsilon} \frac{1}{(4\pi)^{d/2}} ( \frac{\mu}{m})^{\epsilon} \Gamma( 2- \frac{d}{2}) 
        $$
        $$
        \hat V^{(4)} (0,0,0,0) =  \frac{3 g^2 \mu^{\epsilon}{32 \pi^2} ( \frac{2}{\epsilon} - \gamma + \log ( \frac{4 \pi \mu^2}{m^2}) + O(\epsilon)
        $$
        Introduce a counter-term $- \delta g$ which is a four point vertex with $\delta g = \frac{3 g^2}{32 \pi^2} \frac{2}{\epsilon}$ in MS and $\delta g =  \frac{3 g^2}{32 \pi^2}( \frac{2}{\epsilon} + \log ( \frac{4 \pi \mu^2}{m^2}) )$ in $\bar{MS}$.\\\\
        \textbf{Explore the consequences of the new scale $\mu$}\\\\
        Old fashioned approach (the next chapter will do the new approach).\\\\
        Look at $$\mathfrak{L}_0 = \frac{1}{2} (\partial \phi_0)^2 + \frac{1}{2} m_0 \phi_0^2 + \frac{\lambda_0}{4!} \phi_0^4$$
        and rescale with $\phi_0 = Z_{\phi}^{1/2} \phi$ to get:
        $$
        mathfrak{L}_0 = \frac{Z_{\phi}}{2} (\partial \phi_0)^2 + \frac{Z_{\phi}}{2} m_0 \phi_0^2 + \frac{\lambda_0 Z_{\phi}^2}{4!} \phi_0^4 =   \mathfrak{L}_{rln} + \mathfrak{L}_{ct}
        $$
        $$
\mathfrak{L}_{rln} + \mathfrak{L}_{ct} = \frac{(1+ \delta Z_{\phi})}{2} ( \partial \phi)^2 + \frac{m^2 + \delta m^2}{2} \phi^2 + \frac{(g+ \delta g) \mu^{\epsilon}}{4!} \phi^4
        $$
        Now we need to equate teh coeeficents between these two to give:
        $$
        Z_{\phi} = 1 + \delta Z_{\phi}, m_0^2 = Z^{-1}_{\phi} ( m^2 + \delta m^2), \lambda_0 = Z_{\phi}^{-2} ( g + \delta g) \mu^{\epsilon}
        $$
        Recall each loop brings in an additonal $\hbar$. For the vertex funciton we started off with the term:
        $$
V_{rln}^{(4)} (0,0,0,0) = \frac{1}{\hbar} ( - \lambda + O(\hbar) - \delta g \sim O(\hbar)}
        $$
        Basically as we have been ignoring the $\hbar$s we need to remember that the counterwieght term needs an extra $\hbar$ so it cancels with the loop terms.\\
        Have $\lambda_0 = Z_{\phi}^{-2} ( g + \delta g) \mu^{\epsilon}$ though in $\phi^4$ $Z_{\phi} = 1$ so ignore it.\\
        We define $\beta(g) = \frac{d}{d\log \mu} g(\mu) = \mu \frac{d}{d\mu} (g(\mu))$. This is an old fashioned attitude as there was nothing special about $\lambda_0$ (though it definitely doesn't depend on $\mu$ as it was introduced before we introduced $\mu$):
        $$
        0 = \frac{d}{d\log u} \lambda = \frac{d}{d \log u} ( ( g+ \delta g) \mu^{\epsilon}) 
        $$
        as $\delta g = \frac{3 g^2 \hbar}{16 \pi^2 \epsilon}$:
        $$
        0 = \beta(g) ( 1 + \frac{3 g \hbar}{8 \phi^2 \epsilon}) + g ( 1+ \frac{4 g \hbar}{16 \pi^2 \epsilon})
        $$
        we find that:
        $$
        \beta(g) = ( - \epsilon g - \frac{3 g^2 \hbar}{16 \pi^2} ( 1 + \frac{3 g \hbar}{8 \pi^2 \epsilon})^{-1} = - \epsilon g + frac{3 g^2 \hbar}{8 \pi^2} - \frac{3 g^2 \hbar}{16 \pi^2 + O(\hbar^2}
        $$
        There is a $O(\frac{1}{\epsilon}$ here but it is multipled by $O(\hbar^2)$ but that would be canceled by a 2 loop or three loop divergence so we ignore it as it is a higher order in pertrubation theory. so we get
        $$
        \beta(g) = - \epsilon g + \frac{3 g^2 \hbar}{16 \pi^2 } \rightarrow \frac{3 g^2 \hbar}{16 \pi^2} = \mu \frac{d g}{d\mu}
        $$
        $$
        \int_{g(\mu)}^{g(\mu')} \frac{dg}{g^2} = \frac{3\hbar}{16 \pi^2} \int_{\mu}^{\mu'} \frac{d\tilde \mu}{\tilde \mu} \implies \frac{1}{g(\mu')} = \frac{1}{g(\mu)} - \frac{3 \hbar}{16 \pi^2} \log \frac{u'}{u}
        $$
        so
        $$
        g(u') = \frac{g(\mu)}{1- \frac{3 g \hbar}{16 \pi^2} \log \frac{\mu'}{\mu}} \sim g(\mu) ( 1- \frac{3 g(\mu) \hbar}{16 \pi^2} \log \frac{\mu'}{\mu})
        $$
        For $\mu'> \mu \implies g(\mu') > g(\mu)$ so the coupling "runs" to larger values as $\mu$ increases. \\\\
        We see that is maybe possible that the denomiator may vanish so we coul dhave a landau pole. A Landau pole occurs if $\mu' \rightarrow \Lambda_{\phi^4}$ where 
        $$
        \frac{3 g \hbar}{16 \pi^2} \log \frac{\Lambda_{\phi^4}}{\mu} = 1
        $$
        (at 1-loop order) as $\mu' \rightarrow \Lambda_{\phi^4}$ we have $g(\mu') \rightarrow \infty$. So $\Lambda_{\phi^4}$ can be a referecne scale:
        $$
        g(\mu) = \frac{16 \pi^2}{3 \hbar^2} \frac{1}{\log(\frac{\Lambda_{\phi^4}}{\mu}}
        $$
        So now we are comparing our running coupling to our reference scale. So we have gone from talking about dimensionless couplings to having a reference scale or mass scale ("dimensionfull") this is sometimes called a "Dimensional transmutation". This is not useful in theories with couplings that run like this that get weaker at higher energy scales like QED, but it is very important for strongly coupled theories.
        \section{Lecture 13}
We are now going to recast this technique in terms of the Quantum Effective Action.
\subsubsection{Renormilisation and Quantum Effective action}
Recall $\Phi_{orig} = - \frac{\delta W}{\delta J} = < \phi_0>$\\
After caluculating quantum corrections we want to rescale using $\phi_0 = Z_{\phi}^{\frac{1}{2}} \phi$ so we should have:
$$
\Phi_{orig} = Z^{\frac{1}{2}}_{\phi} < \phi> = Z_{\phi}^{\frac{1}{2}} \Phi
$$
$$
\Gamma_0^{(n)} (x_1,..,x_n) = \frac{\delta^{(n)}\Gamma [\Phi_{orig} ]}{\delta \Phi_{orig} (x_1) ...\delta \Phi_{orig} (x_n)} = Z_{\phi}^{-\frac{n}{2}} \frac{\delta^{(n)}\Gamma[\Phi]}{\delta \Phi(x_1) ... \delta \Phi( x_n)} = Z_{\phi}^{- \frac{n}{2}} \Gamma^{(n)}_{ren}(x_1, ..., x_n)
$$
$\Gamma_0^{(n)}$ is independant of scale $\mu$ and $\mu \frac{d}{d\mu} \Gamma_0^{(n)} = 0$\\
Define the "anomalous dimension" of $\phi$ as 
$$\gamma_{\phi} = \frac{\mu}{2} \frac{d}{d\mu} \log Z_{\phi}$$
such that
$$
\mu \frac{d}{d\mu} Z_{\phi}^{-\frac{n}{2}} = - \frac{n}{2} Z_{\phi}^{-\frac{n}{2}} \mu \frac{d}{d \mu} \log Z_{\phi} = - n \gamma_{\phi} Z_{\phi}^{-\frac{n}{2}}
$$
Then
$$
0 = ( \mu \frac{\partial}{\partial \mu} + \mu \frac{d m^2}{d \mu} \frac{\partial}{\partial m^2} + \beta(g) \frac{\partial}{\partial g} - n \gamma_{\phi}) \Gamma_{ren}^{(n)} (x_1,..., x_n)
$$
Equations of this type are called Callen-Symanzik equations.\\\\
This is an argument made in position space which is useful as many actions are written in sums of local operators, but as we go into the LSZ expression we are normally  in momentum space.\\\\
In momentum space, the C-s equation implies a renormalised vertex $f^{\bm n}$ independant of scale:
$$
\Gamma[ \tilde \Phi] = \frac{1}{2} \int \frac{d^d k}{(2\pi)^d} Z_{\phi} \tilde \Phi(-k) ( k^2+ m^2 - \Pi(k^2)) \tilde \Phi(k) - \sum_{n \geq 3} \frac{Z_{\phi}^{\frac{n}{2}}}{n!} \int \prod_{j=1}^n ( \frac{d^d k_j}{(2\pi)^d}) (2 \pi)^d \delta^{(d)}(k_1 + ... + k_n) V_{rem}^{(n)}(k_1,..,k_n) \tilde \Phi(k_1)... \tilde \Phi(k_n)
$$
In $\phi^4$ theory, found $\Pi_1^{\bar{MS}} = \frac{gm^2}{32 \pi^2} (\log \frac{\mu^2}{m^2} - 1)$. Here we have:
$$
\tilde \Gamma^{(2)} (k^2= 0) = [ \tilde G^{(2)} (k^2 = 0) ]^{-1} = m^2 - \Pi_{1}^{\bar{MS}} =  m^2 - \frac{g m^2}{32 \pi^2} ( \log \frac{\mu^2}{m^2} - 1) 
$$
taking $Z_{\phi} = 1$  and as $\mu \frac{d}{d\mu} \tilde \Gamma^{(2)} (k^2 = 0) = 0$:
$$
\mu \frac{d m^2}{d\mu} = \frac{g m^2}{16 \pi^2} + O(g^2)
$$
We need to restore $\hbar$ by doing $\phi = \frac{\hbar \phi}{\sqrt{\hbar}}$ and $g = \hat g \hbar$. This is also covered in the typed notes.\\\\
$\bar{MS}$ earlier found:
$$
V_{ren}^{(4)} (0,0,0,0) = - g \mu^{\epsilon} + \frac{3 g^2 \mu^{\epsilon}{32 \pi^2} \log \frac{\mu}{m^2}
$$
So by taking $\mu \frac{d }{d\mu} V_{ren}^{(0)} = 0 \implies \beta(g) = \frac{3 g^2}{16 \pi^2} + O(g^2)$ as before.
\subsection{Renormalization group}
The problem that we have been trying to solve is that we were assuming we could intergrate down all the way to $0$. The renormalisation group allows us to do physics at large length scales without knowing what is going on at a small length scale. \\\\$QFT$ is not defined without a regulator. So we have to regulate the theory with a cutoff or some sort of dimensional regulaisation. We also have to impose renormalisation conditions using some external information. A useful theory has a small number of renormalisation conditions. Then you can made predictions. These predictions should be independant of our artificial choices. \\\\
$RG$ (Renormalisation Group) studies how theories of different microscopic details can give the same long distance physics.\\
"Universality" is key to this. The long distance physics should remain the same.\\
Real scalar theory gives a generic action with momentum cutoff $\Lambda_0$  in $d \in \mathbb{Z}^+$ dimensions;
$$
S_{\Lambda_0}[p] = \int d^d x (\frac{Z_{\Lambda_0}}{2} (\partial \phi)^2 + \sum_i \frac{ Z_{\Lambda_0}^{\frac{n_i}{2}}}{\Lambda_{0}^{d_i - d}} g_i( \Lambda_0) O_i(x))
$$
with $O_i(x)$ are local operators $O_i = ( \partial \phi)^n \phi^{s_i}$ with $n_i + r_i + s_i$ fields and with diemsnions $[O_i] = d_i= r_i [ \partial \phi] + s_i [\phi]$. So $g_i( \Lambda_0)$ are dimensionless and the mass is include $m^2 ( \Lambda_0) = g_i (\Lambda_0) \Lambda_0^2$.
\subsubsection{Effective actions}
$$
Z_{\Lambda_0} (\{ g_i (\Lambda_0)\}) = \int^{\Lambda_0} \mathfrak{D} \phi e^{- S_{\Lambda_0}[\phi]}
$$
integrate over fields $\tilde \phi(p)$ with $|p| \leq \Lambda_0$. Split low and igh momentum modes along lines $\Lambda < \Lambda_0$ and integrate out the high ones:
$$
\phi(x) = \int_{|p| < \Lambda_0} \frac{d^d p}{(2\pi)^d} e^{i p x} \tilde \phi(p) = \int_{|p| < \Lambda}...  +  \int_{\Lambda < p < \Lambda_0} ...) = \phi^-(x) + \phi^+ (x)
$$
These are disjoint with $\mathfrak{D} \phi = \mathfrak{D} \phi^+ \mathfrak{D} \phi^-$\\
Integrate over $\phi^+$ to get an effective action (instead of $w$ use $S^{eff}$):
$$
S_{\Lambda}^{eff} ( \phi^-) = - \log \int_{\Lambda}^{\Lamdba_0} \mathfrak{D} \phi^+ e^{- S_{\Lambda_0} (\phi^+ + \phi^-)}
$$
\section{Lecture 12}
$\phi^+(x)$ are UV and $\phi_-(x)$ are IR modes. We want to split up the action into:
$$
 S_{\Lambda_0} [ \phi^- + \phi^+] = S^{free} [ \phi-] + S^{free}[\phi^+] + S_{\Lambda_0} ^{int} [\phi^-, \phi^+]
$$
where 
$$
S^{free}[\phi] = \frac{Z_{\Lambda_0}}{2} \int d^4 x ( (\partial \phi)^2 + m^2 \phi^2)
$$
There are no quadratic terms $\phi^+ \phi^-$ as they have disjoint support on momentum space. $\tilde \phi^-(k) \tilde \phi^+(k') \delta( k+k') vanishes due to "disjoint support" as $\tilde \phi^-(k)$ contains modes less than $\Lamdba$ and $\tilde \phi^+(k)$ contians modes greater than $\Lambda$. Either $k> \Lamdab$ and then $\tilde \phi^- = 0$ and if $k< \Lambda$ then $\phi^+ = 0$. \\\\
We want to write:
$$
S_{\Lambda}^{eff} ( \phi^-) = - \log \int_{\Lambda}^{\Lamdba_0} \mathfrak{D} \phi^+ e^{- S_{\Lambda_0} (\phi^+ + \phi^-)} = S^{free} [ \phi^-] - \log \int_{\Lamdba}^{\Lambda_0} \mathfrak{d} \phi^+ \exp ( - S^{free} [ \phi^+]  - S^{int}_{\Lamdba_0} [\phi^-, \phi^+]) = S^{free} [ \phi^-} + S_{\Lambda}^{int} [\phi^-}
$$
We wrote $S_{\Lambda_0}$ generically so we should be able to drop (-) superscript:
$$
S_{\Lambda}^{eff} [\phi] = \int d^d x ( \frac{Z_{\Lambda}}{2} (\partial \phi)^2 + \sum_i \frac{Z_n^{ \frac{n_i}{2}}}{\Lambda^{d_i-d}} g_i (\Lambda) O_i( x))
$$
The upshot is the integration can be thought of as changing the couplings at normalisation of $\Lambda$\\\\
First lets focuse on the couplings themselves:
\subsubsection{Running couplings}
The partition functions 
$$
 Z_{\Lambda_0} (g_i(\Lambda_0)) = \int^{\Lamdba_0} \mathfrak{D} \phi e^{- S_{\Lambda_0}[\phi]} = Z_{\Lamdba}( g_i (\Lambda)) = \int^{\Lamdba} \mathfrak{D} \phi e^{- S_{\Lamdba}^{eff} [\phi]}
$$
The LHS $$\Lambda \frac{d}{d\Lambda} Z_{\Lambda_0} = 0 \implies \Lamdba \frac{d}{d\Lamdba} Z_{\Lambda} =0 = ( \Lambda \frac{\partial}{\partial \Lamdba} |_{g_i} + \Lambda \frac{dg_i}{d\Lambda} \frac{\partial }{\partial g_i}) Z_{\Lamdba}$$
So we get $\beta_i =Lambda \frac{dg_i}{d\Lambda} $. So now we have a set of $\beta$ functions:
$$
\beta_i = \frac{d g_i}{d \Lambda}
$$
may depend on all $\{ g_j\}$.
$$
\beta_i ( \{ g_j(\Lambda) \}) = ( d_i - d) g_i (\Lamdba)  + \beta^{qu}_i( \{g_j\})
$$
the first term on RHs is refered to as the "clasical" part due to the explicit $\Lambda$ in $S^{eff}$ and the second term refres to "quantum" part due to integrating out the UV modes.\\\\
This discussion implies there is a flow in coupling "constant" space. So we might imagine there is some coupling space and our initial value $\Lamdba_0$ is a particular point in this space and then integrating out the UV modes causes it to flow to someother coupling $\Lamdba$. \\\\
Now lets look at coupling constants.
\subsubsection{Correlation functions}
At the momentum the typed notes are a bit of the mess they are currently awful have a look after wednesday when he should have fixed them. \\\\
Quantum effective action (which somehow contained the result of suming those connected diagrams) should give physical results independant of cutoff, up to field renormalization. This amounts to saying $\Gamma[\Phi_{orig} = \Gamma[ \Phi_{\Lambda_0}] = \Gamma[ \Phi_{\Lambda}]$ 
$$
Z_{\Lambda_0}^{-n/2} \Gamma^{(n)}_{\Lambda_0} ( x_1,..., x_n) = Z_{\Lambda}^{-n/2} \Gamma_{\Lambda}^{(n)}(x_1,...,x_n)
$$
so $Z_{\Lambda_0}^{-n/2} \Phi_{\Lambda_0} = Z_{\Lambda}^{-n/2} \Phi_{\Lamdba}$ and as the LHS is independant of $\Lamdba$ we get:
$$
0 = ( \Lambda \frac{\partial}{\partial \Lambda} + \beta_i \frac{\partial }{\partial g_i} - n \gamma_{\phi}) \Gamma^{(n)}(x_1, x_2,...,x_n)
$$
with $\gamma_{\phi} = \frac{1}{2} \Lambda \frac{d}{d\Lambda} \log Z_{\Lambda}$. With the Legendre transform $W[J]$ we get:
$$
 W_{\Lamdba_0}[ J_{\Lamdba_0}] = \Gamma[ \Phi_{\Lamdba_0}] - J_{\Lambda_0} \Phi_{\Lmabda_0} = W_{\Lambda}[ J_{\Lambda}] = \Gamma[ \Phi_{\Lamdba}] - J_{\Lamdba} \Phi_{\Lambda}
$$
So we ahve $Z_{\Lambda_0}^{-n/2} J_{\Lambda_0} = Z^{-n/2} _{\Lamdba} \Phi_{\Lamdba}$. So 
$$
Z_{\Lambda_0} ^{n/2} G_{\Lambda_0} ^{(n)} (x_1,...,x_n) = Z_{\Lambda_0}^{n/2} \frac{\delta ^{(n)} W_{\Lambda_0} [ J_{\Lamdba_0}]}{ \delta J_{\Lamdba_0}(x_1)... \delta J_{\Lamdba_0}(x_n)} = Z_{\Lambda} ^{n/2} \frac{\delta^{(n)} W_{\Lamdba} ( J_{\Lamdba)}}{ \delta J_{\Lambda}(x_1)... \delta J_{\Lambda}(x_n)} = Z_{\Lamdba}^{n/2} G_{\Lambda}^{n/2} (x_1,...,x_n)
$$
 so gives C-S equation
 $$
 0 = ( \Lamdba \frac{\partial}{\partial \Lamdba} + \beta_i \frac{\partial}{\partial g_i} + n \gamma_{\phi} ) G_{\Lambda}^{(n)} (x_1,...x_n)
 $$\\\\
 Let us integrate a few more modes $p$ with $s \Lamdba < p < \Lamdba $ with $0< s< 1$ with $1-s$ small. Then:
 $$
  Z_{\Lambda}^{n/2} G_{\Lamdba}^{(n)} (x_1, ..., x_n; g(\Lamdba)) = Z_{s \Lamdba}^{n/2} G_{s\Lambda}^{(n)} ( x_1,...,x_n;g(s\Lamdba))
 $$
 Looking at the RHS $G_{s\Lambda}^{(n)} ( x_1,...,x_n;g(s\Lamdba))$ from $W_{s\Lamdba}[J]$ with $$S^{eff}_{s\Lambda} - J \phi = \int d^d x [ Z_{s\phi} ( \partial \phi)^2 + \sum_i \frac{g_i(s \Lambda)}{(s\Lambda)^{d_i-d}} O_i(x) - J(x) \phi(x)]$$
  We wantt o rescale from $x \rightarrow x' = sx$ so $\Lambda \rightarow \Lamdba' = \frac{\Lamdba}{s}$, $\partial \rightarrow \partial' = \frac{1}{s} \partial$. Field has massi dimension $\frac{d}{2} -1 $ so $\phi \rightarrow \phi' = s^{1- \frac{d}{2}} \phi$ and similarly $O_i \rightarrow O'_i = O_is^{-d}$ and $J\rightarrow J' = J s^{d/2 -1}$. So after rescaling we get:
  $$
  W_{s\Lambda} [ J] = W_{\Lambda} [J']
  $$
  So RHS of equation is equal to:
  $$
   G_{s\Lamdba}^{(n)} (x_1,...,x_n; g(s\Lamdba)) = s^{n ( \frac{d}{2} -1)} \frac{\delta^{(n))} W_d[ J']} {\delta J'(x_i} - \delta J'(x'_n)} = s^{n (\frac{d}{2} -1)} G_{\Lamdba}^{(n)} (sx_1, ..., sx_n; g(s\Lamdba))
  $$
  so if we let $y_i = sx_i$ we get:
  $$
  G_{\Lambda}^{(n)} ( \frac{y_1}{s},..., \frac{y_n}{s}; g(\Lamdba)) = ( \frac{Z_{s\Lambda}}{Z_{\Lambda}} s^{d-2} )^{n/2} G_{\Lambda}^{(n)}(y_1,..,y_n; g(s\Lambda))
  $$
  \section{Example Sheet 2}
  We know that $S[\phi]$ is a phase so it must be dimensionless and as $[d^d x] = -d$ we must have $[\frac{1}{2} (\partial \phi)^2 + ..] =d$ and as $\partial$ has length dimension -1 it must have mass dimension 1 so $[\phi] = \frac{d}{2} -1$.\\\\
  The argument of $\log$ or $\exp$ must always be dimensionless which is important when expanding out things like $(\frac{\Delta^2}{4 \pi \mu^2})^{- \epsilon}{2}$.\textbf{ So you need to bring the $\mu^{\epsilon}$ inside the integral before you expand out anything so you only ever expand dimesnionless quantities.}\\\\
  Super useful forumula!:
  $$
   \int \frac{d^d l}{(2\pi)^d} \frac{l^{2a}}{(l^2+ \Delta)^b} = \frac{ \Gamma( b - a -\frac{d}{2}) \Gamma ( a + \frac{d}{2})}{ ( 4 \pi)^{d/2} \Gamma(b) \Gamma( d/2)} \Delta^{-(b-a - \frac{d}{2}}
  $$\\\\
  \textbf{Finally figured out how to think of symmetry factors!!!}: First label the half edges leaving a vertex and then draw the directions of the propogators. If you can swap propogators and half edges at the same time to create a diagram that looks exactly the same (by the same we mean the same edge maps to the same leg and they are topologically equivalent (no twists or knots)). For an example look at the back of the first pad of paper in blue folder.
  \section{Lecture 14}
  $$
  G_{\Lambda}^{(n)} ( \frac{y_1}{s},..., \frac{y_n}{s}; g(\Lamdba)) = ( \frac{Z_{s\Lambda}}{Z_{\Lambda}} s^{d-2} )^{n/2} G_{\Lambda}^{(n)}(y_1,..,y_n; g(s\Lambda))
  $$
  On the LHS external points get further apart as $s$ decreses from 1, whereas the coupling is fixed. On the RHS, the external points are fixed and the coupling runs to lower momentum scale. This us whether interactions appear stronger or weaker at longer distances. Forces like QED have couplings get weaker at longer distances whereas QCD gets stronger at longer distances. So QCD is a strong theory, but at short distances it becomes weakly coupled and perturbation theory starts to work.\\\\
  Now lets investigate the prefactor a bit more. If we know that the mass dimension of $G^{(n)}$ in $n ( \frac{d}{2} -1) = < \phi(x_1)... \phi(x_n) >^{conn}$. We want to expand this prefactor about small $\delta s = 1-s$:
  $$
   f(s) = ( \frac{Z_{s\Lamdba}}{Z_{\Lamdba}} s^{d-2} )^{\frac{1}{2}} = f(1) - f'(1) \delta s 
  $$
  can write 
  $$
   f'(s) = f(s) \frac{d}{ds} \log f(s) = f(s) ( \frac{d+2}{2s} + \fra{1}{2s} s \frac{d}{ds} \log Z_{s\Lambda})
  $$
  $$
   \gamma_{\phi} = \frac{1}{2} \Lambda' \frac{d}{d\Lambda'} \log Z_{\Lambda'} = \frac{s}{2} \frac{d}{ds} \log Z_{s \Lambda}
  $$
  Then 
  $$
   ( \frac{Z_{s \Lambda}}{Z_{\Lambda}} s^{d-2})^{\frac{1}{2}} = 1 - \delta s( \frac{d-2}{2} + \gamma_{\phi})
  $$
  So our scaling dimension is defined as : $\Delta_{\phi} = ( \frac{d}{2} - 1) + \gamma_{\phi}$ which is the mass dimension of the field plus a correction (anomalous dimension of $\phi$.
  \subsection{RG flow}
  Now consider the different types of behaviour of this coupling flow, and we will see that things are connected to the scaling dimensions. We want to consider RG trajectories from $\Lambda_0$ to $\Lambda$ in coupling constant space governed by the $\beta$ functions.When we flow from one point in coupling constant space to another is by integrating out modes with the condition that low energy physics remains the same, so we can pick different initial positions but they will give the same family of flows with different laws of low energy physics. So theories lying on the same trajectory describe equivalent long distance physics (by construction).\\\\
  Interestingly you can have fixed points of this flow, which are points in coupling constant space where all $\beta_i|_{\{g^*_j\}} = 0 \forall i$ vanish. \\
  Recall:
  $$
   \beta_i ( \{ g_j\}) = (d_i - d) g_i + \beta^{qu}( \{ g_i\})
  $$
  There is always the trival ("Gaussian") fixed point with $g_i^* = 0 \forall i$. Free theory with no interactions. To have a non-trival fixed point we need some cancellation between the classical and quantum parts of the beta function. These will probably be a non-peturbative theory which means it is strongly interacting. We are going to look at behaviour near the trivial fixed point. A theory sitting at a fixed point has scale invariance as the $\beta$ functions don't move so the couplings don't run $g_i^*$ are independant of scale. So all functions of these couplings are also independant of scale $\Lambda$ (e.g. $\gamma_{\phi} ( g_i^*) = \gamma_{\phi}^*$. The Callum Sysm equation simplifies for e.g. $G^{(n)}$ so the explict scale dependance of $G$:
  $$
   \Lambda \frac{\partial}{\partial \Lambda} G_{\Lambda}^{(2)} (x,y) = - 2 \gamma_{\phi}^* G_n^{(2)} (x,y)
  $$
Assume we are working in a theory with translational invariance so the two point function only depends on the distance between two points:
$$
 G^{(2)} (x,y) = G^{(2)} (|x-y|)
$$
Like $<\phi(x) \phi(y)>$ $G^{(2)}$ has mass dimension of $d-2$, so we have to have:
$$
G^{(2)}_{\Lambda} (x,y; g_i^*) = \frac{\Lambda^{d-2}}{ \Lambda^{2 \Delta_{\phi}}} \frac{1}{|x-y|^{2 \Delta_{\phi}}}
$$
where $\Delta_{\phi} = \frac{1}{2} (d-2) - \gamma_{\phi}^*$. The above is constructed to satisfy the C-S equation with the correct mass dimension. We see that:
$$
 G^{(2)}_{\Lambda} \sim \frac{1}{|x-y|^{2 \Delta_{\phi}}}
$$
 power-law decay which is very slow compared to exponential decay and this is characterisic of a scale invariant theory.\\\\
 If we have a scale like a mass gap (some other scale in the problem) then this introduces some correlation length that gives us some exponential decay.\\
 Note that in the theory with a mass scale $m$ then generically we expect that scale to come into correlation fucntions as:
 $$
 G^{(2)} \sim \frac{e^{-m |x-y|}}{|x-y|^{2 \Delta_{\phi}}}, m \sim \frac{1}{\zeta}$
 $$
 For correlation length $\zeta$.\\\\
 Whilst we say power law for long distance QED or low energy gravity at short distances quantum corrections ruin that.\\\\
 We now consider theories that have RG flows that go near fixed points (especially the guassian fixed point where the couplings are vanishing). We can linearise the RG equations near a fixed point:\\
 Let $\delta g_j = g_j - g_j^*$ and then the equations become:
 $$
 \Lambda \frac{d g_i}{d\Lambda}|_{g_i^{\dagger} \delta g_j^*} = B_{ij} \delta g_j + O(( \delta g_j)^2
 $$
 Eigenvectors of $B$ are $\sigma_i$ with eigenvalues $\Delta_i - d$. $\Delta_i$ is the scaling dimension of $\simga_i$ which is a linear combination of $\{O_i\}$.\\
 Now let's write down the linearised RG equations:
 $$
 \Lambda \frac{d \sigma_i}{d \Lambda} = ( \Delta_i - d) \sigma_i
 $$
 so the magnitude of our eigenvector is
 $$
 \sigma_i(\Lamdab ) = ( \frac{ \Lamdba}{\Lambda_0})^{ \Delta_i -d}\sigma_i( \Lambda_0)
 $$
 for some intitial $\Lamdba_0$ with $\Lambda < \Lamdba_0$. So we classify behaviour based on the signs of these exponents. First we will take the case $\Delta_i > d \implies \sigma_i ( \Lamdba) < \sigma_i( \Lamdba_0)$ so the flow is towards the fixed point where they don't effect things as they don't contribute any scales. Therefore this is called the irrelevant direction in coupling constant space.  In the case $\Delta_i < d \implies \sigma(\Lambda) > \sigma_i (\Lambda_0)$ then you flow away from that fixed point which we called relevant. Finally we have $\Delta_i = d \implies $ marginal so we must go to a higher order. \\\\
 If we think about coupling constant space there will be a very large subspace of irrelevant couplings that is called the critical surface (this is infinite dimensional). It should contain the fixed points and is the surface that contains all the irrelevant couplings that flow to the fixed point.
 \section{Lecture 15}
 Corrections to previous lecture:
 $$
 \phi'(x') = s^{1 - \frac{d}{2}} \phi(x)
 $$
 $$
J'(x')= s^{- \frac{d}{2} -1} J(x)
 $$
 Also from 
 $$
  \frac{\delta W}{\delta J} = - \Phi, \frac{\delta}{\delta J'(x')} = s^{1-\frac{d}{2}} \frac{\delta }{\delta J(x)}
 $$
\\\\
Relevant couplings are ones that flow away form fixed points as we integrate out more modes, whereas irrelevant couplies flow towards fixed points. In theory we have many couplings that are relevant and irrelevant. If we imagine a critical surface in coupling constant space with a fixed point on it. On the surface the flow is all into the fixed point, however if we draw the renormalised trajectory perpendicular to the critical surface then flows near the cirtical surface will flow towards the fixed point and then away along parallel to the renormalised trajectory. 
\subsubsection{Continiuum limit and renormalizability}
Continium limit is considering if the theory is fine as $\Lambda_0 \rightarrow \infty$. The question of renormalizability is linked to the sensitivity of our initial couplings. There are three cases to consider:
\begin{itemlist}
\item Only irrelevant couplings. Then $g_i(\Lambda_0)$ lies on a critical surface. Flow to fixed point as $\frac{\Lambda_0}{\Lambda} \rightarrow \infty$. In this case the limit $\lim_{\Lambda_0} \rightarrow S_{\Lambda}^{eff}$ exists but it describes a scale invariant theory $g_i(\Lambda) = g_i^*$ so non renormalization conditions are needed to fix the couplings. These theories are called super renormalisable.\\
        \item 'There is at least one relevant couplings'. Flow away from fixed point in a relevant direction but for generic intiial  $g_i(\Lambda_0)$ we flow away from fixed point so we lose control of the calculation, as we think the physics is described by somthing close to the fixed point. The solution to this is to careffully chose the intitial couplings $g_i(\Lamdba_0)$ closer to critical surface as $\Lambda_0 \rightarrow \infty$. Flowis slower near fixed point so $\frac{\Lambda_0}{\Lamdba}$ can become larger and yet $g(\lambda)$ is still clsoe to fixed point. We have to impose renormalisation conditions which are initial conditions that tune $g_i(\Lambda_0)$ closer to the critical surface. Usually we have a finite number of relevenat couplings so have a finite number of renormalisation conditions, and theories that behave like this we call renormalisable.\\
        \item Irrelevant operators need to be tuned for correct description of physics ( Fermi theory with a neutron change into  a proton and a positron and an electron) Here we cannot allow flow into fixed point as $\Lambda_0 \rightarrow \infty$ does not give correct physics so non continumm limit exists. This is called a "nonrenormalizable" theory. There is normally a reason why. In pricniple, an infinite number of renoamlization conditions are needed, in practice using an expansion around small momenta or some other trick you can work with just a few operators up to some finite level of precision. 
\end{itemlist}
\subsection{Path integral quantization and symmetries}
Symmetries led us to useful identities involving correlation functions. use schwinger-dyson equation to link to canonical quantization.
\subsubsection{Schwinger-Dyson equations}
\textbf{Free massless scalar field theory}
$$
 S[\phi] = \frac{1}{2} \int d^4 y \partial_{\mu} \phi \partial^{\mu} \phi = - \frac{1}{2} \int d^4 y \phi \partial^2 \phi
$$
Consider $< \phi(x)> = \frac{1}{Z} \int \mathfrak{D} \phi \phi(x) e^{-S[\phi]}$ and shift $\phi(x) \rightarrow \phi(x) + \epsilon(x)$. As we are integrating over the whole space this should be invariant:
$$
<\phi(x)> = \frac{1}{Z} \int \mathfrak{D} \phi ( \phi(x) + \epsilon(x)) \frac{1}{2} \int d^4 y ( \phi(x) + \epsilon(x)) \partial^2 ( \phi(x) + \epsilon(x))
$$
 Expand the exponential:
 $$
 e^{\frac{1}{2} \int d^4 y \phi  \partial^2 \phi} ( 1 + \frac{1}{2} \int d^4 z ( \phi \partial^2 \epsilon + \epsilon \partial^2 \phi))  = e^{\frac{1}{2} \int d^4 \phi \partial^2 \phi}( 1+ \int d^4 z \epsilon \partial^2 \phi)
 $$
 Then
 $$
  < \phi(x)> = \frac{1}{Z} \int \mathfrak{D} \phi e^{ - S[\phi]} ( \phi(x) + \epsilon(x) + \phi(x) \int d^4 z epsilon(z) \partial_z^2 \phi(z))
 $$
  so
  $$
\frac{1}{Z} \int \mathfrak{D} \phi e^{ - S[\phi]} (  \epsilon(x) + \phi(x) \int d^4 z epsilon(z) \partial_z^2 \phi(z)) = 0
  $$
  so as
  $$
   \epsilon(x) = \int d^4 z \epsilon(z) \delta^{(4)}(z-x)
  $$
  $$
  \frac{1}{Z} \int \epsilon(z) \int \mathfrak{D} \phi e^{ - S[\phi]} ( \phi(x) \partial_z^2 phi(z) +\delta^{(4)}(z-z)) = 0
  $$
  therefore, we get the Schwinger-Dyson equation:
  $$
   \parital_z^2 < \phi(z) \phi(x)> = - \delta^{(4)} (z-x)
  $$
  Previous steps also work for larger $n$-point functions. e.g.
  $$
   \partial^2_z < \phi(z) \phi(x) \phi( y)> = - \delta^{(4)} (z-x) < \phi(y)> - \delta^{(4)} (z-y) < \phi(x)>
  $$
  Interactions 
  $$
  S= \int d^4 y ( - \frac{1}{2} \phi \partial^2 \phi + \mathfrak{L}_{int} [\phi])
  $$
  expand $\mathfrak{L}_{int} ( \phi+ \epsilon) = \mathfrak{L}_{int} [\phi] + \epsilon \mathfrak{L}'( \phi) + O(\epsilon^2)$ with $\frac{\delta \mathfrak{L}}{\delta \phi}$ so
   $$
    \partial_z^2 < \phi(z) \phi(x) > = < \mathfrak{L}'(\phi(z)) \phi(x) > - \delta^{(4)} (z-x)
   $$
   \section{Lecture 15}
   \subsubsection{Schwinger-Dyson + generating functional}
   $$
    Z[J] = \int \mathfrak{D} \phi \exp \int d^4 y ( \frac{1}{2} (\phi+ \epsilon) \partial^2 ( \phi + \epsilon) - \mathfrak{L}_{int} (\phi + \epsilon) + J( \phi - \epsilon)) = \int \mathfrak{D} \phi \exp ( - \int d^4 y ( \mathfrak{L} - J \phi)) ( 1 + \int d^4 z \epsilon(z) ( \partial_z^2 \phi - \mathfrak{L}'_{int} + J) _ ...)
   $$
   so as $O(\epsilon)$ vanishes:
   $$
   \partial_z^2 \int \mathfrak{D} \phi \phi(z) e^{- \int d^4 y ( \mathfrak{L} - J \phi)} =\int \mathfrak{D} \phi ( \mathfrak{L}'_{int} - J) e^{- \int d^4 y ( \mathfrak{L} - J\phi)}
   $$
   $$
    \partial_z^2 \frac{\delta Z [J]}{\delta J(z)} = \{ \mathfrak{L}'_{int} ( \frac{\delta}{\delta J(z)} - J(z) \} Z[J]
   $$
   Diffrentiatial equation for $Z[J]$.\\\\
   Previously we did $\phi(x) \rightarrow \phi(x) + \epsilon(x)$ with completely general $\epsilon(x)$. In the next section we want to think of specific transformations that leave the action invariant:
   \subsection{Symmetries and Wandi-Takashi identities}
   Often $\epsilon(x) = \eta f(\phi, \partial \phi)$ with $\eta$ small. Under $\phi \rightarrow \phi + \epsilon$ we have $\mathfrak{L} \rightarrow \mathfrak{L} + \delta \mathfrak{L}$
   with $\delta \mathfrak{L} = \frac{\delta \mathfrak{L}(y)}{\delta \phi(y)} \epsilon(y) + \frac{\delta \mathfrak{L}}{\delta ( \partial_a \phi(y))} \partial_a \epsilon (y)$\\\\
   If we look at the variation of the aciton
   $$
   S = \int d^4 y \mathfrak{L}
   $$
   $$ 
   \frac{\delta S}{\delta \epsilon(z)} = \int d^4 y \{ \frac{\delta \mathfrak{L}(y)}{\delta \phi(y)}  \delta^{(4)}(z-y) - \partial_{\mu} ( \frac{\delta \mathfrak{L}}{\delta ( \partial_{\mu} \phi) (y)} \delta^{(4)}(z-y)) = \frac{\delta \mathfrak{L}}{\delta \phi(z)} - \partial_{\mu} (\frac{\delta \mathfrak{L}}{\delta ( \partial_{\mu} \phi) (z)})
   $$
   $$
    \delta \mathfrak{L}(y) = \parital_{\mu} ( \frac{\delta \mathfrak{L}}{\delta ( \parital_{\mu} \phi)}\epsilon(y)) + \frac{\delta S}{\delta \epsilon(y)} \epsilon(y) =  j^{\mu} (y)
   $$
    wiht $j^{\mu}(y)$ the Noether current of transformation. this all implies that:
    $$
     \partial_{\mu} j^{\mu} = \delta \mathfrak{L}(y) - \frac{\delta S}{\delta \epsilon(y) }\epsilon(y)
    $$
    Classically if field equation is satisfied then $\partial_{\mu} j^{\mu} = 0$\\\\
    If we take our action and seperate out the interacting part as before:
    $$
     S= \int d^4 y ( - \frac{1}{2} \phi \partial^2 \phi + \mathfrak{L}_{int} (\phi))
    $$
    $$
     \frac{\delta S}{\delta \epsilon(z)} = \frac{\delta \mathfrak{L}}{\delta \phi(z)} - \partial_{\mu} ( \frac{\delta \mathfrak{L}(z)}{\delta ( \partial_{\mu} \phi(z))} = \mathfrak{L}_{int}'(\phi) - \delta^2 \phi
    $$
    Look at S-D equation
    $$
     \partial_z^2 < \phi(z) \phi(x) > - < \mathfrak{L}_{int}' (\phi(z))> = - \delta^{(4)} (z-x)
    $$
    $$
     < \frac{ \delta S}{\delta \epsilon(z)} \phi(x)> = \delta^{(4)} (z-x)
    $$
    If $\delta \mathfrak{L} = 0$ ( transformation is a symmetry of $\mathfrak{L}$):
    $$
    \partial_{\mu} j^{\mu} =  - \frac{\delta S}{\delta \epsilon} \epsilon 
    $$
    $$
     \frac{\partial}{\partial Z^{\mu}}< j^{\mu}(z) \phi(x) < = - \delta^{(4)} (z-x) <\epsilon(x) >
    $$
    whic is hte Wadi-takahashi identity.\\\\
    \subsubsection{Symmetry and effective actions}
    Assume $\phi \rightarrow \phi'(x) = \phi(x) + \epsilon(x)$, $\epsilon(x) = \eta f( \phi, \partial \phi)$ leaves $\mathfrak{D} \phi e^{- S(\phi)}$ invariant. (in most cases these are also separately invariant but they don't need to be).
    $$
    Z[J] = \int \mathfrak{D} \phi' \exp ( - S[ \phi'] + \int d^4 x J(x) \phi'(x)) = \int \mathfrak{D} \phi e^{- S[ \phi] + \int d^4 x J \phi} ( 1 + \int d^4 x J(x) \epsilon(x) + ... ) = Z[J] ( 1 + ... ] \implies \int d^4 x J(x) < \epsilon(x)>_J = 0
    $$
    Now we have gone from a symmetry inside the path integral to now a relation involving expecation values. We turn this into the quatnum effective action\\
    \textbf{Quantum effective action}
    $$
     \frac{\delta \Gamma[\phi]}{\delta \Phi(y)} = J_{\Phi}(y) \implies \int d^4 x \frac{\delta \Gamma[\Phi] }{\delta \Phi(x)} < \epsilon(x)>_{\Phi} = 0
    $$
    This imples that $\Gamma(\Phi)$ is invariant under $\Phi(x) \rightarrow \Phi(x) + < \epsilon(x)>_{\partial \Phi}$ which is the Slavnonv-Taylor identities. \\\\
    If the transformation is actually lienar in the fields $\epsilon = \eta f( \phi, \partial \phi)$ is linear then we can bring the expecation value insdie
    $$
    < \epsilon(x)> _{\partial \Phi} = \eta < f( \phi, \partial \phi)>_{\partial \Phi}> = \eta f( \Phi, \partial \Phi)
    $$
    so $\phi \rightarrow \phi + \eta f( \phi, \partial \phi$ is a symmetry of S implies that $\Phi \rightarrow \Phi + \eta f(\Phi, \partial \Phi)$ is symmetry of $\Gamma$.
    \subsection{Quantum Electrodynamics}
    In Euclidian spacetime we have a classical action $S[\psi, \bar \psi, A] = \int d^4 x ( \frac{1}{4} F_{\mu \nu} F^{\mu \nu} + \bar \psi ( \cancel D + m) \psi$ with $\cancel D = \gamma^{\mu}( \partial_{\mu} - ie A_{\mu})$ and $F_{\mu \nu} = \partial_{\mu} A_{\nu} - \partial_{\nu} A_{\mu}$
    $$
    Z = \int \mathfrak{D} \psi \mathfrak{D} \bar \psi \mathfrak{D} A e^{- S[ \psi, \bar \psi, A]}
    $$
     Gauge transform: $\psi(x) \rightarrow e^{i \alpha(x)} \psi(x)$, $\bar \psi \rightarrow e^{-i \alpha(x)} \bar \psi$ and $A_{\mu} \rightarrow A_{\mu} + \frac{1}{e} \partial_{\mu} \alpha(x)$
     \\\\
     Some stuff that is specific to eulcian spacetime:
     $$
      \{ \gamma_{\mu}, \gamma_{\nu} \} = 2 \delta_{\mu \nu}
     $$
      Choose convention such that $\gamma_{\mu}^{\dagger} = \gamma_{\mu}$ and they are expressed interm sof the Pauli matrices:
      $$
       \gamma_j = \begin{pmatrix} 0 & - i \simga_j \\ i \sigma_j & 0 \end{pmatrix}, \gamma_4 = \begin{pmatrix} 1 & 0 \\ 0 & -1 \end{pmatrix} or \begin{pmatrix} 0 & 1 \\ 1 & 0 \end{pmatrix}
      $$
      \section{Lecture 17}
      \subsection{Feynman rules of QED}
      \subsubsection{Photon propagator}
      Introduce source $J^{\mu}(x)$ so our free propogator:
      $$
       Z_0[J] = \int \mathfrak{D} A \exp ( - \int d^4 x ( \frac{1}{4} F^2 - J^{\mu} A_{\mu}))
      $$
      Gauge action:
      $$
       S_g[A] - \int d^4 x J^{\mu} A_{\mu} = - \int d^4 x ( \frac{1}{2} A_{\mu} ( \delta^{\mu \nu} \partial^2 - k^{\mu} k^{\nu}) A_{\nu} + J^{\mu} A_{\mu})
      $$
      (after integrating by parts)\\\\
      Classical EOM is given by $\delta S = 0$ under $A_{\mu} \rightarrow A_{\mu} + \detla A_{\mu}$ which gives:
      $$
       J^{\mu} = ( \partial^{\mu} \partial^{\nu} - \delta^{\mu \nu} \partial^2) A_{\nu} = \partial_{\nu} ( \partial ^{\mu} A^{\nu} - \partial ^{\nu} A^{\mu}) = \partial_{\nu} F^{\mu \nu}
      $$
      $$
       \partial_{\mu} J^{\mu} = \partial_{\mu} \partial_{\nu} F^{\mu \nu} = 0
      $$
      therefore $F^{\mu \nu} = - F^{\nu \mu}$.\\\\
      Fourier transform:
      $$
      A_{\mu} (x) = \int \frac{d^4 k}{(2\pi)^4} e^{i k \cdot x} \tilde A_{\mu} ( k) 
      $$
      $$
      S_g - \int J^{\mu} A_{\mu} = \frac{1}{2} \int \frac{d^4 k}{(2\pi)^4} ( \tilde A_{\mu} (-k) ( k^2 \delta^{\mu \nu} - k^{\mu} k^{\nu} ) \tilde A_{\nu} (k) - \tilde J^{\mu} (-k) \tilde A_{\mu} (k) - \tilde J^{\mu} (k) \tilde A_{\mu}(-k))
      $$
      Problem is the coeffficent of $A^2$ is proportional to :
      $$
       P^{\mu \nu} (k) = \delta^{\mu \nu} - \frac{k^{\mu} k^{\nu}}{k^2}
      $$
      This is a projection operator $P^{\mu}_{\nu} P^{\nu \sigma} = P^{\mu \sigma}$ so the eigenvalues are wither 0 or 1 as 
      $$
       \lambda v = Pv = P^2 v = P(Pv) = \lambda^2 v
      $$
      In this case $k_{\nu}$ is the zero eginevector $P^{\mu \nu}(k) k_{\nu} = 0$. Therefore, we find that for certain classes of $A$ fields this operator acting on $A$ gives us zero:
      $$
      \frac{1}{2} \int d^4 x \tilde A_{\mu}(-k) k^2 P^{\mu \nu} \tilde A_{\nu} (k) = 0
      $$
      for fields satisfying $\tilde A_{\mu} (k) = k_{\mu} \tilde \alpha(t)$ or in position space $A_{\mu} (x) = \partial_{\mu} \alpha(x)$ that is for fields that gauge equivalent to $A_{\mu}(x) = 0$. Also from momentum space have $k_{\mu} J^{\mu} = 0$. \\\\
      We will come up with a fix for today that works for QED and then later we will need to come up with a fix for more general gauge theories:\\\\
       Since trace of $P^{\mu \nu} (k)$ is:
       $$
        \delta^{\mu \nu} P^{\mu \nu} (k) = 3 = \sum \text{eigenvalues}
       $$
       Restrict our path integral to fields which are transverse (with $\tilde A_{\mu}(k) = k_{\mu} \tilde \alpha(k)$ being longitudal) so we want:
       $$
        k^{\mu} \tilde A_{\mu} = 0 \implies \parital^{\mu} A_{\mu} = 0
       $$
       this is the Lorentz gauge,( or Landau gauge). In this subspace, $P^{\mu \nu}$ is the identity so the inverse is easy to take so:
       $$
       ( k^2 P^{\mu \nu} )^{-1} = \frac{1}{k^2} P^{\mu \nu}
       $$
       Usual steps lead to:
       $$
       Z_0(J) = \exp( \frac{1}{2} \int \frac{d^4 k}{(2 \pi)^4} \tilde J_{\mu} (-k) \frac{P^{\mu \nu}}{k^2}\tilde J_{\nu} (k))
       $$
       This is the Landau gauge propogator $\frac{1}{k^2}$. So canc include $k^{\mu} k^{\nu}$ with any coefficent, say $1- \xi$ so we can write the propogator as:
       $$
       \tilde D^{\mu \nu} (k) = \frac{1}{k^2} ( \delta^{\mu \nu} - (1- \xi) \frac{k^{\mu} k^{\nu}}{k^2})
       $$
       with $\xi = 0$ is landau gauage and $\xi =1$ is Feynman gauage. So we can reverse engineer the action that would have given this propagator. This is the Gauge-fixed action:
       $$
       S_g = \int d^4 ( \frac{1}{4} F^2 + \frac{1}{2\xi} ( \partial^{\mu} A_{\mu})^2)
       $$
       That was the interesting feynmann rule the rest just are the same as in zero dimensions\\\\
       Femion (electron) propogator form $\bar \psi( \cancel \partial + m) \psi$ with:
       $$
       \psi(x) = \int \frac{d^4 p}{(2\pi)^4} e^{i p \cdot x} \tilde \psi(p), \bar \psi = \int \frac{d^4 p}{(2\pi)^4} e^{i p\cdot x} \tilde{\bar \psi}(P)
       $$
       so action is:
       $$
       S_f ( \tilde \psi, \tilde{\bar \psi}) = \int \frac{d^4 p}{(2\pi)^4} \tilde{\bar \psi} (-p) ( i \cancel p + m ) \tilde \psi (p)
       $$
       $$
        Z[ \eta, \bar \eta] = \int \mathfrak{D} \tilde \psi \mathfrak{D} \tilde{\bar \psi} \exp ( - \int_p ( \tilde{\bar \psi} ( i \cancel p + m) \tilde \psi - \tilde{\bar \eta} \tilde \psi + \tilde{\bar \psi} \tilde \eta) = Z[ 0,0] \exp( - \tilde{\bar \eta} ( i \cancel p + m)^{-1} \tidle \eta)
       $$
       So we get Fermion propogator:
       $$
        \tilde S_F (p) = \frac{1}{i \cancel p + m}
       $$
       $$
        S^{\alpha \beta}_F = ( \frac{1}{i \cancel p + m})^{\alpha \beta} = \frac{- i \cancel p^{\alpha \beta} + m \delta^{\alpha \beta}}{p^2 + m^"}
       $$
        Interactions from $\bar \psi \cancel D \psi$ gives 
        $$
        ie A_{\mu}(x) \bar \psi^{\alpha}( x) ( \gamma^{\mu})^{\alpha \beta} \psi^{\beta}( x)
        $$
        this gives interaction term $- ie \gamma^{\mu}$.\\\\
        We also need to include a factor of -1 for every fermion loop. We will do this in canconical quantisation as in path integrals is it is tedious, and you have to use the generating funcitonal:
        $$
         Z[\eta, \bar \eta, J] \sim \exp ( i e\int d^4 x ( \frac{\delta}{\delta J(x)} ( \frac{\delta}{\delta \eta^{\alpha}(x)}) ( \gamma^{\mu})^{\alpha \beta} ( \frac{\delta}{\delta \bar \eta(x)} Z_0 (\eta, \bar \eta) Z_0[J]
        $$
        exercise to show that need a factor of -1 for every fermion loops.
        \subsubsection{Vacuum polarization}
We want to calculate the sum of the amputated 1PI diagrams
$$
\Pi^{\mu \nu} (q) = ~O~ + 2-loop + ...
$$
In $d$-dimensions, we can go through the same sort of dimensional analysis to write $e^2 = \mu^{\epsilon} g^2(\mu)$, $\epsilon = 4 -d$:
$$
 \Pi_1^{\mu \nu} = - \mu^{\epsilon} (i g)^2 \int \frac{d^dp}{(2\pi)^d} tr( \frac{1}{i \cancel p+ m} \gamma^{\mu} \frac{1}{i( \cancel p - \cancel q) +m} - \gamma^{\nu})
$$
$$
\Pi_1^{\mu \nu} = \mu^{\epsilon} g^2 \int \frac{d^d p}{(2\pi)^d} tr \frac{ (-i \cancel p + m) \gamma^{\mu} ( - i ( \cancel p - \cancel q) + m) \gamma^{\nu}}{( p^2 + m^2) ( (p-q)^2 + m^2)}
$$
 We want the demoninator in the form $(l^2 + \Delta)^b$ so we use Feynmanns trick:
 $$ 
 \frac{1}{AB} = \int^1_0 dx \int^1_0 dy \delta(x+y-1) \frac{1}{(Ay+Bx)^2}
 $$
 to write:
 $$
 \Pi_1^{\mu \nu} = \mu^{\epsilon} g^2 \int \frac{d^d l}{(2\pi)^d} \int^1_0 dx \frac{ tr ((-i \cancel p + m) \gamma^{\mu} ( - i ( \cancel p - \cancel q) + m) \gamma^{\nu})}{(p-qx)^2 + m^2 + q^2(x(1-x))}
 $$
 shift the loop momentum $l = p - qx$ and define $\Delta = m^2 + q^2x(1-x)$
 $$
 \Pi_1^{\mu \nu} = \mu^{\epsilon} g^2 \int \frac{d^d l}{(2\pi)^d} \int^1_0 dx \frac{ tr ((-i (\cancel l + \cancel qx) + m) \gamma^{\mu} ( - i ( \cancel l - \cancel q(1-x)) + m) \gamma^{\nu})}{l^2+ \Delta}
 $$
 As the spin traces in the Euclidean case:
 $$
  tr \gamma^{\mu} \gamma^{\nu} = 4 \delta^{\mu \nu}
 $$
 $$
  tr ( \gamma^{\mu} \gamma^{\phi} \gamma^{\nu} \gamma^{\sigma}) = 4 (\delta^{\mu \phi} \delta^{\nu \sigma} - \delta^{\mu \nu} \delta^{\phi \sigma}+ \delta^{\mu \sigma} \delta^{\nu \phi}
 $$
 so we get
 $$
 tr (-i (\cancel l + \cancel qx) + m) \gamma^{\mu} ( - i ( \cancel l - \cancel q(1-x)) + m) = 4 ( - (l + qx)^{\mu} ( l- q(1-x))^{\nu} + (l + qx)(l-q(1-x))\delta^{\mu \nu} - ( l + qx)^{\nu} ( l- q(1-x))^{\mu} + m^2 \delta^{\mu \nu}
 $$
 As $d >4$ we have superfical degree of divergence $0, (1), 2$. Integrals over odd powers of $l$ vanish and similarly only the diagonal parts of $l^{\mu} l^{\nu}$ will give nonvanishing integrals. As integrating over even space, so replace $l^{\mu} l^{\nu} \rightarrow \frac{1}{d} \delta^{\mu \nu} l^2$. So then use the master integral:
 $$
 \int \frac{d^d x}{(2\pi)^d} \frac{(l^2)^a}{(l^2 + \Delta)^b} = \frac{1}{(4\pi)^{d/2}} \frac{\Gamma( b-a- \frac{d}{2}) \Gamma( a + frac{d}{2})}{\Gamma(\frac{d}{2} ) \Gamma( b)} \frac{1}{\Delta ^{b-a- \frac{d}{2}}}
 $$
 So altogether we get:
 $$
 \Pi_1^{\mu \nu} (g) = \frac{4 \mu^{\epsilon g^2}{(4 \pi)^{\frac{d}{2}}\Gamma(\frac{d}{2})} \int^1_0 dx \int^{\infty}_0 \frac{dl^2 ( l^2)^{\frac{d}{2}}}{ (l^2 + \Delta)^2}( l^2 ( 1- \frac{2}{d}) \delta^{\mu \nu} + ( 2q^{\mu \nu} - q^2 \delta^{\mu \nu}) x(1-x) + m^2 \delta^{\mu \nu}) 
 $$
 $$
 \Pi_1^{\mu \nu} (g) = \frac{8 g^2 \mu ^{\epsilon}}{(4\pi)^{\frac{d}{2}}} \Gamma( \frac{\epsilon}{2}) \int^1_0 dx \frac{1}{\Delta^{\frac{\epsilon}{2}}} ( - q^2 \delta^{\mu n\u} + q^{\mu} q^{\nu}) x(1-x) = (q^2 \delta^{\mu \nu} - q^{\mu} q^{\nu} ) \pi_1(q^2)
 $$
 where
 $$
  \pi_1(q^2) = - \frac{8 g^2 \Gamma(\frac{d}{2})}{(4\pi)^{\frac{d}{2}}} \int^1_0 dx x(1-x) ( \frac{\mu^2}}{\Delta})^{\epsilon/2}
 $$
 as $\epilson \rightarrow 0$;
 $$
 \pi_1(q^2) = - \frac{g^2}{(2 \pi^2} \int^1_0 dx x(1-x) ( \frac{2}{\epislon} - \gamma + \log ( \frac{4 \pi \mu^2}{\Delta})) + \O(\epsilon)
 $$
 Renormalise by writing $S_0 + S + S^{ct}$ with $e_0 = Z_e e, m_0 = Z_m m, \psi_0 = Z_2^{\frac{1}{2}} \psi, A_0 = Z_0^{\frac{1}{2}}A$
 $$
  S+S^{ct}= \int d^4 x ( \frac{1}{4} Z_3 F^2 + Z_2 \bar \psi \gamma \psi + Z_m Z_2 m \bar \psi \psi - ie Z_1 \bar \psi \cancel A \psi)
 $$
 where $Z_1 = Z_e Z_2 \sqrt{Z_3}$. Gauge invariance requires that $Z_1 = Z_2$ in order to form a gauge covariant derivative $D_{\mu}$. Let $Z_k =1+ \delta Z_k$ so 
 $$
  \delta Z_e = \delta Z_1 - \delta Z_2 - \frac{1}{2} \delta Z_3 = - \frac{1}{2} \delta Z_3 
 $$
 Counterterm to cancel divergence in $\pi_1(q^2)$
 $$
  \int d^4 x \frac{\delta Z_3}{4} F^2
 $$
 which gives $-( q^2 \delta^{\mu \nu} - q^{\mu} q^{\nu} ) \delta Z_3$. We want to chose $\delta Z_3 $ so it cancels the divergent portion of $\Pi_1^{\mu \nu}$. so In $\bar{MS}$ we have:
 $$
 \delta Z_3 = - \frac{g^2(\mu)}{2\pi^2} ( \frac{2}{\epsilon} - \gamma + \log 4\pi) 
 $$
 so
 $$
 \Pi_1^{ren} (q^2) = \frac{g^2(\mu)}{2 \pi^2} \int^1_0 dx x(1-x) \log ( \frac{\Delta}{m^2})
 $$
 with $\Delta = m^2 + x(1-x) q^2$. Let's look at what is happening with this log:
 $$
 \log \Delta = \log ( m^2 + x(1-x)q)
 $$
so we have a branch cut along $\Delta <0$. As for $x \in [0,1]$ we have $0 \leq x(1-x) \leq \frac{1}{4}$ with Minkowski signature $g_0 = iE$ and the cut corresponds to $x(1-x) ( E^2 - |\bm q|^2) \geq m^2$. So the smallest $E$ on the cut is $E=2m$ so this corresponds to the thershold for creating real electron positron pairs.\\\\
We can now connect this to the $\beta$ function. 
\subsubsection{QED $\beta$ function}
$$g_0 = Z_e g\mu^{\epsilon/2} =  Z_3^{- \frac{1}{2}} g \mu^{\epsilon/2}$$
so
$$
\mu \frac{d g_0}{ d\mu} = 0 = ( \mu \frac{\partial}{\partial \mu} + \beta(\beta) \frac{\partial }{\partial g} ) ( 1 + \frac{g^2}{24 \pi^2} ( \frac{2}{\epsilon} - \gamma + \log 4 \pi))
$$

$$
 \beta(g) = - ( \frac{\epsilon g}{2} + \frac{g^3}{24 \pi^2}) ( 1+ \frac{g^2}{(4 \pi^2 \epsilon})^{-1} = \frac{\epsilon g}{2} + \frac{g^3 }{12 \pi^2} + H.O.
$$
We get a positive $\beta$ function as if you consider the usual integration:
$$
 \frac{1}{g^2(\mu')} = \frac{1}{g^2( \mu)} + \frac{1}{6 \pi^2} \log \frac{\mu}{\mu'}
$$
Let $\Lamdba_{QED}$ be the scale where $g^2( \mu)$ diverges:
 $$
  g^2(\mu) = \frac{6 \pi^2}{\log \frac{\Lamdba_{QED}}{\mu}}
 $$
 Given $m_e = 0.5111 Mev$ we have $\alpha = \frac{g^2(m_e)}{4 \pi} = \frac{1}{137}$ implies $\Lambda_{QED} = 10^{286} GeV$. So despite the coupling running up and diverging it does so slow enough it isn't important until very high energy scales. So QED is irrelevant but it runs so slowly it does describe low energy dynamics.
 \section{Lecture 19}
 Let's see if we can obtain the $\beta$ function from the quantum effective action.\\
 Write the free propagator $D_{\mu \nu} (q)$ = ....
 $$
  G_{\mu \nu}^{(2)} = - + -1PI- + - 1PI - 1PI - + ... = D_{\mu \nu} + D_{\mu \rho} \Pi^{\pi \sigma} D_{\sigma \nu} + ...
 $$
 with $D_{\mu \nu} = \frac{P_{\mu \nu}}{q^2}$ and $\Pi^{\mu \nu} = q^2 P^{\mu \nu} \pi( q^2)$ so we get:
 $$
 G^{\mu \nu}^{(2)} = D_{\mu \nu} ( 1 +  \pi(q^2) + \pi^2(q^2) + ... ) = \frac{D_{\mu \nu}}{1 - \pi(q^2)}
 $$
 $G^{(2)}(q)$ is obtained from $\Gamma$ with:
 $$
 \Gamma [ \tilde{\Psi}, \tilde{\bar \Psi}, A] > \int \frac{d^dp }{ (2\pi)^d} \{ ( 1- \pi(p^2))(p^2 \delta^{\mu \nu} - p^{\mu} p^{\nu} ) \frac{1}{2} \tilde A_{\mu} (p) \tilde A_{\nu} (p) \}
 $$
 To get the $\beta$ function we need to make the coupling more explcity by rescaling the guage field ($A_{\mu} \rightarrow \frac{1}{e} A_{\mu}$) to move the coupling from the covariant derivative to the kinetic term to give:
 $$
 \Gamma [ \Psi, \bar \Psi, A] > \int d^d x \{ \frac{1- \pi(0)}{4 e^2} F_{\mu \nu} F^{\mu n\u} + "\partial^2 F^2 " \text{other terms etc.}\}
 $$
 We use the arguement that the coeffficent of terms in $\Gamma$ should be $\mu$ independant so we should have $\frac{1 - \pi(0)}{e^2} = \frac{1}{\mu^{\epsilon} g^2} ( 1 - \frac{g^2}{2\pi} \int dx x(1-x) \log \frac{\Delta}{\mu^2}$. We could say that this is our physical coupling $\frac{1}{e_{phys}}$, but it doesn't have to be, it just needs to be independant of $\mu$ so can take $\mu \frac{d}{d\mu}$ of both sides to get $\beta(g)$.\\\\
 \subsection{Full one-loop renormalisation of QED}
 First lets look at the fermion self-energy. So we need to look at the terms $\delta Z_2$ and $\delta Z_m$ from electron self-energy.
 $$
  F( \cancel p) = \int d^4 x e^{i p(x-y)} < \psi(x) \bar \psi(y)> = ->- + ->-1PI->- +... = \frac{1}{i \cancel p + m - \Sigma (\cancel p)}
 $$
 where $\Sigma( \cancel p)$ is the ->-1PI->- part. 
 $$
 \Sigma_1(\cancel p) = ->-^n->- = (ie)^2 \int \frac{d^d k}{(2 \pi)^d} \gamma^{\mu} \frac{1}{i (\cancel p + \cancel k) + m} \gamma^{\nu} \frac{\delta_{\mu \nu}}{k^2}
 $$
 $$
 \Sigma_1(\cancel p) = - \frac{g^2}{16 \pi^2} \int^1_0 dx ( (2- \epsilon) x (i \cancel p) + (4- \epsilon)m)( \frac{2}{\epsilon} - \gamma + \log \frac{4 \pi \mu^2}{\Delta})
 $$
 So we have to add conterterms $-( \delta Z_2(i \cancel p) + (\delta Z_2 + \delta Z_m)m$. We need some renormalisation condition e.g. $MS$ or $\bar{MS}$, or on shell scheme.The renormalised propagator with physical mass $m_{phys}$ and unit residue implies:
 $$
  \Simga(\cancel p) |_{\cancel p = i m_{phys}} = 0 \text{ fixes} \delta Z_m
 $$
 and
 $$
  \frac{ d \Sigam}{d \cancel p} |_{\cancel p = i m_{phys}} = 0 \text{fixes} \delta Z_2
 $$\\\\
 We asserted that $\delta Z_1 = \delta Z_2$ and one can check explictly that this is true at one-loop e.g.:
 $$
  V^{\mu} ( p,p') = ie \gamma^{\mu} + 
 $$
 We want to sandwich the vertex between spinors $\bar u(p) V^{\mu} ( p, p') u( p') = i e \bar u(p) ( F_1(q^2)\gamma^{\mu} + \frac{1}{4m} [ \gamma^{\mu}, \gamma^{\nu}] q_{\nu} F_2(q^2))$ and $F_1$ and $F_2$ are form factors. They can be calculated in perturbation theory. \\\\
         At Tree level it is apparent that $F_1(q^2) = 1$, $F_2(q^2) = 0$ but beyond it turns out that the loop integral you get at beyond tree level gives a finite $F_2(q^2)$ so needs no renormalisation. However, $F_1(q^2)$ is divergent and requires a conterterm that is $\sim \delta Z_1$. We can show that the on-shell renormalisation condition is to choose $\delta Z_1$ such that $F_1(0) = 1$.
         \subsection{Schwinger-Dysonn for fermions}
         Transform $$\psi(x) \rightarrow e^{i \alpha(x)} \psi(x)$$ $$\bar \psi \rightarrow e^{- i \alpha(x)} \bar \psi(x)$$\\
         Kinetic term $$\bar \psi \cancel \partial \psi \rightarrow \bar \psi \cancel \partial \psi + i \bar \psi \gamma^{\mu} \psi \partial_{\mu} \alpha$$. So if we have small alpha we have:
         $$
          \psi \rightarrow \psi + i \alpha \psi, \bar \psi \rightarrow - i \alpha \psi
         $$
         Consider $< \psi(x_1) \bar \psi(x_2)>$ before and after variation then the leading order is the correlation number by itself with $O(\alpha)$ terms sum to zero. 
         $$
         0 = \int \mathfrak{D} \psi \mathfrak{D} \bar \psi \{ e^{-S} ( i \int d^4 x \bar \psi(x) \gamma^{\mu} \psi(x) \partial_{\mu} \alpha(x)) \psi( x_1) \bar \psi(x_2) + \int \mathfrak{D}\psi \mathfrak{D} \bar \psi e^{-S} ( i \alpha (x_1) - i \alpha(x_2)) \psi(x_1) \bar \psi(x_2)
         $$
         So on the LHS we integrate by parts to move the derivative:
        $$
         \int d^4 x \alpha(x) \partial_{\mu} ( \int \mathfrak{D} \psi \mathfrak{D} \bar \psi e^{-S} \bar \psi \gamma^{\mu} \psi )(x) \psi(x_1) \bar \psi(x_2)) = - \int d^4 x \alpha(x) ( \delta( x-x_1) - \delta(x-x_2)) \int \mathfrak{D} \psi \mathfrak{D} \bar \psi e^{-S} \psi( x_1) \bar \psi( x_2)
        $$
        Holds for all $\alpha(x)$ so we get the Schwinger-Dyson equation:
        $$
         \partial_{\mu} < j^{\mu} (x) \psi(x_1) \psi(x_2) > = - ( \delta (x-x_1) \delta (x-x_2)) < \psi(x_1) \bar \psi( x_2)> 
        $$
        where $j^{\mu} = \bar \psi(x) \gamma^{\mu} \psi(x)$ which is the Noether current of transformation. So let's consider the fourier transform of this identity.
        $$
        m_3^{\mu} ( p, q_1, q_2) = \int d^4 x d^4 x_1 d^4 x_2 e^{i px} e^{-i q_1 x_1} e^{i q_2 x_2} < j^{\mu} (x) \psi( x_1) \bar \psi(x_2)>
        $$
        $$
m_2(q_1, q_2) = \int d^4 x_1 d^4 x_2 e^{- i q_1 x_1} e^{i q_1 x_1} < \psi(x_1) \bar \psi(x_2)> 
        $$
        here we don't stick to the convention of everything being directed outwards as when we have fermions they are directed so we know we must have one in and one out. \\\\
        The Schwinger-Dyson equation becomes the following in momentum space:
        $$
        i p_{\mu} m^{\mu}_3 (p, q_1, q_2) = - ( m_2 ( q_1 -p, q_2) - m_2( q_1, q_2 + p))
        $$
        \section{Lecture 20}
        $$
         i p_{\mu} m_3^{\mu} ( p ,q_1, q_2) = - ( m_2( q_1 - p, q_2) = m_2( q_1, q_2 + p))
        $$
        \subsection{Wandi-Takahashi identity}
        The variation $\psi \rightarrow ( 1 + i \alpha(x)) \psi(x)$ along with $A_{\mu} (x) \rightarrow A_{\mu} (x) + \frac{1}{2} \partial_{\mu} \alpha(x)$. We need to make a connection between these $m$'s that are just a fourier transform of the correlation functions to the 1-loop functions in QED. \\\\
        define:
        $$
         < \psi(x) \bar \psi(y) > = \int \frac{d^4 q}{(2\pi)^4} e^{i q ( x-y) } G( \cancel q)
        $$
        Looking back at the defintion of $m_2$ in the last lecture we can read off the fact that:
        $$
         m_2 (q_1, q_2) = (2\pi)^4 \delta^{(4)} ( q_1 - q_2) G( \cancel q_1)
        $$
        This will feed into the right hand side of our schwinger dyson equaiton. For the LHS we want to show at the 3 point function $m_3^{\nu}$ is the fourier transform of $< \psi(x_1) A_{\mu}(x) \bar \psi(x_2)> $ with the external propagators amputated. Recall:
        $$
         \mathfrak{L}_{int} =-ie\bar \psi \gamma^{\mu} A_{\mu} \psi = - ie j^{\mu} A_{\mu} 
        $$
        Earlier in this chapter we had a source term $j^{\mu} = ie j^{\mu} = ( \partial^{\mu} \partial^{\nu} - \partial^2 \delta^{\mu \nu} ) A_{\mu}$. This differential operator is the inverse of the propogator so in order to amputate the photon that comes out of the vertex up here, it is sufficent to replace the $A_{\mu}$ by the current $j_{\mu}$. So to tamputate external $\gamma$- propagator replace $A_{\mu} \rightarrow ie j^{\mu}$. So our vertex function:
        $$
        V_3^{\mu} ( p, q_1, q_2)  (2\pi)^4 \delta^{(4)} ( q_1 - q_2 - p) = ie \int d^4 x d^4 x_1 d^4 x_2 e^{i p x} e^{-iq_1 x_1} e^{i q_2 x_2} G^{-1}( \cancel q_1) < j^{\mu}(x) \psi(x_1) \bar \psi(x_2) > G^{-1} ( \cancel q_2)
        $$
        this is schematic the typed notes do it in more details, but the punch line is:
        $$
         V_3^{\mu} ( p, q_1, q_2) (2 \pi)^4 \delta^{(4)} (q_1 - q_2 - p) = G^{-1} ( \cancel q_1) m_3^{\m} ( p, q_1, q_2) G^{-1}(\cancel q_2)
        $$
If we combine this result with the Schwinger-dyson equation we get:
$$
 i p_{\mu} m_3^{\mu}( p, q_1, q_2) = - (2\pi)^4 \delta^{(4)} ( q_1 - q_2 - p) ( G( \cancel q_1 - \cancel p) - G( \cancel q_1))
$$
So we get the Wandi-takasashi identity:
$$
i p_{\mu} V^{\mu} ( p, q_1, q_2) = ie ( G^{-1}( \cancel q_1) - G^{-1} ( \cancel q_1 - p) )
$$
Using the fact that the inverse propogator is related to the self-energy: $G^{-1}( \cancel q) = i \cancel q + m - \Sigma (\cancel q)$ so
$$
i p_{\mu} V^{\mu} ( p, q_1, q_2) =  e ( i \cancel p + \Sigma( \cancel q_1 - p) - \Sigma( \cancel q_1))
$$
Recalling $V_3^{\mu} = ie ( F_1( p^2) \gamma^{\mu} + \frac{1}{4m} F_2(p^2) [ \gamma^{\mu}, \gamma^{\nu}]p_{\nu})$ and chosing the renormalisation condition at $p^2 = 0$, and in the on-shell scheme:
$$
1 = F_1(0)
$$
Look at $p_{\mu} V^{\mu}_3 = ie F_1 (p^2) \cancel p + 0$ so $$F_1(0) = \lim{\cancel p \rightarrow 0} \lim_{\cancel q_1 \rightarrow m_{phys}} ( \frac{ \Sigma( \cancel q_1 - p) - \Sigma(\cancel q_1)}{ i \cancel p} + 1) = i\Sigma' ( i m_{phys}) + 1$$ so $\Sigma(i m_{phys}) = 0$\\\\
On-shell condition for $F_1(0)$ gives condition for $\Sigma' ( i m_{phys})$ which gives $\delta Z_1 = \delta Z_2$ and so $Z_1 = Z_2$.
\subsection{Nonabelian gauge theories}
\subsubsection{Lie Groups}
       Group element $U = \exp ( i \theta^a T^a) I$ with $T^a$ are hermitiant generators, and $\theta^a$ are continously numbers parametirsing U, and use $a$ to index over generators.\\\\
       These $T^a$s form an algebra ( we are simplifying the algebra down to suitable for gorups of SU(n)), so have in mind $SU(2)$ with $U \in SU(2)$ and $T^a = \frac{1}{2} \sigma^a$ with $a = 1,2,3$. So the Lie algebra of $T^a$ is defined by:
       $$
        [T^a, T^b] = i f^{abc} T^c
       $$
       and jacobi idnenity:
       $$
       [ A, [B,C] + [ B, [C,A] + [ C, [A, B] = 0 \implies f^{abd} f^{dc e} + pems = 0
       $$
       We can have different classes of Lie groups: unitary groups, orthogonal groups , sympletic groups and some exceptional groups.\\\\
       For unitary $U^{\dagger} U = I$ for $U \in G = SU(n)$. and $n$ will turn out to be the number of components in our matter field that we care about. $SU(n)$ has $\det U = 1$ and has $n^2 -1$ generators  ($dim d(G) = n^2 -1$)\\\\
       \textbf{Representations}:\\
       Fundemantal represnetaion: smallest non-trivial representation of the algebra. For $SU(n)$ it would be the $n \times n$ traceless hermitian matrices. Under $SU(n)$ scalar field would transform as:
       $$
       \phi \rightarrow e^{i \alpha^a T^a} \phi = \phi_i + i \alpha^a (T^a_{fund})_{ij} \phi_j
       $$
       $i,j$ are the representation indicies and $a,b$ are generator indicies.\\\\
       Anti-fundamental representation: $T_{afund}^a = - ( T_{fund}^a)^*$ so:
       $$
        \phi^*_i \rightarrow \phi_i^* + i \alpha^a(T^a_{afund})_{ij} \phi_j^* = \phi_i^* - i \alpha^a \phi_j^* ( T_{fund}^a)_{ji}
       $$
       then we can drop the fundamental subscript as we are keeping $T$ to mean $T_{fund}$.\\
       Adjoint: acts on a vecotr space spanned by generators:
       $$
        ( T_{adj}^a) _{ij} =- if^{aij}
       $$
        can see that Gauge fields transform in adjoint representation.
        \section{Lecture 22}
        Index of a representation $R$, $T(R)$ is defined as the inner product 
        $$T(R) \delta^{ab} = tr( T_r^a T_r^b) = (T_R^a)_{ij} (T_R^b)_{ji}$$
        For the fundamental:
        $$
         T_{ij}^a T_{ji}^b = \frac{1}{2} \delta^{ab}, T(fund) \neq T_F = \frac{1}{2}
        $$
        For the adjoint: 
        $$
         f^{acd} f^{bcd} = N \delta^{ab}, T(adj) = T_A = N
        $$
        Quadratic Asmir of $R$, $C_i(R)$ is given by:
        $$
         T_R^a T_R^a = C_2(R) I
        $$
        In defintion of index, set $a=b$ so:
        $$
         T(R) d(G) = C_2(R) d(R)
        $$
        so
        $$
         C_2(fund) = C_F = \frac{N^2-1}{2N}, C_2(adj) = C_A = N
        $$
        \subsection{Gauge invariance and Wilson lines}
In typed notes we do this in QED, here we are going to skip ahead.
Consider transformation of $N$-component fermions:
$$
 \psi(x) \rightarrow V(x) \psi(x), V(x) \in G (=SU(N))
$$
$$
 \bar \psi(x) \rightarrow \bar \psi(x) V^{\dagger} (x)
$$
$\bar \psi \cancel \partial \psi$ is not invariant due to $\partial_{\mu} V(x)$ term. OCnsider derivative in direction of unit vector $n^{\mu}$:
$$
 n^{\mu} \partial_{\mu} \psi = \lim_{a \rightarrow b} \frac{1}{a} ( \psi(x+ an) - \psi(x))
$$
We have:
$$
 \psi(x + an) - \psi(x) \rightarrow V(x+an) \psi(x+ an) - V(x) \psi(x)
$$
We want to define a gauge invariant derivative $D_{\mu}$ s.t.:
$$
 D_{\mu} \psi(x) \rightarrow V(x) D_{\mu} \psi(x)
$$
Solution is to introduce Wilson line:
$$
 W(y,x) \rightarrow V(y) W(y,x) V^{\dagger}(x)
$$
with $W(x,x) = I$ of $G$. If we define $D_{\mu}$ as:
$$
n^{\mu} D_{\mu} \psi = \lim_{a \rightarrow 0} \frac{1}{a} ( \psi( x+ an) - W( x+ an, x) \psi(x))
$$
In the infintesimal limit, relate group element to algebra element $A_{\mu} (x+ \frac{a}{2}n)$ where $A_{\mu} (x) = A_{\mu}^a (x) T^a$:
$$
 W( x+an, x) = e^{i g an^{\mu} A_{\mu} ( x+ \frac{a}{2} n)} \approx 1 + i g a n^{\mu} A_{\mu}^a ( x+ \frac{a}{2} n) T^a + O ( g^2)
$$
We want to work with the $A_{\mu}$ field so using how we know $W$ transforms and incisting $D_{\mu} \psi(x) \rightarrow V(x) D_{\mu} \psi(x)$:
$$
 ( \parital_{\mu} - i g A'_{\mu}) V \psi = V(x) ( \partial_{\mu} -ig A_{\mu}) \psi
$$
$$
 \partial_{\mu} V - i g A_{\mu}' V = - i g V A_{\mu}
$$
Solve for $A_{\mu}' = V A_{\mu} V^{-1} - \frac{i}{g} ( \partial_{\mu} V) V^{-1}$. \\Infinitesimal gauge transformations:
$$
V(x) = e^{i \alpha^a(x) T^a} \approx 1 + i \alpha^a (x) T^a
$$
so
$$
 \psi \rightarrow (1+ i \alpha^a T^a) \psi(x)
$$
$$
 A_{\mu}^a (x) \rightarrow A_{\mu}^a(x) + \frac{1}{g} \partial_{\mu} \alpha^a(x) + f^{abc} A_{\mu}^b(x) \alpha^c(x) = A_{\mu}^a(x) - \frac{1}{g} ( \partial_{\mu} \delta^{ac} - ig A_{\mu}^b(x) ( -if^{bac}) ) \alpha^c(x)
$$
$$
 A_{\mu}^a = A_{\mu} ^a(x) + \frac{1}{g} D_{\mu}^{ac} \alpha^c(x)
$$
where
$$
 D_{\mu}^{ac} = \partial_{\mu} \delta^{ac} - ig A_{\mu}^b( T^b_{adj})^{ac}
$$
$D_{\mu} \psi$ transforms line $\psi$ and $D_{\mu} D_{\nu} \psi$ also does so:
$$
 [ D_{\mu} , D_{\nu} ] \psi \rightarrow V [ D_{\mu}, D_{\nu}] \psi
$$
Insert $D_{\mu} = \partial_{\mu} - i g A_{\mu}$ so:
$$
 [D_{\mu}, D_{\nu}] \psi = - ig F_{\mu \nu}^a T^a \psi 
$$
       where $F_{\mu \nu}^a = \partial_{\mu} A_{\nu}^a - \partial_{\nu} A_{\mu}^a + g f^{abc} A_{\mu}^b A_{\nu}^c$ which is not a differential operator.  Under gauge transformation:
       $$
        F_{\mu \nu}^a \rightarrow F_{\mu \nu}^a - f^{abc} \alpha^b F^c_{\mu \nu}
       $$
       so $F^a_{\mu \nu} F^{a, \mu \nu}$ is gauge invariant so the Yang-Mills aciton is given by:
       $$
       S_{YM} [A] = \int d^4 x \frac{1}{4} F_{\mu \nu}^a F^{a \mu \nu}
       $$
       Gauge theory wiht DIrac fermions:
       $$
        \mathfrak{L} = \frac{1}{4} ( F^a)^2 + \bar \psi_i ( \cancel \partial \delta_{ij} - i g \cancel A^a T^a_{ij} + m \delta_{ij}) \psi_j = \frac{1}{2} Tr ( F_{\mu \nu} F^{\mu \nu}) + \bar \psi( \cancel D + m ) \psi
       $$
       \subsection{Fadeev-Popov Gauge-fixing}
       Analogy: go back to zero dimensions with integral:
       $$
       Z \sim \int da db e^{-S(a)}
       $$
       There are some modes here $b$ that do not affect the action so we just ignored them and didn't integrate over the modes. In non-abelian gauge theory we cant be so caviler so need to be more formal:
       $$
       Z = \int da db \delta(b) e^{- S(a)} = \int da db \delta (b - f(a)) e^{- S(a)}
       $$
       Fix $b$ s.t. it is solution to some equation $G(a,b) = 0$ this is our gauge fixing cndition and then use:
       $$ 
       \delta( G(a,b)) = | \frac{\partial G}{\parital b}|^{-1} \delta (b - f(a))
       $$
       so:
       $$
        Z = \int da db \frac{\parital G}{\partial b} \delta( G(a,b)) e^{- S(a)}
       $$
       Assumed that $\frac{\partial G}{\partial b} > 0$  and there exists a unique solution ( Actually "Eribov copies" can exits). This was warm up exercise next time we do this with parth integrals.
       \section{Example sheet 3}
       Remember when transforming to momentum space remember that $\partial_{\mu} \phi \rightarrow i p_{\mu} \phi$. I keep forgetting the $i$.\\\\
       When in euclidean space we have $\int D \phi e^{-S}$ so there is an extra minus sign added to each interaction term that you read off the langrangian.\\\\
       If a diagram has no momentum depedeance than it is sort of like a mass correction term,however if this is in a theory with gauge invariant mass then this is probably cancelling some masslike terms generated in another diagram to make the sum gauge invariant.\\\\
       The reason $\delta Z_1 = \delta Z_2$ when rescaling a scalar QED, is that the renormalised lagrangian after scaling must still be gauge invariant so it must be constructed of gauge invariant things which we only have a limited number including teh covariant derivative and mass terms. So it is useful to figure out how the gauge symmetry restricts the counterterms so you know which terms must renormalise together. Any symmetry of the original lagrangian needs to be preserved under renormalisation (e.g. parity). \\\\\\
       We want to make $g$ small as this is pertrubation method so we want the second order stuff to dissapear. So for $\beta >0$ we have $g$ increases with $\mu$ increasing, so for small energy the theory becomes free. However, this does not always give us the whole picture as sometimes there is a $\mu$ dependance in the self energy terms (the first order and second order etc. terms) can explode dispite $g$ running down. This only occurs in MS or $\bar{MS}$ schemes. in the on shell scheme the corrections vanish so we get a physical renormalised parameter so the coupling won't run as the renormalised lagrangian has no dependance on $\mu$.\\\\
       When asked for 1-loop corrections remember to think about $<\phi>,<\phi^3>, <\phi^4>$ as well as $<\phi \phi>$ terms e.g. in $\phi^3$ theory can have 1-loop 1 particle correction. It is only the $<\phi \phi>$ terms that form the infinite series of $-1PI-1PI-1PI-...$, and the rest have the full $<\phi \phi>$ terms on each incoming propogator. However, not all of these will diverge even if they exist (as we only need counterterms for ones that diverge) .so we need to think about a circle which is a series of sections of scalar and fermion lines and a whole load  of random external lines added which gives in general dimension of $4L$ from the integral measure, $-2$ for every internal scalar propogator ($\frac{1}{p^2}$) and $-1$ for every internal fermion propogator ($\frac{1}{\cancel p}$). So this divergences if $4 -2I_S - I_F \geq 0$. There are only a finite number of potential number of propogators that satisfy this condiiton namely (I_S, I_F): (0,4), (1,2), (0,2), (2,0), (1,0), (1,1) which correspond to diagrams.\\\\
       The defintion of the on-shell scheme is we want to make the correction vanish for a physical particle so $\Pi( p^2 = - m^2)= \frac{\partial \Pi( p^2 = -m^2)}{\partial p^2} = 0$, so when it appears on an external leg it will have a zero contribution (\textbf{don't forget the $\Pi'(p^2 =-m^2) = 0$ condition}. Sometimes this is not possible as setting $p^2 = -m^2$ introduces an imagineary part that cannot be cancels by a counterterm as the counterterm should be real. In this case we use the general on-shell condition $Re \Pi(p^2 = -m^2) = 0$. If you have an onshell condition that does not restrict anything, think about whether th symmetries of the original terms prevents a finite term renormalisation anyway.
       \section{Lecture 22}
       In gauge theory, $b$ variables represent redundant degrees of freedom. We will fix the gauge with some function such as $G^a( x) = \partial^{\mu} A_{\mu} (x)$ so:
       $$
       Z= \int DA \det ( \frac{\delta G}{\delta }) ( \Prod_{x,z} \delta^{(4)}(G^a)) e^{ - S_{YM} [A]}
       $$
       with $S_{YM} = \frac{1}{4} F_{\mu \nu} \cdot F^{\mu \nu}$ with the dot meaning sum over the generator indices.\\\\
       How does $G$ change under change of gauge parameter. Given $A_{\mu}^a \rightarrow A_{\mu}^a + \frac{1}{g} D_{\mu}^{ab} \alpha^b$ we get $G^{(a)} \rightarrow G^{(a)} + \frac{1}{g} \partial^{\mu} D_{\mu}^{ab} \alpha^b$. So we ahve:
       $$
        \frac{\delta G^a (x)}{\delta \alpha^b (y)} = \frac{1}{g} \delta^{(4)}(x-y) \partial^{\mu} D_{\mu}^{ab} 
       $$
       this is a differential operator and it is important to note that the covariant derivate contains $A_{\mu}$ the gauge field.
       \subsubsection{Fadeev-Popov determinant}
       $$
        det \frac{\delta G^a (x)}{\delta \alpha^b(y)} \sim \int D c D \bar c e^{- S_{gh}}
       $$
       where $c, \bar c$ are spinless, Grassmann field with action $S_{gh} = \int d^4 x \mathfrak{L}_{gh}$ for:
       $$
        \mathfrak{L}_{gh} = - \bar c^a \partial^{\mu} D_{\mu}^{ab} c^b = \partial^{\m} \bar c^a D_{\mu}^{ab} c^b = \partial^{\mu} \bar c \cdot D_{\mu} c
       $$
       $$
        \mathfrak{L}_{gh} = \partial^{\mu} \bar c^a \partial_{\mu} c^a - i g \partial^{\mu} \bar c^a A_{\mu}^c ( T_{adj}^c)^{ab} c^b = \partial^{\mu} \bar c \partial_{\mu} c^a - g f^{abc} A_{\mu}^c \partial^{\mu} \bar c^a c^b
       $$
       The first term here is the kinetic temr and the second is the interaction term with ghost, anti-ghost and gauge boson.\\
       Put $\delta-f^n$ into the action via a Lagrange multiplier. Introduce a scalar (commuting) field $B^a(x)$ (auxillary, Nakansihi-Lautrap).
       $$
       \mathfrak{L} = \frac{1}{4} (F^a)^2 + \bar \psi( \cancel D + m ) \psi - \bar c \partial^{\mu} D_{\mu} x + B^a \partial^{\mu} A_{\mu}^a - frac{\xi}{2} ( B^a)^2
       $$
       The last two terms taken together are the gauge-fixing lagrangian. An analogy is:
       $$
        \int dx \delta(f(x)) e^{-S(x)} = \int dx d \lambda e^{-S(
       $$
       \end{document}


:
