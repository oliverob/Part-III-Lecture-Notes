\documentclass{article}
\usepackage[utf8]{inputenc}
\usepackage{bm}
\usepackage{amssymb}
\usepackage{amsmath}
\usepackage{braket}
\usepackage{cancel}
\title{Quantum Information Theory}
\author{oliverobrien111 }
\date{July 2021}

\begin{document}

\maketitle

\section{Lecture 1}
\subsection{Path Integrals in Quantum Mechanics}
Goal is to reformulate Schrodingers equation as a path integral.
$$
\hat H(\hat x, \hat p) \text{ with } [\hat x , \hat p] = i \hbar
$$
Assuming $\hat H = \frac{\hat p^2}{2m} + V(\hat x)$. The schrodinger picture is:
$$
i \hbar \frac{d}{dt} \ket{\psi(t)} = \hat H \ket{\psi(t)}
$$
so
$$
\ket{\psi(t)} = e^{-i \hat H t/\hbar} \ket{\psi(t)}
$$
Wavefunction: $\Psi(x,t) = \bra{x} \ket{\psi(t)}$. We want to solve schordingers equation for this wavefucniton in a way that introduces the path integral:
$$
\Psi(x,t) = \bra{x} \ket{\psi(t)} = \bra{x} e^{- \hat H t/\hbar} \ket{\psi(0)}
$$
$$
\Psi(x,t) = \int_{-\infty}^{\infty} K(x, x_0;t) \Psi(x_0, 0)
$$
where
$$
K(x, x_0;t) = \ket{x} e^{-i \hat Ht/ \hbar} \ket{x_0}
$$
is what we want a path integral expression for\\\\
Lets consider $n$ intermediate times/positions. Let $0 =t_0 <t_1 < t_2 <\ldots < t_n < t_{n+1} = T$:
$$
e^{-i \hat H T/\hbar} = e^{-i \hat H(t_{n+1}-t_n)/\hbar}\cdotse^{-i \hat H(t_{1}-t_0)/\hbar}
$$
Insert identity: $I = \int dx_r \ket{x_r}\bra{x_r}$:
$$
K(x, x_0;t) = \int_{-\infty}^{\infty} \left( \prod_{r=1}^n dx_r \bra{x_{r+1}}e^{-i\hat H(t_{r+1}-t_r)/\hbar} \ket{x_r}\right) \bra{x_1} e^{-i\hat H(t_1 - t_0)/\hbar} \ket{x_0}
$$
Consider fixed $V(\hat x) =0$. $K_0(x, x'; t) = \bra{x} e^{-i \frac{ \hat p^2}{2m \hbar} t} \ket{x'}$. Insert the identity: $I = \int \frac{dp}{2\pi \hbar} \ket{p}\bra{p}$.
$$
K_0(x,x';t) = e^{\frac{im(x-x')^2}{2 \hbar t}} \sqrt{\frac{m}{2\pi i \hbar t}}
$$
For $V(\hat x) \neq 0$, we need very small time steps. Separate kinetic and potential parts (Suzuki-Trotter decomposition). Take $t_{r+1} - t_r = \delta t$ to be small and $n$ large so $n\delta t = T$ (constant).
$$
e^{-i \hat H \delta t / \hbar} = \exp( - \frac{i \hat p^2 \delta t}{2\pi \hbar}) \exp( - \frac{iV(\hat x) \delta t}{\hbar}) ( 1+ O((\delta t)^2))
$$
The last term here is vanishingly small under the Broker-Campbell-Henshorff thingy. Between any 2 position eigenstates:
$$
\bra{x_{r+1}} e^{-i \hat H \delta t/\hbar} \ket{x_r} = e^{-i V(x_r) \delta t/\hbar} K_0(x_{r+1}, x_r; \delta t)
$$
Putting all these pieces together:
\begin{equation}
        K(x,x_0;T) = \int [ \prod_r dx_r] (\frac{m}{2\pi i \hbar \delta t})^{\frac{n+1}{2}} \exp( i \sum_{r=0}^n [ \frac{m}{2\hbar}( \frac{x_{r+1} - x_r}{\delta t})^2  - \frac{1}{\hbar} V(x)] \delta t)
\end{equation}
In the limit $n\rightarrow a, \delta t \rightarrow 0$ the exponent becomes $\frac{1}{\hbar} \int_0^T dt [ \frac{1}{2} m \dot x^2 - V(x)] = \int_0^T dt \mathfrak{L}(x,\dot x)$. So this is classical action in the limit.\\\\
We have no found a path integral (functional integral)
$$
K(x, x_0; T) = \bra{x} e^{- i \hat H T /\hbar} \ket{x} = \int \mathcal{D}x e^{i S/\hbar}
$$
where $$\mathcal{D}x = \lim_{\delta t \rightarrow 0, n \delta t = T} \sqrt{\frac{m}{2 \pi i \hbar \delta t}} \prod_{r=1}^n ( \sqrt{\frac{m}{2 \pi i \hbar \delta t}} dx_r)$$
One way of considering the classic limit is taking $\hbar \rightarrow 0$. For $e^{iS/\hbar}$ thisincreases the phases/frequencies. The Riemann-Lebesque lemma implies tha thte smallest frequncy (i.e. the path whihc minimises $S$ dominates the integral). As smallest $S$ is the hamiltonians principle of least action so this is equivalent to the classical treatment. \\\\
Another way is for $\hbar \neq 0$ the QM amplitude is the sum of all paths each weighted by phase $e^{iS/\hbar}$. This gives the interfence patterns we see in double sli, which is just represented by some classical line with no further diagrams.etc.\\\\
One trick we are going to play is dealing in imaginary time. You can analytically continue to imaginary time. Let $\tau = it$, then $\bra{x} e^{- \hat H t/\hbar} \ket{x_0} = \int \mathcal{D}x e^{-S/\hbar}$. Here the $\hbar \rightarrow 0$ argument is much more clear as the smallest value of $S$ will dominate in the limit. Analogy with statistical mechanics where $e^{-S /\hbar}$ is a botlzmann factor, and $\int \mathcal{D}x $ is the sum over microstates. These (with real exponentials) converge. Not all quantum questions can be answer in imaginary time. e.g. if there is a causality relationship between the initial and final space as we have convereted from mincovski to euclian.
\section{Lecture 2}
We showed that in quantum mechanics a path integral over positions weighted by the classical action:
$$
\int \mathcal{D}x e^{i S[x]/ \pi}
$$
this came from non-relativisitic quantum mechanics whre the position is an operator. As we saw in quantum field theory this mixed treatment of space as an operator and time as a label is not appropriate for satisfying lorentz invariance, so we demote x to be a label so that space and time are treated the same. So we work with the appropriate fields. For much of this course we will work with scalar fields and then will generalise to fermionic fields and gauge fields. QM is 0+1 dimensional field theory.
\subsection{Integrals and their diagram attic expansions}
Goal of next couple lectures is to show mathematics and show that they generate the same diagrams as in QFT (have to take it on a little bit of faith will become clear towards the end of the course). We supress the interesting relationships between space and tiem for the this chapter and just treat them as labels.\\\\
0-dimentsional field: $p: \{ \text{point} \} \rightarrow \mathbb{R}$ $q$ real variable\\\\
Path integrals as if in imaginary time (which makes the integrals better behaved as we get expoentially decaying factors rather than complex integrands):
$$
Z = \int_{\mathbb{R}} d\phi e^{- S(\phi)/\hbar}
$$
For the purposes here just assume it is well-behaved. So assume it is an even polynomial so that as $\phi\rightarrow\pm \infty$ we have $S[\phi] \rightarrow \infty$. Also look at expectation values:
$$
< f(\phi) > = \frac{1}{Z} \int d\phi f(\phi) e^{-S(\phi)/\hbar}
$$
This are sometimes refeerred to as correlation functions. Assume that $f$ does not grow so much it overwhelms the expoential so this is well behaved. \\\\
Now write down action corresponding to the free field theory which we can write down exactly and then we will do one that needs pertubation theory expansion.
\subsubsection{Free theory}
say we have $N$ real scalar fields (variables). Let $a, b \in [1,N]$:
$$
S[\phi] = \frac{1}{2} m_{ab} \phi_a \phi_b = \frac{1}{2} \phi^T m \phi
$$
with $m$ symmetric and positive definite ( $det m > 0$). If $m$ is diagonal it would obviously be a mass term, but it could also couple nearest neighbours and so could contain a discrete approximation to a derivate (so could represent difference operators on a discrte lattice but this isn't important today).\\\\
We can diagonalise $m$ with orthogonal matrices $P$ as $m$ is symmetric and positive definite:
$$
m = P \Lambda P^T
$$
$\Lambda$ is diagonal with elements $\lambda_c$ with $c \in [1,N]$. Let $\chi = P^T Q$, then:
$$
Z_0 = \int d^N \phi \exp( - \frac{1}{2\pi} \phi^T m \phi) = \prod_{c} \sqrt{ \frac{2 \pi \hbar}{\lambda_c}} = \sqrt{ \frac{ (2\pi \hbar)^N}{det M}}
$$
We will need to play some tricks when we do fermionic fields as they are antisymmetric not symmetric so you end up with the detminant in the numerator rahter than the denominator. \\\\
To go from the partition function $Z_0$ to correlation function $f^{\bm n}$ we introduce an external source $J$ (with N compononents) and replace the action $S_0(\phi)$ with $S_0(\phi) - J^T \phi$.
$$
Z_0(J) = \int d^N \phi \exp( -\frac{1}{2 \hbar} \phi^T m \phi + \frac{1}{\hbar} J^T \phi)
$$
Let $\tilde \phi = \phi - m^{-1} J$:
$$
Z_0(J) = \int d^N \phi \exp( - \frac{1}{2 \hbar} \tilde \phi^T m \tilde \phi) \exp( \frac{1}{2\hbar} J^T m^{-1} J)
$$
This is called the generating function or generating functional as it generates the correlation functions.\\\\
\textbf{Example: Generate '2-point' function}:\\
$$
< \phi_a \phi_b> = \frac{1}{Z_0(0)} \int d^N \phi \phi_a \phi_b \exp( - \frac{1}{2\hbar} \phi^T m \phi + \frac{1}{\hbar} J^T \phi)|_{J=0} = \frac{1}{Z_0(0)} \int d^N \phi ( \hbar \frac{\partial}{\partial J_a}) (\hbar \frac{\partial}{\partial J_b}) \exp ( - \frac{1}{2\hbar} \phi^T m \phi + \frac{1}{\hbar} J^T \phi)|_{J=0}
$$
Now the $\phi$ depedance is only in the exponential so can bring out derivatives:
$$
< \phi_a \phi_b> = \hbar^2\frac{\partial}{\partial J_a} \frac{\partial}{\partial J_b} Z_0(J)|_{J=0} = \hbar^2\frac{\partial}{\partial J_a} \frac{\partial}{\partial J_b} Z_0(0) \exp( \frac{1}{2\hbar} J^T m^{-1} J)|_{J=0} = \hbar (m^{-1})_{ab} 
$$
We represent this as a line between two points $a$ and $b$ also called a propagator.\\
\textbf{Example: '4-point' function}\\
$$
< \phi_b \phi_c \phi_d \phi_f > = \hbar^2[ (m^{-1})_{bc} (m^{-1})_{bc} 
 + (m^{-1})_{bd} (m^{-1})_{cf} + (m^{-1})_{bf} (m^{-1})_{cd} ]
$$
These represent the three ways of linking up 4 points with 2 lines. For $2k$ field in <> then there should be $\frac{(2k)!}{2^k k!} = \frac{ \text{permutate all 2k points}}{\text{ (permutte all points inside pairs) (permute pairs)}}$ diagrams.
\subsubsection{Interacting theory}
We just want to go beyond the action we have written down to something a bit more complicated. In cases where exact intergration is not possible. So we seek an expansion about a classical point, with small $\hbar$. Integrals don't end up being convergent they are asymptotic. e.g.
$$
\int d^N \phi f(\phi) e^{-S/\hbar}
$$
wont have a Taylor expansion about $\hbar = 0$. All is not lost we can still make progress. The expansion we are going to look at in many cases is asymptotic which means that the various terms in the series get to be better and better approximations to the full results at least up to a point.
$$
I(\hbar) \sim \sum_{n=0}^{\infty} c_n \hbar^n
$$
iff
$$
\lim_{\hbar \rightarrow 0^+} \frac{1}{\hbar^N} | I(\hbar) - \sum_{n=0}^N c_n \hbar^n | = 0
$$
Series missed out some terms $e^{-\frac{1}{\hbar^2}}$ so there are non-perturbative effects. We won't cover it in this course but these terms do contribute effects in some gauge theories. In some weakly coupled thoeries like QED this is a very good expansions, as shown by how we can very accurate measure magnetic moment of an electron to $10^{-10}$ accuracy.
\section{Lecture 3}
Last time we looked at
$$S(\phi) = \frac{1}{2} m^2 \phi^2 + \frac{\lambda}{4!} \phi^4 = S_0(\phi) + \frac{\lambda}{4!} \phi^4, m^2 > 0, \lambda >0$$
Partition $f^1$ ( generating function with respect to $J=0$):
$$
Z = \int d\phi e^{- S/ \hbar}
$$
Seperate this action into the free part and the interaction part and then expand around classical fields
$$
Z = \int d \phi e^{- S_0 / \hbar} \sum_{v=0}^{\infty} \frac{1}{v!} [ (- \frac{\lambda}{4!\hbar}) \phi^4 ]^v
$$
Truncate, swap order of sum and integral to give an asymptotic expansion.\\\\
In 0-dimensions we can integrate exactly. Let $x = \frac{1}{2\hbar} m^2 \phi^2$:
$$
Z = \frac{\sqrt{2\hbar}}{m} \sum_{v=0}^N \frac{\hbar}{V!} (- \frac{\hbar \lambda}{4! m^4})^v 2^v \int_0^{\infty} e^{-x} x^{2v + \frac{1}{2} -1} = \frac{\sqrt{2\hbar}}{m} \sum_{v=0}^N \frac{\hbar}{V!} (- \frac{\hbar \lambda}{4! m^4})^v 2^v \Gamma(2v + \frac{1}{2})
$$
$$
Z =  \frac{\sqrt{2\hbar}}{m} \sum_{v=0}^N \frac{\hbar}{V!} (- \frac{\hbar \lambda}{ m^4})^v \frac{1}{(4!)^v v!} \frac{(4v)!}{2^{2v}(2v)!}$$
Use stirlings approximation:
$$
V! \sim e^{V \log V}
$$
So this factorial growth is very fast and so will make this an asympotic expansion as it means the terms will not converge. We can see that $\frac{1}{(4!)^v v!} $ comes from expanding $\exp( - S_I / \hbar)$, whereas $\frac{(4v)!}{2^{2v}(2v)!}$ comes from the combinatoric ways of pairing the $4v$ fields.\\\\
\subsubsection{Generating function}
$$
Z(J) = \int d\phi \exp( - \frac{1}{\hbar} [ S_0(\phi) + S_I(\phi) - J \phi] )
$$
Taylor expand $e^{-S_I/\hbar}$ then replace the $\phi$ with $\hbar \frac{\partial}{\partial J}$ then pull it out and sum that infinite series
$$
Z(J) = \exp( - \frac{1}{\hbar} S_I(\hbar \frac{\partial}{\partial_J}) )\int d\phi \exp ( - \frac{1}{\hbar} [S_0(\phi) - J \phi])
$$
Drop multiplicative factor $\exp( - \frac{\lambda}{4! \hbar}( \hbar \frac{\partial }{\partial J})^4) \exp ( \frac{1}{2\hbar} J^T M^{-1} J)$. Here $J$ is just a signle variable $M = m^2$.
\begin{equation}
Z(J) = \sum_{v= 0}^N \frac{1}{V!} [ - \frac{\lambda}{4!\hbar} (\hbar \frac{\partial}{\partial J})^4 ]^v \sum_{p=0} \frac{1}{p!} ( \frac{1}{2 \hbar} J^T M^{-1} J)^p
\end{equation}
We represent this double series by diagrams of propogators and vertices. A propogrator is a line connecting to field. The propogator is $M^{-1}$ which for the moment is just a boring $m^{-2}$. At each vertex we have: $- \frac{\lambda}{\hbar} ( \hbar \frac{\partial}{\partial J})^4$.\\\\
Check $Z(0)$. In order for a term to survive to be nonzero we have to match up the derivatives with the $J$s. When $J=0$, need number of derivates to be equal to the number of sources. Generally, lets call these external sources. e.g. in the case of this field $E = 4 v - 2p$ and we want $E=0$. First nontrivial terms are $(v,p) = (1,2)$ and $(2,4)$. So:
$$
Z(0) = 1 + \text{figure eight} + ...
$$
Product rule in differentiation turns into symmetry factors or combinatorial factors associated with each diagram. Think about the "pre-diagram" with the half edges (corresponding to derivatives) and floating propogators (with ends corresponding to sources) and then we label the sources so there are $4!$ ways of assigning 4 derivates (or half edges) to 4 sources (at $a, a', b, b'$). \\\\
The numerator $A$ is cancelled by denominator $F$. 
$$
F = (v!)(4!)^v (P!)2^p $$
$F$ accounts for all the permuations of all vertices $V!$, and each vertex's legs and all propogators $P!$ and both ends of the propogators $2!$. After dividing $A$ by $F$ we get the symmetry factor. This is important to remove double counting. As some of the ways of assigning derivates to sources are equivalent. Looking for the number of distinct ways of mapping from the half edges to themselves whilst preserving the graph but creating a distigusible set of half edge assignations. Number of ways of changing the labels but keeping the same graph.
\section{Lecture 4}
Now we want to look at diagrams with external terms e.g. $E=2$ terms in the idagrammatic expansion. In the generating function:
$$
Z(J) > | + |o + | \infty + |^o_o + |oo ...
$$
We get both the vaccuum bubbles and the connected diagrams, so can factor out the vacuum bubbles:
$$
Z(J) > (| + 8 + ...)( | + |o + ...)
$$
When we go to calculate expectation values:
$$
< \phi^2 > = \frac{\hbar^2}{Z(0)} )\frac{\partial }{\partial J)^2} Z(J) |_{J=0} = \frac{1}{Z(0)} (| + |o + | 8 + ...) = (| + |0 + |oo)
$$
The generalization is straightforward
$$
< \phi^4 > = || + X + |o o| + ...
$$
\subsection{Effective actions}
In this section we show that we only need to consider the connected vaccum bubbles as the action only deals in the sums of connected bubbles rather than $Z$ which is the sum of all vaccumm bubbles.\\\\
We will show that $W = - \frac{1}{\hbar} \log Z$ Wilson effective action is sum of connected vacuum diagrams. Any diagram $D$ as products of connected diagrams $ D= \frac{1}{S_D}\sum_I(c_I)^{n_I}$
$$
\frac{Z}{Z_0} = \sum_{\{ n_I\}} D = \prod_I \sum_{n_I} (c_I)^{n_I} = \exp ( \sum_I c_I ) = e^{- (W-W_0)/ \hbar}
$$
$$
W= W_0 - \hbar \sum_I C_I
$$
Introduce external sources
$$
- \frac{1}{\hbar} W(J) = \log Z(J)
$$
$$
- \frac{1}{\hbar} \frac{\partial ^2}{\partial J^"} W |_{J=0} = \frac{1}{Z(0)} \frac{\partial^2 Z}{\partial J^2} |_{J=0} - \frac{1}{(Z(0)^2} (\frac{\partial Z}{\partial J}_{J=0} )^2 = \frac{1}{\hbar} ( < \phi^2> -< \phi>^2)
$$
In our theory for even actions the second term is zero. Above you can see we are finding the two point functions and then subtracting off the disconnected one point functions.\\\\
\textbf{Why is W "effective"}:\\
Consider a theory with 2 scalars $\phi$ and $\chi$:
$$
S(\phi, \chi) = \frac{m^2}{2} \phi^2 + \frac{M^2}{2} \chi^2 + \frac{\lambda}{4} \phi^2 \chi^2
$$
Feynmann rules:\\
Propogrators of $\frac{\hbar}{m^2}$ and $\frac{\hbar}{M^2}$, vertex rules of $- \frac{\lambda}{\hbar}$.
Therefore, $- \frac{W}{\hbar} = $ sum of connected diagrams with two dashed lines and two solid lines in coming out of each vertex.
$$
< \phi^2> = | + |0 + (|) + ...
$$
Look in typed notes to see how this works it is too difficult to latex the diagrams. If we want to remove the $\chi$ field then. Define $W(\phi)$
$$
e^{- W(\phi) /\hbar} = \int d\chi e^{- S(\phi, \chi)/\hbar}
$$
Treat $\chi^2 \phi^2$ term as a source for $\chi^2$ ( $J = - \chi^2$) so the correlation funciton of the $\phi$ fields:
$$
< f( \phi)> = \frac{1}{Z} \int d\phi d\chi f(\phi) e^{- S(\phi, \chi)} = \frac{1}{Z} \int d\phi f(\phi) e^{- W(\phi) / \hbar}
$$
As this is simple theory we can use exact integration:
$$
\int d\chi e^{- S(\phi, \chi)/\hbar} = e^{- m^2 \phi^2 /2 \hbar} \sqrt{ \frac{ 2\pi \hbar}{ M^2 + \frac{\lambda \phi^2}{2}}}
$$
$$
W(\phi) = \frac{1}{2} m^2 \phi^2 + \frac{\hbar}{2} \log (1 + \frac{\lambda}{2 M^2} \phi^2) + \frac{\hbar}{2} \log \frac{M^2}{2\pi \hbar}
$$
$$
W(\phi) = ( \frac{m^2}{2} + \frac{\hbar \lambda}{4M^2} )\phi^2 - \frac{\hbar \lambda^2}{16 m^4} \phi^4 + \frac{\hbar \lambda^3}{48 M^6} \phi^6 + ...
$$
 $$
 W(\phi) = \frac{m^2_{eff}}{2} \phi^2 + \frac{\lambda_4}{4!} \phi^4 + .. + \frac{\lambda^6}{6!} \phi^6
 $$
 \section{Lecture 5}
 Now we do it perturbatively using diagrams. Treat $\frac{\lambda}{4} \phi^2 \chi^2$ as a source term:
 $$
 Z = \int d \phi e^{- m^2 \phi^2/2\hbar} \int d\chi \exp ( - \frac{1}{\hbar}( \frac{M^2}{2} \chi^2 - J \chi^2))
 $$
 $J = - \frac{\lambda}{4} \phi^2$
 This leads to the Feynman rules with the propogator of $\frac{\hbar}{M^2}$ and we have self interaction terms with $- \frac{\lambda \phi^2{2 \hbar}}$ from the $\chi^2$ source term. Now we can find the efffective action (given by the sum of the connected diagrams):
 $$
 W(\phi) =  - \hbar(....)
 $$
 (above is all the connected diagrams with more and more vertices each with two legs so effectively looks like the increasing roots of unity)
 $$
 W(\phi) = \frac{m^2}{\phi^2}{2} - \frac{1}{2} \frac{\hbar\lambda}{2 M^2} \phi^2 - \frac{1}{4} \frac{\hbar \lambda^2}{4 M^4} \phi^4 + \frac{1}{3!} \frac{\hbar \lambda^2}{8 M^6} \phi^6 + ..= \frac{m^2_{\eff}}{2} \phi^2 + \frac{\lambda_4}{4!} \phi^4 + ...
 $$
 This is the same result as before. Now we have a theory of how the $\phi$ interacts. Now lets do the calculation with the full theory where we keep the $\phi$ explictly. Using $W(\phi)$:
 $$
 <\phi^2> = \frac{1}{Z} \int d\phi \phi^2 e^{- W(\phi) / \hbar} = | + |o + ... = \frac{\hbar}{m^2_{eff} } - \frac{\lambda_4 \hbar^2}{2 m^6_{eff}} + ...
 $$
This agrees with the full caculation from earlier. We have shown that given a theory of two fields we can intergrate out one of them to get a theory in just one field, this changes the degrees of freedom which therefore changes the coefficent.\\\\
Lets continue on with effective actions, now we want to introduce the quantum effective action\subsubsection{Quantum effective action}
Define average field in the presence of an external source $ < \phi> = \Phi = - \frac{\partial W}{\partial J} = \frac{\hbar}{Z(J)} \frac{\partial}{\partial J} \int d\phi e^{-(S-J\phi)/\hbar}$.
Legendre transform is a transformation from treating the source $J$ as the independant variable to treating the mean field $\Phi$ as the independant variable
$$
\Gamma (\Phi) = W(J_{\Phi}) + \Phi J_{\Phi}
$$
$J_{\Phi}$ is the $J$ which gives the correct expression $\frac{\partial W}{\partial J}|_{J_{\Phi}} = - \Phi$. Lets find the derivative:
$$
\frac{\partial \Gamma}{\partial \Phi} = \frac{\partial W}{\partial \Phi} + J_{\Phi} + \Phi \frac{\partial J_{\Phi}}{\partial \Phi} = - \frac{\partial W}{\partial J}_{J_{\Phi}} \frac{\partial J}{\partial \Phi} + J_{\Phi} + \Phi \frac{\parital J_{\Phi}}{\partial \Phi} 
$$
$$
\frac{\partial \Gamma}{\partial \Phi} = J_{\Phi}
$$
\subsection{$\Gamma(\Phi)$ and Feynmann diagrams}
External lines have one free end, whereas internal lines have no free ends. A bridge is any internal line of a connected graph which if cut would make the graph disconnected. A connected graph is said to be one-particle irreducible (1PI) iff it has no bridges. We are interested in these single particle irreducible graphs. Statement that we want to prove is that $\Gamma(\Phi)$ sums the 1PI graphs of the theory. We might expand about $\Phi = \Phi_0 (J = 0)$. Let $\varphi = \Phi$:
$$
\Gamma(\Phi) = \Gamma^{(0)} + \frac{1}{2} \Gamma^{(2)} \varphi^2 + ... + \frac{1}{n!} \Gamma^{(n)} \varphi^n
$$
Treat $\Phi$ as we did $\phi$ earlier. The quantum path integral for $\Phi$:
$$
e^{-W_{\Gamma}(J)/g}\int d\Phi e^{- (\Gamma(\Phi) - J \Phi)/g}
$$
$g$ is fictious planck constant.
$$
W_{\Gamma}(J) = \text{ sum of connected diagrams } =  \sum_{l=0} g^l W_{\Gamma}^{(l)}(J)
$$
In $g \rightarrrow 0$ limit $W_{\Gamma}(J) = W_{\Gamma}(J) = \Gamma(\Phi) - J \Phi$ "classical" action of the path integral ($W(J)$) which is the effective action of the original theory. $W(J) = -\hbar \log \int d \phi e^{- S(\phi)/\hbar}$
$$
W(J) = - \hbar \log (\int d\phi e^{-(S(\phi) - J\phi)/\hbar)}
$$
The sum of connected diagrams with rules derived from action $S(\phi) - J\phi$ can be obtained as the sum of tree diagrams (no loops only bridges) using $W_{\Gamma}(J)$ rules derived from action $\Gamma(\Phi) - J\Phi$. Therefore the diagrams in the original theory that don't contain any bridges have to be absorbed into the coefficents of the tree diagrams of $W_{\Gamma}(J)$.\\\\
In a theory with several scalar fields: $\phi_a$ $a = 1,..., N$:
$$
<\phi_a \phi_b>_J^{conn} = < \phi_a \phi_b>_J - <\phi_a> <\phi_b> = - \hbar \frac{\partial^2 W}{\partial J_a \partial J_b} = - \hbar \frac{\partial}{\partial J_a}( \frac{\partial W}{\partial J_b}) = \hbar \frac{\partial}{\partial J_a} \phi_B = \hbar (\frac{\partial J_a}{\partial \Phi_b})^{-1} = \hbar (\frac{\partial^2 \Gamma}{\partial \Phi_a \partial \Phi_b})^{-1}
$$
The point is that the two point function which we know in the original thepry working with action $S(\phi)$ we have to sum the diagrams with two external legs so it starts of as just a single propogator plus the loop contributions. This tells us that if we know the quantum effective action then we can just read off the two point function, which is just represented by some classical line with no further diagrams.\\\\
What is basically happening here is we can split every diagram into irreducible parts and each of them can be expressed as a single vertex with the correct them of external lines and will represent all the possible way it can be done like with loads of loops and stuff etc. so basically we can hugely simplify the number of diagrams. We have to figure out what the vertex is.
\section{Lecture 6}
Given $$\Gamma(\Phi) = \Gamma^{(0)} + \frac{1}{2} \Gamma^{(2)}( \Phi - \Phi_0)^2 + ... + \frac{1}{n!} \Gamma^{(n)}(\Phi- \Phi_0)^n +...$$ we have
$$
< \phi_a \phi_b>_J^{conn} = \hbar( \frac{\partial^2 \Gamma}{\partial \Phi_a \partial \Phi_b})^{-1} = \hbar ( \Gamma^{(0)})^{-1}
$$
in $\phi^4$ theory:
$$
< \phi_a \phi_b>_J^{conn} = -- \,+\, -o- \,+\, -o-o- \,+\, -8- \,+ \,- \theta - \,+ .. =  \,--\, +\, -IPI- \,+ \,-IPI-IPI- \,+ \,-IPI-IPI-IPI- =  \frac{1}{1- IPI}
$$
intuitively the last step comes from recognising the geometric series. On the example sheet we will do the same for three point function:
$$
< \phi_a \phi_b \phi_c >_J^{conn} = - \frac{1}{\hbar} ( \frac{\partial^3 \Gamma}{\partial \Phi_d \parital \Phi_e \partial_f})^{-1} <\Phi_a \Phi_d> <\Phi_b \Phi_e> <\Phi_c \Phi_f> 
$$
$$
( \frac{\partial^3 \Gamma}{\partial \Phi_d \parital \Phi_e \partial_f})^{-1}  = \Gamma^{(3)} = - <\Phi_d \Phi_e \Phi_F> ^{IPI}
$$
This is still trivial as you cant imagine having a non trivial bridge so we go to one higher level . If we go to a four point function we could have a bridge within the internal interactions e.g. it could be two 2 particle interactions rather than a single 4 particle interaction. Remember every external line has a two point interaction 1PI on it as there is a particle coming in a one coming out. This section is not in the written notes. It is slightly discussed later on page 41.
\subsection{Fermions}
Now have anticommutation relations rather than commutation relations. We introduce the abstract conscept of Grassmann numbers which are anti-commuting variables.\\
$n$ numbers $\{ \theta_a \} a = 1,..., n$ and they obey:
$$
\theta_a \theta_b = - \theta_b \theta_a \implies \theta_a^2 = 0
$$
for any scalar $\phi_b \in \mathbb{C}$
$$
\theta_a \phi_b = \phi_b \theta_a
$$
Functions can be expressed as finite sums:
$$
f(\theta) = f + \rho_a \theta_a + \frac{1}{2!} g_{ab} \theta_a \theta_b + ... + \frac{1}{n!}h_{a_1a_2...a_n} \theta_{a_1} \theta_{a_2} ... \theta_{a_n}
$$
where $g_{ab},..., h_{a_1...a_n}$ etc. are anti-symmetric in their indicies. This series is finite as adding any more $\theta$s would give a $\theta_a^2=0$ and vanish the term. Note
$$
(\theta_a \theta_b) (\theta_c \theta_d) = (\theta_c \theta_d) (\theta_a \theta_b)
$$
Differentiation is defined to anti commute:
$$
\frac{\partial}{\partial \theta_a} \theta_b + \theta_b \frac{\partial}{\partial \theta_a} = \delta_{ab}, \frac{\partial}{\partial \theta_a}( \theta_b F(\theta)) = \delta_{ab} F(\theta) - \theta_b \frac{\partial F}{\partial \theta_a}
$$
Integration: For a single grassamann $\theta$
$$
F(\theta) = f+ \rho \theta
$$
Define $\int d\theta$ and $\int \theta d\theta$. We want to require translational invariance and consider what it means to intergate over the whole range of $\theta$. So impose requirement that
$$
\int d \theta( \theta + \eta) = \int \theta d\theta 
$$
for constant grossman variable $\eta$. This implies that we want $\int d\theta = 0$ and we want $\int \theta d\theta = 1$ (this includes a choice of normalization). These are called the "Berezin rules". \\\\
Integration by parts is simplified:
$$
\int d\theta \frac{\partial}{\partial \theta} F(\theta) = 0
$$
So we have intoduced the rules for one variable now lets consider $n$ Grassmann variables $\theta_a$. In this case the only non-vanishing integral have to have one and only one $\theta$:
$$
\int d^n \theta \theta_1 \theta_2... \theta_n = 1 \iff \int d \theta_n d\theta_{n-1} ... d\theta_1 \theta_1 \theta_2... \theta_n = 1
$$
A key point is in general if we start commuting these indicies we are going to start picking up signs so
$$
\int d^n \theta \theta_{a_1} \theta_{a_2} ... \theta_{a_n} = \epsilon_{a_1 a_2... a_n}
$$
Now let use consider a change of variables $\theta' = X_{ab} \theta_b$ then:
$$
\theta_a' = X_{ab} \theta_b, X_{ab} \in \mathbb{C}
$$
$$
\int d^n \theta'_{a_1} ... \theta'_{a_n} = X_{a_1 b_1} ... X_{a_n b_n}  \int d^n \theta \theta_{b_1}... \theta_{b_n} = X_{a_1 b_1} ... X_{a_n b_n}   \epsilon^{b_1... b_n} = \det X \epsilon^{a_1...a_n} = \det X \int d^n \theta' \theta'_{a_1}... \theta'_{a_n}
$$
therefore $d^n \theta = det X d^n \theta'$ which compared to scalars where we have $\phi' = Y \phi \implies d^n \phi = \frac{1}{\det Y} d^n \phi'$. So fermions give the converse relationship here to what you would expect from scalars.
\subsection{Free fermion field theory}
$d = 0$, 2 fields $\theta_1, \theta_2$. Need a scalar action, only non-constant action
$$
S(\theta) = \frac{1}{2} A \theta_1 \theta_2, A \in \mathbb{R}
$$
$$
Z_0 = \in d^2 \theta e^{-S(\theta)/\hbar} = \int d^2 \theta ( 1- \frac{A}{2\hbar} \theta_1 \theta_2) = - \frac{A}{2 \hbar}
$$
$n = 2m$ fields
$$
S= \frac{1}{2} A_{ab} \theta_a \theta_b
$$
$ A_{ab}$ is antisymmetric matrix
$$
Z_{0} = \int d\theta^{2m} e^{-S /\hbar} = \int d^{2m} \theta \sum_{j=0}^m \frac{(-1)^j}{(2\hbar)^j j!} ( A_{ab} \theta_a \theta_b)^j
$$
$$
Z_0 = \frac{(-1)^m}{(2\hbar)^m m!} \epsilon^{a_1...a_n} A_{a_1a_2} ... A_{a_{2m-1} a_{2m}} = \frac{(-1)^m}{\hbar} Pf(A) = \pm \sqrt{\frac{\det A}{\hbar^n}}
$$
$Pf(A)$ is the Pfaffian.
\subsubsection{External sources}
Need to be grassman valued $\eta$:
$$
S(\theta, \eta) = \frac{1}{2} A_{ab} \theta_a \theta_b - \eta_a \theta_a
$$
Complete the square, using translation invariance to get the following integral that we can then do:
$$
Z_0(\eta) = \exp( - \frac{1}{2\hbar} \eta^{T} A^{-1} \eta) Z_0(0)
$$
Propogator
$$
< \theta_a \theta_b> = \frac{\hbar^2}{Z_0(0)} \frac{\partial^2 Z_0(\eta)}{\partial \eta_a \partial \eta_b}|_{\eta=0} = \hbar ( A^{-1})_{ab}
$$
\section{Lecture 7}
\subsection{LSZ reduction formula}
This discussion is quite general and not very connected to the earlier section.\\\\
We are going to work through 2-2 scatting of $\phi$ particles. From last term:
$$
\phi(x) = \int \frac{d^3k}{(2\pi)^3 2 E_b} ( a(\bm k) e^{-i k \cdot  x} + a^{\dagger}(\bm k) e^{i  k \cdot x})
$$
We are using the Minkowski metric $(+, -, - ,- )$ so $k \cdot x = E_0 t - \bm k \cdot \bm x$. There is also a convention for the normalisation. We are taking the realtivistic normalisation for $a(\bm k)$. It depends on whether you want the inner product of two functions to be the delta funcitno or the dealt function times $E$.\\\\
We invert to find $a(\bm k)$:
$$
\int d^3 x e^{ik \cdot x} \phi(x) = \frac{1}{2E} a(\bm k) + \frac{1}{2E} e^{2 i Et} a^{\dagger} ( - \bm k)
$$
$$
\int d^3 x e^{ik\cdot x} \partial_a \phi(x) = \frac{-i}{2} a(\bm k) + \frac{i}{2} e^{2 i Et} a^{\dagger} (- \bm k)
$$
These imply:
$$
a(\bm k) = \int d^3 x e^{i k\cdot x} ( i \partial _a \phi(x) + E\phi(x))
$$
$$
a^{\dagger}(\bm k) = \int d^3 x e^{-i k\cdot x} ( -i \partial _a \phi(x) + E\phi(x))
$$
In free theory: the 1 particle state  is given by
$$
\ket{k } = a^{\dagger} (\bm k) \ket{0} , \bra{0} \ket{0} =1, a(\bm k) \ket{0} = 0 \forall \bm kk
$$
Norm:
$$
\bra{k}\ket{k'} = (2\pi)^3 (2E) \delta^{(3)}(\bm k - \bm k')
$$
$$
E^2= \bm k^2 + m^2
$$
We are going to assume that the interacting theory is  close to the free theory.\\\\
Introduce a gaussian wave packet:
$$
a_1^{\dagger} = \int d^3 k f_1( \bm k ) a^{\dagger}(k)
$$
with $f_1(k) \sim \exp( - \frac{(\bm k - \bm k_1)^2}{4 \sigma^2})$ and similary
$$
a_2^{\dagger} = \int d^3 k f_2( \bm k ) a^{\dagger}(k)
$$
with $\bm k_2 \neq \bm k_1$. \\\\
Now we evolve the gaussians backward in time until a time where the particles had no overlap in space and can be consider free. THere is a complication as due to the interaction $a_1^{\dagger}(t)$ and $a_2^{\dagger}(t)$ depend on t. However,  the point is that as $t \rightarrow \pm \infty$ $a_1^{\dagger}$ and $a_2^{\dagger}$ coincide with free theory expressions.\\\\
Consdiring 2-2 scattering the initial state is $\ket{i} = \lim_{t\rightarrow -\infty} a_1^{\dagger}(t) a_2^{\dagger} (t) \ket{\Omega}$ and the final state is $\ket{f} = \lim_{t\rightarrow \infty} a_1^{\dagger}(t) a_2^{\dagger}(t) \ket{\Omega}$. We also have $\bra{i} \ket{i} =  \bra{f} \ket{f} = 1$ and $\bm k_1 \neq \bm k_2, \bm k'_1 \neq \bm k'_2$.\\\\
We want to calculate the scattering amplitude $\bra{f} \ket{i}$. First note that 
$$
a_1^{\dagger}( \infty) - a_1^{\dagger}(- \infty) = \int_{\infty}^{\infty} \partial_0 a_1^{\dagger} (t) = \int d^3 f_1(\bm k) \int d^4 x \partial_0 ( e^{-k \cdot x} ( - i \partial_0 \phi + c \phi)
$$
$$
a_1^{\dagger}( \infty) - a_1^{\dagger}(- \infty) = i \int d^3 k F_1(\bm k) \int d^4 x e^{-ik \cdot x} ( \partial _0^2 + E^2 ) \phi = - i \int d^3 k f_1( \bm k) \int d^4 x e^{-ik \cdot x}( \partial^2 + m^2 ) \phi
$$
In free theory we have $(\partial^2 + m^2)\phi = 0$ (klein gordon equation) so the creation operator doesn't change.
$$
\bra{f} \ket{i} = \bra{\Omega}T  a_{1'}(\infty) a_{2'} (\infty) a_1^{\dagger}(-\infty) a_2^{\dagger}(\infty) \ket{\Omega}
$$
We can relate $$a_j^{\dagger}(-\infty) = a_j^{\dagger}(\infty} + i \int d^3 k f_j \int d^4 x e^{-ikx} (\partial^2 + m^2) \phi
$$
$$a_{j'}^{\dagger}(\infty) = a_{j'}^{\dagger}(-\infty} + i \int d^3 k f_{j'} \int d^4 x e^{ikx} (\partial^2 + m^2) \phi
$$
Time ordering moves $a_j(- \infty)$ right and $a^{\dagger}_j(\infty)$ left. Therefore the t only nonzero term is the \textbf{LSZ reduction formula}:
$$
\bra{f} \ket{i} = (i)^4 \int d^4 x_1 d^4 x_2 d^4 x'_1 d^4 x'_2 e^{-i k_1 \cdot x_1} e^{-i k_2 \cdot x_2} $e^{i k'_3 \cdot x'_3} $e^{-i k'_4 \cdot x'_4} ( \partial_1^2 + m^2 )  ( \partial_2^2 + m^2 )  
( \partial_{1'}^2 + m^2 )  
( \partial_{2'}^2 + m^2 )  \bra{\Omega} T \phi(x_1) \phi(x_2) \phi(x'_1) \phi(x'_2) \ket{\Omega}$$
having taken $\sigma \rightarrow 0$ limit of $f_j(k) \rightarrow \delta^{(3)}(\bm k - \bm k_j)$.\\\\
Generalisatoin to $m\rightarrow n$ scattering is straightforward. You performa a fourier transform to get:
$$
\phi(y) = \int \frac{d^4 q}{(2\pi)^4} \tilde \phi(q) e^{-i q \cdot y}
$$
so}}
$$
\bra{f} \ket{i} = \bra{k'_1 ... k'_n} \ket{ k_1... k_n} = (i)^{m+n} \prod_{j=1}^m ( - k_j^2 + m^2) \prod_{j=1}^n (- k'_j^2 + m^2) \bra{\Omega} T \tilde \phi( k_1) ... \tilde \phi(k_m) \tilde \phi(k'_1)... \tilde \phi(k'_n) \ket{\Omega}
$$
momentum satisfy $k^2 = m^2$ propogators $\frac{k}{k^2 +m^2}$. The $(-k^2 +m^2)$ factors cancel the poles form the external propogators.\\\\
\textbf{LSZ in momentum space}:
$$
\bra{k'_1 .. k'_n}\bra{k_1...k_n} = \bra{\Omega} T \tilde \phi( k_1)...\tilde (k'_n) \ket{\Omega}_{amputated}
$$
Usually we are interested in all momentum being unequal, so all the particles are involved in the scattering. This implies we want connected diagrams.
\section{Example sheet 1}
When calculating symmetry factors you are genuinely looking for symmetries. Rather than trying to think about automorphisms first count the number of symmetries, then consider if any of them are already acounted for by exchange of particles and ignore them. If you get confused remember how it works with a single particle we don't say how many ways of picking which two to connect up is 6 but rather say we can slip those loops or exchange those loops (think a bit like polymod though in some instances this does not hold e.g. if the diagram is symmetric under exchange of two propogators which end on different particles this is a symmetry just like if they ended on the same particle). Try question 3 on ES1 it is a very good test of if you have got it cracked. It might be worth just memorising the symmetry factors of everything less than 3. 
 \end{document}
