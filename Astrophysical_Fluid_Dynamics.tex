\documentclass{article}
\usepackage{bm}
\usepackage{amsmath}
\begin{document}
In kinetic theory they would say there is a distribution function of velocities at every point and time, but in fluids we characterise every point with just one number describing the velocities.\\
If given $L$ (size of the system) and $\lambda$ (mean path of particles), then the fluid approach is valid when $\lambda << L$. Usual formula for $\lambda$ is $\lambda = \frac{1}{n\sigma}$ with $n$ number density and $\delta$ cross-section.\\
\textbf{Neutrals}:
$\delta = 10^{-15} cm^2$ way you figure it out is by roughly taking the cross-sectional area of a hydrogen atom.\\
\textbf{Ionized species}: $\delta = 10^{-4} \left( \frac{k}{T} \right) cm^2$. Interaction is so strong between positive and negative charges they are scattering through 90 degrees. So the higher the temperature the faster the particles move and the small the distance between the particles has to be before they get deflected as much.\\
Consider in the case of the bow and termination shocks of the sun. The gas that has been flowing towards the sum has $n \approx 1 cm^{-3}$ and $T \approx 10^4K$, and the bow shock occurs about about $100-150 AU$. For neutrals this gives $\lambda = 10^{15} cm = 70 AU$ which is quite close to the size of the shock, so therefore the neutrals are weakly collisional so the fluid approach is not going to be very good at describing their behaviour. For ionized particles, this gives $\sigma = 10^{-12} cm$, so $\lambda = 10^{12}cm = 0.1AU$. Therefore, for ionized species $\lambda << L$ so the fluid approach is good.\\
\subsection{Definitions}
Characterise fluid by $\bm u$ (sometimes written $\bm v$), $p$(pressure) and $\rho$ (density) at point $\hat X (\hat r)$ and time $t$\\
\textbf{Eulerian Derivative}: Characterises the derivative at a given point in space\\ $$\frac{\partial}{\partial t}|_{\hat X}$$
\textbf{Lagrangian time Derivative}: Taken while moving with the fluid. $Df = f(x_0 + u\delta t,t+\delta t) - f(x_0, t) =  (\frac{\partial f}{\partial t}  + (\bm u \cdot \nabla) f) \Delta t$\\
$$
\frac{Df}{Dt} = \frac{\partial f}{\partial t}  + (\bm u \cdot \nabla) f 
$$
\subsection{Evolution of a line Element}
$$
\delta x(t+\Delta t) = \delta x(t) + \Delta t (\delta x(t) \cdot \nabla) \bm u 
$$
$$
\frac{D \delta x}{Dt} = (\delta x \cdot \nabla) \bm u 
$$
\subsection{Continuity Equation}
Adopt Eulerian approach. Fixed $\hat x$, fixed volume element, and mass.
$
M = \int_V \rho dV
$ - varies with time with:
$$
\frac{\partial M}{\parital t} = \frac{\partial}{\partial t} \int_V \rho dV = \int_V \frac{\partial \rho}{\parital t} dV
$$
$$
\frac{\partial M}{\parital t} = - \int_{\partial V} \rho \bm u \cdot d\bm S = - \int_V \nabla \cdot (\rho \bm u) dV
$$
$$
\int_V \frac{\partial \rho}{\partial t} dV = - \int_V \nabla \cdot (\rho \bm u) dV
$$
$$
\int_V \left( \frac{\partial \rho}{\partial t} + \nabla \cdot (\rho u)\right) dV = 0
$$
\begin{equation}
        \frac{\partial \rho}{\partial t} + \nabla \cdot (\rho \bm u) = 0
\end{equation}
\subsection{Momentum Equation}
Stick to Lagrangian approach. Follow mass element M as it moves and apply newtons law to this element. $M \frac{d\bm u}{dt} = \frac{D\bm u}{D t} = F$. Possible forces: Pressure, gravity, body forces\\
\textbf{Pressure} - always acts on the surface so : $p = - \int_{\partial V} p d \bm S = -\int_V \nabla p dV$\\
\textbf{Gravity} - given by gravitational potential ($\phi$) $(-\nabla \phi}M = - \nabla \phi \int_V \rho dV = - \int_V \rho \nabla \phi dV$\\
\textbf{Bodily forces} - $f_e = \frac{F_e}{M}$\\
$$
M\frac{D\bm u}{Dt} = \int_V \frac{D \bm u}{dt} \rho dV =- \int_V \nabla p dV - \int_V \rho \nabla \phi dV + \int_V f_e \rho dV$
$$
\begin{equation}
        \rho \frac{D u}{Dt} = - \nabla p - \rho \nabla \phi + \rho f_e
\end{equation}
\begin{equation}
        \frac{\partial u}{\partial t} + (\bm u \cdot \nabla ) \bm u = - \frac{\nabla P}{\rho} - \nabla \phi + f_e
\end{equation}
If we know $\phi(x)$ (Cowling Approximation) and $P=P(\rho)$ (barotopic approximation) then these equations are sufficient to describe the system. 
\subsection{Examples of barotopic fluids}
\textbf{Isothermal fluid}: $P= c_s^2\rho$, $c_s$ is sound speed and $c_(x)=$ constant\\
\textbf{Adiabatic fluid}: has constant entropy everywhere, $S(x) =$ constant\\\\
As $S(P,\rho) = constant$ there must be a unique $P(\rho)$ relation.\\\\
For an ideal gas, $S = c_{\sigma} \ln \frac{P}{\gamma \rho}$ where $c_{\sigma}$ is the specific heat at a constant volume and $\gamma$ is the adiabatic index. $\gamma$ is given by 
$$
\gamma = \frac{c_p}{c_s} = \frac{T\frac{\partial S}{\partial T}_p}{T \frac{\partial S}{\partial T}_v} = \frac{\frac{\partial S}{\partial T}_P}{\frac{\partial S}{\partial T}_{\rho}}
$$
As S is constant $P = k \rho^\gamma$ for $k = e^{\frac{S}{c_v}$ (the adiabatic equation of state). The value of $\gamma$ depends on the type of gas and often depends on the polytropic index $n$ by $\gamma = 1+\frac{1}{n}$.\\\\
\textbf{Monoatomic gas}: $n= \frac{3}{2}$ and $\gamma = \frac{5}{3}$\\
\textbf{Diatomic gas}: $n = \frac{5}{2}$ and $\gamma = \frac{7}{5}$\\\\\\
What if $\phi(x)$ depends on the density distribution. You need to use the Poisson equation:
\begin{equation}
        \nabla^2 \phi = 4\pi G \rho
\end{equation}
$$
\phi(x) = -G \int_V \frac{\rho(\bm x',t)}{|\bm x - \bm x'|} d\bm x' - G \int_{V_{ex}}\frac{\rho(\bm x', t)}{|\bm x - \bm x'|} d\bm x'
$$
\subsection{Drop barotropic assumption}
Still assume ideal fluid so no dissapative effects - no heat transport/radiation transport/conductivity. Then entropy will still be conserved for each fluid element
\\\\
Entropy is a material property so it belongs to a particular particle or element, so entropy being conserved means that its material derivative is 0: $\frac{DS}{Dt} = 0$. No longer have $S(P,\rho) = constant$ but still have one to one relationshi$p$ and $\rho$.
$$
\frac{Dp}{Dt} = \frac{\partial p}{\partial \rho}_S \frac{D\rho}{Dt} = \frac{P}{\rho}\frac{\frac{1}{p}\partial p}{\frac{1}{\rho}\partial \rho} \frac{D\rho}{Dt} = \frac{p}{\rho} (\frac{\partial \ln p}{\partial \ln \rho})_s \frac{D\rho}{Dt} = \frac{p}{\rho} \Gamma_1 \frac{D\rho}{Dt}
$$
$$
\Gamma_1 = (\frac{\partial \ln p}{\partial \ln \rho})_s 
$$
where $\Gamma_1$ is the first adiabatic exponent. For an ideal gas $\gamma = (\frac{\partial \ln p}{\partial \ln \rho})_s \implies$
$$
\frac{Dp}{Dt} = \gamma \frac{p}{\rho} \frac{D\rho}{Dt}
$$
From continuity: $\frac{\partial \rho}{\partial t} + (\bm u \cdot \nabla) \rho + \rho \nabla \cdot \bm u = 0$. So:
\begin{equation}
        \frac{D\rho}{Dt} = - \rho \nabla \cdot \bm u
\end{equation}
So
$$
\frac{Dp}{Dt} = - \gamma p \nabla \cdot \bm u
$$
Can rewrite as energy equation:
\begin{equation}
        \frac{\partial p}{\partial t} + (\bm u \cdot \nabla) p  + \gamma p \nabla \cdot \bm u = 0
\end{equation}
An ideal fluid is described by equations (1), (3), (4) and (6).
\subsection{Departures from $p=\frac{n}{\mu}kT$}
We used the above assumption to get the equation for $\gamma$ for an ideal gas of ($\gamma = (\frac{\partial \ln p}{\partial \ln \rho})_s $). In general $\Gamma_1 = (\frac{\partial \ln p}{\partial \ln \rho})_T \gamma$, and for an ideal gas $(\frac{\partial \ln p}{\partial \ln \rho})_T =1$.\\\\
Whenever T is very high, $p = \frac{\rho}{\mu} k_B T + a T^4$ where $a = \frac{4}{3} \frac{\sigma_{sb}}{c}$.\\\\
Radiation pressure is important in: Early Universe, Centres of stars, Inner parts of acretion disks around neutron stars and black holes\\\\
\section{MHD}
Maxwell's equations
\begin{equation}
        \frac{1}{c} \frac{\partial \bm B}{\partial t} = - \nabla \times \bm E
\end{equation}
\begin{equation}
        \nabla \cdot \bm B = 0
\end{equation}
\begin{equation}
        \nabla \times \bm B = \frac{4 \pi}{c} \bm J + \frac{1}{c}\frac{\partial \bm E}{\partial t}
\end{equation}
\begin{equation}
        \nabla \cdot \bm E = 4 \pi \rho_{g} = 4\pi \Sigma_i q_i
\end{equation}
Ideal MHD assumes that conductivity of the fluid is infinite, $\sigma \rightarrow \infty$.\\\\
Lets switch to a co-moving with a fluid frame (that is moving with velcoity $u$). In this frame $\bm J'$, $\bm E'$ are related by Ohm's Law: $\bm J' = \sigma \bm E'$. If $\sigma \rightarrow 0$ then $\bm E' = 0$ in the co-moving frame.\\
\textbf{Lorentz transformation}
$\bm E' = \frac{\bm E + \frac{\bm u \times \bm B}{c}}{\sqrt{1-\frac{u^2}{c^2}}}$ but we will be considering non-relativistic limit with $\frac{u}{c} <<1$. So 
$$
\bm E' = \bm E + \frac{\bm u \times \bm B}{c} + O(\frac{u^2}{c^2})
$$
As $E'=0$:
\begin{equation}
        \bm E = -\frac{\bm u \times \bm B}{c}
\end{equation}
Say L and T are typical length and time scales of the problem $L \sim u T$:
$$
\nabla \times B \sim \frac{B}{L}
$$
$$
\frac{1}{c}\frac{\partial E}{\partial t} \sim \frac{E}{cT} \sim \frac{B}{cT}\frac{u}{c}
$$
So $\frac{\frac{1}{c}\frac{\partial \bm E}{\partial t}}{\nabla \times \bm B} \sim \frac{u^2}{c^2} << 1$, so:
\begin{equation}
        \nabla \times \bm B = \frac{4 \pi}{c} \bm J 
\end{equation}
Can use 7 and 11 to derive the induction equation:
$$
\frac{1}{c} \frac{\partial \bm B}{\partial t} = - \nabla \times \frac{\bm u \times \bm B}{c}
$$
\begin{equation}
        \frac{\partial \bm B}{\partial t} = \nabla \times (\bm u \times \bm B)
\end{equation}
If $\bm u$ is known then we can solve this equation for $\bm B$. The equation is linear if $\bm u$ is specified. Kinematic limit when the B-field doesn't affect $\bm u$ much.\\\\
Take divergence of induction equation gives: $\frac{\partial}{\partial t}(\nabla \cdot \bm B) = 0$ so if a system starts solenoidal then it stays that way.
\subsubsection{Magnetic force}
per unit volume
$$
\bm F = \frac{1}{c} \sum_{i} f_i \bm u_i \times \bm B = \frac{\bm J \times \bm B}{c} = \frac{c}{4 \pi} \frac{(\nabla \times \bm B) \times \bm B}{c} = \frac{(\nabla \times \bm B) \times \bm B}{4 \pi}
$$
as:
$$
 \frac{\partial u}{\partial t} + (\bm u \cdot \nabla ) \bm u = - \frac{\nabla P}{\rho} - \nabla \phi + f_e =  - \frac{\nabla P}{\rho} - \nabla \phi +\frac{(\nabla \times \bm B) \times \bm B}{4 \pi}
$$
\begin{equation}
         \frac{\partial u}{\partial t} + (\bm u \cdot \nabla ) \bm u = - \frac{\nabla P}{\rho} - \nabla \phi +\frac{(\nabla \times \bm B) \times \bm B}{4 \pi}

\end{equation}
This has no electrostatic term ($\rho_g E$) as in the non-relativistic limit it is negillibe compared to the magentic term.
$$
F_{m,i} = \frac{1}{4\pi} \epsilion_{ijk} \epsilion_{jlm} \frac{\partial B_m}{\partial X_l}B_k = -\frac{1}{4\pi}  \epsilion_{ikj} \epsilion_{jlm}  \frac{\partial B_m}{\partial X_l}B_k = -\frac{1}{4\pi} (\delta_{il}\delta_{km}- \delta_{im}\delta_{kl})  \frac{\partial B_m}{\partial X_l}B_k
$$
$$
F_{m,i} = \frac{1}{4\pi}( \frac{\partial B_i}{\partial X_k} B_k  - \frac{\partial B_k}{\partial X_i} B_k) = \frac{1}{4\pi}( (\bm B \cdot \nabla) \bm B - \frac{1}{2} \nabla \cdot \bm B^2)_i 
$$
gives the \textbf{isotropic magnetic pressure}:
\begin{equation}
        \bm F = \frac{(\bm B \cdot \nabla) \bm B}{4\pi} -  \nabla \cdot \frac{\bm B^2)}{8\pi}
\end{equation}
$$
(\bm B \cdot \nabla) \bm B} =  B \frac{\partial \bm B}{\partial s} = B \frac{\partial B \bm s}{\partial s} =    B  \bm S \frac{\partial B}{\partial s} + B^2  \frac{\partial \bm s}{\partial s}  
$$
Did not understand end of lecture 4 about finding the magnetic force in terms of the perpendicular and parallel gradients.\\\\
As
$$
\nabla \times( \bm u \times \bm B)  = \bm u (\nabla \cdot \bm B) + (\bm B \cdot \nabla) \bm u - (\bm u \cdot \nabla) \bm B - \bm B (\nabla \cdot \bm u) 
$$
We have $\nabla \cdot \bm B = 0$ and $\nabla \times( \bm u \times \bm B) = \frac{\partial \bm B}{\partial t}$ so 
$$
\frac{D\bm B}{Dt} =  (\bm B \cdot \nabla) \bm u - \bm B (\nabla  \cdot u) 
$$
From the continuity equation: $\nabla \cdot \bm u = -\frac{1}{\rho}\frac{D\rho}{Dt}$
$$
\frac{D\bm B}{Dt} - \frac{\bm B}{\rho} \frac{D\rho}{Dt} = (\bm B \cdot \nabla) \bm u
$$
This gives the motion of the field lines.
\begin{equation}
\frac{D}{Dt}(\frac{\bm B}{\rho}) = \frac{\bm B}{\rho} \cdot \nabla} \bm u
\end{equation}
This is equivalent to the motion of laine eleemnet from the start of the course so $\frac{\bm B}{\rho} \sim \delta \bm x$ scale proportionally. \\\\
Consider a cyclinder of length $\delta x$ and cross section $\delta S$ with magnetic field $\bm B$ so $\delta m = \rho \delta S \delta x = \bm B \delta S$. Therefore, as the mass is fixed so must the magentic flux element must stay constant. This also holds more generally, consider magnetic flux throguh a surface $S$
$$
\phi = \int_S \bm B \cdot d \bm S
$$
In a Lagrangian frame, contour gets advected with the fluid. Look at the change of the flux as the contour gets advected.
$$
\frac{D\phi}{Dt} = \int_S \frac{\partial \bm B}{\partial t} \cdot d \bm S + \int_C \bm B \cdot (\bm u \times d\bm l)G
$$
with first term is changing magnetic flux and the second is the changing surface (considered seperately).
$$
\frac{D \phi}{D t} = \int_s \nabla \times (\bm u \times B) \cdot d\bm S + \int_C \bm B \cdot (\bm u \times d\bm l)
$$
$$
\frac{D \phi}{D t} = \int_C (\bm u \times B) \cdot d\bm l + \int_C \bm B \cdot (\bm u \times d\bm l) = 0
$$
This illustrates flux-freezing. In MHD the magnetic flux is conserved as the fluid moves. \\
\textbf{Example: Star formation}\\
Should be able to calculate the magnetic flux in the star from knowing the magnetic flux of the initial cloud. This would lead to an estimate of $10^8 G$ but the actual flux is $10^2 G$ and this is because the MHD assumption is not accurate.\\
\textbf{Critical flux}\\
$$
E_m \sim \frac{\phi^2}{6R_c}
$$
$$
E_{grav} \sim \frac{GM^2}{R_c}
$$
If
$$
\frac{E_m}{E_{grav}} = \frac{\phi}{M_c}^2\frac{1}{G} < 1
$$
then collapse occurs so need $\phi < \phi_{crit} = M_cG^{\frac{1}{2}}$ in order for it collapse. For our sun the flux is too large for the collapse to have occured. The solution to this is the solar formation occured at very cold temperatures so the ionization is very low so the conductivity is not infinite so has other effects to consider such as the hall effect and resisitivity and mainly antipolar diffusion. Antipolar diffusion is when B field couples to charges and charges then couple to neutrals by collisions so the magnetic field does not affect the neturals and so the magnetic field just slips through the cloud of gas. So therefore magentic flux is not conserved and we expexpect to lose a lot of magnetic flux.
\\\\
\textbf{Example: Formation of neutron stars}
A neutron star is only 10 km radius whereas the core of the star that collapses is about $10^{11}$ cm.
$$
\phi \sim B_*R_*^2 \sim B_{NS}R_{NS} \>\> \implies B_{NS} \sim B_*\frac{R_*}{R_{NS}} = 10^{12}G
$$
This is actually very accurate so the model works well for the transition from a star to a neutron star.\\\\
\subsection{Rotating, axisymmetric systems: induction equation}
Use cyclindrical coordinates
$$
\bm u = u_{\psi} \bm e_{\psi} = R \Omega(R,z) \bm e_{\psi}
$$
$$
\bm B = \bm B_p + B_{\psi}\bm e_{\psi}
$$
$$
\bm B_p = B_R \bm e_R + B_z \bm e_z
$$
$$
\frac{\partial \bm B}{\partial t} = \nabla \times (\bm u \times \bm B) = (\bm B \cdot \nabla ) \bm u - (\bm u \cdot \nabla) \bm B - \bm B (\nabla \cdot \bm u)
$$
Last term can be dropped as incompressible behaviour
$$
\nabla = (\frac{\partial}{\partial R}, \frac{\partial}{\partial z}, \frac{1}{R}\frac{\partial}{\parital \psi}
$$$$
\frac{\partial \bm e_R}{\partial \psi} = \bm e_{\psi}, \frac{\partial \bm e_{\psi}}{\partial \psi} = - \bm e_R
$$
$$
(\bm B \cdot \nabla) \bm u = B_r\frac{\partial(u_{\psi} \bm e_{\psi}}{\partial R} + B_z \frac{\partial(u_{\psi} \bm e_{\psi}}{\partial z} + \frac{B_{\psi}}{R} \frac{\partial(u_{\psi} \bm e_{\psi}}{\partial \psi} = B_r \frac{\partial (R\Omega)}{\partial R} \bm e_{\psi}+ B_z \frac{\partial (R\Omega)}{\partial z} \bm e_{\psi} + B_{\psi} \Omega(-\bm e_R) 
$$
$$
(\bm u \cdot \nabla) \bm B = \frac{u_{\psi}}{R} \left( \frac{\partial }{\partial \psi} B_R \bm e_R + \frac{\partial}{\partial \psi}B_z \bm e_z + \frac{\partial }{\partial \psi} (B_{\psi} \bm e_{\psi}\right)
$$
$$
\frac{\partial \bm B}{\partial t} = \bm e_{\psi} R (\bm B_p \cdot \nabla) \Omega` 
$$
$$
\frac{\partial \bm B_p}{\partial t} = 0, \frac{\partial B_{\psi}}{\partial t} = R(\bm B_p \cdot \nabla) \Omega
$$
In steady state have $ \frac{\partial B_{\psi}}{\partial t}  = 0$ $ \implies (\bm B_p \cdot \nabla) \Omega = 0$ so the gradient of $\Omega$ is always orthogonal to $\bm B_p$ which means $\Omega$ is constant on magnetic surfaces. This is called the \textbf{Ferronos' isozotation law}
\subsection{Magnetic Forces - force balance}
$$
\frac{D \bm u}{Dt} = - \frac{\nabla P}{\rho} - \nabla \phi + \frac{(\nabla \times \bm B) \times \bm B}{4 \pi \rho}
$$
L, T are the length and time scales. LHS and first term comparison $\frac{u \rho L}{\rho c_s^2 T} = \frac{v}{c}^2$ if $u<<c_s$ can neglect ineterial LHS compared to 1st term on RHS.\\
Now compare LHS and 3rd term. $\frac{4\pi uL\rho}{TB^2} \sim \frac{u^2}{\frac{B^2}{4\pi \rho}}$. $ \frac{B}{\sqrt{4\pi \rho}}$ is the Alfeven velocity $u_A$. so this ratio is $\frac{u}{u_A}^2$. So if $u << u_A$ can neglect inertial term compared to magnetic force. Consider third term against first term: $\frac{(3)}{(1)} \sim \frac{B^2}{4\pi \rho L}\frac{\rho L}{\rho c_s^2} = \frac{u_A}{c_s}^2$
\\\\
\noindent
When $u << c_s$ neglect inertial vs pressure\\
When $u << u_A$ can neglect inertial vs magnetic stresses\\
When $c_s << u_A$ can neglect thermal pressure vs magnetic pressure\\ \\
\textbf{Plasma $\beta$ }\\
$p$ - thermal pressure\\
$p_B = \frac{B^2}{8\pi}$ - magnetic pressure\\
$$
\beta = \frac{p}{p_B} \sim \frac{c_s}{u_A}^2
$$
So whenever $\beta <<1$ then thermal pressure is negligible, and if $\beta >>1$ then magnetic pressure dominates.\\
\subsection{Magnetic Buoyancy}
If we have a magnetic structure in which the velocity is very small so we can set the magnetic structure to zero:
$$
0 = - \nabla p - \rho \nabla \phi + \frac{(\nabla \times \bm B) \times \bm B}{4 \pi} = -\nabla p - \rho \nabla \phi + \frac{1}{4\pi} B^2 \frac{\bm u}{R_c} - \nabla_{\perp}p_B
$$
Imagine a flux tube with considerably stronger flux density than the surrounding fluid which the magnetic field going in the $-x$ direction with graviational acceleration in $-z$ direction. Look $z$ in a plane. Assume $R_c \rightarrow \infty$ as magnetic field lines straight and parallel, also as $\nabla \phi$ is in $-z$ direction. The only contributions in the $x$ direction is: 
$$
0 = - \nabla_{\perp} (p+p_B) \implies p+p_B = p + \frac{B^2}{8\pi}= c(z)
$$
Let $B_2$ be the much stronger magnetic flux inside the tube and $B_1$ be the flux outside.
$$
 p_1 + \frac{B_1^2}{8\pi} =  p_2 + \frac{B_2^2}{8\pi} \implies p_1 > p_2
 $$
 This means that $\rho_2 < \rho_1$ if we assume the temperature remains constant in the centre of a star as the thermal timescales are to quick and equillibrium will be found quickly. This generates an archimedian force that pushes the flux tube up hence the magnetic buoyancy. So strong areas of magnetic fields will float up to the surface of a star for instance. \\
\texbf{Sunspot}\\
At some point on the surface of the sun you have an area with a much higher magnetic field flux ($B_2$) than the surrounding area ($B_1$). Can again fix $z$ constant we would find the same equations again:$p_1 + \frac{B_1^2}{8\pi} =  p_2 + \frac{B_2^2}{8\pi} \implies p_1 > p_2$ at fixed $z$.Typically the temperature does vary a lot here rather than density, as it is easily possible to lose heat. As emissivity $\sim 6T^4$ we get a magnetically dominated areas being dark sun spots.
\subsection{Force-free equilibrium}
Assume that $\beta <<1$ meaning that $p_B >> p$, then we can neglect gravity, so the magnetic force dominates the force balance. Useful if gravity is very weak. This means $\bm F_k = 0$ which looks weird as it looks like the magnetic force is zero but it actually just means the magnetic terms are so much large than the rest. Remember the decompistion of the magnetic force:
$$
\bm F_k = \frac{(\nabla \times \bm B) \times \bm B}{4 \pi} = \frac{\bm J \times \bm B}{c} = 0
$$
so when $\bm J || \bm B$ we are in the free force state. Therefore, $\bm J = \alpha(\bm u)\bm B$ so $\nabla \times \bm B = \frac{4\pi}{x} \bm J = \frac{4 \pi \alpha(\bm u) }{c} \bm B$.
Take the divergence of this equation:
$$
0 = \frac{4\pi}{c}\left( ( \bm B \cdot \nabla) \alpha + \alpha (\nabla \cdot \bm B) \right) \implies (\bm B \cdot \nabla) \alpha = 0
$$
This looks very similar to the Ferrono's isozotation but here it means that $\alpha(\bm u)$ is constant along the magnetic field lines so not just an arbitrary function. We have
$$
\frac{4\pi}{c} \alpha(\bm u) \bm B = \nabla \times \bm B 
$$
which is linear in $\bm B$. If know distribution of $\alpha(S)$ on a particular surface S then I can solve this whole system. The structure of the upper solar atmosphere is force free, same is true for magnetosphere of pulsars. \\\\
Lets consider the consquences of assuming $\alpha$ is constant. Then we would find that $\nabla \cdot \bm B = \frac{4\pi \alpha}{c} \bm B$ (take the curl):
$$
\nabla \times(\nabla \cdot \bm B) = \nabla(\nabla \cdot \bm B) - \nabla^2 \bm B = \frac{4\pi \alpha}{c} \nabla \times \bm B = (\frac{4\pi \alpha}{c}) ^2\bm B 
$$
This gives a Helmholtz equation (this is a nice linear equation):
$$
\nabla^2 \bm B +  (\frac{4\pi \alpha}{c}) ^2\bm B = 0
$$
\textbf{Cylindrical force free equilibrium}\\
Assume $\alpha$ is constant and cylindrically symmetrical. $\bm r = ( R, z, \psi)$, $\bm B = (B_R, B_z, B_{\psi})$, $\frac{\partial}{\partial \psi} = 0$,$\frac{\partial}{\partial z} = 0$
$$
\nabla \cdot \bm B = 0 \implies \frac{1}{R}\frac{\partial}{\partial R} (R B_R) = 0 \implies RB_R(R) = \text{constant}
$$
This is a solution but it is a bad solution as it diverges at $R=0$ and the only way to get a non-singular system is to set the constant to 0 so $B_R =0$.\\
$$
\nabla \times \bm B = -\frac{\partial B_z}{\partial R}\bm e_{\psi} + \frac{1}{R} \frac{\partial}{\partial R}(RB_{\psi} \bm e_z
$$
$$
\frac{4\pi\alpha}{c}\bm B = \frac{4\pi\alpha}{c}B_{\psi}\bm e_{\psi} + \frac{4\pi\alpha}{c}B_z \bm e_z
$$
$$
B_{\psi} = \frac{c}{4\pi \alpha}(-\frac{\partial B_z}{R}), \alpha B_z = \frac{c}{4\pi} \frac{1}{R} \frac{\partial}{\partial R} (R B_{\psi})
$$
$$
\alpha B_z = \frac{c}{4\pi} \frac{1}{R} \frac{\partial}{\partial R} \left( R \frac{c}{4\pi \alpha} (-\frac{\partial B_z}{\partial R}) \right)
$$
\section{Lecture 8}
$$
\frac{1}{R}\frac{\partial}{\partial R}(R \frac{\partial B_z}{\partial R}) + (\frac{4 \pi \alpha}{c})^2 B_z = 0
$$
This is a version of Helmholtz equation and remember that in cyclindrical geometry $$\nabla^2 f = \frac{1}{R} \frac{\partial}{\partial R} (R \frac{\partial f}{\partial R}) + \frac{\partial^2 f}{\partial z^2} + \frac{1}{R^2}\frac{\partial^2 f}{\partial \phi^2}$$
The solutions to this equation have form:
$$
B_z = H_aJ_a(\tilde \alpha R)
$$
$$
B_{\phi} = B_a J_1(\tilde \alpha R)
$$
where $\tilde \alpha = \frac{4 \pi}{c} \alpha$, $J_o, J_1$ are Bessel functions. Which satisify:
$$
(tJ_o'(t))' + tJ_0(t) = 0, J_1(t) = -J_0'(t)
$$
This length scale $\alpha$ depends on the stength of the current running along the field lines. If the current is very high then the field variations will be on a shorter length scale.
\subsection{Potential Field}
This gets realised whenever $\bm J = 0$ so $\alpha = 0$ everywhere. We know that $\bm J \sim \nabla \times \bm B$ so $\nabla \times \bm B = 0$, so therefore $\bm B = \nabla \phi$ so is a potential field. As we still have $\nabla \cdot \bm B = 0$ we get $\nabla^2 \phi = 0$. This is a familar equation that we know how to solve. e.g. Consider a magnetic star with boundary condition on surface of star if we know $\bm B $ at $r = R_*$, then $\phi = \sum_{l=0}^{\infty} \sum_{m=-l}^l (a_{lm}r^l + b_{lm}r^{-(l+1)})P_l^m(\cos \theta) e^{im\phi}$.

\subsection{Grad-Shafranov Equation}
Lets consider a system with pressure and B-field in equilibrium. The equation of motion of this system is 
$$
-\nabla p + \frac{\bm J \times \bm B}{c} = 0
$$
Take a scalar product with $\bm B$:
$$
(\bm B \cdot \nabla) p = 0
$$
So the gradient of magnetic pressure has to be perpendicular to magnetic pressure/magnetic field lines have to be perpendicular to surfaces where the pressure is constant. If you take scalar product with $\bm J$:
$$
(\bm J \cdot \nabla)p =0
$$ so same result as for $\bm B$. If I construct surfaces of constant $p$: $p(x,y,z) p constant$ then it follows that $\bm B, \bm J$ must lie on these surfaces.\\\\
Assume axisymmetry $\frac{\partial }{\partial \phi} =0$:
$$
\bm B = \bm B_p + B_{\phi} \bm e_{\phi}
$$
Introduce magentic flux:
$$
\psi(R,z) = \int^R_0 B_z 2\pi R dR
$$
Introduce current:
$$
I(R,z) = \int^R_0 J_z 2\pi R dR
$$
Flux can be considered like a label of magnetic surfaces by the value of $\psi$ that they are enclosing.
$$
I = I(\psi), p = p(\psi)
$$
(current and pressure are constant on surfaces of given $\psi$). From defintion of flux:
$$
B_z = \frac{1}{2\pi R} \frac{\partial \psi}{\partial R}
$$
from defintion of current:
$$
J_z = \frac{1}{2\pi R} \frac{\partial I}{\partial R}
$$
From $\nabla \cdot \bm B = 0$:
$$
\frac{1}{R} \frac{\partial R}{\parital R} (R B_R) + \frac{\partial B_z}{\partial z} = 0
$$
$$
\frac{1}{R} \frac{\partial }{\partial R}(R B_R) + \frac{1}{2\pi R} \frac{\partial^2 \psi}{\partial z \parital R}  = 0
$$
$$
RB_R + \frac{1}{2\pi} \frac{\partial \psi}{\partial z} = c(z) \implies B_R = -\frac{1}{2\pi R} \frac{\partial \psi}{\partial z} + \frac{c(z)}{R}
$$
As we want a solution that is non-singular so the last term must be zero
$$
B_R = -\frac{1}{2\pi R} \frac{\partial \psi}{\partial z}
$$
We now know $B_R$ and $B_z$ so 
$$
\bm B_p =  -\frac{1}{2\pi R} (\frac{\partial \psi}{\partial z}, \frac{\partial \psi}{\partial R} ) = \frac{\nabla \psi \times \bm e_{\phi}}{2\pi R}
$$
Lets look at the current:
$$
I = \int_S \bm J \cdot d\bm S = \int_S \frac{c}{4\pi} (\nabla \times \bm B) \cdot d\bm S = \frac{c}{4\pi} \int_C \bm B \cdot d\bm l = \frac{c}{4\pi} B_{\phi} 2 \pi R = \frac{c B_{\phi}R}{2}
$$
$$
B_{\phi} = \frac{2 I(\psi)}{Rc}
$$
$$
\bm B =  \frac{\nabla \psi \times \bm e_{\phi}}{2\pi R} +  \frac{2 I(\psi)}{Rc}\bm e_{\phi}}
$$
For $\bm J = \frac{c}{4\pi} \nabla \times \bm B \implies \nabla \cdot \bm J = 0$. We could do the same derivation the current and we will get exactly the same expression for the poliodal component of hte current:
$$
\bm J_p =  \frac{\nabla I(\psi) \times \bm e_{\phi}}{2\pi R}
$$
For $J_{\phi}$ we can consider $J_{\phi} = \frac{c}{4\pi}( \nabla \times \bm B)_{\phi}$ so
$$
\bm J =  \frac{\nabla I(\psi) \times \bm e_{\phi}}{2\pi R} + \frac{c}{4\pi}( \nabla \times \bm B)_{\phi}\bm e_{\phi}
$$
This expression has second derivatives of $\psi$ in it as we differentiate $B$ again.
\section{Lecture 9}
$$
\nabla p = \frac{\partial p}{\partial \psi} \nabla \psi = \frac{\bm J \times \bm B}{c}
$$
This is normal to the surface of constant pressure, and so it is also normal to the flux surfaces and the current surfaces. As the RHS is also normal to the flux and current surface this equation only has one component, therefore:
$$
\frac{\bm J \times \bm B}{c} = G(\nabla \psi)
$$
This force is non-linear (it is quadratic in $\psi$, $\bm B$ contains first derivatives in $\psi$ and $\bm J$ contains second derivatives of $\psi$). He will not be deriving the expression for $G$, just giving the final result which is:
$$
\frac{\partial p}{\partial \psi} \nabla \psi = G \nabla \psi \implies \nabla \psi(G - \frac{\partial p}{\partial \psi}) = 0
$$
\textbf{Grad-Shafrahov equation}:
$$
G - \frac{\partial p}{\partial \psi} = 0
$$
Often called equation of cross-field balance or trans-field balance.\\
G in particular case: Cyclindrical coordinates (R, z)
$$
\frac{\partial^2 \psi}{\partial R^2} - \frac{1}{R} \frac{\partial \psi}{\partial R} + \frac{\partial ^2 \psi}{\partial z^2} = - 16\pi^3 R^2 \frac{\partial p}{\partial \psi} - \frac{8\pi^2}{c^2} \frac{\partial I^2(\psi)}{\partial \psi}
$$
Don't have to memorise this lol i'm glad. The power of this approach is we can just solve this system and we can then get the behaviour of everything (current and pressure) just from the flux $\psi(R,z)$. LHS does not reduce very nicely unfortunately it is: $R^2 \nabla \cdot( \frac{1}{R^2} \nabal \psi)$. What is interesting is flux function was originally used as just a coordinate for labeling our pressure surfaces and it has become an independant function that we need to solve for. This equation does not have to be linear as $p(\psi)$ and $I(\psi)$ could be anything (normally specified at the beginning of the problem). We can very easily get force free by dropping pressure term.\\\\
Consider force-free cyclindrical equilibrium structure ($\frac{\partial }{\partial z} = 0, p = 0$). Assume that $I(\psi) = \kappa \psi$ so:
$$
\frac{\partial^2 \psi}{\partial R^2} - \frac{1}{R} \frac{\partial \psi}{\partial R} = - \frac{16\pi^2\kappa^2}{c^2}\psi$$
$$
\frac{\partial}{\partial R}(R\frac{1}{2 \pi R} \frac{\partial \psi}{\partial R}) - \frac{1}{2\pi R} \frac{\partial \psi}{\partial R} = - \frac{8 \pi \kappa^2}{c^2} \psi
$$
$$
\frac{\partial}{\partial R}(RB_z) - B_z = R\frac{\partial B_z}{\partial R} = - \frac{8 \pi \kappa^2}{c^2} \psi
$$
$$
\frac{1}{R}\frac{\partial}{\partial R} R \frac{\partial B_z}{\partial R} =  - \frac{16 \pi^2 \kappa^2}{c^2} \frac{1}{2\pi R}\frac{\partial \psi}{\partial R} = - \frac{16 \pi^2 \kappa^2}{c^2} B_z
$$
let $\alpha = \frac{4\pi\kappa}{c}$ and this reduces to the Helmholtz equation
$$
\nabla^2 B_z + \alpha^2 B_z = 0
$$
\section{Conservation Laws}
Some variable has a density (amount per volume) $q(\bm x, t)$ that satisfies an equation:
$$
\frac{\partial q}{\partial t} + \nabla \cdot \bm F = 0
$$
then we say that $\bm F$ is a flux of $q$. By analogy with continiuity equation: $q = \rho$, $\bm F = \rho \bm u$. Amount of $q$ in volume $V$: $Q = \int_V q(\bm x,t) dV$.
$$
\frac{\partial Q}{\partial t} = - \int_V \nabla \cdot \bm F  dV = - \int_S \bm F \cdot d \bm S \text{ flux of $q$ across surface of V}
$$
Define a material invariant as a scalar field $f(\bm x, t)$ for which $\frac{D f}{Dt} = 0$. e.g. Entropy is a material invariant. Then $\frac{\partial f}{\partial t} + (\bm u \cdot \nabla)f = 0$, therefore $\rho\frac{\partial f}{\partial t} + (\rho\bm u \cdot \nabla)f = 0$ and as $f\frac{\partial \rho}{\partial t} + f\nabla \cdot (\rho \bm u) =0$:
$$
\frac{\partial (\rho f)}{\partial t} + \nabla \cdot ( \rho f \bm u) = 0
$$
Consider energy density:
$$
\epsilon = \rho(\frac{u^2}{2} + \phi + e) + \frac{B^2}{8\pi}
$$
We want to find $\bm F_{\epsilon}$ such that $$\frac{\partial \epsilon}{\partial t} + \nabla \cdot \bm F_{\epsilon} = 0$$
$$
\frac{\partial \epsilon}{\partial t} = \frac{\partial \rho}{\partial t} ( \frac{u^2}{2} + \phi + e) + \frac{1}{8\pi} \frac{\partial B^2}{\partial t} + \rho (\frac{1}{2} \frac{\partial u^2}{\partial t} + \frac{\partial \phi}{\partial t} + \frac{\partial e}{\partial t})
$$
$$
\frac{1}{8 \pi } \frac{\partial B^2}{\partial t} = \frac{\bm B}{4 \pi} \frac{\partial \bm B}{\partial t} =^{maxwells} - \frac{\bm B c}{4 \pi} (\nabla \times \bm E) = - \frac{c}{4\pi} \bm B \cdot (\nabla \times \bm E)
$$
$$
\rho \frac{1}{2} \frac{\partial u^2}{\partial t} = \rho \bm u \frac{\partial \bm u}{\partial t} = \rho \bm u ( - (\bm u \cdot \nabla) \bm u - \frac{\nabla p}{\rho} - \nabla \phi \rho + \frac{1}{4\pi \rho}((\nabla \times \bm B) \times \bm B)) 
$$
$$
\frac{De}{Dt} =T \frac{Ds}{Dt} - p\frac{dV}{Dt} = t\frac{Ds}{Dt} + \frac{p}{\rho^2}\frac{D\rho}{Dt}
$$
$$
\frac{\partial e}{\partial t}  = T \frac{Ds}{Dt} -(\bm u \cdot \nabla) e - \frac{p}{\rho}(\nabla \cdot \bm)
$$
\section{Lecture 9}
$$
\frac{\partial \epsilon}{\partial t} = -\frac{\partial \rho}{\partial t} ( \frac{u^2}{2} + \phi + e) - \rho \bm u( \bm u \cdot \nabla) \bm u - (\bm u \cdot \nabla) \phi + \frac{\bm u \cdot( (\nabla \times \bm B) \times \bm B)}{4\pi}  + \frac{\partial \phi}{\partial t} - (\rho \bm u) \cdot \nabla e - p (\nabla \cdot \bm u) + \rho T \frac{Ds}{Dt} 
$$
$$
\frac{\partial \epsilon}{\partial t} =  -\frac{\partial \rho}{\partial t} ( \frac{u^2}{2} + \phi + e) - \rho(\bm \cdot \nabla) \frac{u^2}{2} - )\rho \bm u \cdot \nabla)e - p(\nabla \cdot \bm u) - (\bm u \cdot \nabla )p - (\bm u \cdot \nabla) \phi - \frac{c}{4\pi} \bm B \cdot (\nabla \times \bm E) + \frac{c(\nabla \tims \bm ) \bm B \times \frac{\bm u}{c}}{4\pi} + \rho \frac{\partial \phi}{\partial t} + \rho T \frac{Ds}{Dt} 
$$
$$
\frac{\partial \epsilon}{\partial t} = -\frac{\partial \rho}{\partial t} ( \frac{u^2}{2} + \phi + e) - (\rho \bm u \cdot \nabla) (\frac{u^2}{2} + e + \phi) - \nabla \cdot (p \bm u) - \frac{c}{4\pi}(\bm B \cdot (\nabla \times \bm E)- (\nabla \times \bm B)\bm E)
$$
$$
\frac{\partial \epsilon}{\partial t} = -\nabla \cdot (\rho\bm u (\frac{u^2}{2} + \phi + e) + p \bm u) - \frac{c}{4\pi}(\bm B \cdot (\nabla \times \bm E)- (\nabla \times \bm B)\bm E) + \rho \frac{\partial \phi}{\partial t} + \rho T \frac{Ds}{Dt}
$$
as
$$
\nabla \cdot (\bm M \times \bm N) = \bm N \cdot (\nabla \times \bm M) - \bm M \cdot (\nabla \times \bm N) \implies \bm B \cdot (\nabla \times \bm E)- (\nabla \times \bm B)\bm E = \nabla \cdot( \bm E \times \bm B)
$$
$$
\frac{\partial \epsilon}{\partial t} = -\nabla \cdot (\rho\bm u (\frac{u^2}{2} + \phi + e) + p \bm u) - \nabla \cdot( \frac{c}{4\pi} \bm E \times \bm B) + \rho \frac{\partial \phi}{\partial t} + \rho T \frac{D s}{Dt}  = - \nabla \cdot (\rho\bm u (\frac{u^2}{2} + \phi + e + \frac{P}{\rho}) + \frac{c}{4\pi} \bm E \times \bm B) + \rho \frac{\partial \phi}{\partial t} + \rho T \frac{Ds}{Dt}
$$
so
\begin{equation}
        \frac{\partial \epsilon}{\partial t} + \nabla \cdot \bm F_{\epsilon} = \rho \frac{\partial \phi}{\partial t} + \rho T \frac{Ds}{Dt}
\end{equation}
for $F_{epsilon} = \rho\bm u (\frac{u^2}{2} + \phi +h) + \frac{c}{4\pi} \bm E \times \bm B$, with $e + \frac{P}{\rho} = h$ (enthalpy)
\subsection{Bernoulli invariant}
$$
\frac{\partial \epsilon}{\partial t} + \nabla \cdot \bm F_{\epsilon} = 0
$$
In steady state $\frac{\partial \epsilon}{\partial t} = 0$ so $\nabla \cdot \bm F_{\epsilon} = 0$. Assume hydrodynamic limit, $\bm B \rightarrow 0$ which implis
$$
\nabla\cdot (\rho \bm u(\frac{u^2}{2} + \phi + h) = 0
$$
$$
\nabla\cdot(\rho \bm u) (\frac{u^2}{2} + \phi + h) + \rho (\bm u \cdot \nabla) (\frac{u^2}{2} + \phi + h)) = 0
$$
so $\bm u \cdot \nabla ) B_B = 0, and $ B_B = \frac{u^2}{2} + \phi + h$ so $B_B$ is constant along streamlines of the flow.
\subsection{Magnetic helicity conservation}
Vector potential $\bm A$, $\bm B = \nabla \times \bm A \implies \nabla \cdot \bm B = 0$. Define magnetic helicity:
$$
H_m = \int_V \bm A \cdot \bm B dV
$$
$$
\frac{\partial}{\partial t} \bm A \cdot \bm B = \bm B \cdot \frac{\partial \bm A}{\partial t} + \bm A \cdot \frac{\partial \bm B}{\partial t} = \bm B \cdot \frac{\partial \bm A}{\partial t} + \bm A \cdot ( \nabla \times (\bm u \times \bm B))
$$
Maxwell's equation
$$
- \nabla \times \bm E = \frac{1}{c} \frac{\partial \bm B}{\partial t} = \frac{1}{c} \nabla \times \frac{\partial \bm A}{\partial t} \implies \nabla \times( \frac{1}{c} \frac{\partial \bm A}{\partial t} + \bm E) = 0
$$
$$
\bm E + \frac{1}{c}\frac{\partial \bm A}{\partial t} = - \nabla \phi_l
$$
where $\phi_l$ is th electrostatic potential
$$
\frac{\partial \bm A}{\partial t} = -c(\bm E + \nabla \phi_l)
$$
so 
$$
\frac{\partial }{\partial t} \bm A \cdot \bm B = - c(\bm B \cdot \nabla \phi_l) - c (\bm E \cdot \bm B) + \bm A \cdot (\nabla \times( \bm u \times \bm B))
$$
Second term vanishes as $E = - \frac{\bm u \times \bm B}{c}$ 
$$
\nabla \cdot (\bm A \times (\bm u \times \bm B)) = (\bm u \times \bm B) \cdot (\nabla \times \bm A) - \bm A \cdot \nabla \times (\bm u \times \bm B) = \bm B \cdot (\bm u \times \bm B) - \bm A \cdot \nabla \times (\bm u \times \bm B)
$$
so
$$
\bm A \cdot \nabla \times(\bm u \times \bm B) = - \nabla \cdot (\bm A \times (\bm u \times \bm B))
$$
First term (use fact $\nabla \cdot \bm B = 0$):
$$
(\bm B \cdot \nabla) \phi_e = (\bm B \cdot \nabla) \phi_e + \phi_e \nabla \cdot \bm B = \nabla \cdot(\bm B \phi_e)
$$
so:
$$
\frac{\partial}{\partial t} \bm A \cdot \bm B = - c \nabla \cdot (\bm B \phi_e) - \nabla \cdot (\bm A \times (\bm u \times \bm B)) = - \nabla \times (c \bm B \phi_e + \bm A \times (\bm u \times \bm B))
$$
\begin{equation}
        \frac{\partial H_m}{\partial t} + \nabla \cdot \bm F_{H_m} = 0
\end{equation}
for $\bm F_{H_m} = c \bm B \phi_e + \bm A \times (\bm u \times \bm B)$
(often have to derive conservation laws like this in exams so understand and learn technique)
\end{document}
