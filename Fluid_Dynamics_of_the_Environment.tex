\documentclass{article}
\usepackage[utf8]{inputenc}
\usepackage{bm}
\usepackage{amssymb}
\usepackage{amsmath}
\usepackage{braket}
\usepackage{cancel}
\title{Quantum Information Theory}
\author{oliverobrien111 }
\date{July 2021}

\begin{document}

\maketitle

\section{Lecture 1}
\subsection{Internal waves}
\subsubsection{Minimal maths version}
Imagine we have some sort of stratification that is given by density profile $\hat \rho$. We are going to require that this is smooth and as we will see later and it is only changing the gradient over a length scale that is large compared to the length scale of the waves. Consider a small volume $V$ of fluid of density $\rho_0$ and I will magically displaced it up by a distance of $\zeta$ (assuming it retains volume, density and shape). There will clearly be a buoyancy force $B$, and for small displacements:
$$
B = g V \zeta \frac{d\hat \rho}{d z}  $$
Newton's second law gives
$$
\rho_0 V \ddot \zeta = B = g V \zeta \frac{d\hat \rho}{d z} 
$$
$$
\ddot \zeta + (0 \frac{g}{\rho_0} \frac{d \hat \rho}{dz} ) \zeta = 0
$$
This will clearly have solutions:
$$
\zeta = A \cos Nt + B \sin Nt
$$
where $N$ is the buoyancy (or Brunt-Vaisala frequency) $N = \sqrt{-\frac{g}{\rho_0} \frac{d\hat \rho}{dz}}$\\\\
There is a key ingredient that hasn't been taken into account in this treatment. There hasn't been anything about continuity e.g. how does the fluid get out the way when it falls back down. To highlight this if we displace a long slim slab of fluid instead of a sphere. If this long slim slab is vertical and we displace it upwards then we get exactly the same maths, but we can make it thin enough that we don't need to worry about continuity to the first order. Now what happens if we take a long thin slab of fluid at an angle and move it along itself. Would it fall vertically downards or slide back along itself. It is intuitively hard to fall downwards as this would cause a continuity issue of needing to compress all of the fluid below the line (in order to fall down a lot of fluid needs to be moved past the line which would make a big pressure difference resiting the motion). So the slab will slide back along itself. This will have only displaced each parcle of fluid by $\zeta \cos \theta$ upwards and it will only experience a force of $g \cos \theta$ back along its original path. Therefore this would give:
$$
\ddot \zeta + N^2 \cos^2 \theta \zeta = 0
$$
$$
\ddot \zeta + \omega \zeta = 0
$$
This gives the dispersion relation for internal gravity waves:
$$
|\omega/N| = |\cos \theta|
$$
This logic works if we stack slabs on top of each other and as long as we only move each on by a small amount. They can all oscillate up and down along themselves but have no need to be in the same phase, so you could send a wave though them all perpendicular to the slabs. This would mean each slab is a line of constant phase with energy being transmitted along the slab.\\\\
Now lets think about the $\cos \theta$. If $\bm k = (k,l,m)$ is a vector perpendicular to the slabs, then we have:
$$
\cos \theta = \frac{\sqrt{k^2 + l^2}}{\sqrt{k^2 + l^2 + m^2}}
$$
\subsubsection{More rigorous derivation}}
$$
\nabla \cdot \bm u = 0
$$
$$
(\frac{\partial}{\partial t} + \bm u \cdot \nabla) \rho = \kappa \nabla^2 \rho
$$
For now $\kappa = 0$:
$$
\rho \frac{ \partial \bm u}{\partial t} + \rho (\bm u \cdot \nabla) \bm u = - \nabla p - \rho g \bm \hat z + \rho \nu \nabla^2 \bm u
$$
For now $\nu = 0$. Take $\rho = \rho_0 + \rho'$. Linearise with $\rho' << \rho_0$ and Boussinesq $|\frac{\nabla \bm u}{\nabla t} | << g$. The Boussieneq means that the $\rho'$ contrbutes to the gravity term but not to the interial term. We can rewrite the equations using reduced gravity to make this clear $g' = g \frac{\rho - \rho_0}{\rho} = g \frac{\rho'}{\rho}$:
$$
\\frac{ \partial \bm u}{\partial t} + (\bm u \cdot \nabla) \bm u = - \frac{1}{\rho_0} \nabla(p + \rho g z) - g' \hat{\bm z}
$$
To linearise we take$\rho = \hat \rho(z) + \rho'(x,t)$ and  $\bm u' \sim \eta \omega << \frac{\omega}{|k|} \implies |k| \eta << 1$. By combining this with our Boussineqs condition we have $|\nabla \rho'| << | \frac{d\hat \rho}{d z}|$. This gives the navier stokes equation:
$$
\frac{\partial \rho'}{\partial t} + w \frac{d\hat \rho}{dz} = \frac{\partial \rho'}{\partial t} - w \frac{\rho_0}{g} N^2 = 0
$$
$$
\frac{\partial \bm u}{\partial t} = - \frac{1}{\rho_0} \nabla( p_0 + \hat p - p') - \frac{\hat \rho + \rho'}{\rho_0} \bm z =  - \frac{1}{\rho_0} \nabla (p_0 + \hat p) - g \frac{\hat \rho}{\rho_0} \hat{\bm z} - \frac{1}{\rho_0} \nabla p' - g \frac{\rho'}{\rho_0} \hat{\bm z}
$$
Unperturbed state gives the first two terms as they are much larger than the other terms
$$
0 = - \frac{1}{\rho_0} \nabla (p_0 + \hat p) - g \frac{\hat \rho}{\rho_0} \hat{\bm z}
$$
therefore
$$
p_0 + \hat p = - \int g \hat \rho dz
$$
Let
$$
b = - g \frac{\rho'}{\rho_0} 
$$
Therefore:
\begin{equation}
\frac{\partial b}{\partial t} = - w N^2
\end{equation}
\begin{equation}
        \frac{\partial \bm u}{\partial t} = - \frac{1}{\rho_0} \nabla p' + b \hat{\bm z}
\end{equation}
\begin{equation}
        \nabla \cdot \bm u = 0
\end{equation}
\subsection{Vorticity}: $\bm \zeta = \nabla \times \bm u$
We are going to deal in 2D as it is easier. In 2-D vorictity can be expressed interms of the streamfunction:
$$
\zeta = - \nabla^2 \psi, \bm \psi = \psi \hat{\bm y}, \bm u = \nabla \times \bm \psi
$$
Take curl of momentum equation to remove pressure:
$$
\frac{\partial \bm \zeta}{\partial t} = - \hat{\bm z} \times \nabla b
$$
$$
(\hat{\bm z} \times \nabla) \cdot (\hat{\bm z} \cdot \nabla w) = \nabla^2_H w
$$
so gives voriticity equation:
\begin{equation}
        (\nabla^2 \frac{\partial^2}{\partial t^2} + N^2 \nabla^2_H) w = 0
\end{equation}
Pose a plane wave type solution ansatz and see what happens. Let :
$$
w(\bm x, t) = Re( \hat w(z) e^{i(kx + ly - \omega t)})
$$
$$
\frac{d^2 \hat w}{dz^2} + (k^2 + l^2) (\frac{N^2}{\omega^2} -1) \hat w = 0
$$
So if we let $m^2 = (k^2 + l^2) ( \frac{N^2}{\omega^2}-1)$ then:
$$
\hat w = Re( A e^{imz} + Be^{-imz}
$$
If $\omega > N$ then $m$ is imaginary let $\gamma = \sqrt{1- \frac{N^2}{\omega^2}}$:
$$
w = (\hat A e^{- \gamma k_h z + \hat B e^{\gamma k_h z})e^{i(k x + ly - \omega t)}
$$
This is sort of showing how the velocity field changes with depth away from a distribance on the surface. If we have $N=0$ then this is just a surface wave. If we have $0<N<\omega$ the $\gamma$ is just giving a vertical rescaling of the behaviour beneath the surface wave. This means if we produce a sinosidal distributance with a frequency bigger than the bouyancy frequency then the distrubance looks like potential flow, as we increase the stratification of a fluid it will decrease the decay rate of that motion as we move away from that boundary. The limiting case is when we reach $\omega = N$ the entire water depth is moving in phase and with the same magnitude as the surface.\\\\
In the case $\omega<N$ we have $m$ is real so we get sinsoidal variations in the vertical direction:
$$
w = w_0 e^{i kx + ly + mz - \omega t} = w_0 e^{ i( \bm k \cdot \bm x - \omega t)}
$$
$$
\phi = \bm k \cdot \bm x - \omega t
$$
so
$$
w = w_0 e^{i\phi}
$$
We want to get an idea of the relationships between the different parameters. To start with consider continuity:
$$
\nabla \cdot \bm u = 0 \implies \frac{\partial u}{\partial x} + \frac{\parital w}{\partial z} = 0
$$
So
$$
u = \int \frac{\partial w_0 e^{i \phi}}{\parital z}dx = - \frac{m}{k} w_0 e^{i \phi} = - \frac{\tan \theta}{ \cos \theta} w_0 e^{i \phi}
$$
Considering surface variation $\eta(x,t) = \tilde \eta(x,t) e^{i \phi}$ therefore by differentiing this and matching with $u$ and $w$ at the surface:
$$
\bm u = \frac{\partial \bm \eta}{\partial t} \implies w_0 = i \omega \cos \theta \tilde \eta
$$
Now considering the relationship arising form the bouyancy equation:
$$
\frac{\parital b}{\partial t} =  -w N^2 \implies i \omega \tilde b = - w N^2 
$$
$$
b = - \eta \frac{\omega^2}{\cos \theta }e^{ i \phi} = - \eta \omega N e^{i \phi}
$$
Now consider the momentum equation:
$$
\frac{\partial u}{\partial t} = - \frac{1}{\rho_0} \frac{\partial p'}{\partial x} \implies \tilde p = i \frac{\omega N}{|\bm k|} \eta \sin \theta
$$
\subsection{Wave velocities}
\textbf{Phase velocity}
$$
\phi = \bm k \cdot \bm x - \omega t = k_i x_i - \omega t
$$
The below identity is very obviously zero:
$$
\frac{\partial \phi}{\partial x_i} \frac{\partial \phi}{\partial t} - \frac{\partial \phi}{\partial t} \frac{\partial \phi}{\partial x_i} = 0
$$
$$
k_i \frac{\partial \phi}{\partial t} + \omega \frac{\partial \phi}{\partial x_i} = 0
$$
Divide across by $k_i$:
$$
\frac{\partial \phi}{\partial t} + \frac{\omega}{|k|^2}k_i \frac{\partial \phi}{\partial x_i} = 0
$$
Therefore, $c_p = \frac{\omega}{|\bm k|^2}\bm k$ as:
$$
\frac{\partial \phi}{\partial t} + (\bm c_p \cdot \nabla) \phi = 0
$$
\section{Lecture 3}
\textbf{Group velocity}:
$$
\frac{\partial^2 \phi}{\partial x_i \partial t} - \frac{\partial^2 \phi}{\partial t \partial x_i} = 0
$$
$$
\frac{\partial k_i}{\partial t} + \frac{\partial \omega}{\partial x_i} = 0
$$
As $\omega = \omega(k)$ we have:
$$
\frac{\partial \omega}{\partial x_i} = \frac{\partial \omega}{\partial k_j} \frac{\partial k_j}{\partial x_i} 
$$
$$
\frac{\partial k_j}{\partial x_i} = \frac{\partial ^2 \phi}{\partial x_j \partial x_j} = \frac{ \partial k_i}{\partial x_j}
$$
Therefore:
$$
\frac{\partial k_i}{\partial t} + \frac{\partial \omega}{\partial k_j} \frac{\partial k_i}{\partial x_j} = 0
$$
Therefore, $c_g = \frac{\partial \omega}{\partial k_i}$ as:
$$
\frac{\partial k_i}{\partial t} + \bm c_g \cdot \nabla k_i  = 0
$$
So the wavenumber vector is being advected outwards with the group velocity.\\
\textbf{Surface waves}: $\omega = gk$, $c_g = \frac{ \partial}{\partial k} \sqrt{ gh} = \frac{1}{2} c_p$, $c_p = \frac{\omega}{k} = \sqrt{ \frac{g}{k}}$.
\subsection{Superposition}
$$
\eta = \cos( (k+ \delta k) x - (\omgea + \delta \omega)t) + \cos ((k- \delta k) x - (\omega - \delta \omega)t)
$$
$$
\eta = 2 \cos( \delta k x - \delta \omega t) \cos (kx - \omega t)
$$
As $\delta \omega = \frac{\partial \omega}{\partial k} \delta k$ for $|\delta k| << |k|$ then
$$
\eta = 2 \cos( (x- \frac{\partial \omega}{\partial k} t) \delta k) \cos (k x - \omega t)
$$
This is a wave and envelope speed of $\frac{\partial \omega}{\partial k}$.
\subsection{Internal wave velocities}
As $\frac{\omega^2}{N^2} = \frac{k^2 + l^2}{|k|^2} = \cos^2 \theta$
$$
\bm c_p = \frac{\omega}{|k|^2} \bm k = \frac{ N( k^2 + l^2)^{\frac{1}{2}}}{|k|^{\frac{3}{2}}} \bm k = \frac{N |\cos \theta|}{|k|^2} \bm k 
$$
This is the polar coordinate equation for two circles touching at the origin at every $\phi$ so they sort of form a torus.\\\\
Now lets look at the group velocity:
$$
c_g = \frac{\partial \omega}{\partial k_i} = \frac{1}{2\omega} \frac{\partial \omega^2}{\partial k_i} = \frac{\omega}{|k|^2} ( \frac{N^2}{\omega^2}(\bm k - k_z \hat{\bm z}) - \bm k) = \frac{N|\sin \theta|}{|k|}\begin{pmatrix}\cos \phi \sin \theta\\ \sin \phi \sin \theta \\\ - \cos \theta \end{pmatrix}
$$
$$
|c_g| = \frac{N}{|k|} |\sin \theta|
$$
This means that in the horizontal direction the phase velocity is always perpendicular to the group velocity. They form the same circle just one with $\sin \theta$ and one with $\cos \theta$ and as $\sin \theta = \cos (\pi/2 - \theta)$. As the angles on a semicircle subtend 90 degrees we can sum the two and we will get the opposite side of the circle always. Therefore,
$$
\bm c_p + \bm c_g = \frac{N}{|\bm k|} \begin{pmatrix} \cos \phi\\ \sin \phi \end{pmatrix}
$$
$$
|\bm c_p + \bm c_g| = \frac{N}{|\bm k|}, c_{p,z} = - c_{g,z}, \bm c_p \cdot \bm c_g = 0
$$
\subsection{Equipartition of energy}
$$
\bm u \cdot (\rho_0 \frac{\partial \bm u}{\partial t} + \nabla p' + \rho' g \bm z) = 0
$$
Recalling that $\frac{\partial \rho'}{\partial t} - w \frac{\rho_0}{g} N^2 = 0 \implies w = \frac{g}{\rho_0 N^2} \frac{\partial \rho'}{\partial t}$, and the incompressibility condition to get:
$$
\frac{\partial}{\partial t}( \frac{1}{2} \rho_0 |\bm u |^2 + \frac{1}{2} \frac{g^2}{\rho_0 N^2} p'^2 ) + \nabla \cdot (p' \bm u) = 0
$$
If we go back to the start and consider the dispalcement of a packet of fluid by $\zeta$ we have change in potential energy of:
$$
\Delta PE = \int^{z_0 + \zeta}_{z_0} g \frac{d \hat \rho}{d z} (z - z_0) dz = \frac{1}{2} \rho_0 N^2 \zeta^2
$$
$$
\rho' = - \frac{d \hat \rho}{dz} \zeta = \frac{\rho_0}{g} N^2 \zeta 
$$
$$
PE = \frac{1}{2} N^2 \rho_0 \zeta^2 = \frac{1}{2} \rho_0 \frac{b^2}{N^2}
$$
If we want to consider the total energy equation:
$$
\int_V \frac{\partial}{\partial t} (KE + PE) dV + \int_S p' \bm u \cdot \bm n dS' = 0
$$
$\bm F_E = p'\bm u$ is the flux of energy.\\\\
Lets consider 2D:
$$
u = \eta \omega \sin \theta \sin \phi, w = - \eta \omega \cos \theta \sin \phi, b = \eta \frac{\omega^2}{\cos \theta} \cos \phi, p' = \eta \rho_0 \frac{\omega^2}{|k|} \tan \theta \sin \phi
$$
Subsitute into kinetic energy:
$$
KE = \frac{1}{2} \rho_0 (u^2 + w^2) = \frac{1}{2} \rho_0 \omega \eta^2 \sin^2 \phi
$$
$$
\bar{KE} = \frac{1}{4} \rho_0 \omega^2 \eta^2
$$
$$
PE = \frac{1}{2} \rho_0 \omega^2 \eta^2 \cos^2\phi
$$
$$
\bar{PE} = \frac{1}{4} \rho_0 \omega^2 \eta^2
$$
Also have:
$$
PE = \frac{1}{2} \rho_0 \frac{b^2}{N^2}
$$
So you have equiparition of energy for linear waves $\bar{KE} = \bar{PE}$. We can also write down an expression for the flux of energy:
$$
\bm F_E = p' \bm u = \rho_0 \omega^2 \eta^2 \sin^2 \phi  \frac{N}{|k|} \sin \theta \begin{pmatrix} \sin \theta\\ - \cos \theta \end{pmatrix} = \frac{1}{2} \rho_0 \omega^2 \eta^2 \bm c_g = \bar{E} \bm c_g
$$
\section{Lecture 4}
\subsection{Oscillating cyclinder}
We are interested in the case where the oscillation of the cylinder $a$ is much smaller than the diameter. You might think this is sufficent to make the waves linear, but it is not as we ahve these delta functions on the singularities on the tangent planes to the cylinder (so we will always have the amplitudes being large compared to the wave lengths here but we will ignore this). If everything is at rest to start with it will take a bit of time for the oscillations to propogate out into the whole space. As $|c_g| = \frac{N}{|\bm k|} \sin \theta$ the area of which is influenced by the oscillation will look like two causality envelopes (circles of increasing size touching at the centre). The waves form a st. andrews cross pattern with the waves being bi modal near the cyclinder and unimodal further way this is due to visocity
\subsubsection{Decay along a beam}
We are going to take it being 2D and the bouunancy frequency $N= 1$ and the mass dispersion $\kappa = 0$ but include visocity $\nu \neq 0$.
$$
\nabla \cdot \bm u = 0
$$
$$
\frac{\partial u}{\partial t} + \frac{1}{\rho_0} \frac{\partial p}{\partial x} = \nu \nabla^2 u
$$
$$
\frac{\partial w}{\partial t} + \frac{1}{\rho_0} \frac{\partial p}{\partial z} -b = \nu \nabla^2 w
$$
$$
\frac{\partial b}{\partial t} + N^2 w = 0
$$
To make our lives easier we are going to use a streamfunction:
$$
\bm \psi = (0, \psi, 0), \bm u = (\nabla \times \psi) e^{- i \omega t} = \begin{pmatrix}-\frac{\partial \psi}{\partial z}\\0 \\ \frac{\partial \psi}{\partial x} \end{pmatrix}e^{-i \omega t}
$$
Take $\zeta$ and $\xi$ to be displacment in the wavevector $k$ direction and then group velocity $c_g$ direction. let $\theta$ be the angle of the group velocity from the vertical which is the asame as the angle of the cross. ThereforeL
$$
\zeta = x \cos \theta - z \sin \theta
$$
$$
\xi = x \sin \theta + z \cos \theta
$$
Boynancy equation becomes:
$$
\frac{\partial b}{\partial t} + \frac{\partial \psi}{\partial \zeta}\cos \theta + \frac{\partial \psi}{\partial \xi} \sin \theta = 0
$$
\textbf{Vorticity}: $\nabla \times \bm u = - \nabla^2 \psi e^{-i \omega t}$
$$
-i \omega ( \frac{\partial ^2 \psi}{\partial \xi^2} + \frac{\partial ^2 \psi}{\partial \zeta^2}) - \frac{\partial b}{\partial \xi} \sin \theta - \frac{\partial b}{\partial \zeta} \cos \theta - \nu \nabla^2 \nabla^2 \psi = 0
$$
Let $b =  (b_0 + \epsilon b_1 +...) e^{-i\omega t}$ and $\psi = (\psi_0 + \epsilon \psi_1 + ...)$:
We take small viscoity to make equations nice e.g. small $\nu = 2 \epsilon$ and $\epsilon = \frac{1}{2} \nu$ as this is dimensional it is not clear what we mean by small parameter. As this is dimensional it means the dimensions of $\psi_0$ and $\psi_1$ are going to be different. This is not the ideal way of doing this but it is going to allow us to see more clearly what is happening.
$$
\chi = \frac{\epsilon}{\sin \theta} \xi
$$
Here $\epsilon$ being small means we are interested in gradual changes in the direction of the group velcoity but fast changes in teh wave vector direction so we keep that sinosdiual behaviour. 
$$
\frac{\partial}{\partial \xi} = \frac{\partial \chi}{\partial \xi} \frac{\partial}{\partial \chi} = \frac{\epsilon}{\sin \theta} \frac{\partial }{\partial \chi}
$$
Plug these into the bouyancy  and vorticity equation and compare terms of the same order:
$$
\epsilon_0:  \frac{\partial \psi_0} {\partial \zeta} = - b_0, \frac{\partial^2 \psi_0}{\partial \zeta^2} = i \frac{\partial b_0}{\partial \zeta}
$$
$$
\epsilon_1: \omega \frac{\psi_1}{\partial \zeta} - i \omega b_1 = - \frac{\partial \psi_0}{\partial \chi}, i \omega \frac{\partial^2 \psi_1}{\partial \zeta^2} + \omega \frac{\partial b_1}{\partial \zeta} = i \frac{\partial^2 \psi_0}{\partial \zeta \partial \chi} - 2 \frac{\partial^4 \psi_0}{\partial \zeta^4}
$$
Can eliminate the LHS of both of these to give:
$$
\frac{\partial^4 \psi_0}{\parital \zeta^4} = i \frac{\partial^2 \psi_0}{\partial \zeta \partial \chi}
$$
$$
\frac{\partial ^3 \psi_0}{\partial \zeta^3} = i \frac{\partial \psi_0}{\partial \chi} + f(\chi)
$$
For a point sources as $|\zeta| \rightarrow \infty$ then we expect $\psi \rightarrow% const. then we expect $\psi \rightarrow $const so $f(\chi) = 0$. Now we can do seperation of variables:
$$
\psi_0 = F(\zeta) G(\chi)
$$
$$
\frac{F''}{F} = i \frac{G'}{G} = - i k^3
$$
$$
G(\chi) = e^{-k^3 \chi}
$$
$$
F( \zeta) = e^{ik \zeta}
$$
$$
\psi \approx \psi_0 e^{-i\omega t} = A e^{-k^3 \chi} e^{i(k \zeta - \omega t)}
$$
$\bm k = (k,0)$ in $(\zeta, \xi)$
$$
\chi= \frac{\epsilon}{\sin \theta} \xi = \frac{\nu }{2\sin \theta} \xi = \frac{\nu}{2N \sin \theta} \xi
$$
for any $N$.
If we take a whole spectrum of linear superposition of waves $A(k)$:
\begin{equation}
\psi = e^{- i \omega t} \int_{- \infty}^{\infty} A(k) \exp( i k \zeta - \frac{\nu k^3}{2N \sin \theta} \xi) dk
\end{equation}
If we think about our cyclinder with the two delta functions at the edge of the cyclinder. As the fourier transform of the delta function is just a constant. So $A(k)$ would just be a constant so the higher wavenumber modes would decay more rapidly due to the $k^3$ in the exponential. We want it to be clear that the length scale over which the day is happening is small compared to the length scale of the oscillations. Lets imagine we are going to scale $\xi$:
$$
\frac{\nu k^2}{2N \sin \theta} k \xi
$$
Recall that $|c_g| = \frac{N}{|k|} \sin \theta$ so
$$
\frac{\nu k^2}{2N \sin \theta} k \xi
 = \frac{ \nu k}{2} \frac{1}{|c_g|} k \xi = - \pi Re^{-1} k \xi
$$
With Reynolds number $Re= \frac{\lambda |c_g|}{\nu}$ with $\lambda = \frac{2\pi}{k}$. We we were requiring $\epsilon$ to be small we were really requiring the reynolds number to be large enough.\subsection{Mass difussivity}
$$
(\frac{\partial}{\partial t} + \bm u \cdot \nabla) \rho = \frac{D\rho}{Dt} = \kappa \nabla^2 \rho
$$
Lets think about what could cause a difference in the density in the fluid e.g. $S$ salt concentration, moisture content or temperature $T$. Carbon dioxide that we are breathing out is denser than the other air we are breathing out, which is more or less balanced by the humidity of the exhaled breath which is higher than the surroundings. Water vapour is less dense than air. The diffusivity of salt and the diffusivity of temperature are different so we can't write down an equation like above. What we can write down is equations of the diffusivity of these two:
$$
\frac{D S}{Dt} = \kappa_s \nabla^2 S, \frac{DT}{Dt} = \kappa_T \nabla^2 T
$$
For water $\frac{\kappa_T}{\kappa_s} \approx 100$. If you move hot salty water down into cold nonsalty water then it quickly becomes cold salty water so it rapidly becomes more dense than the fluid around it. Like wise if you took a parcel of cold fresh up into the hot area it will stay fresh but rapidly heat and so will be less dense than its surroundings and want to rise. THis is called salt fingering. Equally if you have cold fresh water above hot salty water. As tempearture diffuses relativitly quickly the gradient will be much steeper for the salt concentration than the temperature, this creates convection.  This is called double-diffusive convection. \\\\
Prandtl number: $\frac{\nu}{\kappa_T}$\\
Schmidt number: $\frac{\nu}{\kappa_S}$ with $\kappa_S$ is the diffusivity of some thing like salt
\subsection{Reflections of waves}
With light the angle of incidence is equal to the angle of reflection.
\section{Lecture 6}
The thing being conserved on relfection is the wavelength (in the case of light) and therefore the colour. If the speed of light in the medium is constant then the frequency is constant, then the wavelength must also be constant. However, in our internal wave system our frequency $\omega$ is not constant as $|\frac{\omega}{N} | = |\cos \theta|$. The angle to the vertical must be conserved not the angle of incidence and angle of reflection in order for the frequency to be the same at the wall. So only in the case of a horizontal wall is the angle of incidence equal to the angle of reflection.  This difference will mean that the wavelength will not be conserved in general.\\\\
Let the displacement on the incident ray $\eta_i$ and reflected rays $\eta_r$. Therefore in example sheet we show that a slope of angle $\alpha$ to the vertical is: 
$$
|\eta_r| = \gamma |\eta_i|, \gamma = |\frac{\sin(\theta + \alpha)}{\sin(\theta - \alpha)}
$$
This comes about as the same volume of fluid must be displaced but the distance between neighbouring rays is smaller, so the dispalcement must be larger.
\subsubsection{Energy density upon reflection}
$$
|\bm k_r|= \gamma |\bm k_i| \implies \lambda_r = \frac{1}{\gamma} \lambda_i
$$
$$
|\tilde{\bm u}_r | = \gamma |\tilde{\bm u}_i|
$$
$$
|\tilde{\bm \eta}_r | = \gamma |\tilde{\bm \eta}_i|
$$
Recall $|c_g|  = \frac{N}{|\bm k|} \sin \theta$:
$$
|c_{g,r}| = \frac{1}{\gamma} |c_{g,i}|
$$
Flux of energy per wavelength must be preserved on reflection. We have to be careful what we mean by flux of energy as we have two different wavelength so we can talk about flux of energy per wavelength or flux of energy per unit length:\\\\
Energy density per wavelength:
$$
\tilde E = \int_0^{\lamba} PE + KE d \zeta
$$
$$
\tilde {\bm F} = \tilde E \bm c_g
$$
Therefore:
$$
\tilde E_r = \gamma \tilde E_i
$$
As $\tilde E_r \sim \lambda( PE + KE) = \lambda ( \tilde{\eta} \tilde{\eta^*} + \tilde u \tilde u^*) \sim \lambda( |\tilde \eta|^2 + |\tilde u|^2)$ and as $\lambda \sim \frac{1}{\gamma}$ and $|\eta|^2 \sim \gamma^2$ so $\tilde E_r \sim \gamma \tilde E_i$.\\
Energy density per unit length:
$$
\bar E = \frac{1}{\lambda} \int_0^{\lambda} PE + KE = \frac{1}{\lambda} \tilde E
$$
$$
|\tilde F| = \tilde E | c_g | = \lambda \bar E | \bm c_g|
$$
Flux of energy per wavelength was conserved:
$$
|\tilde F_r| = \lambda_r \bar E_r|c_{g_r}| = \frac{1}{\gamma} \lambda_i \bar E_r \frac{1}{\gamma} |c_{g,i}| = |\tilde F_i| = \lambda_i \bar E_i |c_{g,i}|
$$
So 
$$
\bar E_r = \gamma^2 \bar E_i
$$
Again consisten with $\bar E \sim PE + KE \sim |\tilde \eta|^2 + | \tilde u|^2$. The typeset notes have hats for energy desnity and flux per wavelength.\\\\
We also might want to think about the total energy of a whole broad region of a wave coming in and going out:
$$
TE_r = \int_{- \frac{L_r}{2}}^{\frac{L_r}{2}} PE_r + KE_r d \zeta = \gamma^2 \int_{- \frac{1}{\gamma} \frac{L_i}{2}}^{\frac{1}{\gamma}\frac{L_i}{2}} PE_i + KE_i = \gamma TE_i
$$
Spectral energy density $S(k)$ if we have a whole lot of waves of different wavenumbers and want to consider how that spectrum will change:
$$
S_r(k) = \gamma S_i(\frac{K}{\gamma})
$$
\subsubsection{Critical reflection}
Under sub crictical reflection the vertical direction of the propogation reverses upon reflection.\\
Under supercitical reflection the vertical direction of propogation is maintained.\\
On the boundary between these modes is the critical reflection where the reflected wave is along the slope of the wall in both directions so we have $\alpha = \theta$. Shifting slightly one way would lead to the wave reflecting slightly above the upper portion of the wall and slightly the other would lead to the the wave reflection slightly above the lower poriton oft he wave.\\\\
As we approach critical reflection, $\tilde \eta \rightarrow \infty$, $k_r \rightarrow \infty$, dissipation which scales like $k^3$ also tends to infinty. Viscosity becomes important and we get non-linearisties as $k\tilde \eta \rightarrow \infty$ as linear waves require $k \tilde \eta << 1$. Viscosity also plays a role away from critical conditions due to no-slip boundary.
\subsection{Ray tracing}
$$
( \nabla^2 \frac{\partial^2}{\partial t^2} + N^2 \nabla_H^2) w = 0
$$
$$
( \nabla^2 \frac{\partial^2}{\partial t^2} + N^2 \nabla_H^2) \psi = 0
$$
In 2D: $\psi = \tilde \psi(x,z) e^{-i \omega t}:
$$
(N^2 - \omega^2) \frac{\partial^2 \tilde \psi}{\partial x^2} - \omega^2 \frac{\partial^2}{\partial z^2} \tilde \psi = 0
$$
$$
\Lambda^2 = \frac{\omega^2}{N^2- \omega^2}
$$
give this Poincare wave equation
$$
(\frac{\partial^2}{\partial x^2} - \Lambda^2 \frac{\partial^2}{\partial z^2}) \tilde \psi=0
$$
If domain bounded with $\tilde \psi = 0$ on boundary this is an ill-posed problem and we will use ray tracing instead.
\section{Lecture 6}
Lets consider a constant slope at the edge of the ocean with free surface with constant bounancy frequncy. Imagine we have some waves entering the systemm in the body of the fluid 1 wavelength apart. These reflect off the free surface and here they remain exactly the same except for a change of vertical orientation. The reflection off the slope shorterns the wavelength as we considered in the previous section. Therefore, the energy reflection goes up, group vecolity is going down, wavenumber goes down and the velocity gets steeper on each reflection off the slope. In this case the energy is trapped into the courner by a sequence of focusing reflections.\\\\
For large Reynolds number $Re = \frac{|c_g|}{|k| \nu}$ nonlinearities will end up dominating leading to wave breaking, mixing and other frequencies and wave numbers coming out of the system.\\\\
Energy density increases:
$$
\tilde E_{n+1} = \gamma \tilde E_{n} = \gamma^{n+1} \tilde E_0
$$
Steepness increases
$$
|k_{n+1}| \tilde \eta_{n+1} = \gamma^{n+1} k_0 \eta_{0}
$$
\subsection{Reflections from rough topography}
How well do we actually know what is down at the bottom of the ocean?\\\\
Lets imagine we have some idealised rough topography, a sine wave of amplitude $h_0$ and wavelength $\lambda_T = \frac{2 \pi}{k_T}$. We can make use of our ray tracing.\\\\
Draw rays one wavelength apart and you can visually see that the reflected wavelength varies depending on which part of the sine wave they reflect off. This means the spectrum has changed so we can no longer just talk about the wavenumber vector. If we going to try and analyse this we can try some sort of linearisation by considering small amplitude variations in topography with $k_T h_0 <<1$.\\\\
Now lets zoom in and consider what is happening at a close scale (here we aren't actually assuming it is small), we have a wave coming in towards $x_i$, then reflects at $x_0$ and then appears to be leaving from $x_r$. We define $\delta x = x_0 - x_r = x_i - x_0$ as the angle of the reflected and incident wave from the vertical are identical. Take the height of the boundary to be $z = h_0 \sin k_T x$. Define $\beta = \cot \theta$. Now geometrically, $h_0 \sin k_T x_0 = \beta \delta x$. Now consider the amplitude of an incident ray 
$$
\eta_i(x,t) = \tilde \eta_i \sin (k_i x - \omega t), \eta_r(x,t) = -\tilde \eta_i \sin (k_i (x + \frac{2 h_0}{\beta} \sin k_T x ) - \omega t)
$$
For small amplitude $\frac{k_i h_0}{\beta} << 1$:
$$
\eta_r(x,t) = - \tilde \eta_i ( \sin (k_i x - \omega ) +k_i (\frac{2h_0}{\beta}) \sin(k_T x) \cos(k_i x - \omega t)) = - \tilde \eta_i \sin (k_i x - \omega ) - \tilde \eta_i \frac{k_i k_0}{\beta}(-\sin ( (k_i - k_T) x - \omega t) + \sin( (k_i + k_T) x - \omega t))
$$
So we have three different wave numbers in the reflection:$k_i = k_R, k_i - k_T = k_B, k_i + k_T = k_F$.\\\\
If $k_B = k_i - k_T < 0$ then we need to start being quite careful as in order to match the boundary condition it seems like the group velocity seems to be moving backwards from the right. This violates causality and the reason for this is we were making an assumption about the direction of the wavenumber vector assoicated with the reflection. What we actually have is two waves reflected in the forwards direction with group velcoity $c_{gR}, c_{gF}$ and a backwards refelcted one with $c_{gB}$. \\\\
In general, for subcritical reflectino we will end up with a backscatter from rough topography.\\\\
What happens if we have super critical reflection when the topography is steeper than the angle of the incident waves. This leads to a very complex spectrum as neighbouring rays could end up with very different end points as the waves can reflect multiple times off the topography before escaping. This problem is not very analytically tractable.
\subsection{Wave attractors}
Now lets look at smooth boundaries but make it a bit more complex by considering a bounded domain.
\subsubsection{Rectangular basin}
With width $X$ and height $Y$\\\\
If we have an eigenmode, then
$$
\frac{X}{Y} \frac{n}{m} = \tan \theta
$$
$m$ is the number of reflection from top boundary, $n$ is the number of reflections of the left-hand boundary.\\\\
The simplest possible case is take $n=m=X=Y=1$ gives $\tan \theta = 1 \implies \theta = \frac{\pi}{4}$ so you get a rectangle reflecting around the inside of the rectangle.
\end{document}
