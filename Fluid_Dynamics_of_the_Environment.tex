\documentclass{article}
\usepackage[utf8]{inputenc}
\usepackage{bm}
\usepackage{amssymb}
\usepackage{amsmath}
\usepackage{braket}
\usepackage{cancel}
\title{Quantum Information Theory}
\author{oliverobrien111 }
\date{July 2021}

\begin{document}

\maketitle

\section{Lecture 1}
\subsection{Internal waves}
\subsubsection{Minimal maths version}
Imagine we have some sort of stratification that is given by density profile $\hat \rho$. We are going to require that this is smooth and as we will see later and it is only changing the gradient over a length scale that is large compared to the length scale of the waves. Consider a small volume $V$ of fluid of density $\rho_0$ and I will magically displaced it up by a distance of $\zeta$ (assuming it retains volume, density and shape). There will clearly be a buoyancy force $B$, and for small displacements:
$$
B = g V \zeta \frac{d\hat \rho}{d z}  $$
Newton's second law gives
$$
\rho_0 V \ddot \zeta = B = g V \zeta \frac{d\hat \rho}{d z} 
$$
$$
\ddot \zeta + (0 \frac{g}{\rho_0} \frac{d \hat \rho}{dz} ) \zeta = 0
$$
This will clearly have solutions:
$$
\zeta = A \cos Nt + B \sin Nt
$$
where $N$ is the buoyancy (or Brunt-Vaisala frequency) $N = \sqrt{-\frac{g}{\rho_0} \frac{d\hat \rho}{dz}}$\\\\
There is a key ingredient that hasn't been taken into account in this treatment. There hasn't been anything about continuity e.g. how does the fluid get out the way when it falls back down. To highlight this if we displace a long slim slab of fluid instead of a sphere. If this long slim slab is vertical and we displace it upwards then we get exactly the same maths, but we can make it thin enough that we don't need to worry about continuity to the first order. Now what happens if we take a long thin slab of fluid at an angle and move it along itself. Would it fall vertically downards or slide back along itself. It is intuitively hard to fall downwards as this would cause a continuity issue of needing to compress all of the fluid below the line (in order to fall down a lot of fluid needs to be moved past the line which would make a big pressure difference resiting the motion). So the slab will slide back along itself. This will have only displaced each parcle of fluid by $\zeta \cos \theta$ upwards and it will only experience a force of $g \cos \theta$ back along its original path. Therefore this would give:
$$
\ddot \zeta + N^2 \cos^2 \theta \zeta = 0
$$
$$
\ddot \zeta + \omega \zeta = 0
$$
This gives the dispersion relation for internal gravity waves:
$$
|\omega/N| = |\cos \theta|
$$
This logic works if we stack slabs on top of each other and as long as we only move each on by a small amount. They can all oscillate up and down along themselves but have no need to be in the same phase, so you could send a wave though them all perpendicular to the slabs. This would mean each slab is a line of constant phase with energy being transmitted along the slab.\\\\
Now lets think about the $\cos \theta$. If $\bm k = (k,l,m)$ is a vector perpendicular to the slabs, then we have:
$$
\cos \theta = \frac{\sqrt{k^2 + l^2}}{\sqrt{k^2 + l^2 + m^2}}
$$
\subsubsection{More rigorous derivation}}
$$
\nabla \cdot \bm u = 0
$$
$$
(\frac{\partial}{\partial t} + \bm u \cdot \nabla) \rho = \kappa \nabla^2 \rho
$$
For now $\kappa = 0$:
$$
\rho \frac{ \partial \bm u}{\partial t} + \rho (\bm u \cdot \nabla) \bm u = - \nabla p - \rho g \bm \hat z + \rho \nu \nabla^2 \bm u
$$
For now $\nu = 0$. Take $\rho = \rho_0 + \rho'$. Linearise with $\rho' << \rho_0$ and Boussinesq $|\frac{\nabla \bm u}{\nabla t} | << g$. The Boussieneq means that the $\rho'$ contrbutes to the gravity term but not to the interial term. We can rewrite the equations using reduced gravity to make this clear $g' = g \frac{\rho - \rho_0}{\rho} = g \frac{\rho'}{\rho}$:
$$
\\frac{ \partial \bm u}{\partial t} + (\bm u \cdot \nabla) \bm u = - \frac{1}{\rho_0} \nabla(p + \rho g z) - g' \hat{\bm z}
$$
To linearise we take$\rho = \hat \rho(z) + \rho'(x,t)$ and  $\bm u' \sim \eta \omega << \frac{\omega}{|k|} \implies |k| \eta << 1$. By combining this with our Boussineqs condition we have $|\nabla \rho'| << | \frac{d\hat \rho}{d z}|$. This gives the navier stokes equation:
$$
\frac{\partial \rho'}{\partial t} + w \frac{d\hat \rho}{dz} = \frac{\partial \rho'}{\partial t} - w \frac{\rho_0}{g} N^2 = 0
$$
$$
\frac{\partial \bm u}{\partial t} = - \frac{1}{\rho_0} \nabla( p_0 + \hat p - p') - \frac{\hat \rho + \rho'}{\rho_0} \bm z =  - \frac{1}{\rho_0} \nabla (p_0 + \hat p) - g \frac{\hat \rho}{\rho_0} \hat{\bm z} - \frac{1}{\rho_0} \nabla p' - g \frac{\rho'}{\rho_0} \hat{\bm z}
$$
Unperturbed state gives the first two terms as they are much larger than the other terms
$$
0 = - \frac{1}{\rho_0} \nabla (p_0 + \hat p) - g \frac{\hat \rho}{\rho_0} \hat{\bm z}
$$
therefore
$$
p_0 + \hat p = - \int g \hat \rho dz
$$
Let
$$
b = - g \frac{\rho'}{\rho_0} 
$$
Therefore:
\begin{equation}
\frac{\partial b}{\partial t} = - w N^2
\end{equation}
\begin{equation}
        \frac{\partial \bm u}{\partial t} = - \frac{1}{\rho_0} \nabla p' + b \hat{\bm z}
\end{equation}
\begin{equation}
        \nabla \cdot \bm u = 0
\end{equation}
\subsection{Vorticity}: $\bm \zeta = \nabla \times \bm u$
We are going to deal in 2D as it is easier. In 2-D vorictity can be expressed interms of the streamfunction:
$$
\zeta = - \nabla^2 \psi, \bm \psi = \psi \hat{\bm y}, \bm u = \nabla \times \bm \psi
$$
Take curl of momentum equation to remove pressure:
$$
\frac{\partial \bm \zeta}{\partial t} = - \hat{\bm z} \times \nabla b
$$
$$
(\hat{\bm z} \times \nabla) \cdot (\hat{\bm z} \cdot \nabla w) = \nabla^2_H w
$$
so gives voriticity equation:
\begin{equation}
        (\nabla^2 \frac{\partial^2}{\partial t^2} + N^2 \nabla^2_H) w = 0
\end{equation}
Pose a plane wave type solution ansatz and see what happens. Let :
$$
w(\bm x, t) = Re( \hat w(z) e^{i(kx + ly - \omega t)})
$$
$$
\frac{d^2 \hat w}{dz^2} + (k^2 + l^2) (\frac{N^2}{\omega^2} -1) \hat w = 0
$$
So if we let $m^2 = (k^2 + l^2) ( \frac{N^2}{\omega^2}-1)$ then:
$$
\hat w = Re( A e^{imz} + Be^{-imz}
$$
If $\omega > N$ then $m$ is imaginary let $\gamma = \sqrt{1- \frac{N^2}{\omega^2}}$:
$$
w = (\hat A e^{- \gamma k_h z + \hat B e^{\gamma k_h z})e^{i(k x + ly - \omega t)}
$$
This is sort of showing how the velocity field changes with depth away from a distribance on the surface. If we have $N=0$ then this is just a surface wave. If we have $0<N<\omega$ the $\gamma$ is just giving a vertical rescaling of the behaviour beneath the surface wave. This means if we produce a sinosidal distributance with a frequency bigger than the bouyancy frequency then the distrubance looks like potential flow, as we increase the stratification of a fluid it will decrease the decay rate of that motion as we move away from that boundary. The limiting case is when we reach $\omega = N$ the entire water depth is moving in phase and with the same magnitude as the surface.\\\\
In the case $\omega<N$ we have $m$ is real so we get sinsoidal variations in the vertical direction:
$$
w = w_0 e^{i kx + ly + mz - \omega t} = w_0 e^{ i( \bm k \cdot \bm x - \omega t)}
$$
$$
\phi = \bm k \cdot \bm x - \omega t
$$
so
$$
w = w_0 e^{i\phi}
$$
We want to get an idea of the relationships between the different parameters. To start with consider continuity:
$$
\nabla \cdot \bm u = 0 \implies \frac{\partial u}{\partial x} + \frac{\parital w}{\partial z} = 0
$$
So
$$
u = \int \frac{\partial w_0 e^{i \phi}}{\parital z}dx = - \frac{m}{k} w_0 e^{i \phi} = - \frac{\tan \theta}{ \cos \theta} w_0 e^{i \phi}
$$
Considering surface variation $\eta(x,t) = \tilde \eta(x,t) e^{i \phi}$ therefore by differentiing this and matching with $u$ and $w$ at the surface:
$$
\bm u = \frac{\partial \bm \eta}{\partial t} \implies w_0 = i \omega \cos \theta \tilde \eta
$$
Now considering the relationship arising form the bouyancy equation:
$$
\frac{\parital b}{\partial t} =  -w N^2 \implies i \omega \tilde b = - w N^2 
$$
$$
b = - \eta \frac{\omega^2}{\cos \theta }e^{ i \phi} = - \eta \omega N e^{i \phi}
$$
Now consider the momentum equation:
$$
\frac{\partial u}{\partial t} = - \frac{1}{\rho_0} \frac{\partial p'}{\partial x} \implies \tilde p = i \frac{\omega N}{|\bm k|} \eta \sin \theta
$$
\subsection{Wave velocities}
\textbf{Phase velocity}
$$
\phi = \bm k \cdot \bm x - \omega t = k_i x_i - \omega t
$$
The below identity is very obviously zero:
$$
\frac{\partial \phi}{\partial x_i} \frac{\partial \phi}{\partial t} - \frac{\partial \phi}{\partial t} \frac{\partial \phi}{\partial x_i} = 0
$$
$$
k_i \frac{\partial \phi}{\partial t} + \omega \frac{\partial \phi}{\partial x_i} = 0
$$
Divide across by $k_i$:
$$
\frac{\partial \phi}{\partial t} + \frac{\omega}{|k|^2}k_i \frac{\partial \phi}{\partial x_i} = 0
$$
Therefore, $c_p = \frac{\omega}{|\bm k|^2}\bm k$ as:
$$
\frac{\partial \phi}{\partial t} + (\bm c_p \cdot \nabla) \phi = 0
$$
\section{Lecture 3}
\textbf{Group velocity}:
$$
\frac{\partial^2 \phi}{\partial x_i \partial t} - \frac{\partial^2 \phi}{\partial t \partial x_i} = 0
$$
$$
\frac{\partial k_i}{\partial t} + \frac{\partial \omega}{\partial x_i} = 0
$$
As $\omega = \omega(k)$ we have:
$$
\frac{\partial \omega}{\partial x_i} = \frac{\partial \omega}{\partial k_j} \frac{\partial k_j}{\partial x_i} 
$$
$$
\frac{\partial k_j}{\partial x_i} = \frac{\partial ^2 \phi}{\partial x_j \partial x_j} = \frac{ \partial k_i}{\partial x_j}
$$
Therefore:
$$
\frac{\partial k_i}{\partial t} + \frac{\partial \omega}{\partial k_j} \frac{\partial k_i}{\partial x_j} = 0
$$
Therefore, $c_g = \frac{\partial \omega}{\partial k_i}$ as:
$$
\frac{\partial k_i}{\partial t} + \bm c_g \cdot \nabla k_i  = 0
$$
So the wavenumber vector is being advected outwards with the group velocity.\\
\textbf{Surface waves}: $\omega = gk$, $c_g = \frac{ \partial}{\partial k} \sqrt{ gh} = \frac{1}{2} c_p$, $c_p = \frac{\omega}{k} = \sqrt{ \frac{g}{k}}$.
\subsection{Superposition}
$$
\eta = \cos( (k+ \delta k) x - (\omgea + \delta \omega)t) + \cos ((k- \delta k) x - (\omega - \delta \omega)t)
$$
$$
\eta = 2 \cos( \delta k x - \delta \omega t) \cos (kx - \omega t)
$$
As $\delta \omega = \frac{\partial \omega}{\partial k} \delta k$ for $|\delta k| << |k|$ then
$$
\eta = 2 \cos( (x- \frac{\partial \omega}{\partial k} t) \delta k) \cos (k x - \omega t)
$$
This is a wave and envelope speed of $\frac{\partial \omega}{\partial k}$.
\subsection{Internal wave velocities}
As $\frac{\omega^2}{N^2} = \frac{k^2 + l^2}{|k|^2} = \cos^2 \theta$
$$
\bm c_p = \frac{\omega}{|k|^2} \bm k = \frac{ N( k^2 + l^2)^{\frac{1}{2}}}{|k|^{\frac{3}{2}}} \bm k = \frac{N |\cos \theta|}{|k|^2} \bm k 
$$
This is the polar coordinate equation for two circles touching at the origin at every $\phi$ so they sort of form a torus.\\\\
Now lets look at the group velocity:
$$
c_g = \frac{\partial \omega}{\partial k_i} = \frac{1}{2\omega} \frac{\partial \omega^2}{\partial k_i} = \frac{\omega}{|k|^2} ( \frac{N^2}{\omega^2}(\bm k - k_z \hat{\bm z}) - \bm k) = \frac{N|\sin \theta|}{|k|}\begin{pmatrix}\cos \phi \sin \theta\\ \sin \phi \sin \theta \\\ - \cos \theta \end{pmatrix}
$$
$$
|c_g| = \frac{N}{|k|} |\sin \theta|
$$
This means that in the horizontal direction the phase velocity is always perpendicular to the group velocity. They form the same circle just one with $\sin \theta$ and one with $\cos \theta$ and as $\sin \theta = \cos (\pi/2 - \theta)$. As the angles on a semicircle subtend 90 degrees we can sum the two and we will get the opposite side of the circle always. Therefore,
$$
\bm c_p + \bm c_g = \frac{N}{|\bm k|} \begin{pmatrix} \cos \phi\\ \sin \phi \end{pmatrix}
$$
$$
|\bm c_p + \bm c_g| = \frac{N}{|\bm k|}, c_{p,z} = - c_{g,z}, \bm c_p \cdot \bm c_g = 0
$$
\subsection{Equipartition of energy}
$$
\bm u \cdot (\rho_0 \frac{\partial \bm u}{\partial t} + \nabla p' + \rho' g \bm z) = 0
$$
Recalling that $\frac{\partial \rho'}{\partial t} - w \frac{\rho_0}{g} N^2 = 0 \implies w = \frac{g}{\rho_0 N^2} \frac{\partial \rho'}{\partial t}$, and the incompressibility condition to get:
$$
\frac{\partial}{\partial t}( \frac{1}{2} \rho_0 |\bm u |^2 + \frac{1}{2} \frac{g^2}{\rho_0 N^2} p'^2 ) + \nabla \cdot (p' \bm u) = 0
$$
If we go back to the start and consider the dispalcement of a packet of fluid by $\zeta$ we have change in potential energy of:
$$
\Delta PE = \int^{z_0 + \zeta}_{z_0} g \frac{d \hat \rho}{d z} (z - z_0) dz = \frac{1}{2} \rho_0 N^2 \zeta^2
$$
$$
\rho' = - \frac{d \hat \rho}{dz} \zeta = \frac{\rho_0}{g} N^2 \zeta 
$$
$$
PE = \frac{1}{2} N^2 \rho_0 \zeta^2 = \frac{1}{2} \rho_0 \frac{b^2}{N^2}
$$
If we want to consider the total energy equation:
$$
\int_V \frac{\partial}{\partial t} (KE + PE) dV + \int_S p' \bm u \cdot \bm n dS' = 0
$$
$\bm F_E = p'\bm u$ is the flux of energy.\\\\
Lets consider 2D:
$$
u = \eta \omega \sin \theta \sin \phi, w = - \eta \omega \cos \theta \sin \phi, b = \eta \frac{\omega^2}{\cos \theta} \cos \phi, p' = \eta \rho_0 \frac{\omega^2}{|k|} \tan \theta \sin \phi
$$
Subsitute into kinetic energy:
$$
KE = \frac{1}{2} \rho_0 (u^2 + w^2) = \frac{1}{2} \rho_0 \omega \eta^2 \sin^2 \phi
$$
$$
\bar{KE} = \frac{1}{4} \rho_0 \omega^2 \eta^2
$$
$$
PE = \frac{1}{2} \rho_0 \omega^2 \eta^2 \cos^2\phi
$$
$$
\bar{PE} = \frac{1}{4} \rho_0 \omega^2 \eta^2
$$
So you have equiparition of energy for linear waves $\bar{KE} = \bar{PE}$. We can also write down an expression for the flux of energy:
$$
\bm F_E = p' \bm u = \rho_0 \omega^2 \eta^2 \sin^2 \phi  \frac{N}{|k|} \sin \theta \begin{pmatrix} \sin \theta\\ - \cos \theta \end{pmatrix} = \frac{1}{2} \rho_0 \omega^2 \eta^2 \bm c_g = \bar{E} \bm c_g
$$
\end{document}
