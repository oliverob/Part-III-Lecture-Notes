\documentclass{article}
\usepackage[utf8]{inputenc}
\usepackage{bm}
\usepackage{amssymb}
\usepackage{amsmath}
\usepackage{braket}
\usepackage{cancel}
\title{Quantum Information Theory}
\author{oliverobrien111 }
\date{July 2021}

\begin{document}

\maketitle

\section{Lecture 1}
\subsection{Internal waves}
\subsubsection{Minimal maths version}
Imagine we have some sort of stratification that is given by density profile $\hat \rho$. We are going to require that this is smooth and as we will see later and it is only changing the gradient over a length scale that is large compared to the length scale of the waves. Consider a small volume $V$ of fluid of density $\rho_0$ and I will magically displaced it up by a distance of $\zeta$ (assuming it retains volume, density and shape). There will clearly be a buoyancy force $B$, and for small displacements:
$$
B = g V \zeta \frac{d\hat \rho}{d z}  $$
Newton's second law gives
$$
\rho_0 V \ddot \zeta = B = g V \zeta \frac{d\hat \rho}{d z} 
$$
$$
\ddot \zeta + (0 \frac{g}{\rho_0} \frac{d \hat \rho}{dz} ) \zeta = 0
$$
This will clearly have solutions:
$$
\zeta = A \cos Nt + B \sin Nt
$$
where $N$ is the buoyancy (or Brunt-Vaisala frequency) $N = \sqrt{-\frac{g}{\rho_0} \frac{d\hat \rho}{dz}}$\\\\
There is a key ingredient that hasn't been taken into account in this treatment. There hasn't been anything about continuity e.g. how does the fluid get out the way when it falls back down. To highlight this if we displace a long slim slab of fluid instead of a sphere. If this long slim slab is vertical and we displace it upwards then we get exactly the same maths, but we can make it thin enough that we don't need to worry about continuity to the first order. Now what happens if we take a long thin slab of fluid at an angle and move it along itself. Would it fall vertically downards or slide back along itself. It is intuitively hard to fall downwards as this would cause a continuity issue of needing to compress all of the fluid below the line (in order to fall down a lot of fluid needs to be moved past the line which would make a big pressure difference resiting the motion). So the slab will slide back along itself. This will have only displaced each parcle of fluid by $\zeta \cos \theta$ upwards and it will only experience a force of $g \cos \theta$ back along its original path. Therefore this would give:
$$
\ddot \zeta + N^2 \cos^2 \theta \zeta = 0
$$
$$
\ddot \zeta + \omega \zeta = 0
$$
This gives the dispersion relation for internal gravity waves:
$$
|\omega/N| = |\cos \theta|
$$
This logic works if we stack slabs on top of each other and as long as we only move each on by a small amount. They can all oscillate up and down along themselves but have no need to be in the same phase, so you could send a wave though them all perpendicular to the slabs. This would mean each slab is a line of constant phase with energy being transmitted along the slab.\\\\
Now lets think about the $\cos \theta$. If $\bm k = (k,l,m)$ is a vector perpendicular to the slabs, then we have:
$$
\cos \theta = \frac{\sqrt{k^2 + l^2}}{\sqrt{k^2 + l^2 + m^2}}
$$
\subsubsection{More rigorous derivation}}
$$
\nabla \cdot \bm u = 0
$$
$$
(\frac{\partial}{\partial t} + \bm u \cdot \nabla) \rho = \kappa \nabla^2 \rho
$$
For now $\kappa = 0$:
$$
\rho \frac{ \partial \bm u}{\partial t} + \rho (\bm u \cdot \nabla) \bm u = - \nabla p - \rho g \bm \hat z + \rho \nu \nabla^2 \bm u
$$
For now $\nu = 0$. Take $\rho = \rho_0 + \rho'$. Linearise with $\rho' << \rho_0$ and Boussinesq $|\frac{\nabla \bm u}{\nabla t} | << g$. The Boussieneq means that the $\rho'$ contrbutes to the gravity term but not to the interial term. We can rewrite the equations using reduced gravity to make this clear $g' = g \frac{\rho - \rho_0}{\rho} = g \frac{\rho'}{\rho}$:
$$
\\frac{ \partial \bm u}{\partial t} + (\bm u \cdot \nabla) \bm u = - \frac{1}{\rho_0} \nabla(p + \rho g z) - g' \hat{\bm z}
$$
To linearise we take$\rho = \hat \rho(z) + \rho'(x,t)$ and  $\bm u' \sim \eta \omega << \frac{\omega}{|k|} \implies |k| \eta << 1$. By combining this with our Boussineqs condition we have $|\nabla \rho'| << | \frac{d\hat \rho}{d z}|$. This gives the navier stokes equation:
$$
\frac{\partial \rho'}{\partial t} + w \frac{d\hat \rho}{dz} = \frac{\partial \rho'}{\partial t} - w \frac{\rho_0}{g} N^2 = 0
$$
$$
\frac{\partial \bm u}{\partial t} = - \frac{1}{\rho_0} \nabla( p_0 + \hat p - p') - \frac{\hat \rho + \rho'}{\rho_0} \bm z =  - \frac{1}{\rho_0} \nabla (p_0 + \hat p) - g \frac{\hat \rho}{\rho_0} \hat{\bm z} - \frac{1}{\rho_0} \nabla p' - g \frac{\rho'}{\rho_0} \hat{\bm z}
$$
Unperturbed state gives the first two terms as they are much larger than the other terms
$$
0 = - \frac{1}{\rho_0} \nabla (p_0 + \hat p) - g \frac{\hat \rho}{\rho_0} \hat{\bm z}
$$
therefore
$$
p_0 + \hat p = - \int g \hat \rho dz
$$
Let
$$
b = - g \frac{\rho'}{\rho_0} 
$$
Therefore:
\begin{equation}
\frac{\partial b}{\partial t} = - w N^2
\end{equation}
\begin{equation}
        \frac{\partial \bm u}{\partial t} = - \frac{1}{\rho_0} \nabla p' + b \hat{\bm z}
\end{equation}
\begin{equation}
        \nabla \cdot \bm u = 0
\end{equation}
\subsection{Vorticity}: $\bm \zeta = \nabla \times \bm u$
We are going to deal in 2D as it is easier. In 2-D vorictity can be expressed interms of the streamfunction:
$$
\zeta = - \nabla^2 \psi, \bm \psi = \psi \hat{\bm y}, \bm u = \nabla \times \bm \psi
$$
Take curl of momentum equation to remove pressure:
$$
\frac{\partial \bm \zeta}{\partial t} = - \hat{\bm z} \times \nabla b
$$
$$
(\hat{\bm z} \times \nabla) \cdot (\hat{\bm z} \cdot \nabla w) = \nabla^2_H w
$$
so gives voriticity equation:
\begin{equation}
        (\nabla^2 \frac{\partial^2}{\partial t^2} + N^2 \nabla^2_H) w = 0
\end{equation}
Pose a plane wave type solution ansatz and see what happens. Let :
$$
w(\bm x, t) = Re( \hat w(z) e^{i(kx + ly - \omega t)})
$$
$$
\frac{d^2 \hat w}{dz^2} + (k^2 + l^2) (\frac{N^2}{\omega^2} -1) \hat w = 0
$$
So if we let $m^2 = (k^2 + l^2) ( \frac{N^2}{\omega^2}-1)$ then:
$$
\hat w = Re( A e^{imz} + Be^{-imz}
$$
If $\omega > N$ then $m$ is imaginary let $\gamma = \sqrt{1- \frac{N^2}{\omega^2}}$:
$$
w = (\hat A e^{- \gamma k_h z + \hat B e^{\gamma k_h z})e^{i(k x + ly - \omega t)}
$$
This is sort of showing how the velocity field changes with depth away from a distribance on the surface. If we have $N=0$ then this is just a surface wave. If we have $0<N<\omega$ the $\gamma$ is just giving a vertical rescaling of the behaviour beneath the surface wave. This means if we produce a sinosidal distributance with a frequency bigger than the bouyancy frequency then the distrubance looks like potential flow, as we increase the stratification of a fluid it will decrease the decay rate of that motion as we move away from that boundary. The limiting case is when we reach $\omega = N$ the entire water depth is moving in phase and with the same magnitude as the surface.\\\\
In the case $\omega<N$ we have $m$ is real so we get sinsoidal variations in the vertical direction:
$$
w = w_0 e^{i kx + ly + mz - \omega t} = w_0 e^{ i( \bm k \cdot \bm x - \omega t)}
$$
$$
\phi = \bm k \cdot \bm x - \omega t
$$
so
$$
w = w_0 e^{i\phi}
$$
We want to get an idea of the relationships between the different parameters. To start with consider continuity:
$$
\nabla \cdot \bm u = 0 \implies \frac{\partial u}{\partial x} + \frac{\parital w}{\partial z} = 0
$$
So
$$
u = \int \frac{\partial w_0 e^{i \phi}}{\parital z}dx = - \frac{m}{k} w_0 e^{i \phi} = - \frac{\tan \theta}{ \cos \theta} w_0 e^{i \phi}
$$
Considering surface variation $\eta(x,t) = \tilde \eta(x,t) e^{i \phi}$ therefore by differentiing this and matching with $u$ and $w$ at the surface:
$$
\bm u = \frac{\partial \bm \eta}{\partial t} \implies w_0 = i \omega \cos \theta \tilde \eta
$$
Now considering the relationship arising form the bouyancy equation:
$$
\frac{\parital b}{\partial t} =  -w N^2 \implies i \omega \tilde b = - w N^2 
$$
$$
b = - \eta \frac{\omega^2}{\cos \theta }e^{ i \phi} = - \eta \omega N e^{i \phi}
$$
Now consider the momentum equation:
$$
\frac{\partial u}{\partial t} = - \frac{1}{\rho_0} \frac{\partial p'}{\partial x} \implies \tilde p = i \frac{\omega N}{|\bm k|} \eta \sin \theta
$$
\subsection{Wave velocities}
\textbf{Phase velocity}
$$
\phi = \bm k \cdot \bm x - \omega t = k_i x_i - \omega t
$$
The below identity is very obviously zero:
$$
\frac{\partial \phi}{\partial x_i} \frac{\partial \phi}{\partial t} - \frac{\partial \phi}{\partial t} \frac{\partial \phi}{\partial x_i} = 0
$$
$$
k_i \frac{\partial \phi}{\partial t} + \omega \frac{\partial \phi}{\partial x_i} = 0
$$
Divide across by $k_i$:
$$
\frac{\partial \phi}{\partial t} + \frac{\omega}{|k|^2}k_i \frac{\partial \phi}{\partial x_i} = 0
$$
Therefore, $c_p = \frac{\omega}{|\bm k|^2}\bm k$ as:
$$
\frac{\partial \phi}{\partial t} + (\bm c_p \cdot \nabla) \phi = 0
$$
\section{Lecture 3}
\textbf{Group velocity}:
$$
\frac{\partial^2 \phi}{\partial x_i \partial t} - \frac{\partial^2 \phi}{\partial t \partial x_i} = 0
$$
$$
\frac{\partial k_i}{\partial t} + \frac{\partial \omega}{\partial x_i} = 0
$$
As $\omega = \omega(k)$ we have:
$$
\frac{\partial \omega}{\partial x_i} = \frac{\partial \omega}{\partial k_j} \frac{\partial k_j}{\partial x_i} 
$$
$$
\frac{\partial k_j}{\partial x_i} = \frac{\partial ^2 \phi}{\partial x_j \partial x_j} = \frac{ \partial k_i}{\partial x_j}
$$
Therefore:
$$
\frac{\partial k_i}{\partial t} + \frac{\partial \omega}{\partial k_j} \frac{\partial k_i}{\partial x_j} = 0
$$
Therefore, $c_g = \frac{\partial \omega}{\partial k_i}$ as:
$$
\frac{\partial k_i}{\partial t} + \bm c_g \cdot \nabla k_i  = 0
$$
So the wavenumber vector is being advected outwards with the group velocity.\\
\textbf{Surface waves}: $\omega = gk$, $c_g = \frac{ \partial}{\partial k} \sqrt{ gh} = \frac{1}{2} c_p$, $c_p = \frac{\omega}{k} = \sqrt{ \frac{g}{k}}$.
\subsection{Superposition}
$$
\eta = \cos( (k+ \delta k) x - (\omgea + \delta \omega)t) + \cos ((k- \delta k) x - (\omega - \delta \omega)t)
$$
$$
\eta = 2 \cos( \delta k x - \delta \omega t) \cos (kx - \omega t)
$$
As $\delta \omega = \frac{\partial \omega}{\partial k} \delta k$ for $|\delta k| << |k|$ then
$$
\eta = 2 \cos( (x- \frac{\partial \omega}{\partial k} t) \delta k) \cos (k x - \omega t)
$$
This is a wave and envelope speed of $\frac{\partial \omega}{\partial k}$.
\subsection{Internal wave velocities}
As $\frac{\omega^2}{N^2} = \frac{k^2 + l^2}{|k|^2} = \cos^2 \theta$
$$
\bm c_p = \frac{\omega}{|k|^2} \bm k = \frac{ N( k^2 + l^2)^{\frac{1}{2}}}{|k|^{\frac{3}{2}}} \bm k = \frac{N |\cos \theta|}{|k|^2} \bm k 
$$
This is the polar coordinate equation for two circles touching at the origin at every $\phi$ so they sort of form a torus.\\\\
Now lets look at the group velocity:
$$
c_g = \frac{\partial \omega}{\partial k_i} = \frac{1}{2\omega} \frac{\partial \omega^2}{\partial k_i} = \frac{\omega}{|k|^2} ( \frac{N^2}{\omega^2}(\bm k - k_z \hat{\bm z}) - \bm k) = \frac{N|\sin \theta|}{|k|}\begin{pmatrix}\cos \phi \sin \theta\\ \sin \phi \sin \theta \\\ - \cos \theta \end{pmatrix}
$$
$$
|c_g| = \frac{N}{|k|} |\sin \theta|
$$
This means that in the horizontal direction the phase velocity is always perpendicular to the group velocity. They form the same circle just one with $\sin \theta$ and one with $\cos \theta$ and as $\sin \theta = \cos (\pi/2 - \theta)$. As the angles on a semicircle subtend 90 degrees we can sum the two and we will get the opposite side of the circle always. Therefore,
$$
\bm c_p + \bm c_g = \frac{N}{|\bm k|} \begin{pmatrix} \cos \phi\\ \sin \phi \end{pmatrix}
$$
$$
|\bm c_p + \bm c_g| = \frac{N}{|\bm k|}, c_{p,z} = - c_{g,z}, \bm c_p \cdot \bm c_g = 0
$$
\subsection{Equipartition of energy}
$$
\bm u \cdot (\rho_0 \frac{\partial \bm u}{\partial t} + \nabla p' + \rho' g \bm z) = 0
$$
Recalling that $\frac{\partial \rho'}{\partial t} - w \frac{\rho_0}{g} N^2 = 0 \implies w = \frac{g}{\rho_0 N^2} \frac{\partial \rho'}{\partial t}$, and the incompressibility condition to get:
$$
\frac{\partial}{\partial t}( \frac{1}{2} \rho_0 |\bm u |^2 + \frac{1}{2} \frac{g^2}{\rho_0 N^2} p'^2 ) + \nabla \cdot (p' \bm u) = 0
$$
If we go back to the start and consider the dispalcement of a packet of fluid by $\zeta$ we have change in potential energy of:
$$
\Delta PE = \int^{z_0 + \zeta}_{z_0} g \frac{d \hat \rho}{d z} (z - z_0) dz = \frac{1}{2} \rho_0 N^2 \zeta^2
$$
$$
\rho' = - \frac{d \hat \rho}{dz} \zeta = \frac{\rho_0}{g} N^2 \zeta 
$$
$$
PE = \frac{1}{2} N^2 \rho_0 \zeta^2 = \frac{1}{2} \rho_0 \frac{b^2}{N^2}
$$
If we want to consider the total energy equation:
$$
\int_V \frac{\partial}{\partial t} (KE + PE) dV + \int_S p' \bm u \cdot \bm n dS' = 0
$$
$\bm F_E = p'\bm u$ is the flux of energy.\\\\
Lets consider 2D:
$$
u = \eta \omega \sin \theta \sin \phi, w = - \eta \omega \cos \theta \sin \phi, b = \eta \frac{\omega^2}{\cos \theta} \cos \phi, p' = \eta \rho_0 \frac{\omega^2}{|k|} \tan \theta \sin \phi
$$
Subsitute into kinetic energy:
$$
KE = \frac{1}{2} \rho_0 (u^2 + w^2) = \frac{1}{2} \rho_0 \omega \eta^2 \sin^2 \phi
$$
$$
\bar{KE} = \frac{1}{4} \rho_0 \omega^2 \eta^2
$$
$$
PE = \frac{1}{2} \rho_0 \omega^2 \eta^2 \cos^2\phi
$$
$$
\bar{PE} = \frac{1}{4} \rho_0 \omega^2 \eta^2
$$
Also have:
$$
PE = \frac{1}{2} \rho_0 \frac{b^2}{N^2}
$$
So you have equiparition of energy for linear waves $\bar{KE} = \bar{PE}$. We can also write down an expression for the flux of energy:
$$
\bm F_E = p' \bm u = \rho_0 \omega^2 \eta^2 \sin^2 \phi  \frac{N}{|k|} \sin \theta \begin{pmatrix} \sin \theta\\ - \cos \theta \end{pmatrix} = \frac{1}{2} \rho_0 \omega^2 \eta^2 \bm c_g = \bar{E} \bm c_g
$$
\section{Lecture 4}
\subsection{Oscillating cyclinder}
We are interested in the case where the oscillation of the cylinder $a$ is much smaller than the diameter. You might think this is sufficent to make the waves linear, but it is not as we ahve these delta functions on the singularities on the tangent planes to the cylinder (so we will always have the amplitudes being large compared to the wave lengths here but we will ignore this). If everything is at rest to start with it will take a bit of time for the oscillations to propogate out into the whole space. As $|c_g| = \frac{N}{|\bm k|} \sin \theta$ the area of which is influenced by the oscillation will look like two causality envelopes (circles of increasing size touching at the centre). The waves form a st. andrews cross pattern with the waves being bi modal near the cyclinder and unimodal further way this is due to visocity
\subsubsection{Decay along a beam}
We are going to take it being 2D and the bouunancy frequency $N= 1$ and the mass dispersion $\kappa = 0$ but include visocity $\nu \neq 0$.
$$
\nabla \cdot \bm u = 0
$$
$$
\frac{\partial u}{\partial t} + \frac{1}{\rho_0} \frac{\partial p}{\partial x} = \nu \nabla^2 u
$$
$$
\frac{\partial w}{\partial t} + \frac{1}{\rho_0} \frac{\partial p}{\partial z} -b = \nu \nabla^2 w
$$
$$
\frac{\partial b}{\partial t} + N^2 w = 0
$$
To make our lives easier we are going to use a streamfunction:
$$
\bm \psi = (0, \psi, 0), \bm u = (\nabla \times \psi) e^{- i \omega t} = \begin{pmatrix}-\frac{\partial \psi}{\partial z}\\0 \\ \frac{\partial \psi}{\partial x} \end{pmatrix}e^{-i \omega t}
$$
Take $\zeta$ and $\xi$ to be displacment in the wavevector $k$ direction and then group velocity $c_g$ direction. let $\theta$ be the angle of the group velocity from the vertical which is the asame as the angle of the cross. ThereforeL
$$
\zeta = x \cos \theta - z \sin \theta
$$
$$
\xi = x \sin \theta + z \cos \theta
$$
Boynancy equation becomes:
$$
\frac{\partial b}{\partial t} + \frac{\partial \psi}{\partial \zeta}\cos \theta + \frac{\partial \psi}{\partial \xi} \sin \theta = 0
$$
\textbf{Vorticity}: $\nabla \times \bm u = - \nabla^2 \psi e^{-i \omega t}$
$$
-i \omega ( \frac{\partial ^2 \psi}{\partial \xi^2} + \frac{\partial ^2 \psi}{\partial \zeta^2}) - \frac{\partial b}{\partial \xi} \sin \theta - \frac{\partial b}{\partial \zeta} \cos \theta - \nu \nabla^2 \nabla^2 \psi = 0
$$
Let $b =  (b_0 + \epsilon b_1 +...) e^{-i\omega t}$ and $\psi = (\psi_0 + \epsilon \psi_1 + ...)$:
We take small viscoity to make equations nice e.g. small $\nu = 2 \epsilon$ and $\epsilon = \frac{1}{2} \nu$ as this is dimensional it is not clear what we mean by small parameter. As this is dimensional it means the dimensions of $\psi_0$ and $\psi_1$ are going to be different. This is not the ideal way of doing this but it is going to allow us to see more clearly what is happening.
$$
\chi = \frac{\epsilon}{\sin \theta} \xi
$$
Here $\epsilon$ being small means we are interested in gradual changes in the direction of the group velcoity but fast changes in teh wave vector direction so we keep that sinosdiual behaviour. 
$$
\frac{\partial}{\partial \xi} = \frac{\partial \chi}{\partial \xi} \frac{\partial}{\partial \chi} = \frac{\epsilon}{\sin \theta} \frac{\partial }{\partial \chi}
$$
Plug these into the bouyancy  and vorticity equation and compare terms of the same order:
$$
\epsilon_0:  \frac{\partial \psi_0} {\partial \zeta} = - b_0, \frac{\partial^2 \psi_0}{\partial \zeta^2} = i \frac{\partial b_0}{\partial \zeta}
$$
$$
\epsilon_1: \omega \frac{\psi_1}{\partial \zeta} - i \omega b_1 = - \frac{\partial \psi_0}{\partial \chi}, i \omega \frac{\partial^2 \psi_1}{\partial \zeta^2} + \omega \frac{\partial b_1}{\partial \zeta} = i \frac{\partial^2 \psi_0}{\partial \zeta \partial \chi} - 2 \frac{\partial^4 \psi_0}{\partial \zeta^4}
$$
Can eliminate the LHS of both of these to give:
$$
\frac{\partial^4 \psi_0}{\parital \zeta^4} = i \frac{\partial^2 \psi_0}{\partial \zeta \partial \chi}
$$
$$
\frac{\partial ^3 \psi_0}{\partial \zeta^3} = i \frac{\partial \psi_0}{\partial \chi} + f(\chi)
$$
For a point sources as $|\zeta| \rightarrow \infty$ then we expect $\psi \rightarrow% const. then we expect $\psi \rightarrow $const so $f(\chi) = 0$. Now we can do seperation of variables:
$$
\psi_0 = F(\zeta) G(\chi)
$$
$$
\frac{F''}{F} = i \frac{G'}{G} = - i k^3
$$
$$
G(\chi) = e^{-k^3 \chi}
$$
$$
F( \zeta) = e^{ik \zeta}
$$
$$
\psi \approx \psi_0 e^{-i\omega t} = A e^{-k^3 \chi} e^{i(k \zeta - \omega t)}
$$
$\bm k = (k,0)$ in $(\zeta, \xi)$
$$
\chi= \frac{\epsilon}{\sin \theta} \xi = \frac{\nu }{2\sin \theta} \xi = \frac{\nu}{2N \sin \theta} \xi
$$
for any $N$.
If we take a whole spectrum of linear superposition of waves $A(k)$:
\begin{equation}
\psi = e^{- i \omega t} \int_{- \infty}^{\infty} A(k) \exp( i k \zeta - \frac{\nu k^3}{2N \sin \theta} \xi) dk
\end{equation}
If we think about our cyclinder with the two delta functions at the edge of the cyclinder. As the fourier transform of the delta function is just a constant. So $A(k)$ would just be a constant so the higher wavenumber modes would decay more rapidly due to the $k^3$ in the exponential. We want it to be clear that the length scale over which the day is happening is small compared to the length scale of the oscillations. Lets imagine we are going to scale $\xi$:
$$
\frac{\nu k^2}{2N \sin \theta} k \xi
$$
Recall that $|c_g| = \frac{N}{|k|} \sin \theta$ so
$$
\frac{\nu k^2}{2N \sin \theta} k \xi
 = \frac{ \nu k}{2} \frac{1}{|c_g|} k \xi = - \pi Re^{-1} k \xi
$$
With Reynolds number $Re= \frac{\lambda |c_g|}{\nu}$ with $\lambda = \frac{2\pi}{k}$. We we were requiring $\epsilon$ to be small we were really requiring the reynolds number to be large enough.\subsection{Mass difussivity}
$$
(\frac{\partial}{\partial t} + \bm u \cdot \nabla) \rho = \frac{D\rho}{Dt} = \kappa \nabla^2 \rho
$$
Lets think about what could cause a difference in the density in the fluid e.g. $S$ salt concentration, moisture content or temperature $T$. Carbon dioxide that we are breathing out is denser than the other air we are breathing out, which is more or less balanced by the humidity of the exhaled breath which is higher than the surroundings. Water vapour is less dense than air. The diffusivity of salt and the diffusivity of temperature are different so we can't write down an equation like above. What we can write down is equations of the diffusivity of these two:
$$
\frac{D S}{Dt} = \kappa_s \nabla^2 S, \frac{DT}{Dt} = \kappa_T \nabla^2 T
$$
For water $\frac{\kappa_T}{\kappa_s} \approx 100$. If you move hot salty water down into cold nonsalty water then it quickly becomes cold salty water so it rapidly becomes more dense than the fluid around it. Like wise if you took a parcel of cold fresh up into the hot area it will stay fresh but rapidly heat and so will be less dense than its surroundings and want to rise. THis is called salt fingering. Equally if you have cold fresh water above hot salty water. As tempearture diffuses relativitly quickly the gradient will be much steeper for the salt concentration than the temperature, this creates convection.  This is called double-diffusive convection. \\\\
Prandtl number: $\frac{\nu}{\kappa_T}$\\
Schmidt number: $\frac{\nu}{\kappa_S}$ with $\kappa_S$ is the diffusivity of some thing like salt
\subsection{Reflections of waves}
With light the angle of incidence is equal to the angle of reflection.
\section{Lecture 6}
The thing being conserved on relfection is the wavelength (in the case of light) and therefore the colour. If the speed of light in the medium is constant then the frequency is constant, then the wavelength must also be constant. However, in our internal wave system our frequency $\omega$ is not constant as $|\frac{\omega}{N} | = |\cos \theta|$. The angle to the vertical must be conserved not the angle of incidence and angle of reflection in order for the frequency to be the same at the wall. So only in the case of a horizontal wall is the angle of incidence equal to the angle of reflection.  This difference will mean that the wavelength will not be conserved in general.\\\\
Let the displacement on the incident ray $\eta_i$ and reflected rays $\eta_r$. Therefore in example sheet we show that a slope of angle $\alpha$ to the vertical is: 
$$
|\eta_r| = \gamma |\eta_i|, \gamma = |\frac{\sin(\theta + \alpha)}{\sin(\theta - \alpha)}
$$
This comes about as the same volume of fluid must be displaced but the distance between neighbouring rays is smaller, so the dispalcement must be larger.
\subsubsection{Energy density upon reflection}
$$
|\bm k_r|= \gamma |\bm k_i| \implies \lambda_r = \frac{1}{\gamma} \lambda_i
$$
$$
|\tilde{\bm u}_r | = \gamma |\tilde{\bm u}_i|
$$
$$
|\tilde{\bm \eta}_r | = \gamma |\tilde{\bm \eta}_i|
$$
Recall $|c_g|  = \frac{N}{|\bm k|} \sin \theta$:
$$
|c_{g,r}| = \frac{1}{\gamma} |c_{g,i}|
$$
Flux of energy per wavelength must be preserved on reflection. We have to be careful what we mean by flux of energy as we have two different wavelength so we can talk about flux of energy per wavelength or flux of energy per unit length:\\\\
Energy density per wavelength:
$$
\tilde E = \int_0^{\lamba} PE + KE d \zeta
$$
$$
\tilde {\bm F} = \tilde E \bm c_g
$$
Therefore:
$$
\tilde E_r = \gamma \tilde E_i
$$
As $\tilde E_r \sim \lambda( PE + KE) = \lambda ( \tilde{\eta} \tilde{\eta^*} + \tilde u \tilde u^*) \sim \lambda( |\tilde \eta|^2 + |\tilde u|^2)$ and as $\lambda \sim \frac{1}{\gamma}$ and $|\eta|^2 \sim \gamma^2$ so $\tilde E_r \sim \gamma \tilde E_i$.\\
Energy density per unit length:
$$
\bar E = \frac{1}{\lambda} \int_0^{\lambda} PE + KE = \frac{1}{\lambda} \tilde E
$$
$$
|\tilde F| = \tilde E | c_g | = \lambda \bar E | \bm c_g|
$$
Flux of energy per wavelength was conserved:
$$
|\tilde F_r| = \lambda_r \bar E_r|c_{g_r}| = \frac{1}{\gamma} \lambda_i \bar E_r \frac{1}{\gamma} |c_{g,i}| = |\tilde F_i| = \lambda_i \bar E_i |c_{g,i}|
$$
So 
$$
\bar E_r = \gamma^2 \bar E_i
$$
Again consisten with $\bar E \sim PE + KE \sim |\tilde \eta|^2 + | \tilde u|^2$. The typeset notes have hats for energy desnity and flux per wavelength.\\\\
We also might want to think about the total energy of a whole broad region of a wave coming in and going out:
$$
TE_r = \int_{- \frac{L_r}{2}}^{\frac{L_r}{2}} PE_r + KE_r d \zeta = \gamma^2 \int_{- \frac{1}{\gamma} \frac{L_i}{2}}^{\frac{1}{\gamma}\frac{L_i}{2}} PE_i + KE_i = \gamma TE_i
$$
Spectral energy density $S(k)$ if we have a whole lot of waves of different wavenumbers and want to consider how that spectrum will change:
$$
S_r(k) = \gamma S_i(\frac{K}{\gamma})
$$
\subsubsection{Critical reflection}
Under sub crictical reflection the vertical direction of the propogation reverses upon reflection.\\
Under supercitical reflection the vertical direction of propogation is maintained.\\
On the boundary between these modes is the critical reflection where the reflected wave is along the slope of the wall in both directions so we have $\alpha = \theta$. Shifting slightly one way would lead to the wave reflecting slightly above the upper portion of the wall and slightly the other would lead to the the wave reflection slightly above the lower poriton oft he wave.\\\\
As we approach critical reflection, $\tilde \eta \rightarrow \infty$, $k_r \rightarrow \infty$, dissipation which scales like $k^3$ also tends to infinty. Viscosity becomes important and we get non-linearisties as $k\tilde \eta \rightarrow \infty$ as linear waves require $k \tilde \eta << 1$. Viscosity also plays a role away from critical conditions due to no-slip boundary.
\subsection{Ray tracing}
$$
( \nabla^2 \frac{\partial^2}{\partial t^2} + N^2 \nabla_H^2) w = 0
$$
$$
( \nabla^2 \frac{\partial^2}{\partial t^2} + N^2 \nabla_H^2) \psi = 0
$$
In 2D: $\psi = \tilde \psi(x,z) e^{-i \omega t}:
$$
(N^2 - \omega^2) \frac{\partial^2 \tilde \psi}{\partial x^2} - \omega^2 \frac{\partial^2}{\partial z^2} \tilde \psi = 0
$$
$$
\Lambda^2 = \frac{\omega^2}{N^2- \omega^2}
$$
give this Poincare wave equation
$$
(\frac{\partial^2}{\partial x^2} - \Lambda^2 \frac{\partial^2}{\partial z^2}) \tilde \psi=0
$$
If domain bounded with $\tilde \psi = 0$ on boundary this is an ill-posed problem and we will use ray tracing instead.
\section{Lecture 6}
Lets consider a constant slope at the edge of the ocean with free surface with constant bounancy frequncy. Imagine we have some waves entering the systemm in the body of the fluid 1 wavelength apart. These reflect off the free surface and here they remain exactly the same except for a change of vertical orientation. The reflection off the slope shorterns the wavelength as we considered in the previous section. Therefore, the energy reflection goes up, group vecolity is going down, wavenumber goes down and the velocity gets steeper on each reflection off the slope. In this case the energy is trapped into the courner by a sequence of focusing reflections.\\\\
For large Reynolds number $Re = \frac{|c_g|}{|k| \nu}$ nonlinearities will end up dominating leading to wave breaking, mixing and other frequencies and wave numbers coming out of the system.\\\\
Energy density increases:
$$
\tilde E_{n+1} = \gamma \tilde E_{n} = \gamma^{n+1} \tilde E_0
$$
Steepness increases
$$
|k_{n+1}| \tilde \eta_{n+1} = \gamma^{n+1} k_0 \eta_{0}
$$
\subsection{Reflections from rough topography}
How well do we actually know what is down at the bottom of the ocean?\\\\
Lets imagine we have some idealised rough topography, a sine wave of amplitude $h_0$ and wavelength $\lambda_T = \frac{2 \pi}{k_T}$. We can make use of our ray tracing.\\\\
Draw rays one wavelength apart and you can visually see that the reflected wavelength varies depending on which part of the sine wave they reflect off. This means the spectrum has changed so we can no longer just talk about the wavenumber vector. If we going to try and analyse this we can try some sort of linearisation by considering small amplitude variations in topography with $k_T h_0 <<1$.\\\\
Now lets zoom in and consider what is happening at a close scale (here we aren't actually assuming it is small), we have a wave coming in towards $x_i$, then reflects at $x_0$ and then appears to be leaving from $x_r$. We define $\delta x = x_0 - x_r = x_i - x_0$ as the angle of the reflected and incident wave from the vertical are identical. Take the height of the boundary to be $z = h_0 \sin k_T x$. Define $\beta = \cot \theta$. Now geometrically, $h_0 \sin k_T x_0 = \beta \delta x$. Now consider the amplitude of an incident ray 
$$
\eta_i(x,t) = \tilde \eta_i \sin (k_i x - \omega t), \eta_r(x,t) = -\tilde \eta_i \sin (k_i (x + \frac{2 h_0}{\beta} \sin k_T x ) - \omega t)
$$
For small amplitude $\frac{k_i h_0}{\beta} << 1$:
$$
\eta_r(x,t) = - \tilde \eta_i ( \sin (k_i x - \omega ) +k_i (\frac{2h_0}{\beta}) \sin(k_T x) \cos(k_i x - \omega t)) = - \tilde \eta_i \sin (k_i x - \omega ) - \tilde \eta_i \frac{k_i k_0}{\beta}(-\sin ( (k_i - k_T) x - \omega t) + \sin( (k_i + k_T) x - \omega t))
$$
So we have three different wave numbers in the reflection:$k_i = k_R, k_i - k_T = k_B, k_i + k_T = k_F$.\\\\
If $k_B = k_i - k_T < 0$ then we need to start being quite careful as in order to match the boundary condition it seems like the group velocity seems to be moving backwards from the right. This violates causality and the reason for this is we were making an assumption about the direction of the wavenumber vector assoicated with the reflection. What we actually have is two waves reflected in the forwards direction with group velcoity $c_{gR}, c_{gF}$ and a backwards refelcted one with $c_{gB}$. \\\\
In general, for subcritical reflectino we will end up with a backscatter from rough topography.\\\\
What happens if we have super critical reflection when the topography is steeper than the angle of the incident waves. This leads to a very complex spectrum as neighbouring rays could end up with very different end points as the waves can reflect multiple times off the topography before escaping. This problem is not very analytically tractable.
\subsection{Wave attractors}
Now lets look at smooth boundaries but make it a bit more complex by considering a bounded domain.
\subsubsection{Rectangular basin}
With width $X$ and height $Y$\\\\
If we have an eigenmode, then
$$
\frac{X}{Y} \frac{n}{m} = \tan \theta
$$
$m$ is the number of reflection from top boundary, $n$ is the number of reflections of the left-hand boundary.\\\\
The simplest possible case is take $n=m=X=Y=1$ gives $\tan \theta = 1 \implies \theta = \frac{\pi}{4}$ so you get a rectangle reflecting around the inside of the rectangle.
\section{Lecture 7}
One of the questions that came up in the background on the chat on wednesday is imagine we have this undular boundary to our domain and it is close to having critical slopes to our domain. A critical slope will lead to a massive increase in wavenumber and in steepnesss of the wave. If we have a near horizontal surface with near critical surfaces then we will end up filling the gaps between the undulations with denser fluid than above due to the mixing so a lot of the wave energy won't end up geting into the valleys as it will reflect off the changing stratificaiton. To some approximaiton if the water can pool in it then we can ignore the fluxuations in the surface, however if the slope is at a gradient and so the water in the pools can flow away then they remain important. \\\\
Returning the immediate discussion, if we don't have an eigenmode ( $\frac{X}{Y} \frac{n}{m} \neq \tan \theta$ then reflections were space-filling. Remember we defined $m = \frac{1}{2}$ number of subcritical reflections, and $n = \frac{1}{2}$ number of supercritical reflections (reflection from a vertical rule will always be a supercritical refleciton but if there is an angle it will depend on $\theta$\\\
\textbf{Trapezoidal basin}:
Length of one side of 2, length of other side of 1 and width of 2. At 45 degrees, we converge to a corner to corner attractor. At 40 degrees we saw roughly diamonded shaped stuff. (the trick of drawing attractors is to draw the attractor and then draw the boundary on, this method allows us to easily see the other possible domains that would give the same attractor.  $\theta = \tan^{-1} \frac{1}{2}$ gives the attractor between the other two corners. Any angles between 45 degrees and $\theta = \tan^{-1} \frac{1}{2}$ will give nice diamond shapes, and outside this range we will get more complex shapes like figure of 8s. Eventually as we get higher and higher modes we have more different directions of waves in the same space so get lots more disspation, we also start to see viscoity having an impact as the scale of the waves gets smaller.\\\\
Define a trapezium with bottom length $X$ and height $Z$ and define the bottom left to the the origin and the angle of the left hand side to be $\alpha$ to the vertical. Define the starting point of the attractor as $(x_0, z_0)$ and let every point of reflection thereafter be labeled $(x_i, z_i)$. Lets label $a = \cot \alpha$ and $b = \cot \theta$ to clean up algebra. First off we can see that $z_0 = a x_0$, $z_1 = Z$ and $x_1 = \frac{Z + (b-a)x_0}{b}$, $x_2 =2$, $z_2 = 2 Z - bX -(b-a) x_0$ ... $x_4 = \frac{2(bX - Z) + (a-b) x_0}{a+b}$, $y_4 = a x_4$ simply from basis trigonometry. So we can now follow the packet of energy as it goes round once and we can repeat this as it goes around a second time until we find where it converges. Alternatively to find the attractor we just set $x_4 = x_0$ which gives $x_0 =  X - \frac{Z}{b}$. This looks a bit like a general iteration where you are trying to find the roots of $f(x) = 0$. One of these techniques is to rewrite this as $x- g(x) = 0$ and then take $x_n = g(x_{n-1})$ which is like what we have done here. This technique only converges under the right circumstances when $|g'(x)| <1$. Here once we get close enough to the attracotr this is equivalent to $\frac{| a-b|}{|a+b|}<1 \iff |\frac{ \sin(\theta - \alpha)}{\sin(\theta + \alpha)} = \frac{1}{\gamma} <1 \iff \gamma > 1$. Possible typo here check with typed version.\\\\
One of the things we have seen here is that viscoity will play a role here as the wave number goes up and up as we go round and dissaption grows with the cube of the wavenumber. So we can start thinking about what the equilibrium energy spectrum looks like:
\subsubsection{Energy spectrum for attractor}
$$
\gamma = \frac{\sin (\theta + \alpha)}{\sin(\theta - \alpha)}
$$
Therefore with each reflection 
$$
k_n = \gamma k_{n-1}
$$
$$
\tilde E_n = \gamma \tilde E_{n-1} = \gamma^n \tilde E_0 = \frac{k_n}{k_0} \tilde E_0
$$
provided there is no dissipation.
$$
|\tilde \eta_n|\sim \frac{k_n}{k_0} |\tilde \eta_0|
$$
The wave steepness is $|k_n \tilde \eta_n| = \frac{k_n^2}{k_0} |\tilde \eta_n|$.\\\\
Recall:
$$
|\bm u| \sim e^{- \frac{k^2 \nu \zeta}{2 N \sin \theta}, \tilde E \sim |\tilde \bm u|^2 \sim e^{- k^3 \nu \zeta }{N \sin \theta}
$$
$L$ - length once around one circuit of the attractor\\
$$
\frac{\tilde E_n^{(end)}}{\tilde E_n^{(start)}} = e^{- \frac{k^3 \nu L}{N\sin \theta}}
$$
$$
\tilde E_n^{(end)} = \frac{k_n}{k_0} e^{- \Gamma ( ( \frac{k_n}{k_0})^3 - 1) }\tilde E_0
$$
with $\Gamma = \frac{\nu k_0^3 L}{(\gamma^3 - 1) N \sin \theta}$.
\section{Examples Class 1}
Does $\nabla \cdot \bm u = 0$ always hold?
$$
\frac{\partial u}{\partial t} = - \frac{1}{\rho_0} \frac{\partial p'}{\partial x}
$$
\section{Lecture 8}
\subsection{Non-linear stratification}
If we have a stratification that changes slowly compared to the wavelength, then we can treat the bouyancy frequency as being almost constant using the WKB approximation were we keep $N$ constant locally and vary it along a ray:
$$
\frac{\omega}{N}= \cos \theta \rightarrow \frac{\omega}{N(z)} = \cos \theta(z)
$$
For example if:
$$
N(z) = N_0 e^{-\frac{z}{H}}, \cos \theta(z) = \frac{\omega}{N_0} e^{z/H}
$$
If we take some rays form a source at $z=0$ wtih $x= X(z)$ then:
$$
X(z) = \int_0^z \frac{dX}{dz} dz = \int^z_0 \tan \theta dz
$$
use substitution $\cos \theta = \frac{\omega}{N_0} e^{z/H}$ to get
$$
X(z) = H( \theta - \theta \tan \theta - (\theta_0 - \tan \theta_0)) = H ( \cos^{-1} ( e^{z/H} \cos \theta_0) - \frac{\sqrt{1 - e^{2 z/H} \cos^2 \theta}}{e^{z/H} \cos \theta_0})
$$
\subsection{Lee waves}
\subsubsection{Extended range of hills}
Imagine we have a sinsuodial topography with a wavelength of $\lambda_T$ and amplitude $\eta_0$ and we have a uniform wind of speed $U$ moving past.\\\\
First we change the frame of reference so the fluid is stationary, which effectively means the mountain range is moving the opposite direction with speed $U$. We know $\omega = k_T U = N \cos \theta \leq 1$ then we have waves. If $\frac{k_T U }{N} >1$ then we have no internal waves just forced oscillations with exponentially decaying distubrances. For $\frac{k_T U}{N} \leq 1$ we will have internal waves.\\\\
Firstly, if these hills are making the waves then the vertical component of the waves must be upwards so $c_{gz} > 0$. A point of constant phase on a hill that moves to the left with speed $-U$ so the phase velocity will also be moving to the left so $c_{px} <0$. We also know that $\bm c_g \cdot \bm c_p = 0$. So we can combine these two facts to tell us which quadrant the group velocity must be in. So the phase velocity will point down to the left and and the group velocity will point up to the left. \\\\
Now consider how this works with the frame with mountains at rest. It will look the same. We will still have lines of constant phase coming from the mountains but now the directions of the velocity will change. The change of direction of speed will change the direction of the horizontal component of the velocities. As the lines of constant phase do not move the phase velocity must be parallel to the line of constant phase, therefore the group velocity must be perpendicular to the lines of constant phase. we have transformed the velocities by adding $U$ like:
$$
\bm c_p = \bm c_p' - \bm U, \bm c_g = \bm c_g - \bm U
$$
\subsection{Kelvin ship waves}
$$
\phi = \bm k \cdot \bm x - \omega t
$$
$$
d \phi = \frac{\partial \phi}{\partial k_i} dk_i + \frac{\partial \phi}{\partial x_i} dx_i + \frac{\partial \phi}{\partial \omega} d\omega + \frac{\partial \phi}{\partial t} dt
$$
Principle of stationary phase: $d\phi = 0, d\bm k =0, \omega = \omega(\bm k) \implies d \omega = 0$
$$
\frac{\partial \phi}{\partial t} + (\bm u \cdot \nabla) \phi = 0
$$
If $\bm u = \bm U$ then $\frac{\partial \phi}{\partial t} + ( \bm U \cdot \nabla ) \phi$\\
Kelvin ship waves: surface waves $\bm U = (U,0,0)$
$$
- \omega + U \frac{\partial \phi}{\partial x} = 0 = - \omega + k U = 0
$$
We need $\omega = k U$ for stationary phase:\\
Deep water waves $\omega^2 = |\bm k| g$ $\bm k = (k,l,0)$ and we also have $|\bm c_g| = \frac{1}{2} |\bm c_p| = \frac{1}{2} \frac{\omega}{|\bm k|}$ but $\omega = k W = U |\bm k | \cos \theta = \frac{1}{2} U \cos \theta$\\\\
Therefore for a ship moving on the surface of the water we get a angle of $\tan \alpha = \frac{\cos \theta \sin \theta}{2 - \cos^2 \theta}$ between the point the wave has propogated from and the ship for waves transmitted with angle $\theta$.
\section{Lecture 9}
This is calculated by knowing that the group velocity is $\frac{1}{2} U \cos \theta$ so the waves make a circle of radius $\frac{1}{2} U \cos \theta$ with its centre $UT$ behind the ship. We have a whole lot of these circles for waves produced at different times and we want to figure out the envolope that contains them all so we want to look for the maximium angle by differentiaing this value to get $\theta_{max} = \frac{1}{2} \cos^{-1} \frac{1}{3}$. \\\\
Something very similar happens in the case of internal waves
\subsection{Stationary phase for internal waves}
Imagine you stand on top of an isolated mountain with wind going past at speed $U$. As before:
$$
(\frac{\partial}{\partial t} + U \frac{\partial}{\partial x}) \phi = 0
$$
$$
\omega = k U
$$
This time $\omega = N \cos \theta = N\frac{k}{|\bm k|}$ in 2D so we have
$$
|\bm k| = \frac{N}{U}
$$
for the phase to be stationary. We can also remember that we have 
$$
|\bm c_g| = \frac{N}{|\bm k|} \sin \theta, |\bm c_p| = \frac{N}{|\bm k|} \cos \theta
$$
Won't be an exam question on 3D stationary waves.
\subsection{Shear Flows}
\subsubsection{Sheared base state}
$$
\bm u = U \hat{\bm x} + \bm u'
$$
Linearise about $U\hat{\bm x}$:
$$
\frac{\partial b}{\partial t} + U \frac{\partial b}{\partial x}  = - w' N^2
$$
$$
\frac{\partial u'}{\partial t} + U \frac{\partial u'}{\partial x} + \frac{dU}{dz} w' = - \frac{1}{\rho_0} \frac{\partial p'}{\partial x}
$$
$$
\nabla \cdot \bm u'=0
$$
Combining all this gives:
$$
\left( ( \frac{\partial}{\partial t} + U \frac{\partial}{\partial x}) ( \frac{\partial^2}{\partial x^2} \frac{\partial ^2}{\partial z^2}) + N^2 \frac{\partial^2}{\partial x^2} - U'' ( \frac{\partial}{\partial t} + U \frac{\partial}{\partial x}) \frac{\partial }{\partial x} \right) w' = 0
$$
In the limit $U = 0$:
$$
( \frac{\partial^2}{\partial t^2} \nabla^2 + N^2 \nabla_H^2 ) w' = 0
$$
as before. \\
If we choose a frame where the flow is steady with $\frac{\partial}{\partial t} = 0$ then
$$
(\frac{\partial^2}{\partial x^2} \nabla^2 + ( \frac{N^2}{U^2} - \frac{U''}{U}) \frac{\partial ^2}{\partial x^2} ) w' = 0
$$
Provided $w'$ is bounded at infinity:
$$
(\nabla^2 + \frac{N^2}{U^2} - \frac{U''}{U}) w' = 0
$$
Consider the case:
$$
w' = \tilde w e^{i(kx + mz)}
$$
Consider easier case with no curvature $U'' = 0$ then
$$
( - (k^2 + m^2) + \frac{N^2}{U^2}) w' = 0
$$
For non-trivial solution $|\bm k| = \frac{N}{U}$ \\\\
\subsubsection{Critical layer reflection}
Let's return to ray tracing to show how we can use this. In the simple case we have already looked at with a single mountain with a constant speed wind. We know that this gives quarter circles enclosed in semicircular casuality envolopes. Lets consider a steady state so when the wind has been blowing forever. The phase velocity is along the lines of constant phase which is necessary for them to be stationary as then you cannot see the phase move.\\\\
If we want to know how the position of one of these rays is moving with time we need to consider
$$
\frac{dZ}{dx} = \frac{c'_{gz}}{c'_{gx}} = \frac{c_{gz}}{c_{gx} + U} = \frac{m}{k}
$$
and as $|\bm k|^2 = k^2 + m^2 = \frac{N^2}{U^2}$.\\
We take the assumption that $k$ is preserved as we move alon a ray. However, if $U \neq $constant or $N \neq $const then we must have $m \neq$ const. Therefore, the orientation of $\bm k$ changes and $|\bm k|$ changes:
$$
m^2 = \frac{N^2}{U^2} - k^2
$$
Need to take negative square root as the phase velocity is perpendicular to the group velocity which has positive vertical motion. So
$$
\frac{dZ}{dx} = \frac{m}{k} = \frac{ ( \frac{N^2}{U^2} - k^2)^{\frac{1}{2}}}{|\bm k|} =  ( (\frac{N}{k U})^2 - 1)^{\frac{1}{2}}
$$
as $k$ is also negative.\\\\
If $\frac{N}{kU} = 1$ then the ray stops propogating vertically upwards and so cannot propogate through height at which $\frac{N}{k U} =1$. Consider $N= $ const and $U$ increases with height. Clearly at the level of the mountain we produce waves as we can always find some $k$ that works as $\frac{N}{k U}> 0$. So waves will always be produced for an isolated mountain. So the rays will slowly bend round until at a certain height $z_c$ (where $\frac{N}{kU} =1$) they are horizontal.\\\\
\section{Lecture 10}
Near this critical height $z_c$ we can do a taylor series expansion to find out what happens if it isn't quite attained.\\
$$
\frac{dZ}{dx} = ( ( Z- z_c ) \frac{d}{dz}( \frac{N^2}{k^2 U^2}))^{\frac{1}{2}}
$$
So:
$$
Z = z_c + \frac{1}{4} \frac{d}{dz} ( \frac{N^2}{k^2 U^2})|_{z=z_c} ( x- x_c)^2
$$
This will give us a quadratic behaviour where the waves rise up to $z_c$ then the reflect back down in a parabolic curve. The lines of constant phase are prependicular to this parabolic $c_g$. So the waves go from having almost horizontal phase lines and then completely vertical phase lines. We have been a bit naughty here as we have made use of the WKB approximation but we need to ask if this is a valid approximation.\\\\
While $U(z)$ may change over a length scale that is large comapred with $\lambda = \frac{ 2 \pi}{|k|}$, the vertical extent of the wave is going towards infinity as $m \rightarrow 0$ as $z \rightarrow z_c$. So at the top we are saying the wave crests are vertical so at the top there must be some variation in velocity across the wave crest. Luckily whilst it is not clear that WKB is valid it works well in practice. Approximations often work well even when they have no right to do so. \\\\
Important to note that the height at which this reflection occurs depends on the wavenumber of the waves, so a spectrum of waves will reflect at differing heights.
\subsubsection{Critical layer absorption}
This is the opposite case to the above. What happens if $\frac{U}{N}$ reduces with height and at some point reduces to zero. This time we can think about $\frac{dz}{dx} = \frac{(\frac{N^2}{U^2} - k^2)^{\frac{1}{2}}}{|\bm k|} = \frac{m}{k}$ so as $U \rightarrow 0$, we get $m \rightarrow \infty$ and $|\bm k|\rightarrow \infty$. This means that this time the ray is going to bend upwards and become vertical at $U = 0$, and also as we become more and more vertical the wave number increases and so the wave crests get closer and closer together (this is because $k$ remains the same as $m$ increases). As $|\bm k| \rightarrow \infty$, $|c_g| = \frac{N}{|\bm k|} \sin \theta \rightarrow 0$ and $|c_p| = \frac{N}{|\bm k|} \cos \theta \rightarrow 0$. We also remember that the dissapation of the waves ( the change of energy per wavelength) will be $\fra{d \tilde E}{d \zeta} = e^{- k^3 \nu \zeta}{N \sin \theta}$ so the energy will be dissapating out very quickly which is refered to as critical layer absorption. Critical layer absorption occurs when $U$ changes sign.\\\\
We aren't covering 3D stuff this year or columnar wave modes which were focused on in previous year. Part of the reason for this is it is more or less impossible to ask exam questions on as it is too algebraically messy.
\subsection{Resonant triads}
\subsubsection{Weakly nonlinear internal waves}
Q1 on Example Sheet 1 goes part of the wave towards getting the background for this. Here we should that the Linear wave solution satisfies the nonlinear equations but you cannot use linear superposition as addition of linear wave solutions does not solve non-linear equations.
$$
\bm u = \tilde{\bm u}_1 e^{i \phi_1} + \tilde{\bm u}_2 e^{i \phi_2} + \tilde{\bm u}_3 e^{i \phi_3}
$$
$$
\phi_1 = \bm k_1 \cdot \bm x - \omega t, ... 
$$
$$
b = \tilde b_1 e^{i \phi_1} + \tilde b_2 e^{i \phi_2 } + \tilde b_3 e^{i \phi_3}
$$
Linear terms in equations, we are going to allow $\tilde{\bm u}_i = \tilde{\bm u}_i(t)$ to be a slowly varying function allowing time dependance in phase and amplitude:
$$
\frac{\partial}{\partial t} ( \bm u) = ( \dot{\tilde{\bm u}_1} - i \omega_1 \tilde{\bm u}_1 ) e^{i \phi_1} + \dot{\tilde{\bm u}_2} - i \omega_2 \tilde{\bm u}_2 ) e^{i \phi_2} + \dot{\tilde{\bm u}_3} - i \omega_3 \tilde{\bm u}_3 ) e^{i \phi_3} 
$$
We are going to take the case with no viscosity and no mass diffusivity $\nu = \kappa = 0$. So in the linear equations this would end up with $\dot{\tilde{\bm u}}$ would vanish but because we have the non-linear terms this time they might not.
$$
(\bm u \cdot \nabla) \bm u = i (\tilde{\bm u}_1 \cdot \bm k_1 ) \tilde{\bm u_1} e^{2 i \phi_1} +i (\tilde{\bm u}_2 \cdot \bm k_2 ) \tilde{\bm u_2} e^{2 i \phi_2} + i (\tilde{\bm u}_3 \cdot \bm k_3 ) \tilde{\bm u_3} e^{2 i \phi_3} + i ( ( \tilde{\bm u_1} \cdot \bm k_2) \tilde{\bm u_2} + ( \tilde{\bm u_2} \cdot \bm k_1) \tilde{\bm u_2}) e^{i(\phi_1 \pm \phi_2)} + ... e^{i (\phi_2 \pm \phi_3)} + ... e^{i (\phi_3 \pm \phi_1)} 
$$
The first self interactions just reperesent sort of local forcing but the cross over terms are much more interesting as they show how three different waves interact.\\\\
Triadic resonance condition to allow a sustatined interaction between $\tilde{\bm u_1},\tilde{\bm u_2},\tilde{\bm u_3}$ over sustained time and space. Order for this to happen the cross over terms must feedback into the part of the interaction not included in themselves so must have $\pm \phi_4 = \pm \phi_1 \pm \phi_2$. So if we put two waves into the system they will produce a third waves in additoin to the forcing waves. As there are two waves you could produce a new wave we need to further consideration to descern which of the potential waves are sustained.\\\\
The two waves generated have: $\pm \phi_3 = \pm \phi_1 \pm \phi_2$ so $\pm \bm k_3 = \pm \bm k_1 \pm \bm k_2$ and $\omega_3 = \pm \omega_1 \pm \omega_2$\\
For sustained interaction need:
$$
\frac{\omega_1}{N} = \frac{k_1}{|\bm k_1|} = \cos \theta_1
$$
$$
\frac{\omega_2}{N} = \frac{k_2}{|\bm k_2|} = \cos \theta_2
$$
$$
\frac{\omega_3}{N} = \frac{k_3}{|\bm k_3|} = \cos \theta_3
$$
This means if we are going to draw things on wavenumber diagram only part of the space is going to marry up.
\section{Example sheet 1}
Linerised equation of motion with non zero visocity and mass diffusion:
$$
( \frac{\partial}{\partial t} = \kappa \nabla^2) ( \frac{\partial }{\partial t} - \nu \nabla^2) \nabla^2 \psi = - N^2 \frac{\partial^2 \psi}{\partial x^2}
$$
which is derived by straight away combining:
$$
\frac{\partial u}{\partial t} = \frac{\partial b}{\partial x} + \nu \nabla^2 u, \frac{\partial b}{\partial t} = -N^2w - \kappa \nabla^2 b
$$
In general $\frac{D}{Dt}$ and $\frac{\partial }{\partial x}$ do not commute in general.\\\\
If it doesn't say derive the dispersion relation you can just write it down. So worth memorising some dispersion relationships like the deep water one $\omega^2 = gh \tanh k H$. I made mistake of thinking the dipersion relation did not change so need to practice deriving this dispersion relation.\\\\
Marks questions out of 40.\\\\
Define potential energy for deep water as:
$$
\bar{PE} = PE_0 + \int_{-H}^{\eta}  \rho g z dz
$$
we can define reference state to be $PE_0 = - \int_{-H}^0 \rho g z dz$ so
$$
\bar{PE} = \int_0^{\eta} \rho g z dz = \frac{1}{4} \rho_0 g \eta_0^2
$$
When determining the structure of a decaying wave we can take the integration constant to be zero, as we want solutions oscillating around zero?? Email the guys about this.\\
Won't set anything as algebraically messy as question 5 in the exam again. Try quesiton 5 again i completely didn't get it at all!\\\\
He uses a completely different technique to me for figuring out the reflection of waves. His is a geometric appoarch where you consider the continuity at the boundary.\\\\
Question 8 is also bizarre, seems to be using techniques from fluids II to do with boundary layers. You are meant to assume the form $\eta = \eta_0 e^{i\omega t} e^{- (1+i) \epsilon/ \delta}$ try it again with this.
\section{Lecture 11}
We force wtih $\phi_1$, $\phi_2$ and nonlinearly generate $\phi_3$:
$$
\phi_3 = \pm \phi_1 \pm \phi_2
$$
$$
\omega_3 = N \frac{k_3}{|\bm k_3|}
$$
\subseection{Triadic resoncance transfer term}
Take $\psi = \tilde \psi_1 e^{i\phi_1} + \tilde \psi_2 e^{i \phi_2} + \tilde \psi_3 e^{i \phi_3}$. Also take $\kappa = 0$ and $\nu$ is small. Assume $\tilde \psi_j$ are functions of $t$. Now lets just look at the non linear $\bm u \cdot \nabla \bm u$ terms and relate them to the linear terms $\frac{\partial \bm u}{\partial t}$ then we find:
$$
\dot{\tilde \psi_1} = I_1 \tilde \psi_2 \tilde \psi_3 - \frac{1}{2} \nu |\bm k_1 |^2 \tilde \psi_1
$$
$I_1$ is in the typeset version of the notes but he doesn't want us to memorise it just have an idea that it exists. We are going to take this body and use the euler equations for this body, this gives a mecahnical analogy:\\\\
Euler equations for a rigid body:
$$
J \dot{ \bm \omega} + \bm \omega \times ( J \bm \omega) = \bm M
$$
wtih $\bm J$ is the Inertia tensor and $\bm M$ is the applied torques. Take 
$$
J = \begin{pmatrix} J_{11} & 0 & 0\\ 0& J_{22} & 0 \\ 0 & 0 & J_{33} \end{pmatrix}
$$
with $J_{11} < J_{22} < J_{3}$.\\\\
If $\bm M = 0$ then conserve energy and we have:
$$
 \bm \omega \cdot J \dot{\bm \omega} + \bm \omega \cdot ( \bm \omega \times (J \bm \omega)) = 0
$$
$$
J_{11} \dot{\omega_1} + (J_{33} - J_{22} ) \omega_2 \omega_3 = 0
$$
$$
J_{22} \dot{\omega_2} + (J_{11} - J_{33}) \omega_3 \omega_1 = 0
$$
$$
J_{33} \dot{\omega_3} + (J_{22} - J_{11}) \omega_1 \omega_2 = 0
$$
If $| \omega_1 | >> | \omega_2| , |\omega_3|$ then $J_{11} \dot{\omega_1} = 0$ and $J_{22} \dot \omega_2 + ( J_{11} - J_{33} ) \omega_3 \omega_1 = 0$ ...\\
Elminate $\omega_3$ to get:
$$
J_{22} J_{33} \ddot{\omega_2} - (J_{11} - J_{33} ) (J_{22} - J_{11}) \omega_1^2 \omega_2 = 0
$$
There is a similar result for $\ddot{\omega_3}$, the RHS coefficent can be easily seen to be less than 0 so this gives harmonic oscillation so if pertubred it will oscillate slightly but be pretty stable. Similarly if $|\omega_3| >> |\omega_1|, |\omega_2|$ we get harmonic osciallation. However, for $|\omega_2| >> |\omega_1| |\omega_3|$ we get:
$$
J_{11} J_{33} \ddot{\omega_1} - (J_{33} - J_{22} ) ( J_{22} - J_{11}) \omega_2^2 \omega_1 = 0
$$
so $\omega_2$ will grow (and decay) exponentially if perturbed so this is an instability.
\subsubsection{Triadic resonance instability}
$$
|\tilde \psi_1| >> |\tilde \psi_2 | | \tilde \psi_3|
$$
We have oscillatory behaviours in the central regions of the sustained graph of solutions, and exponentially growing solutions on the top and bottom.
The TRI is a linear instability of internal waves when we consider weakly non-linear behaviour.\\\\
It would be too hard to write an exam question on this area, he tried to write an exam question on this last year and it was too algebraically nasty to do any more than we just covered.\\\\
We want to figure out what the growth rate actually looks like for a plane wave on the expoentially growing portion of the curve. We can do this analytically to figure out:
$$
\sigma_{\pm} = - \frac{1}{4} \nu( |\bm k_2 |^2 + |\bm k_3|^2) \pm \sqrt{ \frac{1}{16} \nu^2 ( |\bm k_2|^2 + |\bm k_3|^2 + I_2 I_3 |\tilde \psi_1|^2}
$$
we have $\sigma_+ \geq 0$ if $|\tilde \psi_1| > 0$. We actually find that the maximium of the growth rate occurs when $\omega_2 = - \omega_3 = \frac{1}{2} \omega_1$. THis has been known for a while and is known as the parametric subharmonic instability (PSI). In practice we don't tend to see this as in practice we are consdiering beams of waves rather than plane waves.
\section{Lecture 12}
\subsection{Shallow water}
There is a strong analogy between shallow water flow and flows in incompressible fluids. We can write down the rules for incompressible fluid flow:
$$
\nabla \cdot \bm u = 0, \frac{\partial u}{\partial x} + \frac{\partial v}{\partial y} + \frac{\partial w}{\partial z} = 0
$$
If we throw away a dimension we get the shallow water flow:
$$
\frac{\partial u}{\partial x} + \frac{\partial v}{\partial y} \neq 0
$$
though in the shallow water flow the flow is still 3D.\\
In the examples sheet 1 we derived the dispersion of interfacial waves to be:
$$
\omega^2 = (\rho_1 - \rho_2) g k ( \frac{\rho_1}{\tanh kH_1} + \frac{\rho_2}{\tanh kH_2})^{-1}
$$
We are going to be interested in one specific limit of this dispersion relation but first lets consdier a different limit. The short wave limit $kH_1 , kH_2 >>1$ so $\tanh kH_i \rightarrow 1$ this gives $\omega^2 = \frac{\rho_1 - \rho_2}{\rho_1 + \rho_2} g k = A gk$. Therefore in the limit $ A \rightarrow 1$ we recover the deep water dispersion relation $\omega^2 = gk$.\\\\
Introduce the 'reduced gravity' $ g' = \frac{\Delta \rho}{\bar \rho} g = 2 A g$ so we get:
$$
\omega^2 = \frac{1}{2} g' k, c_p = \sqrt{ \frac{Ag}{k}} = \sqrt{ \frac{g'}{2k}}, c_g = \frac{1}{2} c_p
$$
We can clearly see from graphs of height against phase velocity and frequency that there are two clear limits. One for short waves (where the frequency depends strongly on the wavelength) and one for when one hiehgt is much bigger than the other for long waves ( the long wave limit).\\\\
$kH_1, kH_2 << 1$ so $\tanh kH_i \rightarrow kH_i$ so 
$$
c_p = c_g = \sqrt{ \frac{ H_1H_2}{H_1 + H_2} g'}
$$
can think of $H_E = \frac{H_1 H_2}{H_1 + H_2}$ as an effective depth. We only need one of $kH_1 $ or $kH_2$ to be small for waves to become non-dispersive. Let $H_s$ be the depth of the shallow layer and $H_d$ be the depth of the deep layer with $\frac{H_s}{\rho_s} << \frac{H_d}{\rho_d}$ so we get $c_p = c_g = \sqrt{ H_s g'}$ with $g' = \frac{\rho_1 - \rho_2}{\rho_s} g$. \\\\
In a lot of what we are going to be doing we are going to be taking the BOussinesq limit of $\Delta \rho << \bar \rho$ so $g' = \frac{\Delta \rho}{\bar \rho} g$ therefore:
$$
\omega^" = \frac{g' k}{\coth kH_1 + \coth k H_2} 
$$
with $\coth kH_i \rightarrow \frac{1}{kH_i}$ for $kH_i <<1$ and $\coth kH_i \rightarrow 1 $ for $kH_i>>1$. To start with mainly consider one shallow layer to make the mathematics easier.\\\\
\textbf{Equipartion of energy}
$$
\bar E = \bar{KE} + \bar{PE}, \bar{F_E} = c_g \bar E
$$
\subsubsection{Shallow water equations}
We have just considered linear waves but we need to think of non-linear waves to be useful so need to derive the shallow ater equations.\\\\
Consider a shallow body of water with height $h$ and horizontal scale $X$ with $h << X$.
$$
\nabla \cdot \bm u = 0
$$
$$
 \frac{\partial \bm u}{\partial t} + ( \bm u \cdot \nabla) \bm u = - \frac{1}{\rho} \nabla ( p + \rho g z) + \nu \nabla^2 \bm u
$$
for homogenous fluid. Take scales $U, W, X, Z, P, T$ with:
$$
 \frac{U}{X} + \frac{W}{Z} = 0
$$
 in shallow water $\frac{X}{Z} \rightarrow \infty$ and so $\frac{W}{U} \rightarrow 0$
 $$
 \frac{W}{T} + \frac{UW}{X} + \frac{W^2}{Z} = \frac{1}{\rho}\frac{P}{Z} + g + \nu ( \frac{W}{X^2} \frac{W}{Z^2})
 $$
 as the vertical pressure gradient is predominantly hydrostatic so the other terms become small and $\frac{1}{\rho} \frac{P}{Z} \sim g$ so:
 $$ 
 \frac{\partial p}{\partial x} \approx - \rho g \implie p = p_0  - \rho g z
 $$
 and 
 $$
  \frac{\partial w}{\partial z} << g
 $$
 now look at horizontal momentum equation:
 $$
  \frac{U}{T} + \frac{U^2}{X} + \frac{ WU}{Z} = \frac{\rho g Z}{\rho X} + \nu ( \frac{U}{X^2} + \frac{U}{Z^2})
 $$
 can neglect the viscoity term as we are dealing with high reyonlods number. Physcailly insight allows us to neglect the $\frac{WU}{Z}$ term as we can see that $W$ should be small as it is a shallow flow so what would be generating a large velocity up or down. this gives:
 $$
  \frac{\partial u}{\partial t} + u \frac{\partial u}{\partial x} = - \frac{1}{\rho} \frac{\partial}{\partial x} ( p + \rho gz) + \nu \frac{\partial^2 u}{\partial z^2}
 $$
 \textbf{From first prinicples}:\\
 We now cover how to do this from first principles as often we will be asked to derive the shallow water equations for bizarre channels like triangluar tubes or something and this gives us an insight into how to do that.\\\\
 Take a flat bottom, unit width, slowly varying depth of water with $u(x,t)$ under the hydrostatic limit and $Re >>1$.\\\\
 The volume flux is given by $Q = uh$ so $2 \delta x \frac{\delta h}{\delta t} = Q_{x- \delta x} - Q_{x+ \delta x}$. So as $\delta t \rightarrow 0$ we get:
 $$
 \frac{\partial h}{\partial t} + \frac{\partial Q}{\partial x} = 0
 $$
 $$
  \frac{\partial h}{\partial t} + u \frac{\partial h}{\partial x} + h \frac{\partial u}{\partial x} = 0
 $$
 If cross-section of channel is not rectangular, then this expression changes.Here the cross-sectional area is just $h$ times unit width but if say a triangular channel then the cross-sectional area would be scaling with $h^2$ instead.\\\\
  We can also consider the momentum flux and the pressure forces. So the pressure force towards the right on the right hand face
  $$
\int_0^{h- \frac{\partial h}{\partial x} \delta x} \rho g( h - \frac{\partial h}{\partial x} \delta x - z) dz  = \rho ( \frac{h^2}{2} - h \frac{\partial h}{\partial x} \delta x) + O( \delta x^2)
  $$
   we also get:
   $$
   \frac{\partial}{\partial t} ( uh) + \frac{\partial}{\partial x} ( u^2 h + \frac{1}{2} gh^2) = 0
   $$
    with $F_M = u(uh) = uM$ for 1D shallow water. Note the simularityes with 
    $$
     \frac{\partial}{\partial t} ( \rho u) + \frac{\partial}{\partial x} ( \rho u^2 + p) = 0
    $$
    which is 1 D compressible. So you can sse that $h$ and $\rho$ play similar roles.\\\\
    In Example sheet 2, we vary $b(x,z)$ as the wdith of the channel. THis is still a 1D shallow awter flow provided $\frac{u^2}{b} \frac{ d^2 b}{dx^2} <<g$\\\\
    \textbf{ Boussineq vs non-Boussinesq}:\\
    $$
    g' = \frac{\rho_1- \rho_2}{\frac{1}{2}(\rho_1 + \rho_2)}= \frac{\Delta \rho}{\bar \rho} g
    $$
    $$ 
    \frac{\partial u}{\partial t} + u \frac{\partial u}{\partial x} + g' \frac{\partial h}{\partial x} = 0
    $$
     For most things a single-layer shallow ater flow ( Boussinesq flow) and a single-layer free surface flow, the equations are the same except for $g' <-> g$. This will not be the case where the shallow water assumptions are violated - we will look at some of these.
     \section{Lecture 13}
     One instance of this not working is if the lower layer is deep as then we could plausibly have much lower velocities in the lower layer than the upper layer. If we were to take the pressure being lower in the lower layer then we would end up having lhigher pressure where the upper layer is larger which would force the pressure to be funnneled in the direction of the narrowest upper layer. If the upper layer has a free surface or a surface with say air then it can rise up slightly allowing the pressure to be unifrom along the boundary with the deep layer below.\\\\
     \textbf{Averaging}\\
     $\bm u = \bar{\bm u} + \bm u', \bm u = \bm u(x,y,z,t)$ with $\bar{\bm u} = \bar{\bm u})(x,t)$ and $h = h(x,t)$ with:
     $$
     \bar{\bm u} = \frac{1}{A} \int^{b/2}_{-b/2} \int_0^h \bm u dz dy
     $$
     wtih $A= bh = A(x,t)$. If we take the diveregence equation:
     $$
      \int^{b/2}_{-b/2} \nabla \cdot \bm u dz dy = 0
     $$
     Consider:
     $$
      \int^{b/2}_{-b/2} \frac{\partial w}{\partial z} dz dy = \int_{-b/2}^{b/2} w|_{z=h} - w_{z=0} dy
     $$
     using the kinematic boundary condition
     $$
      w(z=h) = \frac{\partial h}{\partial t} + u|_{z=h} \frac{\partial h}{\partial x}
     $$
     gives
     $$
     b \frac{\partial h}{\partial t}  + \frac{\partial h}{\partial x} \int^{b/2}_{-b/2} u|_{z=h} dy = 0
     $$
     similar result holds for $\frac{\partial v}{\partial y}$ so these together gives:
     $$
      \int^{b/2}_{-b/2} \nabla \cdot \bm u dz dy = b h \frac{\partial \bar u}{\partial x} + \bar u( b \frac{\partial h}{\partial x} + h \frac{db}{dx} ) + b \frac{\partial h}{\partial x} = 0
     $$
     If we do the same for the momentum equation:
     $$
     \int \int_A \frac{\partial u}{\partial t} + (\bm u \cdot \nabla) u dA 
     $$
     which gives:
     $$
     \frac{\partial \bar b}{\partial t} + \bar u \frac{\partial u}{\partial x} +g \frac{\partial h}{\partial x} - \nu \frac{\partial^2 \bar u}{\partial x^2} = \nu[ \frac{1}{b} ( \bar{\frac{\partial u'}{\partial y}|_{y = \frac{b}{2}}} - \bar{\frac{\partial u'}{\partial y} |_{y=- \frac{b}{2}}}) + \frac{1}{h} ( \bar{ \frac{\partial u'}{\partial z}|_{z=h} } - \bar{ \frac{\partial u'}{\partial z}|_{z=0}} )]- ( \bar{\bar{ u' \frac{\partial u'}{\partial x}}} + \bar{\bar{ v' \frac{\partial u'}{\partial y}}} + \bar{\bar{w' \frac{\partial u'}{\partial z}}})
     $$
     The equations for $\bar u$ requre a knowledge of the equations for $\bar{u'^2}$. This is generally referred to as the closure problem.\\\\
     At high Re, then $\nu$ may be negligible but $\bar{\bar{u'_i u_j'}}$ may not be.\\
     We can try to model the unknown terms on the RHS by consdiering the scaling of the different terms with $\frac{\nu}{h} \frac{\partial u'}{\partial z} \sim \frac{\nu \bar u}{h^2}$ and $\frac{\nu}{b} \fraC{\partial u'}{\partial y} \sim \frac{2\nu \bar u}{b^2}$. So could model the RHS as:
     $$
     \nu[ \frac{1}{b} ( \bar{\frac{\partial u'}{\partial y}|_{y = \frac{b}{2}}} - \bar{\frac{\partial u'}{\partial y} |_{y=- \frac{b}{2}}}) + \frac{1}{h} ( \bar{ \frac{\partial u'}{\partial z}|_{z=h} } - \bar{ \frac{\partial u'}{\partial z}|_{z=0}} )] = C_L \nu \frac{\bar u}{h^2} (1 + 2 \frac{h^2}{b^2})
     $$
     where $C_L$ is an O(1) laminar drag coefficent. In the slow viscous flow course we could find that $C_L= \frac{2}{3}$ for Re <<1 and $b>>h$. However for high Re, $\nu$ is neglible and we can think about the other terms:
     $$
     - ( \bar{\bar{ u' \frac{\partial u'}{\partial x}}} + \bar{\bar{ v' \frac{\partial u'}{\partial y}}} + \bar{\bar{w' \frac{\partial u'}{\partial z}}}) \approx C_t \frac{\bar u | \bar u|}{h} ( 1+ 2 \frac{h}{b})
     $$
     The reason for this form instead of just $\bar u^2$ is we want it to retrarding the flow so we need to retain the sign. The simplest way we can model this is is with $C_T$ is an O(1) turbulent drag coefficent that dependson roughness of boundaries. For smooth boundaries it can be small $ C_T \approx 0.03$ for rough boundaries it is roughly 0.1. Often $h << b$ so $C_t \frac{\bar u |\bar u|}{h}$. Dropping overbars.
     $$
      \frac{\partial h}{\partial t} + b^{-1} \frac{\partial}{\partial x} ( bhu) = 0
     $$
     $$
      \frac{\partial u}{\partial t} + u \frac{\partial u}{\partial x} + g \frac{\partial h}{\partial x} = - C_L \nu \frac{u}{h^2} - C_T \frac{u |u|}{h}
     $$
     Do Q13 and 17  on ES2.\\
     \subsection{Hyperbolic system}
     These equations support wavelike solutions of the form: 
     $$
     F_i(x-c_it)
     $$
     where $c_i$ is the wave speed and $F(..)$ expresses the evolution following a wave. The equations we are looking at are quasi linear.\\
     \textbf{Recap linearity}:\\ LInear equations: $a(x,t) \frac{\partial u}{\partial x} + b (x,t) \frac{\partial u}{\partial t} = c(x,t) u + d(x,t)$\\
     Semi-linear equations: $ a(x,t) \frac{\partial u}{\partial x} + b(x,t) \frac{\partial u}{\partial t} = c(x,t,u)$\\
      Quasi-linear equations: $a(x,t,u) \frac{\partial u}{\partial x} + b(x,t,y) \frac{\partial u}{\partial t} = c(x,t,u)$
      \subsubsection{A model for traffic flow}
      We have a road with speed depending only on the local traffic density $u = u(\rho)$. So can easily write down continuity equation:
      $$
       \frac{\partial \rho}{\partial t} + \frac{\partial}{\partial x} (u \rho) = 0
      $$
Now lets just imagine we have got helicopters flying over the traffic proving traffic reports with some speed $\lambda$ so:
$x = x_0 + \lambda s$, $t = t_0 + s$. The helicopter will see:
$$
 \frac{df}{ds} = \frac{dt}{ds} \frac{\partial f}{\partial t} + \frac{dx}{ds} \frac{\partial f}{\partial x}
$$
as $ \frac{d}{ds} = \frac{\partial}{\partial t} + \lambda \frac{\partial }{\partial x}$ and so $\frac{d \rho}{ds} = \frac{\partial \rho}{\partial t} +\lambda \frac{\partial \rho}{\partial x}$.\\\\
Imagine $u(\rho) = 1 -\rho$ then we will get a quadratic flux. Also our continuity equation would give us:
$$
\frac{\partial \rho}{\partial t} + \frac{\partial}{\partial x} ( (1- \rho) \rho) = 0 = \frac{\partial \rho}{\partial t} + ( 1- 2 \rho) \frac{\partial \rho}{\partial x} = 0
$$
If the pilot flies at speed $\lambda = 1- 2 \rho$ then $\frac{d \rho}{ds} =0$ and so $\rho =$ const. So $\lambda $ is the characteristic speed of the problem. We have reduced the pde to ode by choosing this moving view point. When characteristics intersect we get a shock. These two different traffic densities are producing the same flux as we still must have continuity so we have:
$$
 \rho_l ( u_l - u_s) = \rho_r( u_r - u_s)
$$
 so in this particular case of constant traffic desnities on each side of the shock we get:
 $$
 u_s = 1 - (\rho_r + \rho_l) = \frac{1}{2} (\lambda_r + \lambda_l)
 $$
 \section{Lecture 14}
 \subsection{Shallow water characteristic treatment}
 $$
  \frac{\partial h}{\partial t} + u \frac{\partial h}{\partail x} + h \frac{\partial u}{\partial x} = 0
 $$
 $$
  \frac{\partial u}{\partial t} + u \frac{\partial u}{\partial x} + g \frac{\partial h}{\partial x} =0
 $$
 $$
 x = x_0 + \lambda s, t = t_0 + s
 $$
 with stopwatch time $s$
 $$
  x= x_0 + \zeta, t = t_0 + \lamdba^{-1} \zeta
 $$
 with tape-measure distance $\zeta$.
 As we have $\frac{\partial}{\partial t} = \frac{d}{ds} - \lambda \frac{\partial }{\partial x}$ so our equations transform to:
 $$
  \frac{dh}{ds} + ( u - \lambda) \frac{\partial h}{\partial x} + h \frac{\partial u}{\partial x} = 0
 $$
 $$
  \frac{du}{ds} + (u- \lamdba) \frac{\partial u}{\partial x} + g \frac{\partial h}{\partial x} = 0
 $$
 multiply the first by $u - \lamdba$ and the second by $h$ and add them together to get:
 $$
 (u-\lambda) \frac{dh}{ds} - h \frac{du}{ds} + ( (u- \lambda)^2 - gh) \frac{\partial h}{\partial x} = 0
 $$
 To make into ODE set $(u- \lambda)^2 - gh = 0$ which occurs when $\lamdba = u \pm \sqrt{gh} = u \pm c$ as $\sqrt{gh} = c_p = c_g = c$ for long waves on stationary fluid.\\\\
 Now lets do the same manipulations a bit differently using matrix formulism:
 $$
\begin{pmatrix} u & h \\ g & u \end{pmatrix} \begin{pmatrix} h \\ u \end{pmatrix}_x + \begin{pmatrix} 1 & 0 \\ 0 & 1 \end{matrix} \begin{pmatrix} h \\ u \end{pmatrix}_t = 0
 $$
 $$
  A \bm v _x + B \bm v_t = \bm f
 $$
 Using transformation $\frac{\partial}{\partial t} = \frac{d}{ds} - \lambda \frac{\partial}{\partial x}$:
 $$
\begin{pmatrix} h \\ u\end{pmatrix}_s + \begin{pmatrix} u & h \\ g & u \end{pmatrix} \being{pmatrix} h \\ u\end{pmatrix} _x - \lambda \begin{pmatrix} 1 &0 \\ 0 & 1\end{pmatrix} \begin{pmatrix} h \\ u \end{pmatrix}_x = 0
 $$
 We then premultiply by $\bm q^T$ and we want to chose $\bm q$ so it eliminates the RHS quantitiy
 $$
  \bm q^T ( A - \lambda B) = \bm 0^T
 $$
 Therefore we can think about the \textbf{general approach}. In general we have:
 $$
 A \bm v_x + B \bm v_t = \bm f
 $$
 These are quasi linear as $A = A(x,t, \bm v), B = B( x,t, \bm v), f = f(x,t \bm v)$. Consider a linear combination of equations:
 $$
  \bm q^T A \bm v_x + \bm q^T B \bm v_t = \bm q^T \bm f
 $$
 The observer $\bm v_s = \lambda \bm v_x + \bm v_t$. Take a linear combination of these as well:
 $$
 \bm m^T \bm v_s = \bm m^T ( \lambda \bm v_x + \bm v_t) = ?
 $$
 Compare these two equations it would make sense to equate the following:
 $$
\lambda \bm m^T = \bm q^T A, \bm m^T = \bm q^T B
 $$
 Then we can eliminate $\bm m$ from these coupled equations to give;
 $$
  \bm q^T A = \lamdba \bm q^T B \iff \bm q^T ( A - \lamdba B) = \bm 0^T
 $$
 This is our generalised left-hand eigenvalue problem. For a non-trial solution we need:
 $$
  |A- \lambda B| = 0
 $$
This is how we chose $\lambda$. We can now go back in the other direction as we also know that :
$$
 \bm  m^T \bm v_s = \bm q^T \bm f
$$
and
$$
 \bm m^T = \frac{1}{\lambda} \bm q^T A = \bm q^T B
$$
so
$$
 \bm q^T( A \bm v_s - \lambda f) = 0, \bm q^T ( B \bm v_s - \bm f) = 0
$$
If we very simply sketch what was happening to our simple shallow water, our eigenvalue problem is:
$$
 \bm q^T \begin{pmatrix} u-\lambda & h \\ g & u - \lamdba \end{pmatrix} = \bm 0^T
$$
 So need
 $$
  | A - \lamdba B| = ( u- \lambda)^2 - g h = 0 \implies \lambda = u \pm c, c = \sqrt{gh}
 $$
 We also have:
 $$
  \begin{pmatrix} 1 & q \end{pmatrix} \begin{pmatrix} \pm c & h \\ g & \pm c \end{pmatrix} = \bm 0 ^T
 $$
  so $\bm q^T = \begin{pmatrix} 1 & \pm \frac{c}{g} \end{pmatrix}$
  $$
   \bm q^T ( A\bm u_s - \lambda \bm f) = ( 1 \pm \frac{c}{g}) ( A \bm u_s - \lambda \bm 0) = 0
  $$
  Use $h = \frac{c^2}{g}$ and solving this gives:
  $$
   0= \begin{cases} (u +c ) \frac{d}{ds} ( 2c + u) & \lambda = u + c\\
   (u-c) \frac{d}{ds} ( 2  c- i) & \lamdba = u -c\end{cases}
  $$
  $u \pm 2c$ = const on $\lambda = u \pm c$\\\\
  \textbf{Higher-order equations}:
  $$
  \frac{\partial^2 \eta}{\partial t^2} + \gamma \frac{\partial^2 \eta}{\partial x^2} = 0
  $$
  Let $v = \frac{\partial \eta}{\partial x}$ and $w = \frac{\partial \eta}{\partial t}$ then:
  $$
   \frac{\partial v}{\partial t} - \frac{\partial w}{\partial x} = 0
  $$
  $$
   \frac{\partial w}{\partial t} - \gamma \frac{\partial v}{\partial x} = 0
  $$
  for $\lambda = \pm \gamma^{\frac{1}{2}}$.
  \subsubsection{Implications of being hyperbolic}
  What makes things hyperbolic? Well we required that $|A- \lambda B| = 0$ and it is hopefully obvious that if $\lambda$ is representing a trajectory in space time then we need real eigenvalues.
  $$
  \bm q^T ( B \bm u_s - \bm f ) = 0
  $$
  If we want to form the ODE along these characteristics we need to know the eigenvectors so we can form two ODEs. So we need a complete set of eigenvectors. There will be some ocassions with repeated eigenvalues but still have a full set of eigenvectors. If we cannot find a full set of eigenvalues the problem is not hyperbolic. The eigen values (velocity) are telling us how the information is propogating. There is an anaolgy between shallow water and compressible gas:\\
  For compressible gas we ahve the Mach number:
  $$
  M = \frac{\text{velocity}}{\text{speed of sound}}
  $$
In the case of shallow water the equivalent is the Froude number :
$$
 F_r = \frac{\text{fluid velocity}}{\text{wave speed}} = \frac{u}{c} = \frac{u}{\sqrt{gh}}
$$
We are cosndiering if the information can propogate in two directions or just one direction. If the fluid is moving fast enough then the waves will only every propogate in the direction of the flow.\\\\
If $|F_r| <1$ then waves can propogate in both directions (subcritical flow)\\
If $|F_r|>1$ the waves can only propogate in one direction ( supercritical flow)\\
If $|F_r| =1$ then flow is critical so either $\lambda^+ = u + c$ or $\lambda_ = u -c$ are zero. One ( or both) of the the waves do not propogate relative to the observer.\\\\
Imagine we know the initial speed and depth at two points then we can follow the characteristics leaving these points and meeting at a point. We can find the new speed and depth by considering the conservation along these characteristics :
$$
 u_1 + 2 c_1 = u_a + 2 c_a, u_1 - 2c_1 = u_b - 2c_b
$$
so 
$$
 u_1 = \frac{1}{2} ( u_a + u_b) + (c_a - c_b)
$$
$$
 c_1 = \frac{1}{4} ( u_a - u)b + \frac{1}{2} (c_a + c_b)
$$
Zone of dependance fo $u_1,c_1$ is everything between $u_a, c_a$ and $u_b, c_b$. The zone of influce is future times between $\lambda_+ = u+c$ and $\lambda_- = u-c$ characteristics.
\subsubsection{The Saint-Verant dam-break problem}
This is an example of the Riemann problem. If we have a dam of depth $H$ and length $L$ if we move one of the walls away at an accelerating rate $V$, at what point does the water not keep up with the dam.
On $C^+$ we have $\lambda_+ = u+c$ and at initial time $u + 2c = 2 c'$. So as $c= \sqrt{gH} >0$ so the maximium value that $u$ can have is $2c'$ when we ahve $c=0$ and therefore $h=0$. So when $V = 2c'$ the water will no longer be able to keep up. Everywhere before that point the $C^+$ characteristics will be able to reach the dam, so we will be able to figure out the depth of the water at the dam by using $u= V$ and $u+c = 2c'$. Therefore, we can figure out the value of $\lamdba_-$ at every point on the dam, and therefore as $u+2c$ is constant everywhere and $u-2c$ is constant along $C^-$ we have $u$ and $c$ are constant along $C^-$ so the line is straight. It is nice to think of $C^-$ as $C^+$ characteristics reflected off the dam.\\\\
If we shrink this accelerating movement down to nothing to model an instaneous removal of the dam. Then the information about the removal of the dam will move backwards along characteristic $\lamdba_- = - c'$. We will have squeezed every point on the exapnsion fan to originate from the same point and as they are all straight this means the characteristics are $\lambda_- = \frac{x}{t}$. We can start to build up the solution as along $\lambda_- = u-c$ we have $u-2c = -2c'$ in the quiecent region and along $\lambda_+ = u+c$ we have $u+2c = 2c'$ everywhere in the fluid. The fastest moving fluid is at the front with $u = 2 c'$ form limit $h \rightarrow 0, c \rightarrow 0$ so on $\lambda_+ = u + 2c = 2c'$ so $u \rightarrow 2 c'$. In the expansion fan $\lambda_- = \frac{x}{t}$ along which $u-c = \lambda_- = \frac{x}{t}$. So as $u -c = \frac{x}{t}$ and $u+ 2c = 2c'$ we have $3c = 2 c' - \frac{x}{t} \implies h = \frac{1}{9g} ( 2c' - \frac{x}{t})^2$.\\\\
But do we see this for shallow water. We predicited its maximium velocity at the thinnest point. We have also violated shallow water. At $t=0$ we have violated the assumption that $h$ and $u$ change only over length scales large compared to $H$. We have also ignored the desnity of the fluid above and ignored viscosity. \\
What would be the effect of drag on the boundary. We have been assuming that $u \rightarrow 2 c'$  as $h \rightarrow 0$ with the st. vennant solution. In any real solution this is going to cause a problem, as soon as this causes a problem we are going to break shallowater as shallow ater requires tat the front goes smoothly to zero and $w$ is asymptotically small. If we actually have the velocity going to zero because of any drag what so ever at the front we will end up with a stagnation point forming at the front. This means a discountity in slope must be forming at the front then this requires vertical acceleration of the fluid and this violates the shallow ater assumptions. If we did an expirment of water into air you would find that you don't get an infintely thin film travelling on, though it is a thin front due to the large density difference.\\\\
\textbf{Moving dam problem}\\
Lets have the speed of the dam being less than $2 c'$. Now in this case when the charcteristics make it to the dam they are still going to have a non-zero height so the reflected ones can propogate backwards relative to the dam, and as everythign is propogating at constant speed, then all of these will be reflected with the same speed and same depth. So we get three areas the undisturbed fluid, the rarefaction wave travelling back and a region of constant depth at the front.\\\\
Because of the constant depth region the Froude number of rthe dam is constant $Fr_f = \frac{u_f}{c_f}$. This can be used to characterise the problem.\\\\
What if we were using a finite volume of fluid rather than the fluid behind the dame extending forever? Then you have the reflected rarefaction wave coming back towards the dam.
\subsection{Gravity Currents}
Brief description: They have some common features, they have a well defined front with large density gradients and there is often mixing behind the front at $Re >>1$. The no-slip boundary condition causes the ambinent fluid to be over run. THe no-slip condition means the fluid has to roll over the ground so it ends up over running some lighter fluid which sets up a static instability. THis means it ends up convecting and forming 'lobes and clefts'. So if we look from the front we start seeing wave like lobes and clefts. This is caused by the lighter fluid trying to rise up through the front. This part of the current is often refered to as the head, and in this region shallow water odes not strictly apply but it does work well in the 'tail'.\\\\
Frontogensis - the formation of the front. Let's consider what would happen if we had a fluid that started dense on the left and as we moved to the right it gets less dense iwth $\frac{\partial \rho}{\partial x}|_{t=0} \neq 0$. If we make our lives simple and make it 2D then in the Bousseqs case the vroticity equation gives:
$$
 \frac{\partial \omega}{\partial t} + ( \bm u \dot \nabla) \omega = - \frac{g}{\rho_0}\frac{\partial \rho}{\partial x}
$$
           \section{Example sheet 1&2}
           Can just state the vorticity equation in 2D and bousinesq case:
           $$
            \frac{\partial \zeta}{\partial t} + (\bm u \cdot \nabla) \zeta = - \frac{g}{\rho_0} \frac{\partial \rho}{\partial x} + \nu \nabla^2 \zeta
           $$
           with $\zeta = - \nabla^2 \psi$. If we wasn't bousinesq we have to worry about how we are doing the derivaties as we are taking teh curl of $\rho \frac{\partial u}{\partial t}$ and we will also get a $\nabla \rho \times \nabla p$. If it was 3D we would also have a $(\bm \zeta \cdot \nabla) \bm u$ term. It is worth actually doing this computation as it looks quite straight forward but unless you remember the write order it turns out to be a bit more of a can of worms.\\\\
           $$
           KE = \int_{-H}^{\eta} \frac{1}{2} \rho ( u^2 + v^2 + w^2)
           $$
           $$
           PE = \int_{-H}^{\eta} \rho g z dz - \int_{-H}^{0} \rho g z dz
           $$
           $$
            \bm F = \int_{-H}^{\eta} u(p + \rho gz) dz
           $$
           but what we really want is $KE = \int_{-H}^0 \frac{1}{2} \rho (u^2 + v^2 + w^2) dz$ as the $u, v, w$ already have $\eta$ baked in so the contribution from the bit at the top is another order of $\eta$ further down. No need to remember the results of these integrals as the book work parts of the question will almost always end in book work to make his marking easier.\\\\
           $$
            ( 2 \frac{\partial}{\partial \chi} + i \frac{\partial^2}{\partial \xi^2}) \psi_0 = f(\chi)
           $$
           The reason we can discard the $f(\chi)$ with $\zeta, \chi$ in direction of group velocity and $\xi$ in the prependicular direction. Is because we assume a plane solution in the $\zeta$ direction and so the integrating up from $( 2 \frac{\partial}{\partial \chi} + i \frac{\partial^2}{\partial \xi^2}) \frac{\partial \psi_0}{\partial \xi} = 0$ would average to zero.\\\\
           For reflection questions draw a diagram with two incoming waves, and draw on all the wavevectors, wavelengths and group velcoity. Then resolve velocities so you have zero normal velocity at the surface. Express the distance between the points of impact interms of both wavelengths to get the relationship between the two of them. This is some standard exam question book work so make sure you can do this. Can quote:
           $$
            | \bm F| = \bar E |\bm c_g|
           $$
           We also know that the flux of energy is conserved per wavelength. \\\\
           When thinking about attractors we can often simplify the algebra by noticing a symmetry, also standard practice to define $\beta = \frac{1}{\tan \theta}$ and therefore the line though a point has equation: $z= z_0 + \beta (x- x_0)$. In order to figure out the magnification factor either you figure out the slopes at the positions of the reflections and then use the $\gamma$ expressions or you could use the iterative forumla for getting back to the same position.\\\\
           Make sure to always draw on wavevector to diagrams remembering it is prependicular to the group velocity, and it is often tidier to write $\bm k = |\bm k| (\cos \theta, - \sin \theta)$ than $\bm k = (k,m)$.\\\\
           Refamilarise with stokes boundary layer condition from Fluids II. Remember form of solution.\\\\
           When finding expressions for $|\bm k|$ to find envelopes remember that we need $|\bm k|$ so there are some values of $\theta$ that aren't possible. Equally remember to think about the direction of phase velocity of stationary waves, it must be in the same direction as the observer in order for them to exist.
           \section{Lecture 16}
           \subsection{Early models for gravity currents}
           The starting point for a lot of this is dimensional analysis. At high Reynolds number we will ahve $u_f^2 \sim gH$ and as we found with the St. Venant solution we found that $u_f^2 = 2 gH$ but this require $h \rightarrow 0$ smoothly. Generally we find this is not the case as at the boundary we have factors like drag or surface tension, so the region close to the boundary will be non-hydrostatic as we need to have a vertical velocoity so we violate shallow ater at the front (doesn't mean it cant be applied elsewhere). Experiementally it has been found that $F_r = \frac{u_f}{\sqrt{g h_f}} = f(\frac{h}{H})$.
           \subsection{Cavity Flows}
           First understood in 1960s by Benjamin. He said lets imagine we have a duct with a height $H$ and high Reynolds number. We assume that out at infinty we have velocity zero and we assume we have a cavity extending into the duct with height $h$ and density of 0 and moving left with velocity $U$. Therefore, the volume flux of the fluid must be equal to the volume dispalced by the cavity in the opposite direction. It is easier to think about this in frame of reference of the front of the cavity. We pick a control volume with boundaries far enough from the front in either direction that we can assume the velocity is parallel to the duct. By continuity of volume flux we must have $u_2 = \frac{UH}{H-h}$ with $u_2$ being the speed under the cavity. Let us assume it is inviscid, irrotational and energy conserving and steady. This means we can use Bernoulli:
           $$
            \frac{1}{2} |\bm u|^2 + \rho g z + p = const
           $$
           consider pressure at the top of the fluid before the cavity and under the cavity on the same steamline.           $$
            p_1 = p_0 - \frac{1}{2} \rho U^2
           $$
           $$
           p_2 = p_0  = p_0 + \rho g h - \frac{1}{2} \rho u_2^2
           $$
           The first of these is the pressure of the cavity and the second is given by the Bernoulli constant \\
           $$
            u_2^ = 2 gh
           $$
           "Flow force" on the upsteam side is:
           $$
           \int_0^H p dz + \rho U^2 H = p_0 H + \frac{1}{2} \rho U^2 H + \frac{1}{2} \rho g H^2
           $$
           "Flow force" on the downstream is:
           $$
            2\rho g h( H-h) + p_0 H + \frac{1}{2} \rho g( H-h)^2
           $$
           So these balancing gives: $h^3 = \frac{1}{2} Hh^2$. This clearly gives two energy-conserving solutions $h=0$ and $h = \frac{1}{2} H$.\\\\
        If we allow some turbulence to develop then we can get an adverse pressure gradient at the back of the head. We want to draw our control volume far enough downstream of these turblence that all the flow is parrallel again. Now the height is $h - \Delta$ which is effectively the loss of pressure caused by the energy loss. This adverse pressure gradient often leads to instabilies, potentially to rotational flow and dissipation of energy. Provided we make the box long enough we can ignore all that and just think of the head loss ( loss of height of the cavity). Now
        $$
         u_2^2 = 2g(h- \Delta))
        $$
        This is our energy loss represented as a reduction in hydrostatic head. Unless there is an internal source of energy we must have $\Delta \geq 0$. Now work through the same maths as before to find:
        $$
         \Delta = \frac{(H- 2h)h^2}{2(H-h) (H+h)}
        $$
        Note $h \rightarrow \frac{1}{2}H \implies \Delta 0$ and as $\frac{h}{H} \rightarrow 0 \implies \Delta \rightarrow \frac{h}{2H} \rightarrow 0$\\\\
        We can write the Frude number as:
        $$
         F_f = \frac{u_f}{ \sqrt{g h_f}} = ( \frac{(1- \frac{h}{H})(2- \frac{h}{H})}{1 + \frac{h}{H}})^{\frac{1}{2}}
        $$
        \subsection{Bonden and Meiburg (2013)}
        They said we don't need to worry about a lot of these things (e.g. things being hydrostatic) we just need to take a control volume and a current. In the current we have density $\rho_0 + \rho'$ and in the ambinent fluid we have desnity $\rho_0$. We have height $h$ of our cavity and then a boundary layer of height $\delta$ and then we will let $\delta$ shrink to zero. We want to think about the circulation
        $$
         \Gamma = \int_V \omega dV = \int_S \bm u \cdot d \bm x
        $$
        This time we have a Baroclinic torque associated with the density difference across the boundary:
        $$
         G_{\omega} = - \int_V \frac{g}{\rho_0} \frac{\partial \rho'}{\partial x} dV = g' h
        $$
        The flux of vorticity:
        $$
         F_{\omega} = \int_{h- \frac{\delta}{2}}^{h + \frac{\delta}{2}} \omega u dz
        $$
         we have $\omega \sim - \frac{u_2}{\delta}$ and $\bar u \sim \frac{1}{2} u_2$ so therefore:
         $$
          F_{\omega} = - \frac{1}{2} u_2^2 \text{as} \delta \rightarrow 0
         $$
         We have continuity $u_1 H = u_2 (H- h)$ which leads to a different Froud number:
         $$
         F_H = \frac{u_f}{\sqrt{g' H}} = ( 2 \frac{h}{H})^{\frac{1}{2}}( 1- \frac{h}{H})
         $$
         $$
          F_f = \frac{u_f}{\sqrt{g'h}} = 2^{\frac{1}{2}} ( 1- \frac{h}{H})
         $$
         In exam questions we can get a long way by saying the Froude number is some fucntion of the final depth and this works really really well for predicting the overall behaviour of the gravity current.
         \subsection{Simplified models of gravity currents}
         \subsubsection{Integral model}
         This is simply a series of rectangles with conserved volumes. Sometimes we go a step further and pick another shape that more accurately approximates the shape. We will be using this plus a boundary condition on the front based on a constant Froude number.\\\\
         So we will start off with a region $L(t)h(t) = L_0  h_0$. If we now apply a Froude number for the front:
         $$
          u_f = F_f \sqrt{g' h} = \dot L
         $$
         then 
         $$
          \dot L = F_f \sqrt{ g' \frac{h_0 L_0}{L}} \sim t^{\frac{2}{3}}
         $$
         Can do similar calculation for channel with width $b(x)$.\\\\
         We will have the front moving linearly with time $L \sim t$ until the initial rarefaction catches back up with the front after reflecting off the back wall when it will move with $L \sim t^{\frac{2}{3}}$
         \subsubsection{Entrainment into shallow water}
         What is all this mixing doing and how is it changing the shallow water equations. At high Re it is often turbulent which may be from the roughness of the lower boundary and/or shear instability at the interface. \\\\
         Let us imagine we have a shallow water layer, a control volume and densities $\rho_0$ and $\rho_1$ and some process that is causing mixing across the interface. This process is entraining fluid with $\omega_e$ into the lower layer and detraining fluid with $\omega_d$ into the upper layer. We will assume the mizing within layers keeps density constant over the depth of the layer so $\frac{\partial \rho}{\partial z} = 0$ and $\rho = \rho(x)$ within a layer. We will asume the upper layer is very deep so $\rho_2 = $const.\\\\
         Momentum equation with no drag:
         $$
          \frac{\partial}{\partial t} ( \rho_1 h_1 u_1) + \frac{\partial }{\partial x} ( u_1 \rho_1 h_1 u_1 + \frac{1}{2} (\rho_1 - \rho_2) g h_1^2) = \oemga_e \rho_2 u_2 - \omega_d \rho_1 u_1
         $$
         We can ignore $\oemga_e \rho_2 u_2$ as $u_2 = 0$.\\
         Volume conservation gives:
         $$
          \frac{\partial h_1}{\partial t} + \frac{\partial}{\partial x} (u_1 h_1) = \omega_e - \omega_d
         $$
         If we assume 'linear mixing'
         \section{Lecture 17}
         Note the following
         $$
          \frac{\partial}{\partial t} ( \rho_1 h) = \rho_1 \frac{\partial h}{\partial t} + h \frac{\partial \rho_1}{\partial t}
         $$
         $$
         \frac{\partial}{\partial t}( \rho_1 h_1 u_1) = \rho _1 u_1 \frac{\partial h}{\partial t} + \rho_1 h_1 \frac{\partial u_1}{\partial t} + h_1 u_1 \frac{\partial \rho_1}{\partial t} 
         $$
         define:
         $$
          B = \begin{pmatrix} h & \rho_1 & 0\\
                  0 & 1 & 0 \\
          h_1u_1 & \rho_1 u_1 & \rho_1 h_1 \end{pmatrix}
         $$
         $$
         \bm v =
         \begin{pmatrix}
                 \rho_1 h_1 u_1
         \end{pmatrix}
         $$
         so we can simplify:
         $$
          \frac{\partial h_1}{\partial t} + u_1 \frac{\partial h_1}{\partial x} + h_1 \frac{\partial u_1}{\partial x} = \omega_e - \omega_d
         $$
         $$
          g' = g'(x,t) = \frac{\rho_1 - \rho_2}{\bar \rho} g
         $$
         $$
          \frac{\partial g'}{\partial t} + u_1 \frac{\partial g'}{\partial x} = -g' \frac{\omega_e}{h_1}
         $$
         $$
          \frac{\partial u}{\partial t} + u \frac{\partial u}{\partial x} + g' \frac{\partial h}{\partial x} + \frac{1}{2} h \frac{\partial g'}{\partial x} = - \omega_e \frac{u_1 - u_2}{h}
         $$
         $$
          \begin{pmatrix}
                  u_1 & 0 & 0\\
                  0 & u_1 & h_1 \\ \frac{1}{2} h_1 & g' & u_1 \end{pmatrix} \begin{pmatrix} g'\\ h_1 \\ u_1 \end{pmatrix}_x + \begin{pmatrix} 1 & 0 & 0\\ 0 & 1 & 0 \\ 0 & 0 & 1 \end{pmatrix} \begin{pmatrix} g' \\ h_1 \\ u_1 \end{pmatrix}_t = \begin{pmatrix} - g' \frac{\omega_e}{h}\\ \oemga_e - \omega_d \\ - \omega_e \frac{u_1}{h} \end{pmatrix}
         $$
If we define a characteristic $s = x - \lambda t$ this reduces to an ode when:
$$
 | \begin{pmatrix} u_1 - \lamdba & 0 & 0 \\
         0 & u_1 - \lambda & h_1 \\
 \frac{1}{2} h_1 & g' & u_1 - \lambda \end{pmatrix} | = (u_1 - \lambda)((u_1 - \lambda)^2 - g' h) = 0
$$
So we get three characteristics $\lambda_{\pm} = u_1 \pm c$ and $\lambda_{g'} = u_1$ and $c = \sqrt{g' h}$ with ODE $$
\frac{d g'}{ds} = - g \frac{\omega_e}{h_1}
$$
for charactersitics $\lambda_{g'} = u_1$ (left as an excerise for the rest).\\\\
It is very easy to incorperate this into integral models, one of the questions we might ask is what sets our entrainment velocity ($\omega_e$ or $\omega_d$) ( possible exam question). We are only going to worry about $\omega_e$ and not $\omega_d$ (this second one is left for exam questions).\\\\ Using the Buckingham Pie theorem. This quanity will be governed by some number of dimensionless parameters e.g. Froude number $\frac{u}{\sqrt{g' h}}$, Reynolds number $\frac{uh}{\nu}$, schmidt number $\frac{\nu}{\kappa}$, \textbf{Richardson number} $ \frac{g' l}{u'^2}$. Make assumption that $l \sim h$ (depth) and that $u' \sim u$  not tha they are equal but that they scale with them - the exact ratio of these will be to do with things like how rough is the channel which is encapsulated by the drag coefficent $C_T$. We find that provided $Re >>1$ it doesn't matter that we have replaced the Froude number with the drag coefficent and we can also hope that the schmidt number is not important (which is how we used to think but now we know it is more important) so we take the assumption that for sufficently large $S_c$ (schmidt number) we can ignore it. So maybe:
$$ \alpha = \frac{\omega_e}{u} = f( R_i , C_T)
$$
Luckly we are only dealing with a small range of richardson numbers in a shallow flow. We find that for small richarson number it increases linearly and then it sort of asymptotes and it isin the asymptotic region that we have gravity currents. So we have an almost constant entraining coefficent. For the moment we will treat $\omega_i = \alpha u$ with constant $\alpha$. It turns out mixing difficult or quite easy depending on the initial conditions (e.g. if you start with dense stuff on top it is easy and if it is on bottom then it is hard). We can think of mixing of stirring and diffusion. Stirring requires KE from somewher, and diffusion is the result of chemical potential energy and maximising entropy. If we just do horzontal stirring there is no potential penalty for switching parcels of fluids. If we try pulling tendrils of fluid up each time we need potentially energy input, however some ways we can do this easily for instance waves which lift up fluid in a reversible way. So once we have uplifted the fluid mixing horizontally doesn't have any potential cost so if we have an instability type thing we can mix horizontally.\\\\
We could have our gravity current running down a slope which changes our momentum equation:
$$
\frac{\partial u}{\partial t} + u \frac{\partial u}{\partial x} = - g \cos \theta \frac{\partial h}{\partial x} + g \sin \theta}
$$
Imagine if we had a gravity current moving along that then collides with an inclined slope.\\\\
If we had a gravity current going down a slope then without entrainment it would make no sense the intergal model would get a taller and narrower rectangle clearly violating shallow water. 
\section{Lecture 18}
\subsection{Single-layer Hydraulics}
Assume we have some surface with topography given by $H(x)$ and a shallow water flow across it with height $h(x)$, this flow is through a channel with width $b(x)$. The velocity of the fluid is $u(x)$. Assume the pressure field is hydrostatic which is equivalent to saying the vertical accelerations are small.\\\\
We are going to make use of various physical concepts. \\First and foremost continuity which equates conservation of volume flux $Q = uhb = $const. (in exam questions sometimes this hasn't been constant)\\
Momentum - this is too hard for us to make progress in these complex geometries as the pressure distribution on the boundaries will have a streamwise component.\\
Energy - assuming no dissipation (though has reared its head in exam questions)\\\\
Note: It is not uncommon for people to misunderstand the Bernoulli equation which is a statement about energy and not about momentum the first thing to note is:
$$
 \nabla ( \frac{1}{2} |\bm u |^2) = \bm u \times \bm \omega + ( \bm u \cdot \nabla) \bm u
$$
so the momentum equation becomes
$$
 \frac{\partial \bm u}{\partial t} + \nabla ( \frac{1}{2} | \bm u|^2 + \frac{P}{\rho} + g z) = \bm u \times \bm \omega
$$
Energy: $\bm u \cdot $ "momentum equations" gives:
$$
 \frac{1}{2} \frac{\partial | \bm u |^2}{\partial t} + ( \bm u \cdot \nabla) ( \frac{1}{2} | \bm u |^2 + \frac{p}{\rho} + g z) = 0
$$
So for a steady state we have the Bernoulli potential 
$$
 B = \frac{1}{2} | \bm u|^2 + \frac{p}{\rho} + g z
$$
which is advected along the fluid velocity.\\\\
The momentum equation can be written in terms of this:
$$
 \frac{\partial \bm u}{\partial t} + \nabla B = \bm u \times \bm \omega
$$
but this is isnt' that helpful much more helpful in the energy equation:
$$
 \frac{\partial}{\partial t} ( \frac{1}{2} | \bm u |^2) + (\bm u \cdot \nabla ) B = 0
$$
 so for steady flow $B(\psi)$ is conserved along streamlines.\\\\
 If $\bm \omega = 0$ we can return to the momentum equation to get:
 $$
  \frac{\partial \bm u}{\partial t} + \nabla B  =0 \implies \nabla ( \frac{\partial \phi}{\partial t} + B) = 0
 $$
 so $\frac{\partial \phi}{\partial t} + B =$ constant. For hydraulics we will simply be considering steady flows with $B =$ constant.
 \subsubsection{Specific energy}
 This is just a way of rewritting the Bernoulli potential a little bit differently. 
 $$
  E = \frac{1}{2} \frac{Q^2}{b^2 h^2 g} + H + h = const
 $$
 Now consider how this changes as funciton of $h$. As $h \rightarrow 0$ it asymptotes to infintiy and as $h \rightarrow \infty$ it tends towards being linear, so it must have a stationary point at some point.\\\\
 For a channel of uniform width with $b(x) = $const. with varying depth. Now we have:
 $$
 E- H(x) = \frac{ Q^2}{b^2 h(x)^2 g} +h(x)
 $$
 So we can just consider how $H(x)$ varies along the channel, as $E$ is constant. Now draw the $E-H$ curve that we discussed above with its stationary point. Now start labelling the points on this graph and consider how the value of $E-H$ changes as you move along the steamline. So if you move towards a bump $H$ will increase so you would move down the curve to the left, when you move past the maximum of the bump you will start to move back up the curve till you reach your original point. For streamlines too close to the bump you cannot move low enough on the curve as $E-H$ would be below the minimum of $\frac{ Q^2}{b^2 h(x)^2 g} +h(x)$. There is a critical staionary streamline that will exactly hit the minimium of $E-H$ and then has the option of following the left hand of the curve or the right hand, either corresponding to a symmetrical solution or one that rapidly decreases in depth after the bump.
 \subsubsection{Informatoin propagation}
 $$
  \frac{\partial E}{\partial h} = - \frac{Q^2}{b^2 h^3 g} + 1 = \frac{- u^2}{hg} +1 = 1 - F_r^2
 $$
 $$
  F_r = \frac{u}{c} = \frac{u}{\sqrt{gh}}
 $$
 Note that $\lambda_+ = u +c, \lambda_- = u -c$ so $\frac{\lambda_+ \lambda_-}{c^2} = \frac{u^2}{c^2} - 1$ so:
 $$
  F_r = \frac{\lambda_+ \lamdba_-}{c^2} +1
 $$
 So we can see if we draw our specific energy curve then on the left of the minimum we ahve $Fr>1$ supercritical flow and to the right of the minimum we ahve $Fr<1$ subcritical flow and at the minimum we ahve $Fr=1$.\\\\
Often these bumps are referred to as sills or weirs.
\subsubsection{Shocks}
If we have a flow that is subcritical before the bump and then have a supercritical flow after, and for instance there is a smaller bump further downstream we will need to have a hydraulic jump formed.\\\\
In the subcritical region we have $\lambda_-$ characteristic pointing up stream and $\lmabda_+$ points downstream. As we appraoch the top of the bump the $\lamdba_-$ characteristic shrinks to nothing. In the supercritical region to the right of the bump we have $\lamdba_-$ and $\lambda_+$ pointing downsteam. Before the second smaller bump we must have a subcritical region with a $\lamdba_-$ characteristic pointing downstream. So we must have charactersitics of the same type intersecting giving a shock. This jump will dissapate some energy so we will be able to move onto a lower specific energy curve. MSo it jumps from the left hand side of the minimium to the right hadn side of the minimum on a lower curve.\\\\
We have constant $E$ so:
$$
 \frac{d E}{dx} = \frac{\partial E }{\partial x} + \frac{\partial E}{\partial \phi} \frac{d Q}{dx} + \frac{\partial E}{\partial b} \frac{d b}{dx} + \frac{\partial E}{\partial H} \frac{dH}{dx} + \frac{\partial E}{\partial h} \frac{dh }{dx} = 0
$$
As in this case we have assumed $b$ is constant, Q and E are constant so:
$$
 \frac{\partial E}{\partial H} \frac{d H}{dx} + \frac{\partial E}{\partial h} \frac{d h}{dx} = 0
$$
At $\frac{dH}{dx} = 0$ either $\frac{dh}{dx} = 0$ (giving a symmetric solution) or $\frac{\partial E}{\partial h} = 0 \implies Fr = 1$.\\\\
If we had a $Fr >1$ then as it climbs up the slope it is going to become thinner and slower moving. So there are cases where if it is not sufficently supercritical to make it over the bump so there is a jump where it goes subcritical. \\
Across the jump shallow water is violated, momentum is conserved, volume flux is conserved and energy is dissapated.\\
\textbf{Differences between Bousinesq and non-Boussinesq}\\
Are the fluids miscible? and also need to be concerned about the pressure difference across the jump.\\\\
This is not too bad for a stationary hydralic jump ( a moving one is often called a bore). We can write done stationary jump conditions:
$$
u_l h_l = u_r h_r
$$
$$
u_l^2 h_l + \frac{1}{2} gh_l^2 = u_r^2 h_r + \frac{1}{2} g h_r^2
$$
If jump is stationary the pressure distribution in the upper layer must be hydrostatic.\\
Typically takes one of two forms:
If only a small amount of energy needs to be dissipated you get an "undular jump" with nonlinear waves that are stationary in the frame of the jump. If too much neergy for nonlinear waves then you have a sharp turbulent jump.\\\\
For a "bore" ( a moving hydraulic jump) a Boussinesq fluid needs to accelerate ambient fluid out of the way leading to non-hydrostatic pressure across the jump.
\section{Lecture 19}
\subsection{Jets, plumes and thermals}
\subsubsection{Connections with shallow water}
$$
u(x, z,t) = \bar u( x,t) \phi(\frac{z}{h})
$$
$$
\bar u(x,t) = \frac{1}{h} \int_0^{\infty} u(x,z,t) dz
$$
with $\eta = \frac{z}{h}$ so:
$$
 \int^{\infty}_0 \phi(\frac{z}{h}) dz = h \int^{\infty}_0 \phi(\eta) d\eta = h
$$
$$
\bar \rho(x,t) = \rho_0 + \frac{1}{h} \int^{\infty}_0( \rho(x,z,t) - \rho_0) dz = \rho_0 + ( \bar \rho(x,t) - \rho_0) \psi( \frac{z}{h})
$$
$$
 \int^{\infty}_0 \psi( \frac{z}{h}) dz = h, \int_0^{\infty} \psi(\eta) d \eta =1
$$
This is only a useful choice\\\\
\textbf{Top-hat profile}
$$
 \phi = \begin{cases} 1 & \eta \leq 1\\ 0 & \eta >1 \end{cases}
$$
$$
 \psi = \begin{cases} 1 & \eta \leq 1 \\ 0 \eta >1 \end{cases}
$$
Momentum flux  is given by:
$$
 M(x,t) = \int_0^{\infty} \rho u^2 dz = \rho_0 \int_0^{\infty} u^2 dz = \chi \rho_0 \bar u^2 h
$$
Here $\chi$ is the shape factor and satisfies:
$$
 \chi = \int_0^{\infty} ( \phi(\eta))^2 d \eta \geq ( \int_0^{\infty} \phi(\eta) d\eta)^2 \geq 1
$$
So for top-hat profile $\chi = 1$ and for nay other profile we have $\chi >1$ for any real flow $\chi \geq 1$ even if only a little bit.\\\\
The buoyancy flux is given by:
$$
 F(x,t) = \int_0^{\infty} \frac{\rho - \rho_0}{\rho_0} g u dz = \gamma \frac{\bar \rho - \rho_0}{\rho_0} g \bar u
$$
$$
 \gamma = \int^{\infty}_0 \phi(\eta) \psi(\eta) d \eta
$$
Schwarz inequality tells us:
$$
 |\gamma|^2 \leq \chi \int^{\infty}_0 \psi^2 d\eta \geq 1 \implies \gamma \geq 1
$$
\textbf{Mention in examples class this logic is off?}\\\\
Take $h$ to be height of shallow water along a slope of angle $\theta$. Assume it is predominatly hydrostatic as for shallow water with:
$$
\frac{\partial p}{\partial z} = \rho g \cos \theta
$$
across the thickness of the layer. So we are saying the pressure within the layer at a given $x$ is the same as the pressure outside the layar at a given $x$ with $x$ taken along the slope.\\\\
We may not have a well defined interface but entrainment can still increase the volume /decrease the density of the flow. We will still have $\int_0^{\infty} \phi(\eta) d \eta = \int_0^{\infty} \psi( \eta ) d\eta = 1$ by defintion, therefore the volume chang is represnted by an increase in $h$.
$$
 \omega_{e} = \alpha bar u
$$
$\alpha$ is the entrainment coefficent or the 'Batchelor entrainment'. As already discussed $\alpha = \alpha (R_i , R_e, S_c, F_r,..., \theta)$ and for high enough $Re$ and $S_c$ we have $\approx \alpha(R_i, \theta)$. For typical flows we are looking at $R_i \sim \frac{1}{F_r^2}$ and $\cos \theta$ plays a role in $F_r$ so maybe we can take the approximation $\alpha \approx \alpha(\theta)$, but actually if we let the richarson number go all the way to zero then the entrainment is different.\\\\
\textbf{Conservation of volume}:
$$
 \frac{\partial h}{\partial t} + \frac{\partial}{\partial x} ( \bar u h) = \omega_e = \alpha | \bar u|
$$
\textbf{Conservation of mass}: Recall
$$
\int_0^{\infty} \rho u dz = \int^{\infty}_0 ( \rho_0 + ( \bar \rho - \rho_0) \psi) \bar u \phi dz = \rho_0 \bar u h + \gamma ( \bar \rho - \rho_0 ) \bar u h 
$$
let $\Delta \rho = \bar \rho - \rho_0$ so:
$$
 \frac{\partial}{\partial t}( \bar \rho h) + \frac{\partial}{\partial x} ( \rho_0 \bar u h + \gamma \Delta \rho \bar u h) = \alpha \rho_0 | \bar u|
$$
Now expand and make use of the conservation of mass:
$$
 \frac{\partial \Delta \rho}{\partial t} + \frac{\gamma -1}{h} \Delta \rho frac{\partial}{\partial x} ( \bar u h) + \gamma \bar u  \frac{\partial}{\partial x} \Delta \rho = - \alpha \frac{\Delta \rho}{h} | \bar u|
$$
let $g' = \frac{\Delta \rho}{\rho_0} g$ gives:
$$
 \frac{\partial g'}{\partial t} + \gamma \bar u \frac{\partial g'}{parital x} + \frac{\gamma -1}{h} g' \frac{\partial}{\partial x} ( \bar u h) = - \alpha \frac{g'}{h} | \bar u |
$$
In the Boussinesq case the momentum equation becomes:
$$
 \frac{\partial}{\partial t} ( \rho_0 \bar u h) + \frac{\partial}{\partial x} ( \rho_0 \chi \bar u^2 h + \frac{ 1}{2} \bar \rho g' h^2 \cos \theta) = g' h \sin \theta
$$
so this simplifies to
$$
\frac{\partial \bar u}{\partial t} + \chi \bar u \frac{\partial \bar u}{\partial x} + g' \cos \theta \frac{\partial h}{\partial x} + \frac{1}{2} h \cos \theta \frac{\partial g'}{\partial x} = g' \sin \theta - \alpha \frac{\bar u | \bar u|}{h}
$$
Therefore in matrix notation:
$$
 \begin{pmatrix} \bar u & h & 0 \\ g' \cos \theta & \chi \bar u & \frac{1}{2} h \cos \theta \\ ( \gamma -1) g' \frac{\bar u}{h} & ( \gamma -1) g' & \gamma \bar u \end{pmatrix} \begin{pmatrix} h \\ \bar u \\ g' \end{pmatrix}_x + \begin{pmatrix} 1 & 0 & 0 \\ 0 & 1 & 0 \\ 0 & 0 & 1 \end{pmatrix} \begin{pmatrix} h \\ \bar u \\ g' \end{pmatrix}_t = \begin{pmatrix} \alpha | \bar u |\\ g' \sin \theta - \alpha \frac{\bar u | \bar u |}{h} \\ - \alpha g' \frac{| \bar u|}{h} \end{pmatrix}
$$
If we expand across the bottom row:
$$
 ( \gamma \bar u - \lambda) ( ( \bar u - \almbda) ( \chi \bar u - \lamdba) - g' h \cos \theta) - ( \gamma -1) g' ( \bar u - \lamdba) \frac{1}{2} h \cos \theta + ( \gamma -1) g' \frac{\bar u}{h} \frac{1}{2} h^2 \cos \theta = 0
$$
This is too much mess so let us look at the specific case with $\gamma =1$ then:
$$
( \bar u - \lambda) ( (\bar u - \lambda)( \chi \bar u - \lamdba) - g' h \cos \theta) = 0
$$
So $\lambda_1 = \hat u + \hat c, \lambda_2 = \hat u - \hat c, \lambda_3 = \bar u$ with $\hat u = \frac{\chi +1}{2} \bar u$ and $\hat c = \sqrt{ g' h \cos \theta + ( \frac{\chi -1}{2} \bar u)^2}$. So a different velocity is responsible for $\lambda_1$ or $\lamdba_2$ than $\lambda_3$ (which relates to just how the information is being carried on the flow of the fluid.\\\\
In the limit $\chi = 1$ we get $\hat u = \bar u$ and $\hat c = c = \sqrt{g' h \cos \theta}$ which gives $\lambda_1 = \bar u + c, \lamdba_2 = \bar u -c, \lambda_3 = \bar u$. Same as before but now we have the $\cos \theta$ there. Now we need to start asking the question is the system still hyperbolic. The requirement for hyperbolic is that we have three real eigenvalues with a full set of eigenvectors. So to check we need to find the eigenvectors of these eigenvalues.
$$
 \lambda_1: \bar q_1^{T} \begin{pmatrix} u-\lambda & h & 0 \\ g' \cos \theta & u-\lamdba & \frac{1}{2} h \cos \theta \\ 0 & 0 & u - \lambda \end{pmatrix} = ( q , 1, r_1) \begin{pmatrix} - c & h & 0 \\ g' \cos \theta & - c & \frac{h}{2} \cos \theta \\ 0 & 0 & -c \end{pmatrix} = \bm 0
$$
which gives:
$$
 \bm q_1^T = ( \frac{g' \cos \theta}{c} , 1, \frac{h \cos \theta}{2 c}) = ( ( \frac{g' \cos \theta}{h})^{\frac{1}{2}}, 1, ( \frac{h \cos \theta}{4 g'})^{\frac{1}{2}})
$$
Can do the same for $\bm q_2^T $ to get:
$$
 \bm q_2^T = ( 0( \frac{g' \cos \theta}{h})^{\frac{1}{2}}, 1, -( \frac{h \cos \theta}{4 g'})^{\frac{1}{2}})
$$
and 
$$
 \bm q_3^T = (0,0,1)
$$
As $\theta \rightarrow \frac{\pi}{2}$ we get:
$$
 \bm q_1^T = ( 0,1,0), \bm q_2^T = (0,1,0), \bm q_3^T = (0,0,1)
$$
so no longer hyperbolic as we can only form two distinct ODES, so it is now parabolic.\\\\
In the limit $\theta = \frac{pi}{2}$ but with $\chi >1$ and $\gamma >1$ we get:
$$
|A- \lambda B| = (\gamma \bar u - \lamdba) ( (\bar u - \lambda) ( \chi \bar u - \lamdab)) + ...
$$
with $\gamma$ sufficently close to 1 that we can ignore the other bits. this will give us distinct eigenvlaues:
$$
 \lambda_1 = \bar u, \lambda_2 = \chi \bar u, \lamdba_3 = \gamma \bar u
$$
so we have three distinct eigenvalues and willl have a full set of eigenvectors that are still hyperbolic. So as soon as we go away from the top hat profile we get a hyperbolic system.
\subsection{Jet}
A source of momentum - typically with a very confined direction. We are looking for someting that is producing a flow that is quite narrow so we can assume the pressure is uniform across it, very much the same as with the shallow water, though in this case there will be no hydrostatics. They can be considered as boundary layers. We will be looking at the high Reynolds number case and assume we have Batchelor entrainment, noting that $R_i =  0$, so we might expect $\alpha_J > \alpha_{gravity current}$. Indeed it is! The question about how it compares to the entraiment coefficent to a plume (which has a density difference and so has a richardson number), in terms of the entraiment density plays a role but is not stabilising. For a point soruce of momentum $M_0 > 0$ but no volume flux at point $Q_0 = 0$.\\\\
Point source of momentum has no length scale, so the only length scale in the problem is the distance from the point source. So we might expect either the width of the jet remains $0$ or it is going to grow linearly with space. Lets pick a coordinate system with $z$ in the direction we are pointing and $r$ in the radial plane, and we might anticipate that things will be conical, at least if we look at a time or ensemble average to remove turbulent fluctations.\\\\
We can think of this as having a self-similar shape with:
$$
 \bar u = U ( E) \phi( \frac{r}{b})
$$
consider our $\phi$ for a top hat profile:
$$
 \phi(\frac{r}{b}) = \phi( \eta) = \begin{cases} 1 & \eta \leq 1\\ 0 & \text{otherwise} \end{cases}
$$
So efficentively we will have our jet expanding with this circular cross-section, so we can think about our momentum flux:
$$
 \pi M = \frac{1}{T} \int^{T}_0 \int_{-\pi}^{\pi} \int_0^{\infty} w^2 r dr d\theta d t = \pi b^2 w^2 \chi
$$
with $\chi = \int_0^{\infty} (\phi(\eta))^2 d\eta$ for the top hat $\chi =1$ and $M = b^2 w^2$.\section{Lecture 20}
Volume flux:
$$
 \pi Q = \frac{1}{T} \int_0^T \int^{\pi}_{-\pi} \int^{\infty}_0 w r dr d\theta dt = \pi b^2 w
$$
so $Q = b^2 W$ if we define:
$$
 \int_{-\pi}^{\pi} \int_0^T w d t d\theta = 2 \pi T W \phi(\frac{r}{b})
$$
The momentum flux is going to be constant so:
$$
 \frac{dM}{dz} = 0 \implies M = M_0
$$
but the volume flux will be increasing due to entrainemnt so:
$$
 \frac{dQ}{dz} = \alpha W 2 b
$$
This comes from $\frac{d}{dz}(\pi) Q ) = 2 \pi b u_e = 2 \pi b \alpha W$. If we note we can write:
$bW = M^{\frac{1}{2}}$ then we can write:
$$
 \frac{dQ}{dz} = 2 \alpha M_0^{\frac{1}{2}} = const.
$$
so our volume flux increases linearly with distance from source:
$$
Q = 2 \alpha (M_0^{\frac{1}{2}} + z_v)
$$
If $Q=0$ at $z=0$ then $z_v = 0$. we have:
$$
 b= \frac{Q}{M^{\frac{1}{2}}} = \frac{Q}{M_0^{\frac{1}{2}}} = 2 \alpha ( z + z_v)
$$
so our jet is conical. We have assumed here the the jet is self-similar, but of course a real source is unlikely to coincide exactly with that. Imagine if we had a tube ending at $z=0$ then initially it will come out a smooth flow, but then as we progressively grow shear instabilities it will start to look like our turbulent jet. We could be tempted to put $z_v$ at the edge of the tube, but that won't be accuract so we actually label a virtual origin as it will asymptotically approach the solution in the far field as though it is issuing from $z_v$ wiht $M_0$ and $Q_0 = 0$.\\\\
Going to motivate this in part by looking briefly at what a plume really looks like.
\subsection{General equations for a plume}
Steady plume: Morton, Taylor, Turner (1956). however most of the ideas here come from a paper by Batchelor (1954).\\\\
Have plume coming from point source, assume a self-similar solution, steady consider velocity profile across it $w(r,z,t)$ which tends to 0 as $r \rightarrow \infty$. Take center lines suppor to be $W$ and density profile $\rho(r,z,t)$ with compact support and take width of plume $b(z,t)$.\\\\
In homogeneous environemnt, we need to consider the bouyancy flux $F = F_0$ if the plume is steady, in general $F(z=0,t) = F_0(t)$. We also need to consider the volume/mass flux by taking the volume flux at the point source to be zero so $Q(z=0,t) = 0$ and $M(z=0,t) = 0$:\\
$$
 \phi(\frac{r}{b}) = \begin{cases} 1 & |r|\leq b\\ 0 & \text{otherwise} \end{cases}
$$
In the example sheet we will take this to be a guassian.\\\\
We can simply write down what our mass flux is for our top hat profile:
$$
 \pi Q = \pi \rho b^2 W
$$
the momentum flux will be:
$$
 \pi M = \pi \rho b^2 W^2
$$
Note: if not top-hat would need $\chi$ shape factor.
This time we also have buoyancy:
$$
\pi F = \pi ( \rho_0 - \rho) g b^2 W
$$
this is $>0$ for a buoyant plume. Note: if not top-hat then $\gamma \neq 1$.\\\\
Now lets do some reverse engineering to write down an expression for $W$ in terms of $M$,$Q$ and $F$:
$$
 W = \frac{M}{Q}
$$
$$
 b = \frac{Q}{(\rho M)^{\frac{1}{2}}}
$$
$$
 g' = \frac{F}{Q}
$$
As before with the jet we will use Batchelors entrainment hypothesis:
$$
 u_e = \alpha ( \frac{\rho}{\rho_0})^{\frac{1}{2}} W
$$
If Boussinesq $u_e = \alpha W$ so $\rho_0 u_e^2 = \alpha \rho W^2$. We can think about the $\frac{\rho}{\rho_0})^{\frac{1}{2}}$ as some statement of energy.\\\\
Here we are interested in a plume version of $\alpha$ rather than a jet version of $\alpha$. We can experiementally verify that for a top hat profile we would have $\alpha$ lieing roughly between 0.1 to 0.16 with typically $\alpha = 0.13$. For comparsion in a jet typically $\alpha \appox 0.065 - 0.08$. This is surprising as in the plume we have less dense fluid above a dense fluid so we would expect it to be more stable and therefore have less entrainment. However, the reason for this is when you introduce turbulent ins and outs at the edges you get lots of places with light fluid below dense fluid, so the presence of this static instability (related to Rayleigh-Talor instability) leads to greatly increased mixing so $\alpha_{plume} > \alpha_{jet}$.
\subsubsection{Time-dependent plume}
Assume linear mixing, top-hat profile, thinking about something vaguely conical in shape then we get volume changing with:
$$
 \frac{\partial}{\partial t} (\pi b^2) + \frac{\partial}{\partial z} ( \pi b^2 w) = 2 \pi b u_e = 2 \alpha \pi b W
$$
compared with the gravity current which has:
$$
 \frac{\partial h}{\partial t} + \frac{\partial}{\partial z} (hu) = u_e
$$
Mass:
$$
\frac{\partial}{\partial t} ( \pi \rho b^2) + \frac{\partial}{\partial z} ( \p \rho b^2 W) = 2 \pi \rho_0 b u_e = 2 \alpha \pi \rho_0 b W
$$
Momentum:
$$
 \frac{\partial}{\partial t} ( \pi \rho b^" W) + \frac{\partial}{\partial z} ( \pi \rho b^2 W^2) = \pi b^2 ( \rho_0 - \rho) g
$$
Now we have our equations for a time dependant top hat plume and it is an example sheet question to work out the solution.
\subsubection{Plumes in homogeneous environment}
\textbf{Steady Boussinesq plume}\\
Boussinesq means that we can ignore $\frac{\rho}{\rho_0}^{\frac{1}{2}}$ so $u_e = \alpha W$, $Q = \rho b^2 W$, $M= \rho b^2 W^2$, $F = (\rho_0 - \rho) b^2 W g$.\\\\ 
The steady state version of the time dependant plume equations are:\\
Volume:
$$
 \frac{d}{dz} ( \pi b^2 W) = 2 \alpha \pi b W
$$
Mass:
$$
 \frac{d}{dz} ( \pi \rho b^2 W) = 2\alpha \pi \hat \rho b W
$$
Momentum:
$$
 \frac{d}{dz} ( \pi \rho b^2 W^2) = \pi b^2 (\hat \rho - \rho) g
$$
Mass flux:
$$
 \frac{dQ}{dz} = 2 \alpha \hat \rho b W
$$
Momentum flux:
$$
 \frac{dM}{dz} = 2 \alpha \pi \hat \rho bW
$$
Buoyancy flux requires more work:
$$
 \hat \rho \frac{d}{dz} ( \pi b^2 w) = \frac{d}{dz}( \hat \rho \pi w) - \pi b^2 w \frac{d\hat \rho}{dz} = 2 \alpha \hat \rho \pi b w
$$
$$
 \pi \frac{d}{dz} ( ( \hat \rho - \rho) b^2 w) = - \pi b^2 w \rho_0 N^2
$$
$$
 \frac{d}{dz} F = - b^2 w \rho_0 N^2
$$
If $\hat \rho = \rho_0$ then $N^2 = 0$ and we have:
$$
 \frac{d}{dz} F = 0
$$
\\\\
\textbf{Plume in a homogenous environemnt}:
$[ F] = R L^4 T^3, [ Q] = R L^3 T^{-1}, [ M] = R L^4 T^{-2}$
Therefore, in order to express $Q$ and $M$ interms of the only input which is $F_0$ we have:
$$
 [Q] = ( R L^4 T^{-3})^{\frac{1}{3}} R^{\frac{2}{3}} z^{\frac{5}{3}}, Q \sim ( F_0)^{\frac{1}{3}} \rho_0^{\frac{2}{3}} z^{\frac{5}{3}}
$$
$$
 [M] = ( RL^4 T^{-3})^{\frac{2}{3}} R^{\frac{1}{3}} z^{\frac{4}{3}}, M \sim ( F_0)^{\frac{2}{3}} \rho_0^{\frac{2}{3}} z^{\frac{4}{3}}
$$
Let
$$
 Q= \tilde Q z^q, M = \tilde M z^m
$$
then the equations become:
$$
\frac{dM}{dz} = \alpha b^2 ( \hat \rho - \rho ) g = \frac{F_0 Q}{M} \implies m \tilde M z^{m-1} = \frac{F_0 \tilde Q z^q}{\tilde M z^m} = \frac{F_0 \tilde Q}{\tilde M} z^{q-m}
$$
$$
 \frac{d Q}{dz} = 2 \alpha \hat \rho b w = 2 \alpha \rho_0^{\frac{1}{2}} M^{\frac{1}{2}} \implies q \tilde Q z^{q-1} = 2 \alpha \rho_0^{\frac{1}{2}} \tilde M^{\frac{1}{2}} z^{\frac{m}{2}}
$$
Therefore we get $q -1 = \frac{m}{2}$ and $ m-1 = q- m$ so $q = \frac{5}{3}$ and $m = \frac{4}{3}$. we aslo have $q \tilde Q = \alpha \rho_0^{\frac{1}{2}} \tilde M^{\frac{1}{2}}$ and $m \tilde M = \frac{F_0 \tilde Q}{\tilde M}$ so:
$$
 \tilde M = ( \frac{9 \alpha}{10})^{\frac{2}{3}} \rho_0^{\frac{1}{3}} F_0^{\frac{2}{3}}, \tilde Q = ( \frac{6 \alpha}{5}) ( \frac{9 \alpha}{10})^{\frac{1}{3}} \rho_0^{\frac{1}{3}} F_0^{\frac{1}{3}}
$$
so 
$$
 Q = \tilde Q z^{\frac{5}{3}}, M = \tilde M z^{\frac{4}{3}}
$$
 and then we get:
 $$
 W = \frac{M}{Q} = \frac{\tilde M}{\tilde Q} z^{- \frac{1}{3}}
 $$
 $$
  b = \frac{Q}{( \rho_0 M)^{\frac{1}{2}}} = \frac{\tilde Q}{( \rho_0 \tilde M)^{\frac{1}{2}}} z = \frac{6 \alpha}{ 5} z
 $$
  $$
   g' = \frac{F_0}{Q} = \frac{F_0}{\tilde Q} z^{- \frac{5}{3}}
  $$
\subsubsection{Steady non-Boussinesq plume}
                 Clearly the Boussinesq assumption is violated by a point source since $g' \rightarrow \infty$ so $\frac{ \rho - \rho_0}{\rho_0}$ is not small.\\\\
                 For the entrainment velocity:
                 $$
                  u_e = \alpha ( \frac{\rho}{\rho_0})^{\frac{1}{2}} W
                 $$
                 $$
                  b = \frac{Q}{(\rho M)^{\frac{1}{2}}}
                 $$
                 This generates the same equations as in the Boussinesq case with $b = \frac{Q}{(\rho_0 M)^{\frac{1}{2}}}$ so the solution is the same, however $b$ has changed:
                 $$
                 b= \frac{Q}{(\rho M)^{\frac{1}{2}} }= \frac{6 \alpha}{5} ( \frac{\rho_0}{\rho})^{\frac{1}{2}} z = \frac{6 \alpha}{5} ( 1 + ( \frac{z_b}{z})^{\frac{5}{3}} )^{\frac{1}{2}} z
                 $$
                 so
                 $$
                 z_b = \frac{5}{3} ( \frac{F_0^2}{20 \alpha^4 \rho_0^2 g^3})^{\frac{1}{3}}
                 $$
                 For large $z$, $b \rightarrow \frac{6\alpha}{5} z$. But this will be different for small $z$ near the plume source. So for $z << z_b$:
                 $$
                 b = \frac{6 \alpha}{5} ( \frac{z_b}{z})^{\frac{5}{6}} z =  \frac{6 \alpha}{5} z_b^{\frac{5}{6}} z^{\frac{1}{6}}
                 $$
                 So near the source, there will be a very rapid adjustment of the plume
                 \subsubsection{Plume in stably stratified environment}
                 We have some buoyancy source with $F_0$ and as the plume rises it will entrain in fluid, this increases the desnity and decreases the byoyancy force. At some height $z_f$, $F(z_f) = 0$. But as $M(z_f) \neq 0$ then the plume will overshoot $z_f$ and there will be some further mixing and the plume desnity will reduce a little below $\hat \rho ( z_F)$. Above $z=z_f$, the bouyancy force is reversed decelerating the plume. Evanutally it will come to rest. \\\\It is easy to figure out how high it can go from dimensional terms but very hard otherwise. The only things that matter are $F_0$, $N$, $\alpha$. $[ F_0] = L^4 T^{-3}$, $[N] = T^{-1}$. Predict that $z_{max} = c ( \frac{F_0}{\rho_0}_^{\frac{1}{4}} N^{- \frac{3}{4}}$ experiementally we can verify that $c \sim 5$.\\\\
                 We can't do this fully analytically but we can numerically solve the plume equations to get:
                 $$
                  z_{max} = 2.572 ( \frac{F_0}{4 \alpha^2 \rho_0 N^3})^{\frac{1}{4}}
                 $$
                 There is a series solution to the plume equations in the notes that we won't be covering this year.
                 \subsubsection{Thermal in a homogeneous environment}
                 A thermal is a discrete parcel of fluid with buoyancy released far form a boundary\\\\
                 First we think about the Buoyancy which is given by: $(\rho - \rho_0) g V$. Then we have the drag force for a rigid body $\frac{1}{2} \rho_0 c_D A W^2$ and $Re >>1$. If we pretned we can apply this rigid body drag then we will eventaully get that the drag will balance the buoyancy at some velocity $W$:
                 $$
                  W= ( \frac{2}{c_D} \frac{\rho- \rho_0}{\rho_0} g \frac{V}{A})^{\frac{1}{2}} = ( g' L)^{\frac{1}{2}} F
                 $$
                 $L = \frac{V}{A}$ with Froude number now given by $F = \frac{W}{\sqrt{g' L}}$ then we predict that the Froude number is ascending with constant velocity. This not a Froude number talking about waves. For a rigid body we would expect $F \approx 1$. If we had a sphere then $V = \frac{4}{3} \pi R^3$ and $A = \pi R^"$ and $c_D = 0.4$ then we owuld get $F \approx 1.8$.
                         \section{Lecture 22}
                         \subsection{Vortex rings}
                         There is a lot more in the printed notes than in the lectures. A vortex ring in an annular ring of vorticity with fluid spinning around a ring with fluid moving down inside the ring and up just outside the ring. In the frame of refrence of the ring it has a cross section with two sets of closed trajectory loops with a stagnation point at the top and bottom of the set.\\\\
                         What happens if we add density. The first thing to note is that at the top of the circular region of closed tajectories the density inside the region containing the ring is greater than the density outside the ring so we get Rayleigh-Taylor instability. We have $\omega^2 = \frac{1}{2} g' k$ as $g' <0$ we have $\omega = \pm i \sqrt{ \frac{1}{2} |g'| k}$ so $e^{i \omega}$ leads to expnentially growing or decaying solutions. At the bottom we get wave like solutions to the perturbations at the bottom of the region, so disturbances here don't decay. So small perturbations that start at the top grow and then as they are accelerated (which decreases the amplitude a bit),around to the bottom of the region, as they get squeezed into the bottom of the region the wavelength reduces and so they become larger perturbations. This ends up drawing some of the ambient fluid into the vortex ring like structure at the top and expelling some of the fluid in the vortex ring like structure. We can consider the time to advect the fluid around as being roughly $t_A = \frac{a}{w}$ and the growth rate is roughly $\sqrt{ |g'|h}$so in this time the amount things grow is : $\simga t_A = \sqrt{ g k} \frac{a}{w}$. These become of similar order to the vortex ring when $\lamdba \sim a$ so $k \sim \frac{1}{a}$ so $\sigma t_A = \frac{\sqrt{g' a}}{w} = \frac{1}{F}$. This says that once the Froude number of the vortex ring becomes of order 1 we start to see these rayleigh taylor instabilities start to have a big impact, this is when it stops behaving like a vortex ring and starts behaving like a thermal.
                                 \subsubsection{Entraining thermal}
                                 We are going to release a volume $V_0$ and it is going to have a reduced gravity $g'_0$. This produces a buoyancy $B= V_0 g'_0$ (not the buoyancy flux). This will be conserved in a homogeneous environment. \\
                                 Consider a thermal arising from a point source ten the only length scale is the distance from the point source (in this situtation you could potentially use the length scale $V_0^{\frac{1}{3}}$ but we are not going to make use of this for now) so as $[B] = L^4 T^{-2}$ and $[w] = \frac{L}{T}$ as we have:
                                 $$
                                  w \sim \frac{B^{\frac{1}{2}}}{z}
                                 $$
We are also going to make use of the idea of entrainment with $u_e = \alpha w$. This is entraining over the surface area of the thermal $S^!$ so:
$$
 \frac{d V}{dt} = u_e S^1 = \alpha w S^1
$$
If we assume the thermal is self-similar then:
$$
 V = v R^3, S^1 = s R^2
$$
in the case of a sphere then $v = \frac{4}{3} \pi$ and $s = 4 \pi$.\\\\
Let's see how far we can get by just considering volume:
$$
 \frac{dV}{dt} = \alpha w S^1
$$
If we take $z$ to be the position of the centre of the thermal then $w = \frac{dz}{dt}$ so:
$$
 \frac{dV}{dz} \frac{dz}{dt} = w \frac{dV}{dz} = \alpha w S^1
$$
$$
 3 v R^2 \frac{dR}{dz} = \alpha s R^2
$$
so
$$
 \frac{dR}{dz} = \frac{\alpha s}{3 v}
$$
in the case of the sphere $\frac{dR}{dz} = \alpha$. We have:
$$
 R = \alpha \frac{s}{3v} (z-z_0), V = \frac{\alpha^3 s^3}{27 v^2} ( z- z_0)^3
$$
Now let's consider mass with $\bar \rho$ the density within the thermal and $\rho_0$ the density of the ambinent fluid:
$$
 \frac{d}{dt} ( \bar \rho V) = \alpha \rho_0 w S^1
$$
We can separate this out in to $\frac{d\bar \rho}{dt}$ and $\frac{d V}{dt}$ and we end up finding that
$$
 g' = \frac{\bar \rho - \rho_0}{\bar \rho} g = \frac{B}{V}
$$
is conserved for all times.\\\\
Now let's consider momentum:
$$
 \frac{d}{dt} ( ( 1+ C_A) \rho V w) = ( \bar \rho - \rho_0) g V
$$
we need to think about the momentum of the mass around the thermal which we have introduced here by adding the added mass coefficent $C_A$. For high Re we have $C_A$ for a sphere is about $\frac{1}{2}$, experimentally for a thermal it is about $0.2$ (generally we decide that $0.2$ is small enough compared to 1 we can sort of forget about it. We can rewrite this again using $\frac{d}{dt} = w \frac{d}{dz}$:
$$
 \frac{d w^2}{dz} = 2 ( \frac{g'}{1 + C_a} - \alpha w^2 \frac{S^1}{V}) = 2 ( \frac{g'}{1+ C_A} - \lapha w^2 \frac{s}{vR})
$$
We have already seen that $[B] = L^4 T^{-2}$  so could try $w^2 = k \frac{B}{z^2}$ as a solution. If we do this we find that:
$$
w = ( \frac{27 v^2}{2 \alpha^2 s^2 ( 1+ C_A)})^{\frac{1}{2}} B^{\frac{1}{2}} ( z- z_0)^{-1} = \frac{dz}{dt}
$$
$$
z = ( \frac{ 54 v^2}{\alpha^3 s^3 ( 1+ C_A)})^{\frac{1}{4}} B^{\frac{1}{4}} (t - t_0)^{\frac{1}{2}}
$$
So this corresponds to a point source at $z=z_0$ when $t=t_0$. The Froude number $F= \frac{w}{\sqrt{g' R}} $. So we will start with a high Froude number and a vortex ring that gradually decreases and then platouts to a constant when it becomes a thermal.\\\\
Thermals are really efficent at entraining fluid as the Rayleigh-Talor instability is very effient and leads thermal to mix in all of the fluid it encounters.
\subsection{Thermal in stratified environment}
$$
 \frac{dV}{dt} = \alpha S^1 w
$$
Not changed by stable straigication provided shape remains self-similar. As the fluid rises up it will be entraining fluid denser than it so it will eventually reach a height at which the buoyancy force disappears.\\\\
If we have a constant buoyancy frequency for the environemnt $N = \sqrt{ - \frac{g}{\rho_0} \frac{d \hat \rho}{dz}}$ which has $[N] = T^{-1}$ so 
$$
 ( \frac{B}{N^2})^{\frac{1}{4}} \sim z_{max}
$$
Of course we won't actually have a self-similar solution as we won't have a large spherical mass at this height, the buoyancy force will want it to spread out. However, until it reaches that height we will still have rayleigh-taylor on the front. A solution based on a self-similar thermal will over-predict the rise in height.
$$
 \frac{d \bar \rho V}{dt} = \alpha \hat \rho S^1 w \implies \frac{d g'}{dt} = - ( \alpha g' \frac{S^1}{V} + N^2) w
$$
if we assume $C_A = 0$ then the momentum change is:
$$
 \frac{d \rho V w}{dt} = \rho_0 g' V
$$
From all this we can bring it together to get an equation for the thermal if we assume a constant Froude number:
$$
\frac{dz}{dt} = F ( 2 \alpha ( z+ z_0) g')^{\frac{1}{2}}
$$
which we are going to need to solve numerically.
\section{Example Sheet 3}
The momentum equation never cares about geometry so can always just quote the following (even if it says find the momentum equation!):
$$
\frac{\partial u}{\partial t} + u \frac{\partial u}{\partial x} + g \frac{\partial }{\partial x}( H +h) = 0
$$
A bousinesq case is when $\Delta \rho$ is small enough we can ignore variation of $\rho$ in the inertial term compared to gravity ( so occurs in large gravity or small variation. Non-bousinesq case is when there is big variaiton in $\rho$ e.g. one is negillbible or zero. \\\\
Hydrostatic is when pressure is the same at every level. Occurs in any fluid at rest. $\frac{\partial p}{\partial z} = - \rho g$. Can cause a pressure gradient in the fluid below if the fluid below has a variable height as then the pressure along the surface of the lower fluid will vary (unless the fluid above has negillible density).\\\\
If you have a constriction and so the flow crosses from the subcritical to the supercritical branch then you must have $Fr =1$ at the narrowest or shallowest point. If you have both variations in depth and width that aren't coincident then it could happen somewhere.\\\\
If you are considering a steady state solutoin and your answer doesn't make sense think about whether your boundary conditions are too strict and whether information may have been gained from somewhere else that changes them. E.g. in question 17 we overspecified the start height and froude number, in reality we needed to takeinto account what happened further down the channel at some sort of hydralic control that would have $Fr =1$.\\\\
Momentum equation includes $\frac{\partial}{\partial x}( P + \rho_1 g z)$ and here the $\rho_1 g z$ is not part of the pressure but rather just another term in the momentum equation due to the acceleration of gravity. Think about the vertical momentum equation which has $0 = \frac{\partial}{\partial z} ( p + \rho_1 g z) \implies \frac{\partial p}{\partial z} = - \rho_1 g$.\\\\
When writing out the momentum equation for plumes rememeber that $\frac{d p}{dt} = F$ so the difference in momentum flux and pressure is equal to the change in momentum which is equal to the buoyancy force on the volume of the region considered. Also remember that fluid advected in does not contribute any momentum. Best way to think about it is momentum before minus momentum after gives change in momentum, and change in momentum is force.\\\\
It is worth remembering the trick on page 3 of Q24, also on page 15 of plume notes.\\\\
Batchelors entrainment theorem is:
$$
 u_e = \alpha \sqrt{\frac{\rho}{\rho_0}} |\bar u|
$$
                    \end{document} 
