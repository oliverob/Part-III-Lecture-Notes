\documentclass{article}
\usepackage[utf8]{inputenc}
\usepackage{bm}
\usepackage{amssymb}
\usepackage{amsmath}
\usepackage{braket}
\usepackage{cancel}
\title{Quantum Computation}
\author{oliverobrien111 }
\date{July 2021}

\begin{document}

\maketitle
\section{Lecture 1}
\subsection{Review of Shor's algoirthm/quantum period finding algorithm}
\textbf{Polynomial time hierarchy}://
Computation with input of size $n$, and we are interested in the number of steps/gates (classical or quantum). When we say $O(poly(n))$ steps we regard this as an "efficent computation".\\\\
Shor's algorithm solves the factoring problem:\\
Given an integer $N$ needing $O(log N)$ bits, we want to find a non-trival factor in $O(poly n)$ time.\\\\
The best known classical algorithm (number sieve): $e^{O(n^{\frac{1}{3}} (\log n)^{\frac{1}{3}} )}$\\
Shor's alogrithm takes $O(n^3)$\\
\subsubsection{Quantum factoring algorithm (summary)}
\begin{enumerate}
        \item First, convert factoring into periodicity determination. Given $N$, choose $a< N$ s.t. $a$ is coprime (this is easy classically can be seen in part II lecture notes). Consider $f: \mathbb{Z} \rightarrow \mathbb{Z}_N$ $f(x) = a^x \mod N$. \textbf{Euler's Theorem}: if $f$ is periodic with period $r$, then it is called 'order of $a \mod N$'. 
        \item In order to find $r$ we need a quantum implementation of $f$. We are always workingon finite size registers so restricting $x \in \mathbb{Z}$ to $x \in \mathbb{Z}_M$ (for some large enough $M$): $f: \mathbb{Z}_M \rightarrow \mathbb{Z}_N$. $f$ will no longer be exactly preriodic but this would have neglible effect if $M$ is sufficently large e.g. $M=O(N^2)$
        \item Using the classical theory of continued fractions. Define Hilbert spaces $\mathcal{H}_M \rightarrow \{\ket{i}\}_{i \in \mathbb{Z}_M}, \mathcal{H}_N\rightarrow \{\ket{i}\}_{i \in \mathbb{Z}_N}$.
        \item $\ket{x} \rightarrow \ket{f(x)}$ is not generally a valid quantum operator, so we make it a unitary operation which can be implemented:
                $$
                U_f: \mathcal{H}_M \otimes \mathcal{H}_N \rightarrow \mathbb{H}_M \otimes \mathbb{H}_N
                $$
                $$
                U_f: \ket{i} \ket{k} \rightarrow \ket{i} \ket{k + f(i)}
                $$
        \item if $x \rightarrow f(x)$ can be classically computed in $O(poly(m))$ time ($m = \log M$)), then $U_f$ can be implemented in poly($m$) time quantumly too
        \item We will sometimes view $U_f$ as a black box/oracle and we will count the number of times the algorithm invokes the oracle.
        \item Back to facotoring to get $r$ we'll use the quantum algorithm for periodicity determination:
        \item Given an oracle $U_f$ with the promise that $f$ is periodic of some unknown period $r \in \mathbb{Z}_N$ so that $f(x+r) = f(x)$ and $f$ is one-to-one in this period (for all $0 \leq x_1 < x_2<r f(x_1) \neq f(x_2)$)
        \item To find $r$ in $O(poly n)$ with any persecribed success probability $1-\epsilon$ we use the following alogirthm:
                \begin{itemize}
                        \item Step 1: Create the state 
                                $$
                                \frac{1}{\sqrt{M}} \sum_{i=0}^{M-1} \ket{i} \ket{0}
                                $$
                        \item Step 2: Apply $U_f$ to get
                                $$
                                \frac{1}{\sqrt{M}} \sum_{i=0}^{M-1} \ket{i} \ket{f(i)}
                                $$
                        \item Step 3: Measure the 2nd register to get $y$. By the born rule the first register collapses to all those $i$: $f(i) =y$ i.e. $i = x_0, x_0 + r, x_o + 2r,..., x_0 + (A-1)r$, $0 \leq x_0 <r$.\\\\Discard the second register to get the following state:
                                $$
                                \ket{per} = \frac{1}{\sqrt{A}} \sum_{j=0}^{A-1} \ket{x_0 + jr}
                                $$
                         If we measure $\ket{per}$ in computation basis we will get a value of one of these states $x_0 + jr$ for uniformly random $j$. This only gives us a random element of $\mathbb{Z}_M$ with no information about $r$.
                        \item Step 4: Apply quantum fourier transform mod $M$ (QFT). Lets recap what QFT does:
                                $$
                                \ket{x} \rightarrow \frac{1}{\sqrt{M}} \sum_{y=0}^{M-1} \omega^{xy} \ket{y}, \forall x \in \mathbb{Z}_M, \omega = e^{2\pi i/ M}
                                $$
                                This can be implement in $O(m^2)$ time and gives state:
                                $$
                                QFT \ket{per} = \frac{1}{\sqrt{MA}} \sum_{j=0}^{A-1} \sum_{y=0}^{M-1}  \omega^{ (x_0 + jr)y} \ket{y} =  \frac{1}{\sqrt{MA}} \sum_{y=0}^{M-1} \omega^{x_0y}\left[  \sum_{j=0}^{A-1} \omega^{jry} \ket{y} \right]
                                $$
                        The square brackets will be:
                        $$
                        \begin{cases}
                                A & \text{if }y = KA= k\frac{M}{r}, x = 0, 1,..., r-1\\
                                0 & \text{otherwise}
                        \end{cases}
                        $$
                        So gives final state:
                        $$
                        QFT \ket{per} = \sqrt{\frac{A}{M}} \sum_{k=0}^{A-1} \omega^{x_0 k \frac{N}{r}} \ket{k \frac{M}{r}}
                        $$
                        Now the random shift $x_0$ only appears in the phase not in the ket labels. So now the measurement probabilities will be indepedant of $x_0$. When we measure this we get some value $c = \frac{k_0M}{r}$ with $k_0$ uniformly random in range $0 \leq k_0 < r$
                        $$
                        \frac{k_0}{r} = \frac{c}{M}
                        $$
                        As $c$ and $M$ are known, and $k_0$ is unknown but random in the given range. We want to find $r$ and so we recall several classical facts.\\\\
                        \textbf{Co-primality Theorem}: The number of integers less than $r$ that are coprime to $r$ grows with $O(\frac{r}{\log \log r})$\\\\
                        Therefore, the probability of $k_0$ being coprime to $r$ is $O(\frac{1}{\log \log r})$.\\\\
                        \textbf{Lemma}: If a single trial has success probability $p$ then if one repeats it $M^*$ times, for any $0<1-\epsilon<1$. We get probability of at least one success in $M^*$ trails is greater than $1-\epsilon$ if $M^* = \frac{- \log \epsilon}{p}$
                \end{itemize}
                
        \item From learning the period $r$ we can use number theory to find a factor of $N$
\end{enumerate}
\end{document}
