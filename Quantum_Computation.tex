\documentclass{article}
\usepackage[utf8]{inputenc}
\usepackage{bm}
\usepackage{amssymb}
\usepackage{amsmath}
\usepackage{braket}
\usepackage{qcircuit}
\usepackage{cancel}
\title{Quantum Computation}
\author{oliverobrien111 }
\date{July 2021}

\begin{document}

\maketitle
\section{Lecture 1}
\subsection{Review of Shor's algoirthm/quantum period finding algorithm}
\textbf{Polynomial time hierarchy}://
Computation with input of size $n$, and we are interested in the number of steps/gates (classical or quantum). When we say $O(poly(n))$ steps we regard this as an "efficent computation".\\\\
Shor's algorithm solves the factoring problem:\\
Given an integer $N$ needing $O(log N)$ bits, we want to find a non-trival factor in $O(poly n)$ time.\\\\
The best known classical algorithm (number sieve): $e^{O(n^{\frac{1}{3}} (\log n)^{\frac{1}{3}} )}$\\
Shor's alogrithm takes $O(n^3)$\\
\subsubsection{Quantum factoring algorithm (summary)}
\begin{enumerate}
        \item First, convert factoring into periodicity determination. Given $N$, choose $a< N$ s.t. $a$ is coprime (this is easy classically can be seen in part II lecture notes). Consider $f: \mathbb{Z} \rightarrow \mathbb{Z}_N$ $f(x) = a^x \mod N$. \textbf{Euler's Theorem}: if $f$ is periodic with period $r$, then it is called 'order of $a \mod N$'. 
        \item In order to find $r$ we need a quantum implementation of $f$. We are always workingon finite size registers so restricting $x \in \mathbb{Z}$ to $x \in \mathbb{Z}_M$ (for some large enough $M$): $f: \mathbb{Z}_M \rightarrow \mathbb{Z}_N$. $f$ will no longer be exactly preriodic but this would have neglible effect if $M$ is sufficently large e.g. $M=O(N^2)$
        \item Using the classical theory of continued fractions. Define Hilbert spaces $\mathcal{H}_M \rightarrow \{\ket{i}\}_{i \in \mathbb{Z}_M}, \mathcal{H}_N\rightarrow \{\ket{i}\}_{i \in \mathbb{Z}_N}$.
        \item $\ket{x} \rightarrow \ket{f(x)}$ is not generally a valid quantum operator, so we make it a unitary operation which can be implemented:
                $$
                U_f: \mathcal{H}_M \otimes \mathcal{H}_N \rightarrow \mathbb{H}_M \otimes \mathbb{H}_N
                $$
                $$
                U_f: \ket{i} \ket{k} \rightarrow \ket{i} \ket{k + f(i)}
                $$
        \item if $x \rightarrow f(x)$ can be classically computed in $O(poly(m))$ time ($m = \log M$)), then $U_f$ can be implemented in poly($m$) time quantumly too
        \item We will sometimes view $U_f$ as a black box/oracle and we will count the number of times the algorithm invokes the oracle.
        \item Back to facotoring to get $r$ we'll use the quantum algorithm for periodicity determination:
        \item Given an oracle $U_f$ with the promise that $f$ is periodic of some unknown period $r \in \mathbb{Z}_N$ so that $f(x+r) = f(x)$ and $f$ is one-to-one in this period (for all $0 \leq x_1 < x_2<r f(x_1) \neq f(x_2)$)
        \item To find $r$ in $O(poly n)$ with any persecribed success probability $1-\epsilon$ we use the following alogirthm:
                \begin{itemize}
                        \item Step 1: Create the state 
                                $$
                                \frac{1}{\sqrt{M}} \sum_{i=0}^{M-1} \ket{i} \ket{0}
                                $$
                        \item Step 2: Apply $U_f$ to get
                                $$
                                \frac{1}{\sqrt{M}} \sum_{i=0}^{M-1} \ket{i} \ket{f(i)}
                                $$
                        \item Step 3: Measure the 2nd register to get $y$. By the born rule the first register collapses to all those $i$: $f(i) =y$ i.e. $i = x_0, x_0 + r, x_o + 2r,..., x_0 + (A-1)r$, $0 \leq x_0 <r$.\\\\Discard the second register to get the following state:
                                $$
                                \ket{per} = \frac{1}{\sqrt{A}} \sum_{j=0}^{A-1} \ket{x_0 + jr}
                                $$
                         If we measure $\ket{per}$ in computation basis we will get a value of one of these states $x_0 + jr$ for uniformly random $j$. This only gives us a random element of $\mathbb{Z}_M$ with no information about $r$.
                        \item Step 4: Apply quantum fourier transform mod $M$ (QFT). Lets recap what QFT does:
                                $$
                                \ket{x} \rightarrow \frac{1}{\sqrt{M}} \sum_{y=0}^{M-1} \omega^{xy} \ket{y}, \forall x \in \mathbb{Z}_M, \omega = e^{2\pi i/ M}
                                $$
                                This can be implement in $O(m^2)$ time and gives state:
                                $$
                                QFT \ket{per} = \frac{1}{\sqrt{MA}} \sum_{j=0}^{A-1} \sum_{y=0}^{M-1}  \omega^{ (x_0 + jr)y} \ket{y} =  \frac{1}{\sqrt{MA}} \sum_{y=0}^{M-1} \omega^{x_0y}\left[  \sum_{j=0}^{A-1} \omega^{jry} \ket{y} \right]
                                $$
                        The square brackets will be:
                        $$
                        \begin{cases}
                                A & \text{if }y = KA= k\frac{M}{r}, x = 0, 1,..., r-1\\
                                0 & \text{otherwise}
                        \end{cases}
                        $$
                        So gives final state:
                        $$
                        QFT \ket{per} = \sqrt{\frac{A}{M}} \sum_{k=0}^{A-1} \omega^{x_0 k \frac{N}{r}} \ket{k \frac{M}{r}}
                        $$
                        Now the random shift $x_0$ only appears in the phase not in the ket labels. So now the measurement probabilities will be indepedant of $x_0$. When we measure this we get some value $c = \frac{k_0M}{r}$ with $k_0$ uniformly random in range $0 \leq k_0 < r$
                        $$
                        \frac{k_0}{r} = \frac{c}{M}
                        $$
                        As $c$ and $M$ are known, and $k_0$ is unknown but random in the given range. We want to find $r$ and so we recall several classical facts.\\\\
                        \textbf{Co-primality Theorem}: The number of integers less than $r$ that are coprime to $r$ grows with $O(\frac{r}{\log \log r})$\\\\
                        Therefore, the probability of $k_0$ being coprime to $r$ is $O(\frac{1}{\log \log r})$.\\\\
                        \textbf{Lemma}: If a single trial has success probability $p$ then if one repeats it $M^*$ times, for any $0<1-\epsilon<1$. We get probability of at least one success in $M^*$ trails is greater than $1-\epsilon$ if $M^* = \frac{- \log \epsilon}{p}$. i.e. roughly $O(1/p)$ trials suffice to achieve probability of success $> 1- \epsilon$
                \item After step 4 cancel $\frac{c}{M}$ down to an irredicible algorithm $\frac{a}{b}$ there is an efficent algorithm ($O(poly n)$) for this. This will give us $r$ as denominator $b$ if $k_0$ is coprime to $r$ with probablity $O(\frac{1}{\log \log r})$. So check $b$ value by computing $f(0)$ and $f(b)$ and $b = r \iff f(0) = f(b)$.\\\\
                        By repeating this process $M^* = O( \log \log r)$ times this will give us $r$ with any desired probability $1- \epsilon$. Since $r < M$ the whole algorithm takes $O(poly m)$ time!
                \end{itemize}
                
        \item From learning the period $r$ we can use number theory to find a factor of $N$
\end{enumerate}
\subsubsection{Further insights to QFT}
Now lets think about the implications of $QFT$. What does applying quantum fourier transform really achieve?\\\\
Lets consider a function: $f: \mathbb{Z}_M \rightarrow \mathbb{Z}_N$ with period $r \in \mathbb{Z}_M$, $A= \frac{M}{r}$. Define:
$$
R = \{ 0, r, 2r, 3r,..., (A-1)r\} < \mathbb{Z}_M
$$
$$
\ket{R} = \frac{1}{\sqrt{A}} \sum_{k=0}^{A-1} \ket{kr}
$$
$$
\ket{per} = \ket{x_0 + R} = \frac{1}{\sqrt{A}} \sum_{k=0}^{A-1} \ket{x_0 + rk}
$$
The problem was this random shift $x_0$ when measuring $\ket{per}$. For each $x_0 < \mathbb{Z}_M$ consider a mapping $k \rightarrow k+ x_0$. "Shift by $x_0$". It is a 1-1 invertible map, and can define a unitary version $U(x_0)$ on $\mathcal{H}_M$: $U(x_0) \ket{k} = \ket{k+x_0}$.
$$
\ket{x_0 + R} = U(x_0) \ket{R}
$$
Since $(\mathbb{Z}_M, +)$ is an abelian group $U(x_0)U(x_1) = U(x_0+x_1) = U(x_1)U(x_0)$. So all $U(x_i)$ commute as operators on $\mathcal{H}_M$. Therefore they have an orthonomal basis of common eigenvectors $\{ \ket{\chi_k}\}_{k \in \mathbb{Z}_M}$. These are called shift invariant states as $U(x_0) \ket{\chi_k} = \omega( x_0, k) \ket{\chi_k}$ for all $x_0, k \in \mathbb{Z}_M$ with the important caveat that $|\omega( x_0, k)| = 1$.\\\\
Consider $\ket{R}$ written in $\{ \ket{\chi_r}\}$ basis:
$$
\ket{R} = \sum_{k=0}^{M-1} a_k \ket{\chi_k}
$$
$a_k$ only depend on $r$ not on $x_0$. Then:
$$
\ket{per} = U(x_0) \ket{R} = \sum_{k=0}^{M-1} a_k \omega(x_0, k) \ket{\chi_k}
$$
Here it can be seen that the probability of measuring $k$ is 
$$
prob(k) = |a_k \omega(x_0, k)|^2= |a_k|^2
$$
So this is all indepedant of $x_0$ and depends only on $r$. So measuring in this basis gives us some information about $r$. So one can think of QFT as the unitary mapping that rotates $\chi$ basis into the standard computational basis. So can define QFT as:
$$
QFT \ket{\chi_k} = \ket{k}
$$
How do these mysterious shift invariant states look?
\subsubsection{Explicit form of shift invariant shapes}
$$
\ket{\chi_k} = \frac{1}{\sqrt{M}} \sum_{l=0}^{M-1} e^{-2\pi il \frac{k}{M}} \ket{l}
$$
$$
U(x_0) \ket{\chi_k} = \fraC{1}{\sqrt{M}} \sum_{l=0}^{M-1} e^{-2\pi i l \frac{k}{M}} \ket{l + x_0} =  \fraC{1}{\sqrt{M}} \sum_{\tilde l=0}^{M-1} e^{-2\pi i (\tilde l - x_0) \frac{k}{M}} \ket{\tilde l}  = e^{2\pi i k \frac{x_0}{M}} \ket{\chi_k}
$$
giving eigenvalue: $\omega(x_0, k) = e^{2\pi i k \frac{x_0}{M}}$. From this we could reconstruct the basis of QFT:
$$
[QFT]_{kl} = \frac{1}{\sqrt{M}} e^{2\pi i \frac{kl}{M}}
$$
\section{Lecture 3}
\subsection{Hidden Subgroup Problem}
Let $G$ be a finite group of size $|G|$. We are given an oracle $f: G \rightarrow X$ with $X$ just some set. We are promised there is a subgroup $K<G$ s.t.
\begin{itemise}
\item $f$ is constant on (left) cosets of $K$ in $G$
\item $f$ is distinct on distinct cosets
\end{itemise}\\\\
\textbf{Problem}: 'Determine' the 'hidden subgroup' $K$ (e.g. output a set of generators or sample uniformly from elements of $K$)\\\\
We want to solve in time $O(poly ( \log |G|))$ (efficent algorithm) with anuy consitent probability $1-\epsilon$.
\textbf{Examples of problems that can be cast as HSP}\\
Periodicity finding $f: \mathbb{Z}_M \rightarrow X$ periodic, period r 1-1 in period
$$
G = \mathbb{Z}_M, K = \{ 0,r,2r,..., (A-1)r\} < G
$$
Discrete Logarithm Problem: $p$ - prime number, $\mathbb{Z}^*_p$ group of integers with multiplication mod $p$, $g \in \mathbb{Z}_p^*$ to be a generator (or primitive root mod p). If $\mathbb{Z}^*_p = \{ g^0, g^1, ..., g^{r-2} \}$ and we have $g^{p-1} = 1$ (mod p). Fact: These always exist for $p$ is prime. Any $x \in \mathbb{Z}_p^*$ can be written as $x = g^y$ for some $y \in \mathbb{Z}_{p-1}$, $y = \log_g x$ is called the discrete log of $x$ to base $g$. Discrete log problem is given a generator $g$, $x \in \mathbb{Z}^*_p$ we want to compute $y = \log_g x$. To express this as the HSP:
$$
f: \mathbb{Z}_{p-1} \times \mathbb{Z}_{p-1} \rightarrow \mathbb{Z}_p^*
$$
$$
f(a,b) = g^a x^{-b} \mod p = g^{a-yb} \mod p
$$
Can check if $f(a_1, b_1) = f(a_2, b_2) \iff (a_1, b_1) = (a_2, b_2) + \lambda( y, 1), \lmbda \in \mathbb{Z}_{p-1}$:
$$
G = \mathbb{Z}_{p-1} \times \mathbb{Z}_{p-1}
$$
$$
K = \{ \lambda(y,1): \lambda \in \mathbb{Z}_{p-1} \} < G
$$
Then $f$ is constant and distinct on cosets of $K$ and generator $(y,1)$ of $K$ gives $y = \log_g x$\\\\
\textbf{Graph Problems}:\\
So we can solve problems like those above where $G$ is abelian, but we can also solve graph problems.\\\\
Consdier graph $A = (V,E)$, $|V| = n$ lets say that the graph is undirected and there is at most one edge between any two vertices. Vertices here are labelled by numbers from 1 to $n$.\\\\
Lets define an adjacency matrix $M_A$: $[M_a]_{ij} = \begin{cases} 1 & \iff (i,j)\\ 0 & \text{otherwise} \end{cases}$. The permuation group of $[n]$, $|P_n| = n!, \log |P_n| \sim O(n \log n)$. Define a group of automorphisms of group $A$ which is a set of permuations with the following property: $\pi \in P_n$ s.t. $\forall i,j (i,j)$ is an edge in $A$ $\iff (\pi(i), \pi(j))$ is also an edge in $A$.\\\\
An associated HSP (the case of non-abelian $G$):
$$
G = P_n, X = \text{set of all labelled graphs on }n\text{vertices}
$$
For any $A \in X$, define $f_A: G \rightarrow X$, $f_A(\pi) = $"A with vertex labels permuted by $\pi$"
$$
K = Aut(A)
$$
(Check $f(K)$ is constant and disctint on cosets of $Aut(A)$)\\\\
\textbf{Applications}:\\
If we can sample uniformly from $K$, then we can solve Graph Isomorphism problem (GI).This has a number of different applications in areas of computer science. Two labelled graphs $A$ and $B$ with $n$ vertices are isomorphic if there is a 1-1 map (i.e. permutation) $\pi[n] \rightarrow [n]$ s.t. $\forall i,j \in [n] (i,j)$ is an edge in $A$ $\iff$ $(\pi(i), \pi(j))$ is an edge in $B$. The GI problem is given to graphs $A$ and $B$ and deciding if they are isomorphic. This can be represented as a non-abealian HSP. There is no known poly(m) time classical algorithm to solve this problem, so $GI$ is clearly in $NP$ but not believed to be $NP$-complete (a class of problems such that every problem in NP can be reduced to an NP-complete problem these are the hardest NP problems).
In 2017, L Babai presented a quasi-polynomial algorithm for GI runtime $n^O(\log n)^2)$. This ranks in between polynomial runtime and exponential algorithms.
\section{Lecture 4}
\textbf{Quantum algorithm for finite abelian HSPs}
- Generalisation of period-finding algorithm\\
Write our abelian group $(G, +)$ additively\\
Construction of shift-invariant states and Fourier transform for $G$.\\
Representations of abelian $G$:\\
Consdier the mapping $\chi: G \rightarrow \mathbb{C}^* = \mathbb{C}- \{ 0\}$ with multiplication that satisfies:
$$
\chi( g_1 + g_2) = \chi(g_1) \chi(g_2), \forall g_1,g_2 \in G 
$$
$\chi$ is a group homomorphism from $G$ to $\mathbb{C}^*$. Such $\chi$'s are called irreducible representations of $G$. They have the following properties:
\textbf{Theorem 1}:\\
1) any value $\chi(g)$ is a $|G|$-th root of unity ($\chi \in \rightarrow S^1$ the unit circle)\\
2) Schur's lemma (orthogonality): If $\chi_i,\chi_j$ satisfy (HOM) then 
$$
\frac{1}{|G|} \sum_{g \in G} \chi_i (g) \bar {\chi_j} (g) = \delta_{ij}
$$
3) There are always exactly $|G|$ different functions $\chi$ satisfying (HOM).\\\\
\textbf{Examples}: $\chi(g) = 1, \forall g \in G$ is an irrep/ called a trival irrep\\
Label the trivial irrep as $\chi_0$, $0\in G$. Then for any other irrep $\chi \neq \chi_0$ orthonality to $\chi_0$ gives:
$$
\sum_{g \in G} \chi(g) = 0 \text{ if } \chi \neq \chi_0
$$
Going back to constructing shift-invariant states
\subsubsection{Shift-invariant states}
Consider a state space $\mathcal{H}_G, dim \mathcal{H}_G = |G|$ wtih basis $\{ \ket{g}\}_{g \in G}$. Now introduce shift operators $U(k)$ for $k \in G$ defined as follows:
$$
U(k): \ket{g} \rightarrow \ket{g+k}, g, k \in G
$$
All shift operators commute so there exists a simultaneous eigenbasis.\\
For each $\chi_k, k \in G$:
$$
\ket{\chi_k} = \frac{1}{\sqrt{|G|}} \sum_{g \in G} \bar{\chi_k} (g) \ket{g}
$$
By thereom 1 $\{ \chi_k \}$ form an orthonormal basis.
$$
U(g) \ket{\chi_k} = \chi_k(g) \ket{\chi_k}
$$
\textbf{Proof}:$$U(g) \ket{\chi_k} = \frac{1}{\sqrt{|G|}} \sum_{h \in G} \bar{\chi_k} (h) \ket{h + g}$$
        $$
        \{h' = h+ g\} = \frac{1}{\sqrt{|G|}} \sum_{h' \in G} \bar{\chi_k} (h' - g) \ket{h'}
        $$
        using HOM $\chi_k(-g) = \chi_k(g)^{-1} = \bar \chi_k(g) \implies \bar{\chi_k(h'-g)} = \bar \chi_k(h') \bar \chi_k(-g) = \bar \chi_k(h') \chi_k(g)$. Therefore,
        $$
        U(g) \ket{\chi_k} = \frac{1}{\sqrt{|G|}}\sum_{h' \in G} \chi_k(g) \bar \chi_k(h') \ket{h'} = \chi_k(g) \ket{\chi_k}
        $$
        So $\ket{\chi_k}$'s form a common eigenbasis\\\\
        Introduce Fourier transform QFT for a group $G$\\
- consider a unitary mapping on $\mathcal{H}_G$ mapping $\ket{\chi_k}$ basis to $\ket{g}$ basis
$$
QFT \ket{\chi_g} = \ket{g}, \forall g \in G
$$
$$
QFT^{-1} \ket{g} =  \ket{\chi_g}
$$
k-th column of $QFT^{-1}$ in $\ket{g}$ basis is mode of components of $\ket{\chi_k}$:
        $$
        [QFT^{-1}]_{gk} = \frac{1}{\sqrt{|G|}} \bar{\chi_k}(g)
        $$
        \textbf{Example}: $G = \mathbb{Z}_M$L\\
        Check $\chi_a(b) = e^{\frac{2\pi i a b}{M}}, a, b \in \mathbb{Z}_M$ satsifies HOM and has its irreps labelled by $a \in \mathbb{Z}_M$ with $\chi_0(b) = 1 \forall b \in \mathbb{Z}_m$.
        $$
        G = \mathbb{Z}_{M_1} \times ... \times \mathbb{Z}_{M_r}
        $$
        $$
        (a_1, ..., a_r ) = g_1, (b_1, ..., b_r) = g_2
        $$
        $$
        \chi_{g_1} (g_2) = e^{2\pi i ( \frac{a_1 b_1}{M_1} +... + \frac{a_r b_r}{M_r}}
        $$
This satifies HOM and our $QFT_G = QFT_{M_1} \otimes ... \otimes QFT_{M_r}$ on $\mathcal{H}_G = \mathcal{H}_{M_1} \otimes ... \otimes \mathcal{H}_{M_r}$.\\\\
This second example is exhaustive since we have a classification theroem:\\
\textbf{Classification theorem}: Any fintie abelian group $G$ is isomorphic to a direct product of the form $G = \mathbb{Z}_{M_1} \otimes ... \otimes \mathbb{Z}_{M_r}$. So $M_1$ can be taken in a form $p_1^{s_1},... p_r^{s_r}$.
\subsubsection{Quantum algorithm}
$$
f: G \rightarrow X
$$
with hidden subgroup $K$ and cosets $k = 0 + k, g_2 + k, ... g_m +k$, $m = \frac{|G|}{|K|}$. we will work on $\mathcal{H}_{|G|} \otimes \mathcal{H}_{|X|}, $\{\ket{g}\ket{x}\}_{g \in G, x \in X}$.\\
\begin{itemlist}
\item Create a state $\frac{1}{\sqrt{|G|}} \sum_{g \in G} \ket{g} \ket{0}\\
\item Apply $U_f$ and $\frac{1}{\sqrt{|G|}}\sum_{g \in G} \ket{g} \ket{f(g)}\\
        \item Measure the second register to get $f(g_0)$. The first register will not give the coset state:
                $$
                \ket{g_0 + k} = \frac{1}{\sqrt{|k|}} \sum_{k \in K} \ket{g_0 + k} = U(g_0) \ket{K}
                $$
        \item apply QFT and measure to get a result $g \in G$
\end{itemlist}
\section{Lecture 6}
We can write $\ket{K}$ in the shift-invariant basis $\{ \chi_g \}_{g \in G}$
$$
\ket{K} = \sum_g a_g \ket{\chi_g}
$$
$$
\ket{g_0+K} = U(g_0) \ket{K} = \sum_g a_g \chi_g(g_0) \ket{\chi_g}
$$
as $QFT \ket{\chi_g} = \ket{g}$ so after we apply QFT
$$
prob(g) = |a_g \chi_g(g_0)|^2 = |a_g|^2, |\chi_g(g_0) | = 1
$$
$$
QFT \ket{K} = \frac{1}{\sqrt{|G|}} \frac{1}{\sqrt{|K|}} \sum_{l \in G} ( \sum_{k \in K} \chi_l (k) \ket{l})$$
$\sum_{k \in K} \chi_l (k) \ket{l}$ involves irreps $\chi_l$ of $G$ restricted to subgroup $K < G$, and each such object is itself an irrep in $K$. Hence we have the following relation:
$$
\sum_{k \in K} \chi_l(k) = \begin{cases} |k| & if \chi_l \text{ restricts to the trival irrep of K}\\ 0 & \text{otherwise} \end{cases}
$$
$$
QFT \ket{K} = \sqrt{\frac{|K|}{|G|}} \sum_{l \in G} \ket{l}
$$
Then a measurement gives a uniformly random choice of $l$ s.t. $\chi_l(k) = 1$.\\
If $k$ has generators $k_1, ..., k_n$ where $M = O(\log(K)) = O(\log |G|)$. Then the output of a measurement gives us $\chi_l(k) = 1 \forall i$.\\\\
It can be shown that if $O(\log(|G|)$ values of $l$ chosen uniformly at random then with probability $>\frac{2}{3}$ they will suffice to determine a generating set for $k$ via the equations $\chi_l(k) = 1$.\\\\
\textbf{Example}: $G = \mathbb{Z}_{M_1} \times ... \times \mathbb{Z}_{M_l}$\\
$l = (l_1, ..., l_q) \in G$, $g = (b_1, ..., b_q) \in G$ gives $\chi_l(g) = e^{2\pi i( \frac{l_1 b_1}{M_1} + ... + \frac{l_q b_q}{M_q}$\\
        For $k = (k_1, ..., k_q) \in K$ with $\chi_l(k) =1 \implies \frac{l_1 k_1}{M_1} + ... + \frac{l_q k_q}{M_q} = 0 mod 1$. This is a homogenous linear equation on $k$ and $O(\log(k))$ such equations determine $k$ as null space.
        \subsubsection{Remarks on HSP for non-abelian groups $G$}
        Now we will consider multiplicative shifts. As before we can generate a bunch of coset states but it is curious to investigate what breaks down. 
        $$
        \ket{g_0 K} = \frac{1}{\sqrt{|K|}} \sum_{k \in K} \ket{g_0 k}, g_0 \in G\text{ is chosen randomly}
        $$
        The real problem with QFT construction is that there is no good basis of shift invariant states. This is because $U(g_0)$ don't commute. \\\\
        \textbf{Construction of non-abelian QFT}\\
        Consider a $d$-dimensional representation of $G$ and a group homomorphism $\chi: G \rightarrow U(d)$\\
        $\chi$ is irreducible if no subset of $\mathbb{C}^d$ is left invariant by all matrices $\chi(g), g \in G$. (i.e. we cannot simulatenoulsy block-diagonalize all of $\chi(g)$'s by a simple basis change)\\
        Let's define a complete set of irreps. It is a set $\chi_1,..., \chi_m$ s.t. that any irrep is unitarily equivalent to one of them. e.g. $\chi \sim \chi' = V \chi CV^{-1}, V \in U(d)$\\
        \textbf{Example}:  $G$ is abelian , all irreps have $d=1$, since all $\chi(g)$ commute. Theorem( non-abelian analogue of Theorem 1) (consullt Fulton and Hardes "Representation Theory" fo more information)\\
        If $d_1,...,d_m$ are the dimensions of a complete set of irreps $\chi_1, ..., \chi_M$ then:\\
        1) $d_1^2 +... + d_m^2 = |G|$\\
        2) $\chi_{i,jk} (g) $ is $(j,k)$ th matrix entry of $\chi_i(g)$ then by Schur orthogonality:
        $$
        \sum_g \chi_{i,jk}(g) \bar{\chi}_{i',j'k'}(g) = |G|\delta{ii'}\delta_{jj'}\delta{kk'}
        $$
Now if we look at the states that correspond to these irreps $\chi_{i,jk} = \sum_g\in G} \bar{\chi}_{i,jk} (g) \ket{g}$ they form an orthonormal basis.\\
QFT on $G$ is defined to be a unitary rotation between two basis of $\{ \chi_{i,jk} \}$ basis $\rightarrow \{ \ket{g} \}_{g \in G}$.\\
These takes $\ket{\chi_{i,jk}}$ are not shift-invariant for all $U(g_0)$ so this implies that measuring coset state $\ket{g_0 k}$ in the $\{ \ket{\chi}\}$ basis results in an output distribution that is not independant of $g_0$.\\
A "partial" shift-invariance survives. Consider a measuremnt $M_{rep}$ on $\ket{g_0 k}$ this measurement will only distinguish the irreps ($i$ values) and not all $(i,j,k)$'s. the outcome $i$ will be associated with $d_i^2$ dimensional orthogonal subspaces that are spanned by $\{\chi_{i,jk}\}_{j,k = 1}^{d_i}$.\\
Then $\chi_i(g_1 g_2) = \chi_i(g_1) \chi_i(g_2) \implies$ the output distribution of $i$ values is indeed independant of $g_0$.\\
So this gives us direct (but incomplete) information about $k$. For instance, conjugate subgroups $k$ and $L = g_0 K g_0', g_0 \in G$. This measurement will give us the same statistics.\\
$M_{rep}$ will result in the same output statistics\\
Not everything is lost there are some cases when this information is enough.\\\\
The reason HSP is good in the abelian case is we have an efficent $QFT$ transform. In other words QFT can be implemented in $poly(\log(|G|)$ times. This is true for abelian groups and some non-abelian groups (e.g. $P_n$).\\
\texbtf{Some partial results}:\\
For normal subgroups $gk = kg \for g \in G$ we have a theorem proven by Hall green, russel Ta shma in 2003 SIAM J Comp 32, p 916-934:\\
Suppose $G$ has QFT that is efficently implementable. Then if a hideen subgroup $k$ is a normal subgroup, then there is an efficent quantum HSP\\\\
Theorem( Edinsguin, Hoyer, Knill, 2004)\\
For general non-abelian HSP, $M = O(poly \log(|G|))$ then random coset states $\ket{g_1k}$,..., $\ket{g_nk}$ suffice to determine $k$ . But it is not known how to efficently determine $k$ from the $M$ coset states.
\section{Lecture 6}
\subsection{Phase estimation Algorithm}
- Unifying principle for quantum algorithms based on QFT\\
- also gives an alternative way of factoring (orginally discovered  by Kitaev)\\
The fact the phase estimation algorithm became so wide spread and you can cast every algorithm that is tangenetially ralted to QFt in QPE\\
Givem a unitary operator $U$. and eigenstates $\ket{v_{\phi}} = U\ket{v_{\phi}} = e^{2\pi i \phi} \ket{v_{\phi}}$\\
Want to estiamte the phase $\phi$ $0< \phi< 1$  ( up to $n$ bits of precision $\phi = 0. i_1 i_2,...i_n = i_1/2 + i_2/4 + ...$ for any given $n$)\\
We will have to implement controlled unitary opearotrs and in particular we will need Controlled-$U^k$ for integers $k$
$$
C-U^k \ket{0} \ket{\xi} = \ket{0}\ket{\xi}, C-U^k \ket{1}\ket{\xi} = \ket{1} U^k \ket{\xi}
$$
$\ket{\xi}$ has a general dimension $d$:
$$
U^k \ket{v_{\phi}} e^{2\pi i k \phi} \ket{v \phi}, C-U^k = (C-U)^k
$$
If we are given $U$ as a circuit description, we can easily implement $C-U$ by controlling each gate in $U$'s circuit. However, if $U$ is given as black box (e.g. a physical operation in the lab) we need further information as there is an inherent ambiguity as we have to account for local phase $e^{i\theta U}$ as it has no effect normally unless you use a controlled operation. \\\\
        If the unitary is specified in this ambiguous way we need to figure out what to do. It suffices to have an eigenstate $\ket{l}$ with a known eigenvalue $U\ket{l} = e^{i \alpha} \ket{l}$ then $e^{i \theta} U$ will map $\alpha \rightarrow \alpha + \theta$. Consider the following circuit:
       
\centerline{ \Qcircuit @C=1em @R=.7em {
\lstick{\ket{a}} & \qw & \ctrl{1} & \qw & \ctrl{1} & \gate{X} & \gate{P(-l)} & \gate{X}\\
\lstick{\ket{\xi}} & \qw & \qswap & \qw & \qswap & \qw & \qw & \qw \\
\lstick{\ket{\alpha}} & \qw & \qswap \qwx & \gate{U} & \qswap \qwx & \qw & \qw & \qw
               }}
               with $P(-l) = \begin{pmatrix}1 & 0 \\ 0 & e^{-il} \end{pmatrix}$. This correctly gives $C-U$. We'll want a "generalised controlled-U" that gives:
               $$
               \ket{x} \ket{\xi} \rightarrow \ket{x} U^x \ket{\xi} x \in \mathbb{Z}_{2^n}
               $$
               For $x= x_{n-1}...x_1x_0 = 2^0 x_0 + 2^1 x_1 + 2^2 x_2 ... + 2^{n-1} x_{n-1}$:\\
              \centerline{ 
             \Qcircuit @C=1em @R=.7em { 
               \lstick{\ket{x_{n-1}}} & \qw & \qw & \cdots & \ctrl{4} \\
                & \cdots & \cdots & \cdots & \cdots \\
               \lstick{\ket{x_{1}}} & \qw & \ctrl{2} & \cdots & \qw \\
               \lstick{\ket{x_{0}}} & \ctrl{1} & \qw & \cdots & \qw \\
       \lstick{\ket{\xi}} & \gate{U^{2^0}} & \gate{U^{2^1}} & \cdots & \gate{U^{2^{n-1}}} 
               }}
       
               If we input $\ket{\xi} = \ket{ V_{\phi}}$ then we get $e^{2\pi i \phi x} \ket{x} \ket{v_{\phi}}$. Now superpose over all $x = 0,1,2,..., 2^{n-1}$ by applying hadamards to all the qubits before applying the circuit, take $\ket{\xi} = \ket{v_{\phi}}$:
\\\\
This gives output $\ket{A} = \frac{1}{\sqrt{2^n}} \sum_x e^{2\pi i \phi x} \ket{x}$. Applying $QFT^{-1}$ to $\ket{A}$ and measure. We get $y_0 y_1 ... y_{n-1}$. Then output the number $0.y_1y_2...y_{n-1} = \frac{y_0}{2} + \frac{y_1}{4} + ... + \frac{y_{n-1}}{2^{n}}$ as an estimate of $\phi$.\\\\ Now lets assume an idealised situtation where $\phi$ has only $n$ binary digits:
$$
\phi = 0.z_1...z_{n-1}
$$
Then $\phi = \frac{z}{2^n}$ where $z$ is an n-bit integer in $\mathbb{Z}_{2^n}$:
$$
\ket{A} = \frac{1}{\sqrt{2^n}} \sum_2 e^{2\pi i 2^n z/2^n} \ket{x} 
$$
is a QFT of $\ket{z}$. Applying $QFT^{-1} \ket{A} = \ket{z}$ and we get $\phi$ exactly with certainity.\\\\
Note the algorithm up to the final measurements is a unitary operation mapping:
$$
\ket{0} \ket{0} ... \ket{0} \ket{ v_{\phi}} \rightarrow \ket{z_0} \ket{z_1} ... \ket{z_{n-1}} \ket{v_{\phi}}
$$
If $\phi$ has more than $n$ bits $\phi = 0.z_0 z_1... z_{n-1}| z_{n} z_{n+1}$.\\
\textbf{Theorem (PE)}: If measurements in the algorithm give $y_0y_1...y_n$ and the aoutput $\Theta = 0. y_0y_1...y_{n-1}$ then :\\
a) Prob ( $\Theta$ is closeset $n$-binary digit approx to $\phi$) $\geq \frac{4}{\pi^2} = 0.4$\\
b) Prob( $|\Theta - \phi| \geq \epsilon$) $\leq O(\frac{1}{2^n \epsilon}$\\\\
We will show that  Prob( $|\Theta - \phi| \geq \epsilon$) $\leq \frac{1}{2^{n+1} \epsilon})$
\section{Lecture 8}
Today we will prove the Phase estimation theorem. We need to change the defintion of distance as we need a distance on a circle.\\\\
Define $d(\theta, \phi) = \min \{ |\theta -\phi|, |1 + \phi - \theta|, |1+ \theta - \phi|\}$ which is the distance on the circle. Lets consider the normal binary expansion 0.999999 the closest string should be 1.\\\\
\textbf{Theorem (Phase Estimation)}: If the output of PE algorithm with $n$ lines (initited in as zeros) is $\theta = 0.y_0y_1...y_{m-1}$, then:\\
a) Prob ( $\theta$ is closeset $n$-binary digit approx to $\phi$, $d(\theta, \phi) \leq \frac{1}{2^{n+1}}$) $\geq \frac{4}{\pi^2} = 0.4$\\
b) Prob( $|\theta - \phi| \geq \epsilon$) $\leq O(\frac{1}{2^n \epsilon})$ for $\epsilon$ fixed\\\\
Recall: the output is obtained by measuring an $n$-qubit state $QFT^{-1} \ket{A}$ where $\ket{A} = \frac{1}{\sqrt{2^n}} \sum_{x = 0}^{2^n-1} e^{2\pi i l x} \ket{x}$
$$
QFT^{-1} \ket{x} = \frac{1}{\sqrt{2^n}} \sum_{y= 0}^{2^n-1} e^{-2\pi i \frac{y x}{2^n} \ket{y}}
$$
Soon we will change the notation to make sure it is not overloaded with these powers of $2^n$
$$
QFT^{-1} \ket{A} = \frac{1}{2^n}\sum_{y=0} \sum_{x=0} e^{2 \pi i ( \psi - \frac{y}{2^n})x} \ket{y}
$$
Let $\{ \phi = 2^n \psi \}$:
$$
QFT^{-1} \ket{A} = \frac{1}{2^n}\sum_{y=0} \sum_{x=0} e^{2 \pi i  \frac{\phi - y}{2^n})x} \ket{y} = \frac{1}{2^n}  \sum_y \frac{ 1 - e^{2\pi i (\phi - y)}}{ 1 - e^{2\pi i \frac{\phi - y}{2^n}}} \ket{y}
$$
In the case $\phi - y \neq 0$:
$$
Prob( see\> y) =  \frac{1}{2^n} |\sum_{x=0}^{2^n-1} e^{2\pi i\frac{(\phi - y)}{2^n} x}|^2 = \frac{1}{2^{2n}} \frac{ |1 - e^{2\pi i (\phi - y)}|^2}{ | 1 - e^{2\pi i \frac{\phi - y}{2^n}}|^2} = \frac{1}{2^{2n}} \frac{| 1 - e^{2\pi i (\psi - \frac{y}{2^n}} 2^n|^2}{| 1 - e^{2\pi i (\psi - \frac{y}{2^n}}|^2}
$$
As $\theta_y = \frac{y}{2^n}$ and observe that $|1- e^{2\pi i(\psi - \theta_y})|^2 = |1 - e^{2\pi i d(\psi, \theta_y)}|^2$ therefore:
        $$
Prob( see\> y) =  \frac{1}{2^{2n}} \frac{|1 - e^{2\pi i 2^n d(\psi, \theta_y)}|^2}{|1 - e^{2 \pi i d(\psi, \theta_y}|^2}
        $$
As $0<d(\psi, \theta_y)\leq \frac{1}{2}$ we will ise the following bounds:\\
               i) $|1-e^{i \alpha}| \leq 2|$\\
               ii) $|1-e^{i \alpha}| \leq |alpha|$\\
               iii) For $|alpha| \leq \phi$ $|1- e^{i \alpha}|= 2|sin( \frac{\alpha}{2})| \geq \frac{2|alpha|}{\pi}$\\\\
               The last of thse comes from the fact that for positive $\alpha$ we hae $\sin(\alpha/2) \geq \frac{\alpha/ \pi}$\\\\
               When $d(\psi, \theta) \leq \frac{1}{2^{n+1}}$ implies that $y$ is the best approxiamtion for $\psi$ and $2\pi d(\psi, \theta_y) 2^n \leq \frac{2^{n+1}}{2^{n+1}} \pi$ so:
               $$
               Prob( y is best approximation) \geq \frac{1}{2^{2n}} | \frac{2}{\pi} \frac{2 \pi d(\psi, \theta_y)}{2 \pi d(\psi, \theta)}|^2 = \frac{4}{\pi^2}
               $$
               The calculations for the above will be on the moodle\\\\
               \textbf{Further remarks}:\\\\
               If $C-U^{2^n}$ is implement as $(C-U)^{2^{\alpha}}$, then PE algorithm needs exponential time ($ 1+ 2+... + 2^{n-1} = 2^{n}-1$). But for some some $U$ implementing $C-U^{2^k}$ requries only polynomial time so we get a poly-time PE algorithm. Harks back to the algorithm for finding powers by repeated squaring, expressing the exponent in binary and then doing repeated squaring. The number of applications of controlled unitaries does not depend on $d$ the dimension of the space. This can be used to rpovide an alternative factoring algorithm ( due to A.Kitaev) ( see example sheet).\\\\
               In many applications we feed an arbitary state to the last register rather than an eigenstate. If instead of $\ket{v_{\phi}}$ we input general state $\ket{\xi}$, expand in eigenbasis of $U$:
               $$
               \ket{\xi} = \sum_j c_j \ket{v_{\phi_j}}, U \ket{v_{\phi_j}} = e^{2 \pi i \phi_j} \ket{v_{\phi_j}}
               $$
               Then we get (before the final measuremnt) a unitary process $U_{PE}$:
               $$
               \ket{00..00} \ket{\xi} \rightarrow^{U_{PE}} \sum_j c_j \ket{\psi_j} \ket{v_{\phi_j}}
               $$
               The Born rule implies that the final measurement will give a choice of $\phi_j$'s (or an approximation) it can be choosen with probability $|c_j|^2$. This is not some average of the $\phi_j$ values.\\\\
               Will be elaborated more in the notes on moodle on the following:\\\\
               If you want to have $n$-qubits and want to get $m$-bits correctly probablility of success $1- \eta$, then must have :
               $$
               n \geq m + \log \frac{1}{\eta}
               $$
               \subsection{Amplitude amplification}
Much like when we multiple up to HSP we revisited shors alogrithm, in this case we revisit grovers algorithm. This is an apotheosis of technique in Grover's algorithm.
\subsubsection{Background}
\textbf{Reflection Operators}:
State $\ket{\alpha}$ in $\mathcal{H}_{d}$ , $n$-dim subspace $L_{\alpha}$ with $(d-1)$ dim orthogonal $l_{\alpha}^{\alpha}$
$$
I_{\ket{\alpha}} = I - 2 \ket{\alpha} \bra{\alpha}
$$
$$
I_{\ket{\alpha}} = - \ket{\alpha}
$$
$$
I_{\ket{\alpha}} \ket{\beta} = \ket{\beta}
$$
for any $\ket{\beta} \prep \ket{\alpha}$\\\\
For any unitary $U$: $UI_{\ket{\alpha}} U^{\dagger} = I_{U\ket{\alpha}}$, $U \ket{\alpha} \bra{\alpha} U^{\dagger} = \ket{\beta} \bra{\beta}$ for $\ket{\beta} = U\ket{\alpha}$.\\\\
Consider a $k$-dimensional subspace $A < \mathcal{H}_d$, any orthonormal basis $\ket{a_1}, ..., \ket{a_k}$. Lets consider a projection operator on to this subspace:
$$
P_A = \sum_{i=1}^k \ket{a_i} \bra{a_i} 
$$
Define a generalised projeciton operator:
$$
I_A = I - 2 P_A
$$
$$
               I_A \ket{\xi} = \begin{cases} \ket{\xi} & \ket{\xi} \in A^{\dagger}\\
               - \ket{\xi} & \ket{\xi} \in A\end{cases}
$$
Now lets recall what grover does very briefly (have a look at Part II course):\\\\
Search for a unique "goal" item  in unstructured database of $N = 2^n$ items.\\\\
Write $B_n =$ set of all n-bit strings. Given an oracle for $f: B_n \rightarrow B_n$ with the promise that there is a unique element $x_0 \in B_1$ with $f(\x_0) = 1$. Problem is to find $x_0$.\\\\
Closely related to class NP and Boolean satisfiability problem. This is explained in part II lecture notes.
\section{Lecture 8}
\textbf{Recap of grovers algorithm}
We are searching for a unique "good" element in an unstructured database, $N= 2^n$ items\\
We are given an oracle $f$ that maps from $B_n \rightarrow B_1$\\
Promise: There is a unique $x_0 \in B_n$ with $f(x_0) = 1$\\
Problem: Find $x_0$\\\\
Consider Grover iteration operator on $n$ qubits
$$
Q = - H_n I_{\ket{0}} H_n I_{\ket{x_0}} = - I_{\ket{\psi_0}} I_{\ket{x_0}}
$$
where $H_n = H\otimes ... \otimes H$ , $\ket{\psi_0} = H_n \ket{00.000} = \frac{1}{\sqrt{2^n}} \sum_{x \in B_n} \ket{x}$\\\\
One application of $Q$ uses 1 query of $U_f$\\\\
\textbf{Thereom (Grover '96)}: In 2-dim span of $\ket{\psi_0}$ and (unknown) $\ket{x_0}$ the action of $Q$ is rotation by angle $2 \alpha$ where $\sin \alpha = \frac{1}{\sqrt{N}}$. Hence grover's algorithm to find $x_0$ given $U_f$ is:
\begin{itemlist}
\item Make $\ket{\psi_0}$\\
\item Apply $Q$ $m$ times where $m = \frac{\arccos \frac{1}{\sqrt{N}}}{2} \arcsin \frac{1}{\sqrt{N}}$ to rotate $\psi_0$ very close to $x_0$ (within the angle $\pm \alpha$\\
\item Measure to see $x_0$ with high probability $1- \frac{1}{N}$
\end{itemlist}
For large $N$ $\arccos \frac{1}{\sqrt{N}} = \frac{\pi}{2}$ and $\arcsin \frac{1}{\sqrt{N}} = \frac{1}{\sqrt{N}}$ so $m = \frac{\pi}{4} \sqrt{N}$ iteractions or queries to $U_f$ needed.\\\\
Classically we need $O(N)$ quereies to find $x_0$ with any constant probability that does not depend on $N$, so this achieves a quadratic speed-up.
\subsection{Amplitude Amplification}
Let $G$ be any subspace ('good subspace') of state space $\mathcal{H}$ $G^{\prep}$ be its orthogonal complement ('bad subspace') $\mathcal{H} = G \oplus G^{\prep}$\\\\
Given any $\ket{\psi} \in \mathcal{H}$, we have unique decomposition with real positive coefficentt $\ket{\psi} = \sin \theta \ket{g} + \cos \theta \ket{b}, \ket{g} \in G, \ket{b} \in G^{\prep}$\\
Introduce reflection operators that flip $\ket{\psi}$ and good vectors:
$$
I_{\ket{\psi}} = I - 2 \ket{\psi} \bra{\psi}, I_G = I - 2 P_G
$$
$\sin \theta  = ||P_G\ket{\psi}|| = $length of good projection of $\ket{\psi}$\\
Introduce $Q = -I_{\ket{\psi}} I_G$
\subsubsection{Amplitude Amplification Theorem}
In the 2-dim space spanned by $\ket{g}$ and $\ket{\psi}$ $Q$ is roation by $2 \theta$ where\\
$\sin \theta = $length of good projection of $\ket{\psi}$\\
\textbf{Proof}: We have $I_G \ket{g} = - \ket{g}, I_G \ket{b} = \ket{b}$:
$$
Q \ket{g} = I_{\ket{\psi}} \ket{g}, Q\ket{b} = - I_{\ket{\psi}} \ket{b}
$$
$$
I_{\ket{\psi}} = I - 2( \sin \theta \ket{g} + \cos \theta \ket{b}) (\sin \theta \bra{g} + \cos \theta \bra{b})
$$
using the fact that $\bra{b}\ket{g} = 0, \bra{g} \ket{g} = \bra{b} \ket{b} = 1$:
$$
Q\ket{b} = \cos 2 \theta \ket{b} + \sin 2 \theta \ket{g}
$$
$$
Q \ket{G} = - \sin 2 \theta \ket{b} + \cos 2 \theta \ket{g}
$$
So
$$
Q \ket{g} = I\ket{g} - 2 \sin^2 \theta \ket{g} - 2 \sin \theta \cos \theta \ket{b} = \cos 2 \theta \ket{g} - \sin 2 \theta \ket{b}
$$
In the $\{ \ket{b}, \ket{g}\}$ basis, the $Q$ matrix is a rotation matrix by $2 \theta$:
$$
                       Q  = \begin{pmatrix} \cos 2 \theta & \sin 2 \theta \\ - \sin 2 \theta & \cos 2 \theta \end{pmatrix}
$$
As we apply $Q$ n $times we get $Q^{n} \ket{\psi} = \sin (2n+1) \theta \ket{g} + \cos (2n+1) \theta \ket{b}. If we measure $Q^{n}$ in $\{ \ket{b}, \ket{g} \}$ basis we have probabiliyt of seeing a good element of $\sin^2 (2n+1) \theta$ so this is maximised when $(2n+1) \theta = \frac{\pi}{2}. So for the nearest integer to $n = \frac{\pi}{4 \theta} - \frac{1}{2}$ we will be within $\theta$ of the element.\\\\
\textbf{Example}: If we have $\theta = \frac{\pi}{6}, n =1$ we can seee that $Q^1$ rotates $\ket{\psi}$ exactly onto $\ket{g}$.\\\\
Generally, for a given $\theta$ $n$ is not an integer, so we use $n=$ nearest integer to $ \frac{4\pi}{\theta} - 1 \approx \frac{4\pi}{\theta} = O(\frac{1}{\theta}) = O(\frac{1}{\sin \theta}) = O(\frac{1}{\text{length of good projection of } \ket{\psi}}$ and then $Q\ket{\psi}$ will be witin angle $\theta$ of $\ket{g}$ so the probability of seeing a good value is: $P \geq \cos \theta = 1 - O(\theta^2)$.\\\\
All this can implement if $I_{\ket{\psi}}I_G$ can be impmeneted efficently. see ES2.\\\\
For $I_G$ is suffices for $G$ to be spanned by computational basis states and have an indicator funciton $f$.
$$
f(x) = 1, x \text{if} x\text{ is good}, f(x) = 0 \text{ if} x \text{is bad}
$$
For $I_{\ket{\psi}}$ we usually have $\ket{\psi} = H_n \ket{00..000}$, then $\ket{\psi}$ can be implemented in $O(n)$ time where $n$ is the number of qubits.\\\\
In the amplitude amplification algorithm the relative amplitudes of good elements remain the same as they were in $\ket{\psi} = \sin \theta \ket{g} + \cos \theta \ket{b}$ so $\ket{g}$ remains the same just the amplitude varies. So AA amplifies overall $\ket{g}$ amplitude at the expense of reducing the amplitude of $\ket{b}$.\\\\
Second remark: Final state is generally not exactly $\ket{g}$, however, if $\sin \theta$ is known then there is a modification of this algorithm that uses a modest amount of resources t omake it exact.\\\\
The rotuine is useful for state prepation e.g.
$$
\sum_{x < N, \text{x is coprime to N}} \ket{x}
$$
\section{Examples Class 1}
Try to prove the same thing in 1(ii) in shors algorithm and it will be slightly nicer language than group theory.\\\\
For $\mathbb{Z}_2$ the irreps are $\chi_a(x) = (-1)^{ax}$ for $a, x \in \mathbb{Z}_2$ therefore $\mathbb{Z}_2^n$ has irreps $\chi_a(x) = (-1)^{a_1 x_1 + a_2 x_2 +...+ a_n x_n}$ so
$$
\ket{\chi} = \frac{1}{\sqrt{|G|}} \sum_g \bar \chi( g) \ket{g} = \frac{1}{\sqrt{|G|}} \sum_{b \in \mathbb{Z}_2^n} (-1)^{a b} \ket{b}
$$
and
$$
[QFT]_{ab} = \frac{1}{\sqrt{|G|}} (-1)^{ab} = \frac{1}{\sqrt{2^n}} (-1)^{a_1 b_1} (-1)^{a_2 b_2} ... (-1)^{a_n b_n}
$$
                       could be written using Hadmards with $QFT = H \otimes H \otimes... \otimes H$ as $H = \frac{1}{\sqrt{2}}\begin{pmatrix} 1 & 1\\ 1& -1 \end{pmatrix}$ with the columns being $a$ and the rows being $b$ so it is only $-1$ if they are both $1$ as otherwise it will have a zero in the exponent.
\\\\
First part of HSP problems is generate coset states and the second part is repeation (how many times you need to repeat the process). In standard HSP we make one query to $f$ to make the following state:
$$
\ket{y \oplus k} = \frac{1}{2^k} \sum_{x \in K} \ket{y \oplus x}, y \in \mathbb{Z}_2^n
$$
Apply $QFT =  H^{\otimes n}$ and measure. This gives an uniformly random output $c \in \mathbb{Z}_2^n$ which is such that the irrep $\chi_c$ of $G$ that is restricted to $K$ is the trivial irrep of $K$. In other words $\chi_c(a) = 1$ for all $a \in K$. In other word, $(-1)^{ac} = 1 \implies c a = 0$ mod 2. We know that this $k$ viewed as a subspace of $\mathbb{Z}_2^n$ has dimension $k$. So we need $(n-k)$ linearly indepdendant $c_i$ with $c_i a = 0$ to determine $K$.\\\\
In order to succeed with probability $1- \epsilon$ when given a constant probability $p$ we run the process some $M$ times. Then you calculate the probablity of $M$ runs failing to determine $k$ and show that for a constant overhead we can get any accuracy. $1- (1-p)^M > 1- \epsilon$.\\\\
Phase gates act with $P(\alpha) : \ket{0} \rightarrow \ket{0}, \ket{1} \rightarrow e^{i \alpha}\ket{1}$. Therefore can apply a fractional phase by first preparing a register with $i_1,...,i_n$ s.t. $y= 0.i_1...i_n$ and then by applying phase gates $P(\frac{1}{2}...P(\frac{1}{2^n}$to get the state $e^{iy}\ket{y}$.\\\\
When inventing algorithms whilst it is useful to think about eigenstates to start with make sure to run it through with a general state as you might need to uncompute at the end. e.g. need to apply an inverse controlled unitary in question 5 after computing the correct eigenvalue.\\\\
\textbf{We can't just discard additoinal registers like $\sum_j\lambda_j \ket{u_j} \ket{c_j}$ to $\sum_j \lambda_j \ket{u_j}$ we need to uncompute to remove these extra stuff.}
\section{Lecture 9}
\subsection{Amplitude amplification}
We have a Hilbert space we can partition into a good part and a bad part: $\mathcal{H} = G_{good} \oplus G_{good}^{\dagger}$, for every $\ket{\psi}$ we have $\ket{\psi} = \sin \theta \ket{g} + \cos \theta \ket{b}$. We have:
$$
I_{\ket{\psi}} = I - 2\ket{\psi} \bra{\psi}, I_G = I - 2 P_G, Q = - I_{\ket{\psi}} I_G
$$
We proved AA Thereoem: That in the plane spanned by $\ket{b}$ and $\ket{g}$ $Q$ is a rotation by $2\theta$.\\\\
Typically we are given some $\ket{\psi}$ and we use this property to rotate it to $\ket{g}$.
\subsection{Applications of Amplitude Amplification}
\textbf{Grover search with one or more (k) good items in N}:\\
Re look over Part II lecture notes and reprove for $k$ good elemetns. Maybe try problem sheet again.
$$
\ket{\psi} = \frac{1}{\sqrt{2^n}} \sum_{x \in B_n} \ket{x} = \sqrt{\frac{k}{N}} ( \frac{1}{\sqrt{k}} \sum_{good elem} \ket{x}) + \sqrt{\frac{N-k}{N}} ( \frac{1}{\sqrt{N-k}} \sum_{bad} \ket{x})
$$
G is spanned by $k$ good $\ket{x}$'s, $\sin \theta = \sqrt{\frac{k}{N}}$ so Q is rotation by $2 2 \theta$, $\theta = \arcsin \sqrt{\frac{k}{N}} \implies O(\sqrt{\frac{N}{k}})$ queries.\\\\
Note that for 2-bit case $N=4$ and $k=1$ good element we get $\theta = \arcsin \frac{1}{2} = \frfrac{\pi}{6}$ and one application of Q rotates $\ket{\psi_0}$ exactly onto $\ket{g}$\\\\
\textbf{Square-root speedy of general quantum algorithms}:\\
Let $A$ be a quantum algorithm/circuit(sequence of 'basic' unitary gates) on input states 
\ket{00.00}. The final state is $A \ket{00...000}$.\\
Good labels = desired computation outcomes.\\
$A\ket{0...0} = \alpha \ket{a} + \beta \ket{b}, \alpha = \sin \theta$  with $\ket{a}$ normalised but generally unequa superposition $\sum_{good x} c_x \ket{x}$.\\
So probability of success in 1 run is $|\alpha|^2$ so need to do $O(\frac{1}{|\alpha|^2})$ repetitionsof A to succeed with any given const probablity $1- \epsilon$.\\\\
Now lets try amplitude amplification instead of repeated measurement. We need to check we satissify some assumptions.\\
Assume we can check if the answer is good or bad.\\
                       So that we can implement $I_G \ket{x} \rightarrow \begin{cases} - \ket{x} & \text{x is good}\\ \ket{x} & \text{x is bad} \end{cases}$\\
                       Consider $\ket{\psi} = A \ket{0.00}$
                       $$
                       Q = - I_{A\ket{00.000}} I_G = - A I_{\ket{0..00}} A^{\dagger} I_G
                       $$
                       so all parts are implementable\\\\
                       By the amplitude amplifacation theorem $Q$ is a rotation by $2 \theta$ with $\sin \theta = |\alpha|$. So after $n = \frac{\pi}{4 \theta} = O(\frac{1}{|\theta|}) = O(\frac{1}{\sin \theta}) = O(\frac{1}{|\alpha|}$ repetitions $A\ket{0..0}$ will be rotated very near $\ket{g}$ and the final measuremnt will succeed with high probability. So we get a square root time speed up over a discrete repetition method.\\\\
                       Each application of $Q$ need one $A$ and one $A^{\dagger}$. and can think of $A^{\dagger}$ as inverse gates in reverse order so it has a similar complexity to $A$.\\\\
                       If success probability of $A$ is known then we can do even better and we can cook up the rotation so exactly get to $\ket{g}$ (this exact method will be covered in ES2). This will convert the probabilistic algorithm $A$ into a deterministic one.\\\\
                       \textbf{Quantum counting}:\\
                       Here we apply amplitude amplification and phase estimation.\\
                       Given $f: B_n \rightarrow B_1$ with $k$ good $x$'s we want to estiamte $k$  with (instead of finding a good $x$).\\
                       Let's recall the grover operator: $Q_G$ for $f$ is a rotation by $2\theta$ in this 2D plane spanned by $\ket{\psi_0} = \frac{1}{\sqrt{2^n}} \sum_{x \in B_n}$ and its good projeciton $\ket{g} = \frac{1}{\sqrt{k}} \sum_{good x} \ket{x}$ with $\sin \theta = \sqrt{\frac{k}{N}} \approx \theta$\\
                       Recall that rotations in $\{ \ket{b}, \ket{g} \}$ palne have eigenvaleus and eigenvectors:
                       $$
                       \ket{e_{\pm}} = \frac{1}{\sqrt{2}} (\ket{b} \pm i \ket{g})
                       $$
                       $$
                       \lambda_{\pm} = e^{\pm 2 i \theta}
                       $$
                       Then check that $\ket{\psi_0} = \sin \theta \ket{g} + \cos \theta \ket{b} = \frac{1}{\sqrt{2}} ( e^{i \theta} \ket{e_+} + e^{- i \theta} \ket{e_-})$\\
       Writing $e^{\pm 2 i \theta} $ as $e^{2 \pi i \phi_{\pm}}$ with $\phi_{\pm} \in (0,1)$:
               $$
               \phi_+ = \frac{2\theta}{2\pi} = \frac{\theta}{\pi}
               $$
               $$
               \phi_- = \frac{-2 \theta + 2 \pi}{2 \pi} = 1 - \frac{\theta}{\pi}
               $$
               $\frac{\theta}{\pi} = \frac{1}{\pi} \sqrt{\frac{k}{N}}$\\
               Run QPE algorithm with $U =$ "Grover Q" and estimate register set to $\ket{\psi_0}$: will output $\frac{\theta}{\pi}$ or $1- \frac{\theta}{\pi}$ with probability $\frac{1}{2}

               \section{Lecture 10}
               $$
               f: B_n \rightarrow B_1
               $$
               which has $k$ good $x$'s and we want to estimate $k$. This is harder than just finding a solution.
               $$
               \ket{\psi_0} = \frac{1}{\sqrt{2^n}} \sum_{x \in B_n} \ket{x}, \ket{g} = \frac{1}{\sqrt{k}} \sum_{good x} \ket{x}
               $$
               $$
               \sin \theta = \sqrt{\frac{k}{N}}
               $$
               for $k << N$ so
               $$
               \ket{\psi_0} =  \sin \theta \ket{g} + \cos \theta \ket{b} = \frac{1}{\sqrt{2}} (e^{i\theta} \ket{e_+} + e^{-i \theta} \ket{e_-})
               $$
       Run QPE -> $\frac{\theta}{\pi}$ or $1- \frac{\theta}{\pi}}$ with probability $\frac{1}{2}$. We can get approximately $\theta$ in either case. For any $m$, running $QPE$ with $m$ qubit lines will give an $m$-bit approximation to $\sqrt{\frac{k}{N}}$ this uses $2^n$ C-Q gates.\\
       By Theroem PE we learn $\sqrt{\frac{k}{N}}$ to an additive error with constant probability $\frac{4}{\pi^2}$ using $O(2^m)$ queries. Write $\frac{1}{2^n}$ as $\frac{\delta}{\sqrt{N}}$ $\delta >0$. Thus we can learn $\sqrt{k}$ to additive error $O(\delta)$ using $O(2^n) = O(\frac{\sqrt{N}{\delta}})$ queries.\\\\
       Classically the same approximation (obtained with constant probability) requires $O(\frac{N}{\delta^2})$ so requires quadratically more effort.
       \subsection{Hamiltonian simulation}
       One of the most promising applications of qunatum computers is simulating quantum systems. This is difficult for classical computers as these physical systems have so many degrees of freedom. We won't do the nitty gritty but cover some of the core fundamental ideas and talk about the complexitiy of it.\\\\
       We want to use a quantum computer to simulate the evolution/dynamics of a quantum system given its Hamiltonian $H$.\\
       For $n$ qubits generally requires $O(2^n)$ time on a classical computer. We will do it in poly(n) time for a class of Hamiltonian. A lot of good hamiltonains that corrospond to real problems do admit a good representations in quantum computers. First we will focus on evolution and dynamics and then we will move on to studying the ground state properties of these hamiltonians.
       \subsubsection{Hamiltonain and quantum evolution}
       Consider a physical system  in state $\ket{\psi}$, with hamiltonain $H$: selfadjoint (Hermitian) operator/matrix - the quantum energy observable. Time evolution given by Schrodinger equation $(\hbar = 1)$ 
       $$
       \frac{d}{dt} \ket{\psi(t)} = - i H \ket{\psi(t)}
       $$
       We will consider time-independant Hamiltonians $H(t) = H$
       $$
       \ket{\psi(t)} = e^{-iHt} \ket{\psi(0)}
       $$
       $$
       e^A = I + A + \frac{A^2}{2!} + ...
       $$
       Thus given $H$ and time $t$ we want to simulate the action of a unitary $U(t) = e^{-i Ht}$ to a suitably good approximation.\\
       \textbf{Approximation (closeness) of unitary operators}
       Operator norm (spectral norm) $||A|| = \max_{||\psi|| = 1} ||A\ket{\psi}|| = |$maximium eigenvalue|$ (if $A$ is diagonalisable)$.
       $$
       ||A+B|| \leq ||A|| + ||B||
       $$
       We say that $U$ approximates $\tilde U$ to within error $\epsilon$ if $||U - \tilde U|| \leq \epsilon$ (for any $\ket{\psi}$ the action of $U$ $\tilde U$ are at most $\epsilon$ apart.\\\\
       In general this hamiltonain will be a huge $2^n\times 2^n$ matrix for an $n$qubit system so hard to write down so we will consider a special class of $k$-local hamiltonains. We will want to simulate $U = e^{- i tH}$ with a circuit of poly(n,t) "basic" unitary gates (this is the defintino of efficent simulation on a quantum computer).\\\\
       Not all $H$'s can be efficently simulated. One such class of efficent hamiltonains is k-local hamiltonains:
                       \textbf{$k$-local Hamiltonain}: $H$ is $k$-local on $n$ qubits if $H= \sum_{j=1}^m H_j$ where $H_j$ is a hermitain matrix acting on at most $k$ qubits. This action does not have to be contiouns a.k.a they don't need to be qubits 1-k it can be any subset of $k$ qubits.
                       $$
                       H_j = \tilde H_j \otimes I
                       $$
                       with $\tilde H_j$ encoding the hamiltonain on the $k$ qubits and the $I$ is on the rest of the system.\\\\
                       $m \leq (^n_k) = O(n^k) = poly(n)$ terms in $H$. We will see that many imporotant classes of hamiltonains fall into this category e.g.:\\
                       1. (3 qubits, 2-local)
                       $$
                       H = X \otimes I \otimes I - 5 Z \otimes I \otimes Y
                       $$
                       2. Write $M_{(k)}$ to denotes an operator $M$ acting on $k$ qubits and $I$ on the rest. This can correspond to the Ising model on $n\times n$ square lattice of qubits. 
$$H = J \sum_{i,j=1}^{n-1} Z_{(i,j)} Z_{(i, j+1)} + Z_{(i,j)}Z_{(i+1,j)}$$
Heisenberg model on a line:
$$
H = \sum_{i=1}^{n-1} J_x X_{(i)} X_{(i+1)}  + J_y Y_{(i)} Y_{(i+1)} + J_z Z_{(i)} Z_{(i+1)}
$$
with $J_x, J_y, J_z$ are real coeff.\\\\
Remember that $e^{-\sum_i H_jt} \neq  \prod_j e^{- H_j t}$ if $H_j$ don't commute\\
Also remember that $e^{-H_j t}$ are local unitaries acting on $k$ qubits.
\section{Lecture 11}
If we want to use some standard universal gate set, then we will invoke the theorem: \\\\
\textbf{Solovay-Kitaev Theorem}: Let $U$ be a unitary operator on $k$ (const) qubits, and $S$ be any universal set of quantum gates. Then $U$ can be approximated to within accuracy $\epsilon$ using a logarithm sequence of gate $O(\log^c (\frac{1}{\epsilon}))$ gates from $S$ with $c<4$. \\ We don't prove this result but it is in Nielsen and Chuang.\\\\
\textbf{Lemma A} about accumulation of errors: We need to prove this on ES2. Let $\{\bm u_i\}$, $\{ \bm v_i\}$ be sets of $m$ unitary operators such that $|| U_i - V_i || \leq \epsilon$. So if we want to approximate a whole sequence of $U_i$s: $||U_m...U_1 - V_m...V_1|| \leq m \epsilon$ so errors will accumulate linearly. This is kind of the worst case error as it is the maximium error over all $\ket{\psi}$.\\\\
Proof: using induction on $m$.\\ First lets consider a little warm up with the (easy) commuting case.\\
Consider the case:
$$
H = \sum_{j=1}^m H_j, \text{ any k-local Hamiltonian with commuting }H_j
$$
Then for any power $t$, $e^{-iHt}$ can be approximated to within $\epsilon$ by a circuit $O(m poly(\log (\frac{m}{\epsilon})))$ from any given universal set. (Note that $m = (^n_k) = O(n^k)$, this is poly$(n_1 \log \frac{1}{\epsilon})$ too and $\log \frac{1}{\epsilon}$ is the number of digits in the precision of the approximation\\
Proof: Using $SK$ theorem each $e^{-iH_jt}$ can be approximated to within $\frac{\epsilon}{m}$ with $O(poly(\log \frac{m}{\epsilon}))$ gates.\\
Lemma A implies the full product $\prod_{i=1}^m e^{-i H_j t}$ is then approximated to within $m \frac{\epsilon}{m} = \epsilon$ using a total of $O(m poly(\log \frac{m}{\epsilon}))$ gates.\\\\
\textbf{The full non-commuting case}\\
For any matrix $X$ we write $X + O(\epsilon)$ for $X + \epsilon$, where $||E|| = O(\epsilon)$\\
\textbf{Lemmma B}(Lee-Trotter formula):\\
Let $A, B$ be the following matrices with $||A|| \leq K$ and $||B || \leq K$ with $K<1$ (small). Then:
$$
e^{-iA} e^{-iB} = e^{-i(A+B)} + O(K)^2
$$
Proof: $e^{-iA} = I - iA + \sum_{k=2} \frac{(-iA)^k}{k!} = I - iA + (iA)^2 \sum_{k=0} \frac{(-iA)^k}{(k+2)!} $ we want to show that the sum in the last term has norm $< e^{-K}<1$. Therefore, $e^{-iA} = I - iA + O(K^2)$ so therefore:
$$
e^{-iA} e^{-iB} = (I - iA + O(K^2))(I- iB + O(K^2)) = I - i(A+B) + O(K^2) = e^{-i(A+B)} + O(K^2)
$$
Now, apply Lee-Trotter formula repeatedly to accumulate sums of $H_1...H_m$ in the exponent\\
Note that if each term $||H_j||<K$, then $||\sum_{i=1}^l H_i || < l K$ and we'll want this expression here to be $<1$ for all $l \leq m$. For now, we'll assume that $K< \frac{1}{m}$. For now we will also take $t =1$ (deal with general $t$ later). 
$$e^{-iH_1} e^{-iH_2}... e^{-iH_m} = [ e^{-i(H_1 + H_2)} + O(K^2)] e^{-iH_3}...e^{-iH_m}$$
So as $||H_1 + H_2 ||< 2K$ and the error that is error that is generated at each stage stays the same for subsequent unitaries $U_i$ as $||AU|| = ||A||$  therefore:
$$
e^{-iH_1} e^{-iH_2}... e^{-iH_m} = e^{-i(H_1+H_2)} e^{-iH_3}...e^{-iH_m} + O(K^2)
$$
Then by Lee-Trotter for $(H_1 + H_2)$ and $H_3$
$$
e^{-iH_1} e^{-iH_2}... e^{-iH_m} = [e^{-i(H_1+H_2+H_3)} +O((2K)^2)]e^{-iH_4}...e^{-iH_m} + O(K^2)
$$
as $||H_1 + H_2|| < 2K$. We can continue in the same fashion until we get \textbf{error estimate 1}:
$$
e^{-iH_1} e^{-iH_2}... e^{-iH_m} = e^{-i(H_1 + ...+ H_m)} + O(k^2) + O((2k)^2)) + ... + O( ((m-1)k)^2) =  e^{-i(H_1 + ...+ H_m)} + O(m^3 k^2)
$$
For general finite $||H_j||$'s and $t$ values $||H_jt|| < Kt$ (this value can be large). So we introduce a way of breaking down the sequence by introducing $N$ (that we will fix later) s.t. $\frac{H_j t}{N}$ gives $\tilde K = || \frac{H_jt}{N}|| < \frac{Kt}{N}$. We can think about this as dividing the time $T$ up into small $\frac{1}{N}$ intervals and then our unitary has the value:
$$
U = e^{i(H_1 + ... + H_m)t} = ( e^{i( \frac{H_1 t }{N} + ... + \frac{H_2 t}{N})})^N
$$
We want final error for $U$ to be less than $\epsilon$ so by lemma A we want the error of each step to be less than $\frac{\epsilon}{N}$. So using error estimate 1 we get 
$$
Cm^3 \tilde K^2 < \frac{\epsilon}{N} \implies N > \frac{C m^3 k^2 t^2}{\epsilon}
$$
Then || e^{-i \frac{H_1t}{N}} ... e^{- i\frac{H_nt}{N}} - e^{-i(H_1 t + ... H_n t)} || < \frac{\epsilon}{N}$. BY Lemma A:
$$
|| (e^{-i \frac{H_1t}{N}} ... e^{- i\frac{H_nt}{N}})^N - e^{-i(H_1 t + ... H_n)} || < \epsilon
$$
So the total circuit size is $O(m^4 \frac{(Kt)^2}{\epsilon})$ we are dealing with Hamiltonians that are $k$-local. So for $n$ qubits and $k$-local terms $m = O(n^k)$. Therefore, the total circuit size is $O( n^{4k} \frac{(Kt)^2}{\epsilon})$. You can actually refine this method to get order $t^{1+ \delta}$ but they are much more technical and would need many more lectures to explain.\\\\
We have a circuit of size $|C| = O( \frac{m^4 (kT)^2}{\epsilon})$ and if we want to use gates from the universal set
\sections{Lecture 12}
\subsection{The Local Hamiltonaian Problem and QMA}
Recap: Defintion of NP. There is no known way to solve it in polynomial time but it is easy to verify the solution in polynomial time.\\
We will adopt the language of the part II lecture course. A language is in NP if it has an efficent (poly-time) verifier $V$. A verifier $V$ for a language $L$ is a computation with two inputs $w$, $c$ s.t. if $w \in L$ then for some $c$ $V(w,c)$ halts with "accept". such $c$ is called a certificate/proof (of membership) for $w$. If $w \cancel{\in} L$, then for all $C$, $V(w,c)$ halts outputing "reject". V is a poly-time verifier for all pairs $(w,c)$. V runs in poly(n) time $n = |w|$.
\subsubsection{The satisficatibility problem (SAT)}
Boolean forumlas $\phi(x_1...x_n): \{ 0,1\}^n \rightarrow \{0,1\}$ and every $(b_1...b_n), b_i \in \{0,1\}$ s.t. $\phi(b_1...b_n) = 1$ is called a satisfying assignment. $V(\phi,c)$ evaluates $\phi(c)$. SAT is not known to be in $P$.\\\\
Theory of NP-completeness shows that many different problems that look very different (SAT, Travelling Salesman Problem,  integer linear programming) are essentially the same problem. We can translate instances of one to another in deterministic poly time.\\\\
Now lets try to relax some of the requirements on $NP$. Consider a setting where the prover and verifier may use randomness and sometimes make errors ( allowing for some probability of error say $\frac{1}{3}$). In other words when $w \in L$ a prover shoul dbe able to prepare a certificate/proof s.t. Prob$(V(w,c)$ accepts) $\geq \frac{2}{3}$ and Prob$(V(w,c)$ rejects) $\leq \frac{1}{3}$. This defines complexity class $MA$ (Merlin-Arthur). Merlin is regarded as the omniscient prover and Arthur is a randomised poly time verifier. \\\\
\textbf{Quantum Merlin Arthur class}\\
The direct analgoue of $NP$.\\
It is a class of promise problems. A promise problem $L$ partitions $\{0,1\}^k$ of all binary strings into $L_0, L_1, L_*$.\\
An algorithm is promised that it never recieves inputs from $L_*$. However, if the input is from $L_0/L_1$ then it has to determine $0/1$.\\
\textbf{QMA}: A prmise problem $L = (L_0,L_1,L_*)$ is in the class $QMA$ if there exists a uniform family of circuits $\{C_n\}$ with two input registers and one output qubit and a prolynomial $p()$ s.t. $\forall w \in \{0,1\}^k$\\
Completeness: If $w \in L_1 \cap \{0,1\}^n$ then there exists a $p(n)$- qubit state $\ket{\psi}$ (a proof/witness state) such that $C_n$ outputs 1 with probability $\geq \frac{2}{3}$ when run on $w$ and $\ket{\psi}$\\
Soundness: If $w \in L_0 \cap \{0,1\}^n$ then for every $p(n)$ qubit state $\ket{\psi}$ the circuit $C_n$ outputs 1 with probability $\leq \frac{1}{3}$ when run on $w$ and $\ket{\psi}\\
\textbf{QMA}: A prmise problem $L = (L_0,L_1,L_*)$ is in the class $QMA$ if there exists a uniform family of circuits $\{C_n\}$ with two input registers and one output qubit and a prolynomial $p()$ s.t. $\forall w \in \{0,1\}^k$\\
Completeness: If $w \in L_1 \cap \{0,1\}^n$ then there exists a $p(n)$- qubit state $\ket{\psi}$ (a proof/witness state) such that $C_n$ outputs 1 with probability $\geq \frac{2}{3}$ when run on $w$ and $\ket{\psi}$\\
Soundness: If $w \in L_0 \cap \{0,1\}^n$ then for every $p(n)$ qubit state $\ket{\psi}$ the circuit $C_n$ outputs 1 with probability $\leq \frac{1}{3}$ when run on $w$ and $\ket{\psi}$\\
\textbf{Remarks}:\\
1) If we replace $\ket{\psi}$ with a classical bitstring we get the class QCMA\\
2) If we additionally to (1) force the verifier to be classical MA\\
3) If we additionally to (1),(2) repalce success probability ($\frac{2}{3}$) by 1 -> NP
$$
NP \leq MA \leq QCMA \leq QMA
$$
SAT is $NP$-complete (Cook-Levin theorem). Consider a special case: $k-SAT$\\
$\phi$ is the conjunction of clauses each of which is a disjunction of $k$-literals. e.g. for $k=3$:
$$
( x_1 \cup \bar x_2 \cup x_3) \cap ( \bar x_1 \cup x_2 \cup x_4) \cap (x_1 \cup \bar x_4 \cup x_5)
$$
k-SAT is NP-complete for $x \geq 3$ and k-SAT is in $P$ when $k=2$.\\\\
We will reformulate $k-SAT$ and relate it to a minimal eigenvalue of a certain hamiltonian which is diagonal in compuational basis.\\
Fix $k=3$. Consider a claus $C = x_1 \cup \bar x_2 \cup x_3$ (has one non-satsifiy assignment 010).\\\\
               Lets associate a particular diagonal hamiltonian with 0 everywhere but a 1 at localation indexed by the bit string above.\\
               $H_c$ gives "penalty" of 1 to the bitstring of $x_1...x_n$ if $x$ does not satisfy clause $C$.\\
               We will regard $H_c$ as part of $n$-qubit Hamiltoanin as $H_c = H_c \otimes I$.\\
               When we evaluate this term: $\bra{x} H \ket{x} = 0$ if clause $C$ is satsified or $1$ otherwise.\\\\
               Suppose we have a 3-SAT forumla $\phi = C_1 \cap ... \cap C_n \implies H_{\phi} = \sum_{j=1}^m H_{c_j}$. The eigenvalues of $H_{\phi}$ lie in the interval $[0,m]$ and $H_{\phi}$ is 3-local.\\
                       Each assignment $x \in \{0,1\}^n$ generates the "energy"\\
                       $\bra{x} H_{\phi} \ket{x} = \sum_{j=1}^n \bra{x} H_{C_j} \ket{x}$ this counts the number of unsatisfied clauses under $x$.\\\\
                       We want to find the minimal energy of this hamiltonain which will correspond to the lowest eigenvalue $\lambda_n$ ( the smallest number of unsatsified clauses).
                       \section{Lecture 12}
                       \textbf{The k-local Hamiltonian problem} (LH): Given a classical description of a Hamiltonian of an $n$ qubit Hamiltonian:
                       $$
                       H = \sum_{j=1}^m H_j
                       $$
                       where each $H_j$ is $k$-local and is positive semi-definite. Also given two parameters $a,b \in [0,m]$ with $b-a \geq \frac{1}{poly(n)}$ with a promise that $\lambda_{min}$- minimal eigenvalue of $H$ is either $\leq a$ or $\geq b$. Decide which is the case.
                       $\lambda_{min}$ is often called the ground state energy. Determining $\lambda_{min}$ ($\leq a$ or $\geq b$) is tantemount to approximating the ground state energy $\lambda_{min}$ up to an additive error $O(b-a)$. Further studied on ES.\\\\
                       We would like to prove that $LH$ is $QMA$-complete\\\\
                       We first need to obtain that $LH$ is in $QMA$: Given the corresponding witness states $\ket{\psi}$ for $x \in L_1$, we can efficently approximate the energy to check if it is $\leq a$ or $\geq b$.\\
                       Note: LH is a promise problem, which means that $H$ with $\lambda \in (a,b)$ won't satsify the condition.\\
                       Next step, is we want to show that LH is QMA-hard (any other problem in QMA can be reduced to it)\\\\
                       Take any $L= (L_1, L_0, L_*)$ in QMA and fix $x \in L_1 \cup L_0$ which will be our $n$ bit string.\\
                       Plan: Consider a circuit $C_n$ and convert it to Hamiltonian $H$ s.t. $\lambda_{min}$ is small iff $C_n$ has high acceptance probability on some state $\ket{\psi}$. We have $C_n = U_T...U_1$, each $U_i$ is a 1 or 2 qubit gate. We'll take error probability to be $\frac{1}{4T}$. \\
                       The circuit $C_n$ take $n+s+ p(n)$ qubits as inputs. $n$ qubits encode the classical input $x$. $s$ qubits is the workspace of clean qubits in state $0$. $p(n)$ is a witness state $\ket{\psi}$.\\\\
                       Given $\ket{\psi}$, define $\ket{\psi_0} = \ket{0}^{\otimes S}\ket{\psi}$, $\ket{\psi_t} = U_t \ket{\psi_{t-1}}$ for $i = 1...T$. \\\\
                       We will define the following Hamiltonian which has three parts. $$H_{init} = \sum_{i=1}^s \ket{1} \bra{1}_i \otimes \ket{0} \bra{0}_C$$
                       with $i = 1, ..., s+p(n)$
                       $$
                       H_t = \frac{1}{2} [ I \otimes ( \ket{t-1} \bra{t-1}_C + \ket{t}\bra{t}_C) - U_t \otimes \ket{t} \bra{t-1}_C - U_t^{\dagger} \otimes \ket{t-1} \bra{t}_C]
                       $$
                       $$
                       H_{final} = \ket{0} \bra{0} \otimes \ket{T} \bra{T}_C
                       $$
                       $$H= H_{init} + \sum_{i=1}^T H_i + H_{final}$$
                       We should note that $H$ "follows" the state $\ket{\psi}$ and percribes penalties when deviating form $\ket{\psi_0}, \ket{\psi_1},..., \ket{\psi_T}$\\
                       $H$ also penalizes the zero output in the final measurement\\
                       $C$-is an extra "clock$ register of $\log (T+1)$ qubits\\
                       Check locality of $H$ is $k= \log (T+1) + 2 = O(\log (n))$. We will further reduce $k$ to a constant $k = 5$.\\
                       We need to check that we can distinguish $x\in L_1$ and $x \in L_0$ by looking at $\lambda_{min}$. \\\\
                       Consider first case when $x \in L_1$ this means that there exists a $p(n)$ qubit witness state $\ket{\psi}$ that makes $C_n$ accept with probability $1- \frac{1}{4T}$. Lets consider another state $\ket{\psi'} = \frac{1}{\sqrt{T+1}} \sum_{t = 0}^T \ket{\psi_t} \ket{t}$ this is a "History state". When we look at this state we want to evaluate the energy.
                       State $\ket{\psi'}$ gets no contribution to penalty from $H_{int}$ and $H_t$ terms. Since the probability of measuring the (incorrect) outocme is $\frac{1}{4T}$ we have $\lambda_{min} = \bra{\psi'} H \ket{\psi'} = \frac{1}{T+1} \bra{\psi_T} \bra{T} H_{final} \ket{T} \ket{\psi_T} \leq \frac{1}{T+1} \frac{1}{4T} = a$.\\
                       This will show us completeness.\\\\
                       Now take $x \in L_0$. We want to show that $\lambda_{min}$ is at least $b=2a$. Consider any witness state $\ket{\psi'}= \sum_{t=0}^T \alpha_t \ket{\phi_t} \ket{t}$ with $\alpha_t \geq 0$ and $\{ \ket{\phi_t}\}$ is normalised. Then evaluate $$\bra{\psi'} H \ket{\psi'} = \frac{1}{2} ( \alpha^2_{t-1} + \alpha_t^2 - \alpha_{t-1} \alpha_t \bra{\phi_t} U_t \ket{\phi_{t-1}} - \alpha_t \alpha_{t-1} \bra{\phi_{t-1}} H \ket{\phi_t})$$
                       $$
                       \bra{\psi'} H \ket{\psi'} = \frac{1}{2} || \alpha_t \ket{\phi_t} - \alpha_{t-1} U_t \ket{\phi_{t-1} ||^2
                       $$
                       We will be making several simplifying assumptions:\\
                       1) All $\alpha_t$s are $\frac{1}{\sqrt{T-1}}$\\
                       2) $\ket{\phi} = \ket{0}^{\otimes s} \ket{\psi}$ for some $p(n)$ qubit state $\ket{\psi}$\\
                       3) $\ket{\phi_T}$ has accept probability close to 1.\\\\
                       Using these and the fact that $x \in L_0$, $C_n \ket{\phi_0}$ must have accept probability near 0.\\
                       From 3 we have $\ket{\phi_T}$ and $C_n \ket{\phi_0}$ must be nearly orthogonal:
                       $$
                       1 \leq || \ket{\phi_T} - C_n \ket{\phi_0} || = || sum_{t=1}^T U_T - U_{t-1} \ket{\phi_t} - U_T...U_t\ket{\phi_{t-1}}|| \leq \sum_{t=1}^T || U_T...U_t \ket{\phi_t} - U_T...U_t \ket{\phi_{t-1}}|| = \sum_{t=1}^T || \ket{\phi_t} - U_t \ket{\phi_{t-1}}||
                       $$
                       Using the evaulation of $\bra{\psi'} H \ket{\psi}$ assumptions and setting $\alpha_t = \alpha_{t-1} = \frac{1}{\sqrt{T+1}$:
                               $$
                               \bra{\psi'} H \ket{\psi'} = \sum_{t=1}^T \bra{\psi'} H_t \ket{\psi'} = \frac{1}{2} \frac{1}{T+1} \sum_{t=1}^T ||\ket{\phi_t} - U_t \ket{\phi_{t-1}} ||^2 \geq \frac{1}{2T(T+1)} ( \sum_{t=1}^T || \ket{\phi_t} - U_t \ket{\phi_{t-1}}||^2 ) \geq \frac{1}{2T(T+1)} = b
                               $$
                               \section{Example Sheet 2}
For Grover's search algorithm, if the indicator function is classically efficently computation then $I_G$ is implementable.\\\\
Look at Part II course to recap coprimality conditions. They also show that to factor $N$ with constant probability it suffices to find an approximations $\xi$ to $\frac{s}{r}$ for random $s$ in range $[1,r)$ to $2m+1$ binary digits of accuracy with $m= O(\log N)$ s.t. $| \xi - \frac{s}{r} | < \frac{1}{2N^2}$\\\\
                       \end{document}
