\documentclass{article}
\usepackage[utf8]{inputenc}
\usepackage{bm}
\usepackage{amssymb}
\usepackage{amsmath}
\usepackage{braket}
\usepackage{cancel}
\title{Quantum Computation}
\author{oliverobrien111 }
\date{July 2021}

\begin{document}

\maketitle
\section{Lecture 1}
\subsection{Review of Shor's algoirthm/quantum period finding algorithm}
\textbf{Polynomial time hierarchy}://
Computation with input of size $n$, and we are interested in the number of steps/gates (classical or quantum). When we say $O(poly(n))$ steps we regard this as an "efficent computation".\\\\
Shor's algorithm solves the factoring problem:\\
Given an integer $N$ needing $O(log N)$ bits, we want to find a non-trival factor in $O(poly n)$ time.\\\\
The best known classical algorithm (number sieve): $e^{O(n^{\frac{1}{3}} (\log n)^{\frac{1}{3}} )}$\\
Shor's alogrithm takes $O(n^3)$\\
\subsubsection{Quantum factoring algorithm (summary)}
\begin{enumerate}
        \item First, convert factoring into periodicity determination. Given $N$, choose $a< N$ s.t. $a$ is coprime (this is easy classically can be seen in part II lecture notes). Consider $f: \mathbb{Z} \rightarrow \mathbb{Z}_N$ $f(x) = a^x \mod N$. \textbf{Euler's Theorem}: if $f$ is periodic with period $r$, then it is called 'order of $a \mod N$'. 
        \item In order to find $r$ we need a quantum implementation of $f$. We are always workingon finite size registers so restricting $x \in \mathbb{Z}$ to $x \in \mathbb{Z}_M$ (for some large enough $M$): $f: \mathbb{Z}_M \rightarrow \mathbb{Z}_N$. $f$ will no longer be exactly preriodic but this would have neglible effect if $M$ is sufficently large e.g. $M=O(N^2)$
        \item Using the classical theory of continued fractions. Define Hilbert spaces $\mathcal{H}_M \rightarrow \{\ket{i}\}_{i \in \mathbb{Z}_M}, \mathcal{H}_N\rightarrow \{\ket{i}\}_{i \in \mathbb{Z}_N}$.
        \item $\ket{x} \rightarrow \ket{f(x)}$ is not generally a valid quantum operator, so we make it a unitary operation which can be implemented:
                $$
                U_f: \mathcal{H}_M \otimes \mathcal{H}_N \rightarrow \mathbb{H}_M \otimes \mathbb{H}_N
                $$
                $$
                U_f: \ket{i} \ket{k} \rightarrow \ket{i} \ket{k + f(i)}
                $$
        \item if $x \rightarrow f(x)$ can be classically computed in $O(poly(m))$ time ($m = \log M$)), then $U_f$ can be implemented in poly($m$) time quantumly too
        \item We will sometimes view $U_f$ as a black box/oracle and we will count the number of times the algorithm invokes the oracle.
        \item Back to facotoring to get $r$ we'll use the quantum algorithm for periodicity determination:
        \item Given an oracle $U_f$ with the promise that $f$ is periodic of some unknown period $r \in \mathbb{Z}_N$ so that $f(x+r) = f(x)$ and $f$ is one-to-one in this period (for all $0 \leq x_1 < x_2<r f(x_1) \neq f(x_2)$)
        \item To find $r$ in $O(poly n)$ with any persecribed success probability $1-\epsilon$ we use the following alogirthm:
                \begin{itemize}
                        \item Step 1: Create the state 
                                $$
                                \frac{1}{\sqrt{M}} \sum_{i=0}^{M-1} \ket{i} \ket{0}
                                $$
                        \item Step 2: Apply $U_f$ to get
                                $$
                                \frac{1}{\sqrt{M}} \sum_{i=0}^{M-1} \ket{i} \ket{f(i)}
                                $$
                        \item Step 3: Measure the 2nd register to get $y$. By the born rule the first register collapses to all those $i$: $f(i) =y$ i.e. $i = x_0, x_0 + r, x_o + 2r,..., x_0 + (A-1)r$, $0 \leq x_0 <r$.\\\\Discard the second register to get the following state:
                                $$
                                \ket{per} = \frac{1}{\sqrt{A}} \sum_{j=0}^{A-1} \ket{x_0 + jr}
                                $$
                         If we measure $\ket{per}$ in computation basis we will get a value of one of these states $x_0 + jr$ for uniformly random $j$. This only gives us a random element of $\mathbb{Z}_M$ with no information about $r$.
                        \item Step 4: Apply quantum fourier transform mod $M$ (QFT). Lets recap what QFT does:
                                $$
                                \ket{x} \rightarrow \frac{1}{\sqrt{M}} \sum_{y=0}^{M-1} \omega^{xy} \ket{y}, \forall x \in \mathbb{Z}_M, \omega = e^{2\pi i/ M}
                                $$
                                This can be implement in $O(m^2)$ time and gives state:
                                $$
                                QFT \ket{per} = \frac{1}{\sqrt{MA}} \sum_{j=0}^{A-1} \sum_{y=0}^{M-1}  \omega^{ (x_0 + jr)y} \ket{y} =  \frac{1}{\sqrt{MA}} \sum_{y=0}^{M-1} \omega^{x_0y}\left[  \sum_{j=0}^{A-1} \omega^{jry} \ket{y} \right]
                                $$
                        The square brackets will be:
                        $$
                        \begin{cases}
                                A & \text{if }y = KA= k\frac{M}{r}, x = 0, 1,..., r-1\\
                                0 & \text{otherwise}
                        \end{cases}
                        $$
                        So gives final state:
                        $$
                        QFT \ket{per} = \sqrt{\frac{A}{M}} \sum_{k=0}^{A-1} \omega^{x_0 k \frac{N}{r}} \ket{k \frac{M}{r}}
                        $$
                        Now the random shift $x_0$ only appears in the phase not in the ket labels. So now the measurement probabilities will be indepedant of $x_0$. When we measure this we get some value $c = \frac{k_0M}{r}$ with $k_0$ uniformly random in range $0 \leq k_0 < r$
                        $$
                        \frac{k_0}{r} = \frac{c}{M}
                        $$
                        As $c$ and $M$ are known, and $k_0$ is unknown but random in the given range. We want to find $r$ and so we recall several classical facts.\\\\
                        \textbf{Co-primality Theorem}: The number of integers less than $r$ that are coprime to $r$ grows with $O(\frac{r}{\log \log r})$\\\\
                        Therefore, the probability of $k_0$ being coprime to $r$ is $O(\frac{1}{\log \log r})$.\\\\
                        \textbf{Lemma}: If a single trial has success probability $p$ then if one repeats it $M^*$ times, for any $0<1-\epsilon<1$. We get probability of at least one success in $M^*$ trails is greater than $1-\epsilon$ if $M^* = \frac{- \log \epsilon}{p}$. i.e. roughly $O(1/p)$ trials suffice to achieve probability of success $> 1- \epsilon$
                \item After step 4 cancel $\frac{c}{M}$ down to an irredicible algorithm $\frac{a}{b}$ there is an efficent algorithm ($O(poly n)$) for this. This will give us $r$ as denominator $b$ if $k_0$ is coprime to $r$ with probablity $O(\frac{1}{\log \log r})$. So check $b$ value by computing $f(0)$ and $f(b)$ and $b = r \iff f(0) = f(b)$.\\\\
                        By repeating this process $M^* = O( \log \log r)$ times this will give us $r$ with any desired probability $1- \epsilon$. Since $r < M$ the whole algorithm takes $O(poly m)$ time!
                \end{itemize}
                
        \item From learning the period $r$ we can use number theory to find a factor of $N$
\end{enumerate}
\subsubsection{Further insights to QFT}
Now lets think about the implications of $QFT$. What does applying quantum fourier transform really achieve?\\\\
Lets consider a function: $f: \mathbb{Z}_M \rightarrow \mathbb{Z}_N$ with period $r \in \mathbb{Z}_M$, $A= \frac{M}{r}$. Define:
$$
R = \{ 0, r, 2r, 3r,..., (A-1)r\} < \mathbb{Z}_M
$$
$$
\ket{R} = \frac{1}{\sqrt{A}} \sum_{k=0}^{A-1} \ket{kr}
$$
$$
\ket{per} = \ket{x_0 + R} = \frac{1}{\sqrt{A}} \sum_{k=0}^{A-1} \ket{x_0 + rk}
$$
The problem was this random shift $x_0$ when measuring $\ket{per}$. For each $x_0 < \mathbb{Z}_M$ consider a mapping $k \rightarrow k+ x_0$. "Shift by $x_0$". It is a 1-1 invertible map, and can define a unitary version $U(x_0)$ on $\mathcal{H}_M$: $U(x_0) \ket{k} = \ket{k+x_0}$.
$$
\ket{x_0 + R} = U(x_0) \ket{R}
$$
Since $(\mathbb{Z}_M, +)$ is an abelian group $U(x_0)U(x_1) = U(x_0+x_1) = U(x_1)U(x_0)$. So all $U(x_i)$ commute as operators on $\mathcal{H}_M$. Therefore they have an orthonomal basis of common eigenvectors $\{ \ket{\chi_k}\}_{k \in \mathbb{Z}_M}$. These are called shift invariant states as $U(x_0) \ket{\chi_k} = \omega( x_0, k) \ket{\chi_k}$ for all $x_0, k \in \mathbb{Z}_M$ with the important caveat that $|\omega( x_0, k)| = 1$.\\\\
Consider $\ket{R}$ written in $\{ \ket{\chi_r}\}$ basis:
$$
\ket{R} = \sum_{k=0}^{M-1} a_k \ket{\chi_k}
$$
$a_k$ only depend on $r$ not on $x_0$. Then:
$$
\ket{per} = U(x_0) \ket{R} = \sum_{k=0}^{M-1} a_k \omega(x_0, k) \ket{\chi_k}
$$
Here it can be seen that the probability of measuring $k$ is 
$$
prob(k) = |a_k \omega(x_0, k)|^2= |a_k|^2
$$
So this is all indepedant of $x_0$ and depends only on $r$. So measuring in this basis gives us some information about $r$. So one can think of QFT as the unitary mapping that rotates $\chi$ basis into the standard computational basis. So can define QFT as:
$$
QFT \ket{\chi_k} = \ket{k}
$$
How do these mysterious shift invariant states look?
\subsubsection{Explicit form of shift invariant shapes}
$$
\ket{\chi_k} = \frac{1}{\sqrt{M}} \sum_{l=0}^{M-1} e^{-2\pi il \frac{k}{M}} \ket{l}
$$
$$
U(x_0) \ket{\chi_k} = \fraC{1}{\sqrt{M}} \sum_{l=0}^{M-1} e^{-2\pi i l \frac{k}{M}} \ket{l + x_0} =  \fraC{1}{\sqrt{M}} \sum_{\tilde l=0}^{M-1} e^{-2\pi i (\tilde l - x_0) \frac{k}{M}} \ket{\tilde l}  = e^{2\pi i k \frac{x_0}{M}} \ket{\chi_k}
$$
giving eigenvalue: $\omega(x_0, k) = e^{2\pi i k \frac{x_0}{M}}$. From this we could reconstruct the basis of QFT:
$$
[QFT]_{kl} = \frac{1}{\sqrt{M}} e^{2\pi i \frac{kl}{M}}
$$
\section{Lecture 3}
\subsection{Hidden Subgroup Problem}
Let $G$ be a finite group of size $|G|$. We are given an oracle $f: G \rightarrow X$ with $X$ just some set. We are promised there is a subgroup $K<G$ s.t.
\begin{itemise}
\item $f$ is constant on (left) cosets of $K$ in $G$
\item $f$ is distinct on distinct cosets
\end{itemise}\\\\
\textbf{Problem}: 'Determine' the 'hidden subgroup' $K$ (e.g. output a set of generators or sample uniformly from elements of $K$)\\\\
We want to solve in time $O(poly ( \log |G|))$ (efficent algorithm) with anuy consitent probability $1-\epsilon$.
\textbf{Examples of problems that can be cast as HSP}\\
Periodicity finding $f: \mathbb{Z}_M \rightarrow X$ periodic, period r 1-1 in period
$$
G = \mathbb{Z}_M, K = \{ 0,r,2r,..., (A-1)r\} < G
$$
Discrete Logarithm Problem: $p$ - prime number, $\mathbb{Z}^*_p$ group of integers with multiplication mod $p$, $g \in \mathbb{Z}_p^*$ to be a generator (or primitive root mod p). If $\mathbb{Z}^*_p = \{ g^0, g^1, ..., g^{r-2} \}$ and we have $g^{p-1} = 1$ (mod p). Fact: These always exist for $p$ is prime. Any $x \in \mathbb{Z}_p^*$ can be written as $x = g^y$ for some $y \in \mathbb{Z}_{p-1}$, $y = \log_g x$ is called the discrete log of $x$ to base $g$. Discrete log problem is given a generator $g$, $x \in \mathbb{Z}^*_p$ we want to compute $y = \log_g x$. To express this as the HSP:
$$
f: \mathbb{Z}_{p-1} \times \mathbb{Z}_{p-1} \rightarrow \mathbb{Z}_p^*
$$
$$
f(a,b) = g^a x^{-b} \mod p = g^{a-yb} \mod p
$$
Can check if $f(a_1, b_1) = f(a_2, b_2) \iff (a_1, b_1) = (a_2, b_2) + \lambda( y, 1), \lmbda \in \mathbb{Z}_{p-1}$:
$$
G = \mathbb{Z}_{p-1} \times \mathbb{Z}_{p-1}
$$
$$
K = \{ \lambda(y,1): \lambda \in \mathbb{Z}_{p-1} \} < G
$$
Then $f$ is constant and distinct on cosets of $K$ and generator $(y,1)$ of $K$ gives $y = \log_g x$\\\\
\textbf{Graph Problems}:\\
So we can solve problems like those above where $G$ is abelian, but we can also solve graph problems.\\\\
Consdier graph $A = (V,E)$, $|V| = n$ lets say that the graph is undirected and there is at most one edge between any two vertices. Vertices here are labelled by numbers from 1 to $n$.\\\\
Lets define an adjacency matrix $M_A$: $[M_a]_{ij} = \begin{cases} 1 & \iff (i,j)\\ 0 & \text{otherwise} \end{cases}$. The permuation group of $[n]$, $|P_n| = n!, \log |P_n| \sim O(n \log n)$. Define a group of automorphisms of group $A$ which is a set of permuations with the following property: $\pi \in P_n$ s.t. $\forall i,j (i,j)$ is an edge in $A$ $\iff (\pi(i), \pi(j))$ is also an edge in $A$.\\\\
An associated HSP (the case of non-abelian $G$):
$$
G = P_n, X = \text{set of all labelled graphs on }n\text{vertices}
$$
For any $A \in X$, define $f_A: G \rightarrow X$, $f_A(\pi) = $"A with vertex labels permuted by $\pi$"
$$
K = Aut(A)
$$
(Check $f(K)$ is constant and disctint on cosets of $Aut(A)$)\\\\
\textbf{Applications}:\\
If we can sample uniformly from $K$, then we can solve Graph Isomorphism problem (GI).This has a number of different applications in areas of computer science. Two labelled graphs $A$ and $B$ with $n$ vertices are isomorphic if there is a 1-1 map (i.e. permutation) $\pi[n] \rightarrow [n]$ s.t. $\forall i,j \in [n] (i,j)$ is an edge in $A$ $\iff$ $(\pi(i), \pi(j))$ is an edge in $B$. The GI problem is given to graphs $A$ and $B$ and deciding if they are isomorphic. This can be represented as a non-abealian HSP. There is no known poly(m) time classical algorithm to solve this problem, so $GI$ is clearly in $NP$ but not believed to be $NP$-complete (a class of problems such that every problem in NP can be reduced to an NP-complete problem these are the hardest NP problems).
In 2017, L Babai presented a quasi-polynomial algorithm for GI runtime $n^O(\log n)^2)$. This ranks in between polynomial runtime and exponential algorithms.
\section{Lecture 4}
\textbf{Quantum algorithm for finite abelian HSPs}
- Generalisation of period-finding algorithm\\
Write our abelian group $(G, +)$ additively\\
Construction of shift-invariant states and Fourier transform for $G$.\\
Representations of abelian $G$:\\
Consdier the mapping $\chi: G \rightarrow \mathbb{C}^* = \mathbb{C}- \{ 0\}$ with multiplication that satisfies:
$$
\chi( g_1 + g_2) = \chi(g_1) \chi(g_2), \forall g_1,g_2 \in G 
$$
$\chi$ is a group homomorphism from $G$ to $\mathbb{C}^*$. Such $\chi$'s are called irreducible representations of $G$. They have the following properties:
\textbf{Theorem 1}:\\
1) any value $\chi(g)$ is a $|G|$-th root of unity (\chi \in \rightarrow S^1$ the unit circle)\\
2) Schur's lemma (orthogonality): If $\chi_i,\chi_j$ satisfy (HOM) then 
$$
\frac{1}{|G|} \sum_{g \in G} \chi_i (g) \bar {\chi_j} (g) = \delta_{ij}
$$
There are always exactly $|G|$ different functions $\chi$ satisfying (HOM).\\\\
\textbf{Examples}: $\chi(g) = 1, \forall g \in G$ is an irrep/ called a trival irrep\\
Label the trivial irrep as $\chi_0$, $0\in G$. Then for any other irrep $\chi \neq \chi_0$ orthonality to $\chi_0$ gives:
$$
\sum_{g \in G} \chi(g) = 0 \text{ if } \chi \neq \chi_0
$$
Going back to constructing shift-invariant states
\subsubsection{Shift-invariant states}
Consider a state space $\mathcal{H}_G, dim \mathcal{H}_G = |G|$ wtih basis $\{ \ket{g}\}_{g \in G}$. Now introduce shift operators $U(k)$ for $k \in G$ defined as follows:
$$
U(k): \ket{g} \rightarrow \ket{g+k}, g, k \in G
$$
All shift operators commute so there exists a simultaneous eigenbasis.\\
For each $\chi_k, k \in G$:
$$
\ket{\chi_k} = \frac{1}{\sqrt{|G|}} \sum_{g \in G} \bar{\chi_k} (g) \ket{g}
$$
By thereom 1 $\{ \chi_k \}$ form an orthonormal basis.
$$
U(g) \ket{\chi_k} = \chi_k(g) \ket{\chi_k}
$$
\textbf{Proof}:$$U(g) \ket{\chi_k} = \frac{1}{\sqrt{|G|} \sum_{h \in G} \bar{\chi_k} (h) \ket{h + g}$$
        $$
        \{h' = h+ g\} = \frac{1}{\sqrt{|G|} \sum_{h' \in G} \bar{\chi_k} (h' - g) \ket{h'}
        $$
        using HOM $\chi_k(-g) = \chi_k(g)^{-1} = \bar \chi_k(g) \implies \bar{\chi_k(h'-g)} = \bar \chi_k(h') \bar \chi_k(-g) = \bar \chi_k(h') \chi_k(g)$. Therefore,
        $$
        U(g) \ket{\chi_k} = \frac{1}{\sqrt{|G|} \sum_{h' \in G} \chi_k(g) \bar \chi_k(h') \ket{h'} = \chi_k(g) \ket{\chi_k}
        $$
        So $\ket{\chi_k}$'s form a common eigenbasis\\\\
        Introduce Fourier transform QFT for a group $G$\\
- consider a unitary mapping on $\mathcal{H}_G$ mapping $\ket{\chi_k}$ basis to $\ket{g}$ basis
$$
QFT \ket{\chi_g} = \ket{g}, \forall g \in G
$$
$$
QFT^{-1} \ket{g} =  \ket{\chi_g}
$$
k-th column of $QFT^{-1}$ in $\ket{g}$ basis is mode of compoents of $\ket{\chi_k}$:
        $$
        [QFT^{-1}]_{gk} = \frac{1}{\sqrt{|G|}} \bar{\chi_k}(g)
        $$
        \textbf{EXample}: $G = \mathbb{Z}_M$L\\
        Check $\chi_a(b) = e^{\frac{2\pi i a b}{M}}, a, b \in \mathbb{Z}_M$ satsifies HOM and has its irreps labelled by $a \in \mathbb{Z}_M$ with $\chi_0(b) = 1 \forall b \in \mathbb{Z}_m$.
        $$
        G = \mathbb{Z}_{M_1} \times ... \times \mathbb{Z}_{M_r}
        $$
        $$
        (a_1, ..., a_r ) = g_1, (b_1, ..., b_r) = g_2
        $$
        $$
        \chi_{g_1} (g_2) = e^{2\pi i ( \frac{a_1 b_1}{M_1} +... + \frac{a_r b_r}{M_r}}
        $$
This satifies HOM and our $QFT_G = QFT_{M_1} \otimes ... \otimes QFT_{M_r}$ on $\mathcal{H}_G = \mathcal{H}_{M_1} \otimes ... \otimes \mathcal{H}_{M_r}$.\\\\
This second example is exhaustive since we have a classification theroem:\\
\textbf{Classification theorem}: Any fintie abelian group $G$ is isomorphic to a direct product of the form $G = \mathbb{Z}_{M_1} \otimes ... \otimes \mathbb{Z}_{M_r}$. So $M_1$ can be taken in a form $p_1^{s_1},... p_r^{s_r}$.
\subsubsection{Quantum algorithm}
$$
f: G \rightarrow X
$$
with hidden subgroup $K$ and cosets $k = 0 + k, g_2 + k, ... g_m +k$, $m = \frac{|G|}{|K|}$. we will work on $\mathcal{H}_{|G|} \otimes \mathcal{H}_{|X|}, $\{\ket{g}\ket{x}\}_{g \in G, x \in X}$.\\
\begin{itemlist}
\item Create a state $\frac{1}{\sqrt{|G|}} \sum_{g \in G} \ket{g} \ket{0}\\
\item Apply $U_f$ and $\frac{1}{\sqrt{|G|} \sum_{g \in G} \ket{g} \ket{f(g)}\\
        \item Measure the second register to get $f(g_0)$. The first register will not give the coset state:
                $$
                \ket{g_0 + k} = \frac{1}{\sqrt{|k|} \sum_{k \in K} \ket{g_0 + k} = U(g_0) \ket{K}
                $$
        \item apply QFT and measure to get a result $g \in G$
\end{itemlist}
\end{document}
