\documentclass{article}
\usepackage[utf8]{inputenc}
\usepackage{bm}
\usepackage{amssymb}
\usepackage{amsmath}
\usepackage{braket}
\usepackage{qcircuit}
\usepackage{cancel}
\title{Quantum Computation}
\author{oliverobrien111 }
\date{July 2021}

\begin{document}

\maketitle
\section{Lecture 1}
\subsection{Review of Shor's algoirthm/quantum period finding algorithm}
\textbf{Polynomial time hierarchy}:\\
Computation with input of size $n$, and we are interested in the number of steps/gates (classical or quantum). When we say $O(poly(n))$ steps we regard this as an "efficent computation".\\\\
Shor's algorithm solves the factoring problem:\\
Given an integer $N$ needing $O(log N)$ bits, we want to find a non-trival factor in $O(poly n)$ time.\\\\
The best known classical algorithm (number sieve): $e^{O(n^{\frac{1}{3}} (\log n)^{\frac{1}{3}} )}$\\
Shor's alogrithm takes $O(n^3)$\\
\subsubsection{Quantum factoring algorithm (summary)}
\begin{enumerate}
        \item First, convert factoring into periodicity determination. Given $N$, choose $a< N$ s.t. $a$ is coprime (this is done by picking $a$ at random and then using Euclids algorithm to compute the greatest common divisor $b= gcd(a,N)$, if $b>1$ we have found a factor else they are coprime). Consider $f: \mathbb{Z} \rightarrow \mathbb{Z}_N$ $f(x) = a^x \mod N$.\\
                \textbf{Euler's Theorem}: \textsl{If $a$ and $N$ are coprime then there is a least power $1 < r < N$ such that $a^r = 1$ mod $N$. We call $r$ the order $a$ mod $N$.}\\ So $f(r) =1$ so $f(k+r) = f(k)$ so $f$ is periodic with period $r$, with $f$ one-to-one within each period.
        \item In order to find $r$ we need a quantum implementation of $f$. We are always working on finite size registers so restricting $x \in \mathbb{Z}$ to $x \in \mathbb{Z}_M$ (for some large enough $M$): $f: \mathbb{Z}_M \rightarrow \mathbb{Z}_N$. $f$ will no longer be exactly periodic but this would have negligible effect if $M$ is sufficiently large e.g. $M=O(N^2)$
        \item Using the classical theory of continued fractions. Define Hilbert spaces $\mathcal{H}_M \rightarrow \{\ket{i}\}_{i \in \mathbb{Z}_M}, \mathcal{H}_N\rightarrow \{\ket{i}\}_{i \in \mathbb{Z}_N}$.
        \item $\ket{x} \rightarrow \ket{f(x)}$ is not generally a valid quantum operator, so we make it a unitary operation which can be implemented:
                $$
                U_f: \mathcal{H}_M \otimes \mathcal{H}_N \rightarrow \mathcal{H}_M \otimes \mathcal{H}_N
                $$
                $$
                U_f: \ket{i} \ket{k} \rightarrow \ket{i} \ket{k + f(i)}
                $$
        \item If $x \rightarrow f(x)$ can be classically computed in $O(poly(m))$ time ($m = \log M$)), then $U_f$ can be implemented in poly($m$) time quantumly too
        \item We will sometimes view $U_f$ as a black box/oracle and we will count the number of times the algorithm invokes the oracle.
        \item Back to factoring to get $r$ we'll use the quantum algorithm for periodicity determination:
        \item Given an oracle $U_f$ with the promise that $f$ is periodic of some unknown period $r \in \mathbb{Z}_N$ so that $f(x+r) = f(x)$ and $f$ is one-to-one in this period (for all $0 \leq x_1 < x_2<r f(x_1) \neq f(x_2)$). To find $r$ in $O($poly $n)$ with any prescribed success probability $1-\epsilon$ we use the following algorithm:
                \begin{itemize}
                        \item Step 1: Create the state 
                                $$
                                \frac{1}{\sqrt{M}} \sum_{i=0}^{M-1} \ket{i} \ket{0}
                                $$
                        \item Step 2: Apply $U_f$ to get
                                $$
                                \frac{1}{\sqrt{M}} \sum_{i=0}^{M-1} \ket{i} \ket{f(i)}
                                $$
                        \item Step 3: Measure the 2nd register to get $y$. By the Born rule the first register collapses to all those $i$: $f(i) =y$ i.e. $i = x_0, x_0 + r, x_o + 2r,..., x_0 + (A-1)r$, $0 \leq x_0 <r$.\\\\Discard the second register to get the following state:
                                $$
                                \ket{per} = \frac{1}{\sqrt{A}} \sum_{j=0}^{A-1} \ket{x_0 + jr}
                                $$
                         If we measure $\ket{per}$ in computation basis we will get a value of one of these states $x_0 + jr$ for uniformly random $j$. This only gives us a random element of $\mathbb{Z}_M$ with no information about $r$.
                        \item Step 4: Apply quantum fourier transform mod $M$ (QFT). Lets recap what QFT does:
                                $$
                                \ket{x} \rightarrow \frac{1}{\sqrt{M}} \sum_{y=0}^{M-1} \omega^{xy} \ket{y}, \forall x \in \mathbb{Z}_M, \omega = e^{2\pi i/ M}
                                $$
                                This can be implement in $O(m^2)$ time and gives state:
                                $$
                                QFT \ket{per} = \frac{1}{\sqrt{MA}} \sum_{j=0}^{A-1} \sum_{y=0}^{M-1}  \omega^{ (x_0 + jr)y} \ket{y} =  \frac{1}{\sqrt{MA}} \sum_{y=0}^{M-1} \omega^{x_0y}\left[  \sum_{j=0}^{A-1} \omega^{jry} \ket{y} \right]
                                $$
                        The square brackets will be:
                        $$
                        \begin{cases}
                                A & \text{if }y = KA= k\frac{M}{r}, x = 0, 1,..., r-1\\
                                0 & \text{otherwise}
                        \end{cases}
                        $$
                        So gives final state:
                        $$
                        QFT \ket{per} = \sqrt{\frac{A}{M}} \sum_{k=0}^{A-1} \omega^{x_0 k \frac{N}{r}} \ket{k \frac{M}{r}}
                        $$
                        Now the random shift $x_0$ only appears in the phase not in the ket labels. So now the measurement probabilities will be independent of $x_0$. When we measure this we get some value $c = \frac{k_0M}{r}$ with $k_0$ uniformly random in range $0 \leq k_0 < r$
                        $$
                        \frac{k_0}{r} = \frac{c}{M}
                        $$
                        As $c$ and $M$ are known, and $k_0$ is unknown but random in the given range. We want to find $r$ and so we recall several classical facts.\\\\
                        \textbf{Co-primality Theorem}: The number of integers less than $r$ that are coprime to $r$ grows with $O(\frac{r}{\log \log r})$\\\\
                        Therefore, the probability of $k_0$ being coprime to $r$ is $O(\frac{1}{\log \log r})$.\\\\
                        \textbf{Lemma}: If a single trial has success probability $p$ then if one repeats it $M^*$ times, for any $0<1-\epsilon<1$. We get probability of at least one success in $M^*$ trails is greater than $1-\epsilon$ if $M^* = \frac{- \log \epsilon}{p}$. i.e. roughly $O(1/p)$ trials suffice to achieve probability of success $> 1- \epsilon$
                \item After step 4 cancel $\frac{c}{M}$ down to an irreducible algorithm $\frac{a}{b}$ there is an efficient algorithm ($O($poly $n)$) for this. This will give us $r$ as denominator $b$ if $k_0$ is coprime to $r$ with probability $O(\frac{1}{\log \log r})$. So check $b$ value by computing $f(0)$ and $f(b)$ and $b = r \iff f(0) = f(b)$.\\\\
                        By repeating this process $M^* = O( \log \log r)$ times this will give us $r$ with any desired probability $1- \epsilon$. Since $r < M$ the whole algorithm takes $O($poly $m)$ time!
                \end{itemize}
                
        \item From learning the period $r$ we can use number theory to find a factor of $N$
\end{enumerate}
\subsubsection{Further insights to QFT}
Now lets think about the implications of $QFT$. What does applying quantum fourier transform really achieve?\\\\
Lets consider a function: $f: \mathbb{Z}_M \rightarrow \mathbb{Z}_N$ with period $r \in \mathbb{Z}_M$, $A= \frac{M}{r}$. Define:
$$
R = \{ 0, r, 2r, 3r,..., (A-1)r\} < \mathbb{Z}_M
$$
$$
\ket{R} = \frac{1}{\sqrt{A}} \sum_{k=0}^{A-1} \ket{kr}
$$
$$
\ket{per} = \ket{x_0 + R} = \frac{1}{\sqrt{A}} \sum_{k=0}^{A-1} \ket{x_0 + rk}
$$
The problem was this random shift $x_0$ when measuring $\ket{per}$. For each $x_0 < \mathbb{Z}_M$ consider a mapping $k \rightarrow k+ x_0$. "Shift by $x_0$". It is a 1-1 invertible map, and can define a unitary version $U(x_0)$ on $\mathcal{H}_M$: $U(x_0) \ket{k} = \ket{k+x_0}$.
$$
\ket{x_0 + R} = U(x_0) \ket{R}
$$
Since $(\mathbb{Z}_M, +)$ is an abelian group $U(x_0)U(x_1) = U(x_0+x_1) = U(x_1)U(x_0)$. So all $U(x_i)$ commute as operators on $\mathcal{H}_M$. Therefore they have an orthonomal basis of common eigenvectors $\{ \ket{\chi_k}\}_{k \in \mathbb{Z}_M}$. These are called shift invariant states as $U(x_0) \ket{\chi_k} = \omega( x_0, k) \ket{\chi_k}$ for all $x_0, k \in \mathbb{Z}_M$ with the important caveat that $|\omega( x_0, k)| = 1$.\\\\
Consider $\ket{R}$ written in $\{ \ket{\chi_r}\}$ basis:
$$
\ket{R} = \sum_{k=0}^{M-1} a_k \ket{\chi_k}
$$
$a_k$ only depend on $r$ not on $x_0$. Then:
$$
\ket{per} = U(x_0) \ket{R} = \sum_{k=0}^{M-1} a_k \omega(x_0, k) \ket{\chi_k}
$$
Here it can be seen that the probability of measuring $k$ is 
$$
prob(k) = |a_k \omega(x_0, k)|^2= |a_k|^2
$$
So this is all indepedant of $x_0$ and depends only on $r$. So measuring in this basis gives us some information about $r$. So one can think of QFT as the unitary mapping that rotates $\chi$ basis into the standard computational basis. So can define QFT as:
$$
QFT \ket{\chi_k} = \ket{k}
$$
How do these mysterious shift invariant states look?
\subsubsection{Explicit form of shift invariant shapes}
$$
\ket{\chi_k} = \frac{1}{\sqrt{M}} \sum_{l=0}^{M-1} e^{-2\pi il \frac{k}{M}} \ket{l}
$$
$$
U(x_0) \ket{\chi_k} = \fraC{1}{\sqrt{M}} \sum_{l=0}^{M-1} e^{-2\pi i l \frac{k}{M}} \ket{l + x_0} =  \fraC{1}{\sqrt{M}} \sum_{\tilde l=0}^{M-1} e^{-2\pi i (\tilde l - x_0) \frac{k}{M}} \ket{\tilde l}  = e^{2\pi i k \frac{x_0}{M}} \ket{\chi_k}
$$
giving eigenvalue: $\omega(x_0, k) = e^{2\pi i k \frac{x_0}{M}}$. From this we could reconstruct the basis of QFT:
$$
[QFT]_{kl} = \frac{1}{\sqrt{M}} e^{2\pi i \frac{kl}{M}}
$$
\section{Lecture 3}
\subsection{Hidden Subgroup Problem}
Let $G$ be a finite group of size $|G|$. We are given an oracle $f: G \rightarrow X$ with $X$ just some set. We are promised there is a subgroup $K<G$ s.t.
\begin{itemise}
\item $f$ is constant on (left) cosets of $K$ in $G$
\item $f$ is distinct on distinct cosets
\end{itemise}\\\\
\textbf{Problem}: 'Determine' the 'hidden subgroup' $K$ (e.g. output a set of generators or sample uniformly from elements of $K$)\\\\
We want to solve in time $O(poly ( \log |G|))$ (efficient algorithm) with any consistent probability $1-\epsilon$.
\\\\
\textbf{Examples of problems that can be cast as HSP}\\
Periodicity finding $f: \mathbb{Z}_M \rightarrow X$ periodic, period r 1-1 in period
$$
G = \mathbb{Z}_M, K = \{ 0,r,2r,..., (A-1)r\} < G
$$
Discrete Logarithm Problem: $p$ - prime number, $\mathbb{Z}^*_p$ group of integers with multiplication mod $p$, $g \in \mathbb{Z}_p^*$ to be a generator (or primitive root mod p). If $\mathbb{Z}^*_p = \{ g^0, g^1, ..., g^{r-2} \}$ and we have $g^{p-1} = 1$ (mod p). Fact: These always exist for $p$ is prime. Any $x \in \mathbb{Z}_p^*$ can be written as $x = g^y$ for some $y \in \mathbb{Z}_{p-1}$, $y = \log_g x$ is called the discrete log of $x$ to base $g$. Discrete log problem is given a generator $g$, $x \in \mathbb{Z}^*_p$ we want to compute $y = \log_g x$. To express this as the HSP:
$$
f: \mathbb{Z}_{p-1} \times \mathbb{Z}_{p-1} \rightarrow \mathbb{Z}_p^*
$$
$$
f(a,b) = g^a x^{-b} \mod p = g^{a-yb} \mod p
$$
Can check if $f(a_1, b_1) = f(a_2, b_2) \iff (a_1, b_1) = (a_2, b_2) + \lambda( y, 1), \lmbda \in \mathbb{Z}_{p-1}$:
$$
G = \mathbb{Z}_{p-1} \times \mathbb{Z}_{p-1}
$$
$$
K = \{ \lambda(y,1): \lambda \in \mathbb{Z}_{p-1} \} < G
$$
Then $f$ is constant and distinct on cosets of $K$ and generator $(y,1)$ of $K$ gives $y = \log_g x$\\\\
\textbf{Graph Problems}:\\
So we can solve problems like those above were $G$ is abelian, but we can also solve graph problems.\\\\
Consider graph $A = (V,E)$, $|V| = n$ lets say that the graph is undirected and there is at most one edge between any two vertices. Vertices here are labelled by numbers from 1 to $n$.\\\\
Lets define an adjacency matrix $M_A$: $[M_a]_{ij} = \begin{cases} 1 & \iff (i,j)\\ 0 & \text{otherwise} \end{cases}$. The permutation group of $[n]$, $|P_n| = n!, \log |P_n| \sim O(n \log n)$. Define a group of automorphisms of group $A$ which is a set of permutations with the following property: $\pi \in P_n$ s.t. $\forall i,j (i,j)$ is an edge in $A$ $\iff (\pi(i), \pi(j))$ is also an edge in $A$.\\\\
An associated HSP (the case of non-abelian $G$):
$$
G = P_n, X = \text{set of all labelled graphs on }n\text{vertices}
$$
For any $A \in X$, define $f_A: G \rightarrow X$, $f_A(\pi) = $"A with vertex labels permuted by $\pi$"
$$
K = Aut(A)
$$
(Check $f(K)$ is constant and disctint on cosets of $Aut(A)$)\\\\
\textbf{Applications}:\\
If we can sample uniformly from $K$, then we can solve Graph Isomorphism problem (GI).This has a number of different applications in areas of computer science. Two labelled graphs $A$ and $B$ with $n$ vertices are isomorphic if there is a 1-1 map (i.e. permutation) $\pi[n] \rightarrow [n]$ s.t. $\forall i,j \in [n] (i,j)$ is an edge in $A$ $\iff$ $(\pi(i), \pi(j))$ is an edge in $B$. The GI problem is given to graphs $A$ and $B$ and deciding if they are isomorphic. This can be represented as a non-abealian HSP. There is no known poly(m) time classical algorithm to solve this problem, so $GI$ is clearly in $NP$ but not believed to be $NP$-complete (a class of problems such that every problem in NP can be reduced to an NP-complete problem these are the hardest NP problems).
In 2017, L Babai presented a quasi-polynomial algorithm for GI runtime $n^{O((\log n)^2)}$. This ranks in between polynomial runtime and exponential algorithms.
\section{Lecture 4}
\textbf{Quantum algorithm for finite abelian HSPs}
- Generalisation of period-finding algorithm\\
Write our abelian group $(G, +)$ additively\\
Construction of shift-invariant states and Fourier transform for $G$.\\
Representations of abelian $G$:\\
Consider the mapping $\chi: G \rightarrow \mathbb{C}^* = \mathbb{C}- \{ 0\}$ with multiplication that satisfies:
$$
\chi( g_1 + g_2) = \chi(g_1) \chi(g_2), \forall g_1,g_2 \in G 
$$
$\chi$ is a group homomorphism from $G$ to $\mathbb{C}^*$. Such $\chi$'s are called irreducible representations of $G$. They have the following properties:
\textbf{Theorem 1}:\\
1) any value $\chi(g)$ is a $|G|$-th root of unity ($\chi \in \rightarrow S^1$ the unit circle)\\
2) Schur's lemma (orthogonality): If $\chi_i,\chi_j$ satisfy (HOM) then 
$$
\frac{1}{|G|} \sum_{g \in G} \chi_i (g) \bar {\chi_j} (g) = \delta_{ij}
$$
3) There are always exactly $|G|$ different functions $\chi$ satisfying (HOM).\\\\
\textbf{Examples}: $\chi(g) = 1, \forall g \in G$ is an irrep/ called a trivial irrep\\
Label the trivial irrep as $\chi_0$, $0\in G$. Then for any other irrep $\chi \neq \chi_0$ orthogonality to $\chi_0$ gives:
$$
\sum_{g \in G} \chi(g) = 0 \text{ if } \chi \neq \chi_0
$$
Going back to constructing shift-invariant states
\subsubsection{Shift-invariant states}
Consider a state space $\mathcal{H}_G, dim \mathcal{H}_G = |G|$ with basis $\{ \ket{g}\}_{g \in G}$. Now introduce shift operators $U(k)$ for $k \in G$ defined as follows:
$$
U(k): \ket{g} \rightarrow \ket{g+k}, g, k \in G
$$
All shift operators commute so there exists a simultaneous eigenbasis.\\
For each $\chi_k, k \in G$:
$$
\ket{\chi_k} = \frac{1}{\sqrt{|G|}} \sum_{g \in G} \bar{\chi_k} (g) \ket{g}
$$
By theorem 1 $\{ \chi_k \}$ form an orthonormal basis.
$$
U(g) \ket{\chi_k} = \chi_k(g) \ket{\chi_k}
$$
\textbf{Proof}:$$U(g) \ket{\chi_k} = \frac{1}{\sqrt{|G|}} \sum_{h \in G} \bar{\chi_k} (h) \ket{h + g}$$
        $$
        \{h' = h+ g\} = \frac{1}{\sqrt{|G|}} \sum_{h' \in G} \bar{\chi_k} (h' - g) \ket{h'}
        $$
        using HOM $\chi_k(-g) = \chi_k(g)^{-1} = \bar \chi_k(g) \implies \overline{\chi_k(h'-g)} = \bar \chi_k(h') \bar \chi_k(-g) = \bar \chi_k(h') \chi_k(g)$. Therefore,
        $$
        U(g) \ket{\chi_k} = \frac{1}{\sqrt{|G|}}\sum_{h' \in G} \chi_k(g) \bar \chi_k(h') \ket{h'} = \chi_k(g) \ket{\chi_k}
        $$
        So $\ket{\chi_k}$'s form a common eigenbasis\\\\
        Introduce Fourier transform QFT for a group $G$\\
- consider a unitary mapping on $\mathcal{H}_G$ mapping $\ket{\chi_k}$ basis to $\ket{g}$ basis
$$
QFT \ket{\chi_g} = \ket{g}, \forall g \in G
$$
$$
QFT^{-1} \ket{g} =  \ket{\chi_g}
$$
k-th column of $QFT^{-1}$ in $\ket{g}$ basis is mode of components of $\ket{\chi_k}$:
        $$
        [QFT^{-1}]_{gk} = \frac{1}{\sqrt{|G|}} \bar{\chi_k}(g)
        $$
        \textbf{Example}: $G = \mathbb{Z}_M$L\\
        Check $\chi_a(b) = e^{\frac{2\pi i a b}{M}}, a, b \in \mathbb{Z}_M$ satisfies HOM and has its irreps labelled by $a \in \mathbb{Z}_M$ with $\chi_0(b) = 1 \forall b \in \mathbb{Z}_m$.
        $$
        G = \mathbb{Z}_{M_1} \times ... \times \mathbb{Z}_{M_r}
        $$
        $$
        (a_1, ..., a_r ) = g_1, (b_1, ..., b_r) = g_2
        $$
        $$
        \chi_{g_1} (g_2) = e^{2\pi i ( \frac{a_1 b_1}{M_1} +... + \frac{a_r b_r}{M_r}}
        $$
This satisfies HOM and our $QFT_G = QFT_{M_1} \otimes ... \otimes QFT_{M_r}$ on $\mathcal{H}_G = \mathcal{H}_{M_1} \otimes ... \otimes \mathcal{H}_{M_r}$.\\\\
This second example is exhaustive since we have a classification theorem:\\
\textbf{Classification theorem}: Any finite abelian group $G$ is isomorphic to a direct product of the form $G = \mathbb{Z}_{M_1} \otimes ... \otimes \mathbb{Z}_{M_r}$. So $M_1$ can be taken in a form $p_1^{s_1},... p_r^{s_r}$.
\subsubsection{Quantum algorithm}
$$
f: G \rightarrow X
$$
with hidden subgroup $K$ and cosets $k = 0 + k, g_2 + k, ... g_m +k$, $m = \frac{|G|}{|K|}$. we will work on $\mathcal{H}_{|G|} \otimes \mathcal{H}_{|X|}, $\{\ket{g}\ket{x}\}_{g \in G, x \in X}$.\\
\begin{itemize}
\item Create a state $\frac{1}{\sqrt{|G|}} \sum_{g \in G} \ket{g} \ket{0}$\\
\item Apply $U_f$ and $\frac{1}{\sqrt{|G|}}\sum_{g \in G} \ket{g} \ket{f(g)}$\\
        \item Measure the second register to get $f(g_0)$. The first register will then give the coset state:
                $$
                \ket{g_0 + k} = \frac{1}{\sqrt{|k|}} \sum_{k \in K} \ket{g_0 + k} = U(g_0) \ket{K}
                $$
        \item Apply QFT and measure to get a result $g \in G$
\end{itemize}
\section{Lecture 5}
We can show that the output distribtion of $g$ is independant of $g_0$, so is the same as obtained from $QFT \ket{K}$ ($g_0 = 0$). First, write $\ket{K}$ in the shift-invariant basis $\{ \chi_g \}_{g \in G}$
$$
\ket{K} = \sum_g a_g \ket{\chi_g}
$$
$$
\ket{g_0+K} = U(g_0) \ket{K} = \sum_g a_g \chi_g(g_0) \ket{\chi_g}
$$
as $QFT \ket{\chi_g} = \ket{g}$ so after we apply QFT
$$
prob(g) = |a_g \chi_g(g_0)|^2 = |a_g|^2, |\chi_g(g_0) | = 1
$$
$$
QFT \ket{K} = \frac{1}{\sqrt{|G|}} \frac{1}{\sqrt{|K|}} \sum_{l \in G} ( \sum_{k \in K} \chi_l (k) \ket{l})$$
$\sum_{k \in K} \chi_l (k) \ket{l}$ involves irreps $\chi_l$ of $G$ restricted to subgroup $K < G$, and each such object is itself an irrep in $K$. Hence we have the following relation:
$$
\sum_{k \in K} \chi_l(k) = \begin{cases} |k| & \text{if }\chi_l \text{ restricts to the trival irrep of K}\\ 0 & \text{otherwise} \end{cases}
$$
$$
QFT \ket{K} = \sqrt{\frac{|K|}{|G|}} \sum_{l \in G} \ket{l}
$$
Then a measurement gives a uniformly random choice of $l$ s.t. $\chi_l(k) = 1$.\\
If $k$ has generators $k_1, ..., k_n$ where $M = O(\log(K)) = O(\log |G|)$. Then the output of a measurement gives us $\chi_l(k) = 1 \forall i$.\\\\
It can be shown that if $O(\log(|G|)$ values of $l$ chosen uniformly at random then with probability $>\frac{2}{3}$ they will suffice to determine a generating set for $k$ via the equations $\chi_l(k) = 1$.\\\\
\textbf{Example}: $G = \mathbb{Z}_{M_1} \times ... \times \mathbb{Z}_{M_l}$\\
$l = (l_1, ..., l_q) \in G$, $g = (b_1, ..., b_q) \in G$ gives $\chi_l(g) = e^{2\pi i( \frac{l_1 b_1}{M_1} + ... + \frac{l_q b_q}{M_q}$\\
        For $k = (k_1, ..., k_q) \in K$ with $\chi_l(k) =1 \implies \frac{l_1 k_1}{M_1} + ... + \frac{l_q k_q}{M_q} = 0 mod 1$. This is a homogenous linear equation on $k$ and $O(\log(k))$ such equations determine $k$ as null space.
        \subsubsection{Remarks on HSP for non-abelian groups $G$}
        Now we will consider multiplicative shifts. As before we can generate a bunch of coset states but it is curious to investigate what breaks down. 
        $$
        \ket{g_0 K} = \frac{1}{\sqrt{|K|}} \sum_{k \in K} \ket{g_0 k}, g_0 \in G\text{ is chosen randomly}
        $$
        The real problem with QFT construction is that there is no good basis of shift invariant states. This is because $U(g_0)$ don't commute. \\\\
        \textbf{Construction of non-abelian QFT}\\
        Consider a $d$-dimensional representation of $G$ is a group homomorphism $\chi: G \rightarrow U(d)$\\
        $\chi$ is irreducible if no subset of $\mathbb{C}^d$ is left invariant by all matrices $\chi(g), g \in G$. (i.e. we cannot simulatenoulsy block-diagonalize all of $\chi(g)$'s by a simple basis change)\\
        Let's define a complete set of irreps. It is a set $\chi_1,..., \chi_m$ s.t. that any irrep is unitarily equivalent to one of them. e.g. $\chi \sim \chi' = V \chi V^{-1}, V \in U(d)$\\
        \textbf{Example}:  $G$ is abelian , all irreps have $d=1$, since all $\chi(g)$ commute.\\\\ \textbf{Theorem( non-abelian analogue of Theorem 1)} (consullt Fulton and Hardes "Representation Theory" fo more information)\\
        If $d_1,...,d_m$ are the dimensions of a complete set of irreps $\chi_1, ..., \chi_M$ then:\\
        1) $d_1^2 +... + d_m^2 = |G|$\\
        2) $\chi_{i,jk} (g) $ is $(j,k)$ th matrix entry of $\chi_i(g)$ then by Schur orthogonality:
        $$
        \sum_g \chi_{i,jk}(g) \bar{\chi}_{i',j'k'}(g) = |G|\delta_{ii'}\delta_{jj'}\delta_{kk'}
        $$
Now if we look at the states that correspond to these irreps $\chi_{i,jk} = \sum_{g\in G} \bar{\chi}_{i,jk} (g) \ket{g}$ they form an orthonormal basis.\\
QFT on $G$ is defined to be a unitary rotation between two basis of $\{ \chi_{i,jk} \}$ basis $\rightarrow \{ \ket{g} \}_{g \in G}$.\\
These states $\ket{\chi_{i,jk}}$ are not shift-invariant for all $U(g_0)$ so this implies that measuring coset state $\ket{g_0 k}$ in the $\{ \ket{\chi}\}$ basis results in an output distribution that is not independant of $g_0$.\\
A "partial" shift-invariance survives. Consider a measuremnt $M_{rep}$ on $\ket{g_0 k}$ this measurement will only distinguish the irreps ($i$ values) and not all $(i,j,k)$'s. the outcome $i$ will be associated with $d_i^2$ dimensional orthogonal subspaces that are spanned by $\{\chi_{i,jk}\}_{j,k = 1}^{d_i}$.\\
Then $\chi_i(g_1 g_2) = \chi_i(g_1) \chi_i(g_2) \implies$ the output distribution of $i$ values is indeed independant of $g_0$.\\
So this gives us direct (but incomplete) information about $k$. For instance, conjugate subgroups $k$ and $L = g_0 K g_0', g_0 \in G$. This measurement will give us the same statistics.\\
$M_{rep}$ will result in the same output statistics\\
Not everything is lost there are some cases when this information is enough.\\\\
The reason HSP is good in the abelian case is we have an efficent $QFT$ transform. In other words QFT can be implemented in $poly(\log(|G|)$ times. This is true for abelian groups and some non-abelian groups (e.g. $P_n$).\\
\texbtf{Some partial results}:\\
For normal subgroups $gk = kg \for g \in G$ we have a theorem proven by Hall green, russel Ta shma in 2003 SIAM J Comp 32, p 916-934:\\
Suppose $G$ has QFT that is efficently implementable. Then if a hideen subgroup $k$ is a normal subgroup, then there is an efficent quantum HSP\\\\
Theorem( Edinsguin, Hoyer, Knill, 2004)\\
For general non-abelian HSP, $M = O(poly \log(|G|))$ then random coset states $\ket{g_1k}$,..., $\ket{g_nk}$ suffice to determine $k$ . But it is not known how to efficently determine $k$ from the $M$ coset states.
\section{Lecture 6}
\subsection{Phase estimation Algorithm}
- Unifying principle for quantum algorithms based on QFT\\
- also gives an alternative way of factoring (orginally discovered  by Kitaev)\\
The fact the phase estimation algorithm became so wide spread and you can cast every algorithm that is tangenetially ralted to QFT in QPE\\
Given a unitary operator $U$ and eigenstates $\ket{v_{\phi}} = U\ket{v_{\phi}} = e^{2\pi i \phi} \ket{v_{\phi}}$\\
Want to estiamte the phase $\phi$ $0< \phi< 1$  ( up to $n$ bits of precision $\phi = 0. i_1 i_2,...i_n = i_1/2 + i_2/4 + ...$ for any given $n$)\\
We will have to implement controlled unitary opearotrs and in particular we will need Controlled-$U^k$ for integers $k$
$$
C-U^k \ket{0} \ket{\xi} = \ket{0}\ket{\xi}, C-U^k \ket{1}\ket{\xi} = \ket{1} U^k \ket{\xi}
$$
$\ket{\xi}$ has a general dimension $d$:
$$
U^k \ket{v_{\phi}} e^{2\pi i k \phi} \ket{v \phi}, C-U^k = (C-U)^k
$$
If we are given $U$ as a circuit description, we can easily implement $C-U$ by controlling each gate in $U$'s circuit. However, if $U$ is given as black box (e.g. a physical operation in the lab) we need further information as there is an inherent ambiguity as we have to account for local phase $e^{i\theta U}$ as it has no effect normally unless you use a controlled operation. \\\\
        If the unitary is specified in this ambiguous way we need to figure out what to do. It suffices to have an eigenstate $\ket{l}$ with a known eigenvalue $U\ket{l} = e^{i l} \ket{l}$ then $e^{i \theta} U$ will map $\alpha \rightarrow \alpha + \theta$. Consider the following circuit:
       
\centerline{ \Qcircuit @C=1em @R=.7em {
\lstick{\ket{a}} & \qw & \ctrl{1} & \qw & \ctrl{1} & \gate{X} & \gate{P(-l)} & \gate{X}\\
\lstick{\ket{\xi}} & \qw & \qswap & \qw & \qswap & \qw & \qw & \qw \\
\lstick{\ket{\alpha}} & \qw & \qswap \qwx & \gate{U} & \qswap \qwx & \qw & \qw & \qw
               }}
               with $P(-l) = \begin{pmatrix}1 & 0 \\ 0 & e^{-il} \end{pmatrix}$. This correctly gives $C-U$. We'll want a "generalised controlled-U" that gives:
               $$
               \ket{x} \ket{\xi} \rightarrow \ket{x} U^x \ket{\xi} x \in \mathbb{Z}_{2^n}
               $$
               For $x= x_{n-1}...x_1x_0 = 2^0 x_0 + 2^1 x_1 + 2^2 x_2 ... + 2^{n-1} x_{n-1}$:\\
              \centerline{ 
             \Qcircuit @C=1em @R=.7em { 
               \lstick{\ket{x_{n-1}}} & \qw & \qw & \cdots & \ctrl{4} \\
                & \cdots & \cdots & \cdots & \cdots \\
               \lstick{\ket{x_{1}}} & \qw & \ctrl{2} & \cdots & \qw \\
               \lstick{\ket{x_{0}}} & \ctrl{1} & \qw & \cdots & \qw \\
       \lstick{\ket{\xi}} & \gate{U^{2^0}} & \gate{U^{2^1}} & \cdots & \gate{U^{2^{n-1}}} 
               }}
       
               If we input $\ket{\xi} = \ket{ V_{\phi}}$ then we get $e^{2\pi i \phi x} \ket{x} \ket{v_{\phi}}$. Now superpose over all $x = 0,1,2,..., 2^{n-1}$ by applying hadamards to all the qubits before applying the circuit, take $\ket{\xi} = \ket{v_{\phi}}$:
\\\\
This gives output $\ket{A} = \frac{1}{\sqrt{2^n}} \sum_x e^{2\pi i \phi x} \ket{x}$. Applying $QFT^{-1}$ to $\ket{A}$ and measure. We get $y_0 y_1 ... y_{n-1}$. Then output the number $0.y_1y_2...y_{n-1} = \frac{y_0}{2} + \frac{y_1}{4} + ... + \frac{y_{n-1}}{2^{n}}$ as an estimate of $\phi$.\\\\ Now lets assume an idealised situtation where $\phi$ has only $n$ binary digits:
$$
\phi = 0.z_1...z_{n-1}
$$
Then $\phi = \frac{z}{2^n}$ where $z$ is an n-bit integer in $\mathbb{Z}_{2^n}$:
$$
\ket{A} = \frac{1}{\sqrt{2^n}} \sum_2 e^{2\pi i 2^n z/2^n} \ket{x} 
$$
is a QFT of $\ket{z}$. Applying $QFT^{-1} \ket{A} = \ket{z}$ and we get $\phi$ exactly with certainity.\\\\
Note the algorithm up to the final measurements is a unitary operation mapping:
$$
\ket{0} \ket{0} ... \ket{0} \ket{ v_{\phi}} \rightarrow \ket{z_0} \ket{z_1} ... \ket{z_{n-1}} \ket{v_{\phi}}
$$
If $\phi$ has more than $n$ bits $\phi = 0.z_0 z_1... z_{n-1}| z_{n} z_{n+1}$.\\
\textbf{Theorem (PE)}: If the output is $\Theta = 0. y_0y_1...y_{n-1}$ then :\\
a) Prob ( $\Theta$ is closeset $n$-binary digit approx to $\phi$) $\geq \frac{4}{\pi^2} = 0.4$\\
b) Prob( $|\Theta - \phi| \geq \epsilon$) $\leq O(\frac{1}{2^n \epsilon}$\\\\
We will show that  Prob( $|\Theta - \phi| \geq \epsilon$) $\leq \frac{1}{2^{n+1} \epsilon})$
\section{Lecture 7}
Today we will prove the Phase estimation theorem. We need to change the defintion of distance as we need a distance on a circle.\\\\
Define $d(\theta, \phi) = \min \{ |\theta -\phi|, |1 + \phi - \theta|, |1+ \theta - \phi|\}$ which is the distance on the circle. Lets consider the normal binary expansion 0.999999 the closest string should be 1.\\\\
\textbf{Theorem (Phase Estimation)}: If the output of PE algorithm with $n$ lines (initiated in as zeros) is $\theta = 0.y_0y_1...y_{m-1}$, then:\\
a) Prob ( $\theta$ is closest $n$-binary digit approx to $\phi$, $d(\theta, \phi) \leq \frac{1}{2^{n+1}}$) $\geq \frac{4}{\pi^2} = 0.4$\\
b) Prob( $|\theta - \phi| \geq \epsilon$) $\leq O(\frac{1}{2^n \epsilon})$ for $\epsilon$ fixed\\\\
Recall: the output is obtained by measuring an $n$-qubit state $QFT^{-1} \ket{A}$ where $\ket{A} = \frac{1}{\sqrt{2^n}} \sum_{x = 0}^{2^n-1} e^{2\pi i l x} \ket{x}$
$$
QFT^{-1} \ket{x} = \frac{1}{\sqrt{2^n}} \sum_{y= 0}^{2^n-1} e^{-2\pi i \frac{y x}{2^n}} \ket{y}
$$
Soon we will change the notation to make sure it is not overloaded with these powers of $2^n$
$$
QFT^{-1} \ket{A} = \frac{1}{2^n}\sum_{y=0} \sum_{x=0} e^{2 \pi i ( \psi - \frac{y}{2^n})x} \ket{y}
$$
Let $\{ \phi = 2^n \psi \}$:
$$
QFT^{-1} \ket{A} = \frac{1}{2^n}\sum_{y=0} \sum_{x=0} e^{2 \pi i  \frac{\phi - y}{2^n})x} \ket{y} = \frac{1}{2^n}  \sum_y \frac{ 1 - e^{2\pi i (\phi - y)}}{ 1 - e^{2\pi i \frac{\phi - y}{2^n}}} \ket{y}
$$
In the case $\phi - y \neq 0$:
$$
Prob( \text{see } y) =  \frac{1}{2^n} |\sum_{x=0}^{2^n-1} e^{2\pi i\frac{(\phi - y)}{2^n} x}|^2 = \frac{1}{2^{2n}} \frac{ |1 - e^{2\pi i (\phi - y)}|^2}{ | 1 - e^{2\pi i \frac{\phi - y}{2^n}}|^2} = \frac{1}{2^{2n}} \frac{| 1 - e^{2\pi i (\psi - \frac{y}{2^n}) 2^n|^2}}{| 1 - e^{2\pi i (\psi - \frac{y}{2^n})}|^2}
$$
As $\theta_y = \frac{y}{2^n}$ and observe that $|1- e^{2\pi i(\psi - \theta_y)}|^2 = |1 - e^{2\pi i d(\psi, \theta_y)}|^2$ therefore:
        $$
Prob( see\> y) =  \frac{1}{2^{2n}} \frac{|1 - e^{2\pi i 2^n d(\psi, \theta_y)}|^2}{|1 - e^{2 \pi i d(\psi, \theta_y}|^2}
        $$
As $0<d(\psi, \theta_y)\leq \frac{1}{2}$ we will use the following bounds:\\
               i) $|1-e^{i \alpha}| \leq 2$\\
               ii) $|1-e^{i \alpha}| \leq |\alpha|$\\
               iii) For $|\alpha| \leq \phi$ $|1- e^{i \alpha}|= 2|\sin( \frac{\alpha}{2})| \geq \frac{2|\alpha|}{\pi}$\\\\
       The last of thse comes from the fact that for positive $\alpha$ we have $\sin(\alpha/2) \geq \frac{\alpha}{\pi}$\\\\
               When $d(\psi, \theta) \leq \frac{1}{2^{n+1}}$ implies that $y$ is the best approxiamtion for $\psi$ and $2\pi d(\psi, \theta_y) 2^n \leq \frac{2^{n+1}}{2^{n+1}} \pi$ so:
               $$
               Prob( y\text{ is best approximation}) \geq \frac{1}{2^{2n}} | \frac{2}{\pi} \frac{2 \pi d(\psi, \theta_y)}{2 \pi d(\psi, \theta)}|^2 = \frac{4}{\pi^2}
               $$
               The calculations for the above will be on the moodle\\\\
               \textbf{Further remarks}:\\\\
               If $C-U^{2^n}$ is implemented as $(C-U)^{2^{\alpha}}$, then PE algorithm needs exponential time ($ 1+ 2+... + 2^{n-1} = 2^{n}-1$). But for some some $U$ implementing $C-U^{2^k}$ requries only polynomial time so we get a poly-time PE algorithm. Harks back to the algorithm for finding powers by repeated squaring, expressing the exponent in binary and then doing repeated squaring. The number of applications of controlled unitaries does not depend on $d$ the dimension of the space. This can be used to provide an alternative factoring algorithm ( due to A.Kitaev) ( see example sheet).\\\\
               In many applications we feed an arbitary state to the last register rather than an eigenstate. If instead of $\ket{v_{\phi}}$ we input general state $\ket{\xi}$, expand in eigenbasis of $U$:
               $$
               \ket{\xi} = \sum_j c_j \ket{v_{\phi_j}}, U \ket{v_{\phi_j}} = e^{2 \pi i \phi_j} \ket{v_{\phi_j}}
               $$
               Then we get (before the final measurement) a unitary process $U_{PE}$:
               $$
               \ket{00..00} \ket{\xi} \rightarrow^{U_{PE}} \sum_j c_j \ket{\psi_j} \ket{v_{\phi_j}}
               $$
               The Born rule implies that the final measurement will give a choice of $\phi_j$'s (or an approximation) it can be choosen with probability $|c_j|^2$. This is not some average of the $\phi_j$ values.\\\\
               Will be elaborated more in the notes on moodle on the following:\\\\
               If you want to have $n$-qubits and want to get $m$-bits correctly probablility of success $1- \eta$, then must have :
               $$
               n \geq m + \log \frac{1}{\eta}
               $$
               \subsection{Amplitude amplification}
Much like when we multiple up to HSP we revisited shors alogrithm, in this case we revisit grovers algorithm. This is an apotheosis of technique in Grover's algorithm.
\subsubsection{Background}
\textbf{Reflection Operators}:
State $\ket{\alpha}$ in $\mathcal{H}_{d}$ , $n$-dim subspace $L_{\alpha}$ with $(d-1)$ dim orthogonal $l_{\alpha}^{\alpha}$
$$
I_{\ket{\alpha}} = I - 2 \ket{\alpha} \bra{\alpha}
$$
$$
I_{\ket{\alpha}} = - \ket{\alpha}
$$
$$
I_{\ket{\alpha}} \ket{\beta} = \ket{\beta}
$$
for any $\ket{\beta} \prep \ket{\alpha}$\\\\
For any unitary $U$: $UI_{\ket{\alpha}} U^{\dagger} = I_{U\ket{\alpha}}$, $U \ket{\alpha} \bra{\alpha} U^{\dagger} = \ket{\beta} \bra{\beta}$ for $\ket{\beta} = U\ket{\alpha}$.\\\\
Consider a $k$-dimensional subspace $A < \mathcal{H}_d$, any orthonormal basis $\ket{a_1}, ..., \ket{a_k}$. Lets consider a projection operator on to this subspace:
$$
P_A = \sum_{i=1}^k \ket{a_i} \bra{a_i} 
$$
Define a generalised projeciton operator:
$$
I_A = I - 2 P_A
$$
$$
               I_A \ket{\xi} = \begin{cases} \ket{\xi} & \ket{\xi} \in A^{\dagger}\\
               - \ket{\xi} & \ket{\xi} \in A\end{cases}
$$
Now lets recall what grover does very briefly (have a look at Part II course):\\\\
Search for a unique "goal" item  in unstructured database of $N = 2^n$ items.\\\\
Write $B_n =$ set of all n-bit strings. Given an oracle for $f: B_n \rightarrow B_n$ with the promise that there is a unique element $x_0 \in B_1$ with $f(\x_0) = 1$. Problem is to find $x_0$.\\\\
Closely related to class NP and Boolean satisfiability problem. This is explained in part II lecture notes.
\section{Lecture 8}
\textbf{Recap of grovers algorithm}
We are searching for a unique "good" element in an unstructured database, $N= 2^n$ items\\
We are given an oracle $f$ that maps from $B_n \rightarrow B_1$\\
Promise: There is a unique $x_0 \in B_n$ with $f(x_0) = 1$\\
Problem: Find $x_0$\\\\
Consider Grover iteration operator on $n$ qubits
$$
Q = - H_n I_{\ket{0}} H_n I_{\ket{x_0}} = - I_{\ket{\psi_0}} I_{\ket{x_0}}
$$
where $H_n = H\otimes ... \otimes H$ , $\ket{\psi_0} = H_n \ket{00.000} = \frac{1}{\sqrt{2^n}} \sum_{x \in B_n} \ket{x}$\\\\
One application of $Q$ uses 1 query of $U_f$\\\\
\textbf{Thereom (Grover '96)}: In 2-dim span of $\ket{\psi_0}$ and (unknown) $\ket{x_0}$ the action of $Q$ is rotation by angle $2 \alpha$ where $\sin \alpha = \frac{1}{\sqrt{N}}$. Hence grover's algorithm to find $x_0$ given $U_f$ is:
\begin{itemlist}
\item Make $\ket{\psi_0}$\\
\item Apply $Q$ $m$ times where $m = \frac{\arccos \frac{1}{\sqrt{N}}}{2} \arcsin \frac{1}{\sqrt{N}}$ to rotate $\psi_0$ very close to $x_0$ (within the angle $\pm \alpha$\\
\item Measure to see $x_0$ with high probability $1- \frac{1}{N}$
\end{itemlist}
For large $N$ $\arccos \frac{1}{\sqrt{N}} = \frac{\pi}{2}$ and $\arcsin \frac{1}{\sqrt{N}} = \frac{1}{\sqrt{N}}$ so $m = \frac{\pi}{4} \sqrt{N}$ iteractions or queries to $U_f$ needed.\\\\
Classically we need $O(N)$ quereies to find $x_0$ with any constant probability that does not depend on $N$, so this achieves a quadratic speed-up.
\subsection{Amplitude Amplification}
Let $G$ be any subspace ('good subspace') of state space $\mathcal{H}$ $G^{\prep}$ be its orthogonal complement ('bad subspace') $\mathcal{H} = G \oplus G^{\prep}$\\\\
Given any $\ket{\psi} \in \mathcal{H}$, we have unique decomposition with real positive coefficentt $\ket{\psi} = \sin \theta \ket{g} + \cos \theta \ket{b}, \ket{g} \in G, \ket{b} \in G^{\prep}$\\
Introduce reflection operators that flip $\ket{\psi}$ and good vectors:
$$
I_{\ket{\psi}} = I - 2 \ket{\psi} \bra{\psi}, I_G = I - 2 P_G
$$
$\sin \theta  = ||P_G\ket{\psi}|| = $length of good projection of $\ket{\psi}$\\
Introduce $Q = -I_{\ket{\psi}} I_G$
\subsubsection{Amplitude Amplification Theorem}
In the 2-dim space spanned by $\ket{g}$ and $\ket{\psi}$ $Q$ is roation by $2 \theta$ where\\
$\sin \theta = $length of good projection of $\ket{\psi}$\\
\textbf{Proof}: We have $I_G \ket{g} = - \ket{g}, I_G \ket{b} = \ket{b}$:
$$
Q \ket{g} = I_{\ket{\psi}} \ket{g}, Q\ket{b} = - I_{\ket{\psi}} \ket{b}
$$
$$
I_{\ket{\psi}} = I - 2( \sin \theta \ket{g} + \cos \theta \ket{b}) (\sin \theta \bra{g} + \cos \theta \bra{b})
$$
using the fact that $\bra{b}\ket{g} = 0, \bra{g} \ket{g} = \bra{b} \ket{b} = 1$:
$$
Q\ket{b} = \cos 2 \theta \ket{b} + \sin 2 \theta \ket{g}
$$
$$
Q \ket{G} = - \sin 2 \theta \ket{b} + \cos 2 \theta \ket{g}
$$
So
$$
Q \ket{g} = I\ket{g} - 2 \sin^2 \theta \ket{g} - 2 \sin \theta \cos \theta \ket{b} = \cos 2 \theta \ket{g} - \sin 2 \theta \ket{b}
$$
In the $\{ \ket{b}, \ket{g}\}$ basis, the $Q$ matrix is a rotation matrix by $2 \theta$:
$$
                       Q  = \begin{pmatrix} \cos 2 \theta & \sin 2 \theta \\ - \sin 2 \theta & \cos 2 \theta \end{pmatrix}
$$
As we apply $Q$  $n$ times we get $Q^{n} \ket{\psi} = \sin (2n+1) \theta \ket{g} + \cos (2n+1) \theta \ket{b}$. If we measure $Q^{n}$ in $\{ \ket{b}, \ket{g} \}$ basis we have probability of seeing a good element of $\sin^2 (2n+1) \theta$ so this is maximised when $(2n+1) \theta = \frac{\pi}{2}$. So for the nearest integer to $n = \frac{\pi}{4 \theta} - \frac{1}{2}$ we will be within $\theta$ of the element.\\\\
\textbf{Example}: If we have $\theta = \frac{\pi}{6}, n =1$ we can seee that $Q^1$ rotates $\ket{\psi}$ exactly onto $\ket{g}$.\\\\
Generally, for a given $\theta$ $n$ is not an integer, so we use $n=$ nearest integer to $ \frac{4\pi}{\theta} - 1 \approx \frac{4\pi}{\theta} = O(\frac{1}{\theta}) = O(\frac{1}{\sin \theta}) = O(\frac{1}{\text{length of good projection of } \ket{\psi}}$ and then $Q\ket{\psi}$ will be witin angle $\theta$ of $\ket{g}$ so the probability of seeing a good value is: $P \geq \cos \theta = 1 - O(\theta^2)$.\\\\
All this can implement if $I_{\ket{\psi}}I_G$ can be impmeneted efficently. see ES2.\\\\
For $I_G$ is suffices for $G$ to be spanned by computational basis states and have an indicator funciton $f$.
$$
f(x) = 1, x \text{if} x\text{ is good}, f(x) = 0 \text{ if} x \text{is bad}
$$
For $I_{\ket{\psi}}$ we usually have $\ket{\psi} = H_n \ket{00..000}$, then $\ket{\psi}$ can be implemented in $O(n)$ time where $n$ is the number of qubits.\\\\
In the amplitude amplification algorithm the relative amplitudes of good elements remain the same as they were in $\ket{\psi} = \sin \theta \ket{g} + \cos \theta \ket{b}$ so $\ket{g}$ remains the same just the amplitude varies. So AA amplifies overall $\ket{g}$ amplitude at the expense of reducing the amplitude of $\ket{b}$.\\\\
Second remark: Final state is generally not exactly $\ket{g}$, however, if $\sin \theta$ is known then there is a modification of this algorithm that uses a modest amount of resources to make it exact.\\\\
The rotuine is useful for state prepation e.g.
$$
\sum_{x < N, \text{x is coprime to N}} \ket{x}
$$
\section{Examples Class 1}
Try to prove the same thing in 1(ii) in shors algorithm and it will be slightly nicer language than group theory.\\\\
For $\mathbb{Z}_2$ the irreps are $\chi_a(x) = (-1)^{ax}$ for $a, x \in \mathbb{Z}_2$ therefore $\mathbb{Z}_2^n$ has irreps $\chi_a(x) = (-1)^{a_1 x_1 + a_2 x_2 +...+ a_n x_n}$ so
$$
\ket{\chi} = \frac{1}{\sqrt{|G|}} \sum_g \bar \chi( g) \ket{g} = \frac{1}{\sqrt{|G|}} \sum_{b \in \mathbb{Z}_2^n} (-1)^{a b} \ket{b}
$$
and
$$
[QFT]_{ab} = \frac{1}{\sqrt{|G|}} (-1)^{ab} = \frac{1}{\sqrt{2^n}} (-1)^{a_1 b_1} (-1)^{a_2 b_2} ... (-1)^{a_n b_n}
$$
                       could be written using Hadmards with $QFT = H \otimes H \otimes... \otimes H$ as $H = \frac{1}{\sqrt{2}}\begin{pmatrix} 1 & 1\\ 1& -1 \end{pmatrix}$ with the columns being $a$ and the rows being $b$ so it is only $-1$ if they are both $1$ as otherwise it will have a zero in the exponent.
\\\\
First part of HSP problems is generate coset states and the second part is repeation (how many times you need to repeat the process). In standard HSP we make one query to $f$ to make the following state:
$$
\ket{y \oplus k} = \frac{1}{2^k} \sum_{x \in K} \ket{y \oplus x}, y \in \mathbb{Z}_2^n
$$
Apply $QFT =  H^{\otimes n}$ and measure. This gives an uniformly random output $c \in \mathbb{Z}_2^n$ which is such that the irrep $\chi_c$ of $G$ that is restricted to $K$ is the trivial irrep of $K$. In other words $\chi_c(a) = 1$ for all $a \in K$. In other word, $(-1)^{ac} = 1 \implies c a = 0$ mod 2. We know that this $k$ viewed as a subspace of $\mathbb{Z}_2^n$ has dimension $k$. So we need $(n-k)$ linearly indepdendant $c_i$ with $c_i a = 0$ to determine $K$.\\\\
In order to succeed with probability $1- \epsilon$ when given a constant probability $p$ we run the process some $M$ times. Then you calculate the probablity of $M$ runs failing to determine $k$ and show that for a constant overhead we can get any accuracy. $1- (1-p)^M > 1- \epsilon$.\\\\
Phase gates act with $P(\alpha) : \ket{0} \rightarrow \ket{0}, \ket{1} \rightarrow e^{i \alpha}\ket{1}$. Therefore can apply a fractional phase by first preparing a register with $i_1,...,i_n$ s.t. $y= 0.i_1...i_n$ and then by applying phase gates $P(\frac{1}{2}...P(\frac{1}{2^n}$to get the state $e^{iy}\ket{y}$.\\\\
When inventing algorithms whilst it is useful to think about eigenstates to start with make sure to run it through with a general state as you might need to uncompute at the end. e.g. need to apply an inverse controlled unitary in question 5 after computing the correct eigenvalue.\\\\
\textbf{We can't just discard additoinal registers like $\sum_j\lambda_j \ket{u_j} \ket{c_j}$ to $\sum_j \lambda_j \ket{u_j}$ we need to uncompute to remove these extra stuff.}
\section{Lecture 9}
\subsection{Amplitude amplification}
We have a Hilbert space we can partition into a good part and a bad part: $\mathcal{H} = G_{good} \oplus G_{good}^{\dagger}$, for every $\ket{\psi}$ we have $\ket{\psi} = \sin \theta \ket{g} + \cos \theta \ket{b}$. We have:
$$
I_{\ket{\psi}} = I - 2\ket{\psi} \bra{\psi}, I_G = I - 2 P_G, Q = - I_{\ket{\psi}} I_G
$$
We proved AA Thereoem: That in the plane spanned by $\ket{b}$ and $\ket{g}$ $Q$ is a rotation by $2\theta$.\\\\
Typically we are given some $\ket{\psi}$ and we use this property to rotate it to $\ket{g}$.
\subsection{Applications of Amplitude Amplification}
\textbf{Grover search with one or more (k) good items in N}:\\
Re look over Part II lecture notes and reprove for $k$ good elemetns. Maybe try problem sheet again.
$$
\ket{\psi} = \frac{1}{\sqrt{2^n}} \sum_{x \in B_n} \ket{x} = \sqrt{\frac{k}{N}} ( \frac{1}{\sqrt{k}} \sum_{good elem} \ket{x}) + \sqrt{\frac{N-k}{N}} ( \frac{1}{\sqrt{N-k}} \sum_{bad} \ket{x})
$$
G is spanned by $k$ good $\ket{x}$'s, $\sin \theta = \sqrt{\frac{k}{N}}$ so Q is rotation by $2 2 \theta$, $\theta = \arcsin \sqrt{\frac{k}{N}} \implies O(\sqrt{\frac{N}{k}})$ queries.\\\\
Note that for 2-bit case $N=4$ and $k=1$ good element we get $\theta = \arcsin \frac{1}{2} = \frfrac{\pi}{6}$ and one application of Q rotates $\ket{\psi_0}$ exactly onto $\ket{g}$\\\\
\textbf{Square-root speedy of general quantum algorithms}:\\
Let $A$ be a quantum algorithm/circuit(sequence of 'basic' unitary gates) on input states 
\ket{00.00}$. The final state is $A \ket{00...000}$.\\
Good labels = desired computation outcomes.\\
$A\ket{0...0} = \alpha \ket{a} + \beta \ket{b}, \alpha = \sin \theta$  with $\ket{a}$ normalised but generally unequal superposition $\sum_{good x} c_x \ket{x}$.\\
So probability of success in 1 run is $|\alpha|^2$ so need to do $O(\frac{1}{|\alpha|^2})$ repetitions of A to succeed with any given const probability $1- \epsilon$.\\\\
Now lets try amplitude amplification instead of repeated measurement. We need to check we satisfy some assumptions.\\
Assume we can check if the answer is good or bad.\\
                       So that we can implement $I_G \ket{x} \rightarrow \begin{cases} - \ket{x} & \text{x is good}\\ \ket{x} & \text{x is bad} \end{cases}$\\
                       Consider $\ket{\psi} = A \ket{0.00}$
                       $$
                       Q = - I_{A\ket{00.000}} I_G = - A I_{\ket{0..00}} A^{\dagger} I_G
                       $$
                       so all parts are implementable\\\\
                       By the amplitude amplification theorem $Q$ is a rotation by $2 \theta$ with $\sin \theta = |\alpha|$. So after $n = \frac{\pi}{4 \theta} = O(\frac{1}{|\theta|}) = O(\frac{1}{\sin \theta}) = O(\frac{1}{|\alpha|}$ repetitions $A\ket{0..0}$ will be rotated very near $\ket{g}$ and the final measurement will succeed with high probability. So we get a square root time speed up over a discrete repetition method.\\\\
                       Each application of $Q$ need one $A$ and one $A^{\dagger}$. and can think of $A^{\dagger}$ as inverse gates in reverse order so it has a similar complexity to $A$.\\\\
                       If success probability of $A$ is known then we can do even better and we can cook up the rotation so exactly get to $\ket{g}$ (this exact method will be covered in ES2). This will convert the probabilistic algorithm $A$ into a deterministic one.\\\\
                       \textbf{Quantum counting}:\\
                       Here we apply amplitude amplification and phase estimation.\\
                       Given $f: B_n \rightarrow B_1$ with $k$ good $x$'s we want to estimate $k$ (instead of finding a good $x$).\\
                       Let's recall the grover operator: $Q_G$ for $f$ is a rotation by $2\theta$ in this 2D plane spanned by $\ket{\psi_0} = \frac{1}{\sqrt{2^n}} \sum_{x \in B_n}$ and its good projection $\ket{g} = \frac{1}{\sqrt{k}} \sum_{good x} \ket{x}$ with $\sin \theta = \sqrt{\frac{k}{N}} \approx \theta$\\
                       Recall that rotations in $\{ \ket{b}, \ket{g} \}$ plane have eigenvalues and eigenvectors:
                       $$
                       \ket{e_{\pm}} = \frac{1}{\sqrt{2}} (\ket{b} \pm i \ket{g})
                       $$
                       $$
                       \lambda_{\pm} = e^{\pm 2 i \theta}
                       $$
                       Then check that $\ket{\psi_0} = \sin \theta \ket{g} + \cos \theta \ket{b} = \frac{1}{\sqrt{2}} ( e^{i \theta} \ket{e_+} + e^{- i \theta} \ket{e_-})$\\
       Writing $e^{\pm 2 i \theta} $ as $e^{2 \pi i \phi_{\pm}}$ with $\phi_{\pm} \in (0,1)$:
               $$
               \phi_+ = \frac{2\theta}{2\pi} = \frac{\theta}{\pi}
               $$
               $$
               \phi_- = \frac{-2 \theta + 2 \pi}{2 \pi} = 1 - \frac{\theta}{\pi}
               $$
               $\frac{\theta}{\pi} = \frac{1}{\pi} \sqrt{\frac{k}{N}}$\\
               Run QPE algorithm with $U =$ "Grover Q" and estimate register set to $\ket{\psi_0}$: will output $\frac{\theta}{\pi}$ or $1- \frac{\theta}{\pi}$ with probability $\frac{1}{2}

               \section{Lecture 10}
               $$
               f: B_n \rightarrow B_1
               $$
               which has $k$ good $x$'s and we want to estimate $k$. This is harder than just finding a solution.
               $$
               \ket{\psi_0} = \frac{1}{\sqrt{2^n}} \sum_{x \in B_n} \ket{x}, \ket{g} = \frac{1}{\sqrt{k}} \sum_{good x} \ket{x}
               $$
               $$
               \sin \theta = \sqrt{\frac{k}{N}}
               $$
               for $k << N$ so
               $$
               \ket{\psi_0} =  \sin \theta \ket{g} + \cos \theta \ket{b} = \frac{1}{\sqrt{2}} (e^{i\theta} \ket{e_+} + e^{-i \theta} \ket{e_-})
               $$
       Run QPE -> $\frac{\theta}{\pi}$ or $1- \frac{\theta}{\pi}}$ with probability $\frac{1}{2}$. We can get approximately $\theta$ in either case. For any $m$, running $QPE$ with $m$ qubit lines will give an $m$-bit approximation to $\sqrt{\frac{k}{N}}$ this uses $2^n$ C-Q gates.\\
       By Theroem PE we learn $\sqrt{\frac{k}{N}}$ to an additive error with constant probability $\frac{4}{\pi^2}$ using $O(2^m)$ queries. Write $\frac{1}{2^n}$ as $\frac{\delta}{\sqrt{N}}$ $\delta >0$. Thus we can learn $\sqrt{k}$ to additive error $O(\delta)$ using $O(2^n) = O(\frac{\sqrt{N}}{\delta})$ queries.\\\\
       Classically the same approximation (obtained with constant probability) requires $O(\frac{N}{\delta^2})$ so requires quadratically more effort.
       \subsection{Hamiltonian simulation}
       One of the most promising applications of qunatum computers is simulating quantum systems. This is difficult for classical computers as these physical systems have so many degrees of freedom. We won't do the nitty gritty but cover some of the core fundamental ideas and talk about the complexitiy of it.\\\\
       We want to use a quantum computer to simulate the evolution/dynamics of a quantum system given its Hamiltonian $H$.\\
       For $n$ qubits generally requires $O(2^n)$ time on a classical computer. We will do it in poly(n) time for a class of Hamiltonian. A lot of good hamiltonains that corrospond to real problems do admit a good representations in quantum computers. First we will focus on evolution and dynamics and then we will move on to studying the ground state properties of these hamiltonians.
       \subsubsection{Hamiltonain and quantum evolution}
       Consider a physical system  in state $\ket{\psi}$, with hamiltonain $H$: selfadjoint (Hermitian) operator/matrix - the quantum energy observable. Time evolution given by Schrodinger equation $(\hbar = 1)$ 
       $$
       \frac{d}{dt} \ket{\psi(t)} = - i H \ket{\psi(t)}
       $$
       We will consider time-independant Hamiltonians $H(t) = H$
       $$
       \ket{\psi(t)} = e^{-iHt} \ket{\psi(0)}
       $$
       $$
       e^A = I + A + \frac{A^2}{2!} + ...
       $$
       Thus given $H$ and time $t$ we want to simulate the action of a unitary $U(t) = e^{-i Ht}$ to a suitably good approximation.\\
       \textbf{Approximation (closeness) of unitary operators}
       Operator norm (spectral norm) $||A|| = \max_{||\psi|| = 1} ||A\ket{\psi}|| = |$maximium eigenvalue|$ (if $A$ is diagonalisable)$.
       $$
       ||A+B|| \leq ||A|| + ||B||
       $$
       We say that $U$ approximates $\tilde U$ to within error $\epsilon$ if $||U - \tilde U|| \leq \epsilon$ (for any $\ket{\psi}$ the action of $U$ $\tilde U$ are at most $\epsilon$ apart.\\\\
       In general this hamiltonain will be a huge $2^n\times 2^n$ matrix for an $n$qubit system so hard to write down so we will consider a special class of $k$-local hamiltonains. We will want to simulate $U = e^{- i tH}$ with a circuit of poly(n,t) "basic" unitary gates (this is the defintino of efficent simulation on a quantum computer).\\\\
       Not all $H$'s can be efficently simulated. One such class of efficent hamiltonains is k-local hamiltonains:
                       \textbf{$k$-local Hamiltonain}: $H$ is $k$-local on $n$ qubits if $H= \sum_{j=1}^m H_j$ where $H_j$ is a hermitain matrix acting on at most $k$ qubits. This action does not have to be contiouns a.k.a they don't need to be qubits 1-k it can be any subset of $k$ qubits.
                       $$
                       H_j = \tilde H_j \otimes I
                       $$
                       with $\tilde H_j$ encoding the hamiltonain on the $k$ qubits and the $I$ is on the rest of the system.\\\\
                       $m \leq (^n_k) = O(n^k) = poly(n)$ terms in $H$. We will see that many imporotant classes of hamiltonains fall into this category e.g.:\\
                       1. (3 qubits, 2-local)
                       $$
                       H = X \otimes I \otimes I - 5 Z \otimes I \otimes Y
                       $$
                       2. Write $M_{(k)}$ to denotes an operator $M$ acting on $k$ qubits and $I$ on the rest. This can correspond to the Ising model on $n\times n$ square lattice of qubits. 
$$H = J \sum_{i,j=1}^{n-1} Z_{(i,j)} Z_{(i, j+1)} + Z_{(i,j)}Z_{(i+1,j)}$$
Heisenberg model on a line:
$$
H = \sum_{i=1}^{n-1} J_x X_{(i)} X_{(i+1)}  + J_y Y_{(i)} Y_{(i+1)} + J_z Z_{(i)} Z_{(i+1)}
$$
with $J_x, J_y, J_z$ are real coeff.\\\\
Remember that $e^{-\sum_i H_jt} \neq  \prod_j e^{- H_j t}$ if $H_j$ don't commute\\
Also remember that $e^{-H_j t}$ are local unitaries acting on $k$ qubits.
\section{Lecture 11}
If we want to use some standard universal gate set, then we will invoke the theorem: \\\\
\textbf{Solovay-Kitaev Theorem}: Let $U$ be a unitary operator on $k$ (const) qubits, and $S$ be any universal set of quantum gates. Then $U$ can be approximated to within accuracy $\epsilon$ using a logarithm sequence of gate $O(\log^c (\frac{1}{\epsilon}))$ gates from $S$ with $c<4$. \\ We don't prove this result but it is in Nielsen and Chuang.\\\\
\textbf{Lemma A} about accumulation of errors: We need to prove this on ES2. Let $\{\bm u_i\}$, $\{ \bm v_i\}$ be sets of $m$ unitary operators such that $|| U_i - V_i || \leq \epsilon$. So if we want to approximate a whole sequence of $U_i$s: $||U_m...U_1 - V_m...V_1|| \leq m \epsilon$ so errors will accumulate linearly. This is kind of the worst case error as it is the maximium error over all $\ket{\psi}$.\\\\
Proof: using induction on $m$.\\ First lets consider a little warm up with the (easy) commuting case.\\
Consider the case:
$$
H = \sum_{j=1}^m H_j, \text{ any k-local Hamiltonian with commuting }H_j
$$
Then for any power $t$, $e^{-iHt}$ can be approximated to within $\epsilon$ by a circuit $O(m poly(\log (\frac{m}{\epsilon})))$ from any given universal set. (Note that $m = (^n_k) = O(n^k)$, this is poly$(n_1 \log \frac{1}{\epsilon})$ too and $\log \frac{1}{\epsilon}$ is the number of digits in the precision of the approximation\\
Proof: Using $SK$ theorem each $e^{-iH_jt}$ can be approximated to within $\frac{\epsilon}{m}$ with $O(poly(\log \frac{m}{\epsilon}))$ gates.\\
Lemma A implies the full product $\prod_{i=1}^m e^{-i H_j t}$ is then approximated to within $m \frac{\epsilon}{m} = \epsilon$ using a total of $O(m poly(\log \frac{m}{\epsilon}))$ gates.\\\\
\textbf{The full non-commuting case}\\
For any matrix $X$ we write $X + O(\epsilon)$ for $X + \epsilon$, where $||E|| = O(\epsilon)$\\
\textbf{Lemmma B}(Lee-Trotter formula):\\
Let $A, B$ be the following matrices with $||A|| \leq K$ and $||B || \leq K$ with $K<1$ (small). Then:
$$
e^{-iA} e^{-iB} = e^{-i(A+B)} + O(K)^2
$$
Proof: $e^{-iA} = I - iA + \sum_{k=2} \frac{(-iA)^k}{k!} = I - iA + (iA)^2 \sum_{k=0} \frac{(-iA)^k}{(k+2)!} $ we want to show that the sum in the last term has norm $< e^{-K}<1$. Therefore, $e^{-iA} = I - iA + O(K^2)$ so therefore:
$$
e^{-iA} e^{-iB} = (I - iA + O(K^2))(I- iB + O(K^2)) = I - i(A+B) + O(K^2) = e^{-i(A+B)} + O(K^2)
$$
Now, apply Lee-Trotter formula repeatedly to accumulate sums of $H_1...H_m$ in the exponent\\
Note that if each term $||H_j||<K$, then $||\sum_{i=1}^l H_i || < l K$ and we'll want this expression here to be $<1$ for all $l \leq m$. For now, we'll assume that $K< \frac{1}{m}$. For now we will also take $t =1$ (deal with general $t$ later). 
$$e^{-iH_1} e^{-iH_2}... e^{-iH_m} = [ e^{-i(H_1 + H_2)} + O(K^2)] e^{-iH_3}...e^{-iH_m}$$
So as $||H_1 + H_2 ||< 2K$ and the error that is error that is generated at each stage stays the same for subsequent unitaries $U_i$ as $||AU|| = ||A||$  therefore:
$$
e^{-iH_1} e^{-iH_2}... e^{-iH_m} = e^{-i(H_1+H_2)} e^{-iH_3}...e^{-iH_m} + O(K^2)
$$
Then by Lee-Trotter for $(H_1 + H_2)$ and $H_3$
$$
e^{-iH_1} e^{-iH_2}... e^{-iH_m} = [e^{-i(H_1+H_2+H_3)} +O((2K)^2)]e^{-iH_4}...e^{-iH_m} + O(K^2)
$$
as $||H_1 + H_2|| < 2K$. We can continue in the same fashion until we get \textbf{error estimate 1}:
$$
e^{-iH_1} e^{-iH_2}... e^{-iH_m} = e^{-i(H_1 + ...+ H_m)} + O(k^2) + O((2k)^2)) + ... + O( ((m-1)k)^2) =  e^{-i(H_1 + ...+ H_m)} + O(m^3 k^2)
$$
For general finite $||H_j||$'s and $t$ values $||H_jt|| < Kt$ (this value can be large). So we introduce a way of breaking down the sequence by introducing $N$ (that we will fix later) s.t. $\frac{H_j t}{N}$ gives $\tilde K = || \frac{H_jt}{N}|| < \frac{Kt}{N}$. We can think about this as dividing the time $T$ up into small $\frac{1}{N}$ intervals and then our unitary has the value:
$$
U = e^{i(H_1 + ... + H_m)t} = ( e^{i( \frac{H_1 t }{N} + ... + \frac{H_2 t}{N})})^N
$$
We want final error for $U$ to be less than $\epsilon$ so by lemma A we want the error of each step to be less than $\frac{\epsilon}{N}$. So using error estimate 1 we get 
$$
Cm^3 \tilde K^2 < \frac{\epsilon}{N} \implies N > \frac{C m^3 k^2 t^2}{\epsilon}
$$
Then || e^{-i \frac{H_1t}{N}} ... e^{- i\frac{H_nt}{N}} - e^{-i(H_1 t + ... H_n t)} || < \frac{\epsilon}{N}$. BY Lemma A:
$$
|| (e^{-i \frac{H_1t}{N}} ... e^{- i\frac{H_nt}{N}})^N - e^{-i(H_1 t + ... H_n)} || < \epsilon
$$
So the total circuit size is $O(m^4 \frac{(Kt)^2}{\epsilon})$ we are dealing with Hamiltonians that are $k$-local. So for $n$ qubits and $k$-local terms $m = O(n^k)$. Therefore, the total circuit size is $O( n^{4k} \frac{(Kt)^2}{\epsilon})$. You can actually refine this method to get order $t^{1+ \delta}$ but they are much more technical and would need many more lectures to explain.\\\\
We have a circuit of size $|C| = O( \frac{m^4 (kT)^2}{\epsilon})$ and if we want to use gates from the universal set
\sections{Lecture 12}
\subsection{The Local Hamiltonian Problem and QMA}
Recap: Definition of NP. There is no known way to solve it in polynomial time but it is easy to verify the solution in polynomial time.\\
We will adopt the language of the part II lecture course. A language is in NP if it has an efficient (poly-time) verifier $V$. A verifier $V$ for a language $L$ is a computation with two inputs $w$, $c$ s.t. if $w \in L$ then for some $c$ $V(w,c)$ halts with "accept". such $c$ is called a certificate/proof (of membership) for $w$. If $w \cancel{\in} L$, then for all $C$, $V(w,c)$ halts outputting "reject". V is a poly-time verifier for all pairs $(w,c)$. V runs in poly(n) time $n = |w|$.
\subsubsection{The satisfiability problem (SAT)}
Boolean formulas $\phi(x_1...x_n): \{ 0,1\}^n \rightarrow \{0,1\}$ and every $(b_1...b_n), b_i \in \{0,1\}$ s.t. $\phi(b_1...b_n) = 1$ is called a satisfying assignment. $V(\phi,c)$ evaluates $\phi(c)$. Finding a satisfying assignment is not known to be in $P$, and is $NP$-complete by Cook-Levin theorem.\\\\
Theory of NP-completeness shows that many different problems that look very different (SAT, Travelling Salesman Problem,  integer linear programming) are essentially the same problem. We can translate instances of one to another in deterministic poly time.\\\\
Now lets try to relax some of the requirements on $NP$. Consider a setting where the prover and verifier may use randomness and sometimes make errors ( allowing for some probability of error say $\frac{1}{3}$). In other words when $w \in L$ a prover shoul dbe able to prepare a certificate/proof s.t. Prob$(V(w,c)$ accepts) $\geq \frac{2}{3}$ and Prob$(V(w,c)$ rejects) $\leq \frac{1}{3}$. This defines complexity class $MA$ (Merlin-Arthur). Merlin is regarded as the omniscient prover and Arthur is a randomised poly time verifier. \\\\
\textbf{Quantum Merlin Arthur class}\\
The direct analgoue of $NP$.\\
It is a class of promise problems. A promise problem $L$ partitions $\{0,1\}^k$ of all binary strings into $L_0, L_1, L_*$.\\
An algorithm is promised that it never recieves inputs from $L_*$. However, if the input is from $L_0/L_1$ then it has to determine $0/1$.\\
\textbf{QMA}: A promise problem $L = (L_0,L_1,L_*)$ is in the class $QMA$ if there exists a uniform family of circuits $\{C_n\}$ with two input registers and one output qubit and a polynomial $p()$ s.t. $\forall w \in \{0,1\}^k$\\
Completeness: If $w \in L_1 \cap \{0,1\}^n$ then there exists a $p(n)$- qubit state $\ket{\psi}$ (a proof/witness state) such that $C_n$ outputs 1 with probability $\geq \frac{2}{3}$ when run on $w$ and $\ket{\psi}$\\
Soundness: If $w \in L_0 \cap \{0,1\}^n$ then for every $p(n)$ qubit state $\ket{\psi}$ the circuit $C_n$ outputs 1 with probability $\leq \frac{1}{3}$ when run on $w$ and $\ket{\psi}$\\
\textbf{Remarks}:\\
1) If we replace $\ket{\psi}$ with a classical bitstring we get the class QCMA\\
2) If we additionally to (1) force the verifier to be classical MA\\
3) If we additionally to (1),(2) repalce success probability ($\frac{2}{3}$) by 1 -> NP
$$
NP \leq MA \leq QCMA \leq QMA
$$
SAT is $NP$-complete (Cook-Levin theorem). Consider a special case: $k-SAT$\\
$\phi$ is the conjunction of clauses each of which is a disjunction of $k$-literals. e.g. for $k=3$:
$$
( x_1 \cup \bar x_2 \cup x_3) \cap ( \bar x_1 \cup x_2 \cup x_4) \cap (x_1 \cup \bar x_4 \cup x_5)
$$
k-SAT is NP-complete for $x \geq 3$ and k-SAT is in $P$ when $k=2$.\\\\
We will reformulate $k-SAT$ and relate it to a minimal eigenvalue of a certain Hamiltonian which is diagonal in computational basis.\\
Fix $k=3$. Consider a clause $C = x_1 \cup \bar x_2 \cup x_3$ (has one non-satisfying assignment 010).\\\\
               Lets associate a particular diagonal Hamiltonian with 0 everywhere but a 1 at location indexed by the bit string above.\\
               $H_c$ gives "penalty" of 1 to the bitstring of $x_1...x_n$ if $x$ does not satisfy clause $C$.\\
               We will regard $H_c$ as part of $n$-qubit Hamiltonian as $H_c = H_c \otimes I$.\\
               When we evaluate this term: $\bra{x} H \ket{x} = 0$ if clause $C$ is satisfied or $1$ otherwise.\\\\
               Suppose we have a 3-SAT formula $\phi = C_1 \cap ... \cap C_n \implies H_{\phi} = \sum_{j=1}^m H_{c_j}$. The eigenvalues of $H_{\phi}$ lie in the interval $[0,m]$ and $H_{\phi}$ is 3-local.\\
                       Each assignment $x \in \{0,1\}^n$ generates the "energy"\\
                       $\bra{x} H_{\phi} \ket{x} = \sum_{j=1}^n \bra{x} H_{C_j} \ket{x}$ this counts the number of unsatisfied clauses under $x$.\\\\
                       We want to find the minimal energy of this Hamiltonian which will correspond to the lowest eigenvalue $\lambda_n$ ( the smallest number of unsatisfied clauses). $\phi$ is satisfyable only iff $\lambda_{min} = 0$.
                       \section{Lecture 13}
                       \url{$http://theory.caltech.edu/~preskill/ph219/20_local_Hamiltonian_problem.pdf$}\\
                       \textbf{The k-local Hamiltonian problem} (LH): Given a classical description of a Hamiltonian of an $n$ qubit Hamiltonian:
                       $$
                       H = \sum_{j=1}^m H_j
                       $$
                       where each $H_j$ is $k$-local and is positive semi-definite. Also given two parameters $a,b \in [0,m]$ with $b-a \geq \frac{1}{poly(n)}$ with a promise that $\lambda_{min}$- minimal eigenvalue of $H$ is either $\leq a$ or $\geq b$. Decide which is the case.
                       $\lambda_{min}$ is often called the ground state energy. Determining $\lambda_{min}$ ($\leq a$ or $\geq b$) is tantamount to approximating the ground state energy $\lambda_{min}$ up to an additive error $O(b-a)$. Further studied on ES.\\\\
                       We would like to prove that $LH$ is $QMA$-complete\\\\
                       We first need to obtain that $LH$ is in $QMA$: Given the corresponding witness states $\ket{\psi}$ for $x \in L_1$, we can efficiently approximate the energy to check if it is $\leq a$ or $\geq b$.\\
                       Note: LH is a promise problem, which means that $H$ with $\lambda \in (a,b)$ won't satsify the condition.\\
                       Next step, is we want to show that LH is QMA-hard (any other problem in QMA can be reduced to it)\\\\
                       Take any $L= (L_1, L_0, L_*)$ in QMA and fix $x \in L_1 \cup L_0$ which will be our $n$ bit string.\\
                       Plan: Consider a circuit $C_n$ and convert it to Hamiltonian $H$ s.t. $\lambda_{min}$ is small iff $C_n$ has high acceptance probability on some state $\ket{\psi}$. We have $C_n = U_T...U_1$, each $U_i$ is a 1 or 2 qubit gate. We'll take error probability to be $\frac{1}{4T}$. \\
                       The circuit $C_n$ take $n+s+ p(n)$ qubits as inputs. $n$ qubits encode the classical input $x$. $s$ qubits is the workspace of clean qubits in state $0$. $p(n)$ is a witness state $\ket{\psi}$.\\\\
                       Given $\ket{\psi}$, define $\ket{\psi_0} = \ket{0}^{\otimes S}\ket{\psi}$, $\ket{\psi_t} = U_t \ket{\psi_{t-1}}$ for $i = 1...T$. \\\\
                       We will define the following Hamiltonian which has three parts. $$H_{init} = \sum_{i=1}^s \ket{1} \bra{1}_i \otimes \ket{0} \bra{0}_C$$
                       with $i = 1, ..., s+p(n)$
                       $$
                       H_t = \frac{1}{2} [ I \otimes ( \ket{t-1} \bra{t-1}_C + \ket{t}\bra{t}_C) - U_t \otimes \ket{t} \bra{t-1}_C - U_t^{\dagger} \otimes \ket{t-1} \bra{t}_C]
                       $$
                       $$
                       H_{final} = \ket{0} \bra{0} \otimes \ket{T} \bra{T}_C
                       $$
                       $$H= H_{init} + \sum_{i=1}^T H_i + H_{final}$$
                       We should note that $H$ "follows" the state $\ket{\psi}$ and prescribes penalties when deviating form $\ket{\psi_0}, \ket{\psi_1},..., \ket{\psi_T}$\\
                       $H$ also penalizes the zero output in the final measurement\\
                       $C$-is an extra "clock" register of $\log (T+1)$ qubits\\
                       Check locality of $H$ is $k= \log (T+1) + 2 = O(\log (n))$. We will further reduce $k$ to a constant $k = 5$.\\
                       We need to check that we can distinguish $x\in L_1$ and $x \in L_0$ by looking at $\lambda_{min}$. \\\\
                       Consider first case when $x \in L_1$ this means that there exists a $p(n)$ qubit witness state $\ket{\psi}$ that makes $C_n$ accept with probability $1- \frac{1}{4T}$. Lets consider another state $\ket{\psi'} = \frac{1}{\sqrt{T+1}} \sum_{t = 0}^T \ket{\psi_t} \ket{t}$ this is a "History state". When we look at this state we want to evaluate the energy.
                       State $\ket{\psi'}$ gets no contribution to penalty from $H_{int}$ and $H_t$ terms. Since the probability of measuring the (incorrect) outcome is $\frac{1}{4T}$ we have $\lambda_{min} = \bra{\psi'} H \ket{\psi'} = \frac{1}{T+1} \bra{\psi_T} \bra{T} H_{final} \ket{T} \ket{\psi_T} \leq \frac{1}{T+1} \frac{1}{4T} = a$.\\
                       This will show us completeness.\\\\
                       Now take $x \in L_0$. We want to show that $\lambda_{min}$ is at least $b=2a$. Consider any witness state $\ket{\psi'}= \sum_{t=0}^T \alpha_t \ket{\phi_t} \ket{t}$ with $\alpha_t \geq 0$ and $\{ \ket{\phi_t}\}$ is normalised. Then evaluate $$\bra{\psi'} H \ket{\psi'} = \frac{1}{2} ( \alpha^2_{t-1} + \alpha_t^2 - \alpha_{t-1} \alpha_t \bra{\phi_t} U_t \ket{\phi_{t-1}} - \alpha_t \alpha_{t-1} \bra{\phi_{t-1}} H \ket{\phi_t})$$
                       $$
                       \bra{\psi'} H \ket{\psi'} = \frac{1}{2} || \alpha_t \ket{\phi_t} - \alpha_{t-1} U_t \ket{\phi_{t-1}} ||^2
                       $$
                       We will be making several simplifying assumptions:\\
                       1) All $\alpha_t$s are $\frac{1}{\sqrt{T-1}}$\\
                       2) $\ket{\phi} = \ket{0}^{\otimes s} \ket{\psi}$ for some $p(n)$ qubit state $\ket{\psi}$\\
                       3) $\ket{\phi_T}$ has accept probability close to 1.\\\\
                       Using these and the fact that $x \in L_0$, $C_n \ket{\phi_0}$ must have accept probability near 0.\\
                       From 3 we have $\ket{\phi_T}$ and $C_n \ket{\phi_0}$ must be nearly orthogonal:
                       $$
                       1 \leq || \ket{\phi_T} - C_n \ket{\phi_0} || = || sum_{t=1}^T U_T - U_{t-1} \ket{\phi_t} - U_T...U_t\ket{\phi_{t-1}}|| \leq \sum_{t=1}^T || U_T...U_t \ket{\phi_t} - U_T...U_t \ket{\phi_{t-1}}|| = \sum_{t=1}^T || \ket{\phi_t} - U_t \ket{\phi_{t-1}}||
                       $$
                       Using the evaluation of $\bra{\psi'} H \ket{\psi}$ assumptions and setting $\alpha_t = \alpha_{t-1} = \frac{1}{\sqrt{T+1}}$:
                               $$
                               \bra{\psi'} H \ket{\psi'} = \sum_{t=1}^T \bra{\psi'} H_t \ket{\psi'} = \frac{1}{2} \frac{1}{T+1} \sum_{t=1}^T ||\ket{\phi_t} - U_t \ket{\phi_{t-1}} ||^2 \geq \frac{1}{2T(T+1)} ( \sum_{t=1}^T || \ket{\phi_t} - U_t \ket{\phi_{t-1}}||^2 ) \geq \frac{1}{2T(T+1)} = b
                               $$
                               \section{Example Sheet 2}
For Grover's search algorithm, if the indicator function is classically efficiently computation then $I_G$ is implementable.\\\\
Look at Part II course to recap coprimality conditions. They also show that to factor $N$ with constant probability it suffices to find an approximations $\xi$ to $\frac{s}{r}$ for random $s$ in range $[1,r)$ to $2m+1$ binary digits of accuracy with $m= O(\log N)$ s.t. $| \xi - \frac{s}{r} | < \frac{1}{2N^2}$\\\\
\section{Lecture 14}
The Hamiltonian is not $k$-local for $k$ constant it is of $\log N$ we will fix this today.
\subsubsection{Making Hamiltonian $k$-local}
What is responsible for this non-locality effect is the clock-register. So we can reduce locality from $O(\log n)$ to 5 by making the clock unary. For $t=0$ we will write $\ket{0}^{\otimes T}$ for $t=1$ we write $\ket{1} \ket{0}^{\otimes T-1}$ for $t=3$ we write $\ket{11} \ket{0}^{\otimes T}$. When we had a binary format we used $\log(T)$ qubits in the unary format we will use $T$ qubits. This construction will require an extra bit that we will see shortly. Let's write $C_t$ to denote the $t$-th qubit of the clock register. Now our Hamiltonian $H$ has the following terms:
$$
H_{init} = \sum_{i=1}^s \ket{1} \bra{1}_i \otimes \ket{0} \bra{ 0}_{C_i}
$$
$$
H_t = \frac{1}{2} ( I \otimes \ket{100} \bra{100}_{C_{t-1} C_t C_{t+1}} + I \otimes \ket{110} \bra{110}_{C_{t-1} C_t C_{t+1}} - U_t \otimes \ket{110} \bra{100}_{C_{t-1} C_t C_{t+1}} _ U^{\dagger}_t \otimes \ket{100} \bra{110}_{C_{t-1} C_t C_{t+1}})
$$
$$
H_{final} = \ket{0} \bra{0}_T \otimes \ket{1} \bra{1}_{C_T}
$$
Need an extra term to penalise clock registers that do not conform to the unary format:
$$
H_{clock} = \sum_{t=1}^{T-1} \ket{01} \bra{01} _{C_t C_{t+1}}
$$
$$
H = H_{init} + \sum_{t=1}^T H_t + H_{final} + H_{clock}
$$
So each term accesses at most 5 qubits, so this is a 5 local Hamiltonian. What is the smallest $k$ and still retain $QMA$ completeness. $k$-local Hamiltonian is $QMA$-complete for $k=2$.
\subsubsection{QMA-complete problems}
Fairly exhaustive list in a paper by A Bookatz arxiv 1292.6312. We won't really look through many of them but we will think about a few interesting problems.\\\\
\textbf{Non-identity check}: Given a poly(n)-sized quantum circuit $U$ on n qubits. We want to determine whether $U$ is not close to the identity up to some global phase:\\
1) accept if $\forall \phi \in [0,2\pi]$ $||U - e^{i \phi} I|| \geq b$ or\\
2) reject if there exists $\phi in [0, 2\pi]$ s.t. $|| U - e^{i \phi} I|| \leq a$.\\
This comes with a promise that one of the above options holds and $b- a \geq \frac{1}{poly(n)}$\\\\
\textbf{Excited k-local Hamiltonian}: For some fixed constant $c$ and given a $k$-local Hamiltonian $H$ on n qubits. We want to determine whether:\\
1) ACCEPT: The $c$-th eigenvalue of $H$ is $\leq a$\\
2) REJECT: the $c$-th eigenvalue of $H$ is $\geq b$\\
Have promise that one of these cases hold and $b-a \geq \frac{1}{\poly(n)}$.\\\\
Many Hamiltonians that correspond to real physical systems are QMA-complete so studying the behaviour of these systems is also hard on a quantum computer. An example of this is a 2-local Ising model $H_{zzxx} = \sum_{i=1}^n h_i Z_i + \sum_{i=1}^n \Delta_i X_i + \sum_{i,j=1}^n J_{ij} Z_i Z_j  + \sum_{i,j =1}^n K_{ij} X_i X_j$ for real $h_i, \Delta_i, J, K$. Also true for 2D Heisenberg model or the Rose-Hubbard model.\\\\
\textbf{Group non-membership: Problem that is in QMA, but not known to be QMA-complete}:\\
Given a finite group $G$, subgroup $H < G$, an element $g \in G$ and determine whether:
1) $g \cancel \in H$\\
2) $g \in H$\\
\subsection{Harrow-Hassidian-Lloyd (HHL)}
Quantum algorithm for solving linear systems of equations. We want to solve the linear system of equations: $A\bm x = \bm b$, $\bm b, \bm x \in \mathbb{C}^N$ where dimensions $N$ is potentially very large $N=2^n$ or let $2^n$ be the least power of 2 that is greater than $N$. Rather than outputting the full solution $\bm x$ itself which would take at least $O(N)$ time we want to compute suitable approximations to the value of properties of the solution $x$ such as quadratic expression $\bm x^TM \bm x$ (e.g. the total weight of some subset of components).\\\\
Very large systems becoming increasingly important in applications:\\
- data mining/machine learning on datasets that are very large (petabytes...) to discover pattern properties within the data\\
- numerical solutions to PDEs. Use discretisation techniques (finite element methods). These techniques lead to systems that have the size for larger than the original problem description. \\\\
The best known classical techniques take poly(N) times to solve such problems - in general even to compute the properties of the solutions, no better method is known than computing solutions itself. \\\\
Important parameters/for both classical and quantum algorithms\\
- the system size $N$\\
- the desire approximation tolerance $\epsilon$\\
- the condition number $\kappa$ of the matrix $A$ is defined as $\kappa = | \frac{\lamdba_{max}}{\lambda_{min}}|$ (this tells us how close $A$ is to being non-invertible)\\\\
If we renormalise $A$ to have $\lambda_{max} =1$ then $\lamdbda_{min} = \frac{1}{\kappa}$. This is important as the numerical computation of $A^{-1}$ becomes less stable with increasing $\kappa$. This means we need more significant digits of computation.
\section{Lecture 15}
This algorithm has been improved a lot of times, including as recently as last year. So the exponential improvement in error scaling from $\frac{1}{\epsilon}$ to $\log \frac{1}{\epsilon}$ was shown in Childs, Kothan, Smomann in 2015.\\\\
We have an $N$-dimensional space of $n = \log N$ qubits. Let $\{\ket{i} : i = 0,..., N-1\}$ be the computational basis. Let eigenvectors and corresponding eigenvalues of $A$ be: $\ket{u_j} $ and $\lambda_j$ for $j = 0,...,N-1$. We being by implementing RHS $\ket{b}$ and consider it in the eigenbasis of $A$, $\ket{b} = \sum_{i=0}^{N-1} b_i \ket{i} = \sum_{i=0}^{N-1} \beta_j \ket{u_j}$. Then $\ket{x} = A^{-1} \ket{b} = \sum_{j=0}^{N-1} \beta_j \frac{1}{\lamdba_j} \ket{u_i}$. THis transformation of $\ket{b}$ is linear  but not unitary so it cannot be implemented as a quantum operation. To achieve it probabilistic we will use the phase estimation algorithm for $U= e^{iA}$ with the exponential (and all its powers need for PE) implemented by Hamiltonian simulation. Then a post-selected measurement process will achieve the desired unitary transformation.\begin{itemlist}\item  PE with Hamiltonian simulation (both assumed to work perfectly) give $$\ket{b} \ket{0} \rightarrow \sum_j \beta_j \ket{u_j} \ket{\lambda_j}$$\\
                \item Next we will adjoin an extra ancilla qubit $\ket{0}$ and apply the controlled rotation ( controlled by $\ket{\lamdba_j}$ register)\\
                \item C-rotation does 
                        $$
                        \ket{\lambda_j} \ket{0} \rightarrow \sqrt{1 - \frac{c^2}{\lambda^2_j}} \ket{\lambda_j} \ket{0} + \frac{c}{\lambda_j} \ket{\lambda_j} \ket{1} = \ket{\lamdba_j}( \cos \theta_j \ket{0} + \sin \theta_j \ket{1})
                        $$
                        here $c$ is chosen to have $c \leq \min_j |\lambda_j|$  ($c = \frac{1}{k}$ for definiteness).\\\\ $\theta_i \in (- \frac{\pi}{2}, \frac{\pi}{2})$ and $\theta_i = \arcsin \frac{c}{\lambda_i}$ which is determined by the content of the $\ket{\lambda_j}$ register\\\\Controlled rotation is a fixed operation on $n+1$ qubits (independent of $A$ or $b$) that can be implemented by a poly(n) circuit of 1 and 2 qubit gates.\\
                \item Apply controlled rotation:
                        $$
                        \sum_{j=0}^{N-1} \beta_j \sqrt{1- \frac{c^2}{\lambda_j^2}} \ket{u_j} \ket{u_j} \ket{0} + \beta_j \frac{c}{\lambda_j} \ket{u_j} \ket{\lambda_j}\ket{1}
                        $$
                        We want the part associated with ancilla in state $\ket{1}$. This step is the post-selection step.\\
                \item We measure the ancilla (hoping to get the result 1) with probability $$p = || \sum_j \beta_j \frac{c}{\lambda_j} \ket{u_j} \ket{ \lamdba_j} ||^2 = \sum_j |\beta_j \frac{c}{\lamdba_j}|^2 = \frac{1}{k^2} \sum_j |\frac{\beta_j}{\lambda_j}|^2 \geq \frac{1}{k^2}$$ as $\sum_j |\beta_j|^2 =1$ and $\frac{1}{\lambda_j} \geq 1$. In this case the post-measurment state is:
                        $$
                         \ket{\gamma} = \frac{1}{\sqrt{p}} c \sum_{j=1}^{N-1} \frac{\beta_j}{\lamdba_j} \ket{u_j} \ket{\lambda_j}
                        $$
                        To mitigate the probabilistic nature of this process and obtain the state with any fixed probability with $1-\eta$ for $\eta >0$ and $\eta$ small. We can repeat the whole process $\frac{\log \frac{1}{\eta}}{p} = O(k^2)$ times and the outcome 1 will occur at least once with probability $1- \eta$.\\\\
               \textbf{Fact}: If a single trial has success probability $p$ and we repeat the trial $M$ times independently then for any $0 < 1- \epsilon< 1$  Prob( at least on trial in M succeeds) $> 1- \epsilon$ if $M \geq \frac{- \log \epsilon}{p}$.\\\\
        Remark (amplitude amplification): the $O(k^2)$ repetitions needed to generate $\ket{\gamma}$ can be improve to $O(k)$ by using AA instead of repeated measurements.\\
\item Having obtained $\ket{\gamma}$ we run the inverse phase estimation process to "uncompute" or "erase" the $\ket{\lambda_j}$ register resetting it to $\ket{0}$ and get $\ket{\hat x} = \frac{c}{\sqrt{p}} \sum_{j=0}^{N-1} \frac{\beta_j}{\lamdba_j} \ket{u_j} = \frac{c}{\sqrt{p}}\ket{x}$\\
\item Finally, we will perform a measuremnt of observable $M$ on $\ket{\hat x}$ to estimate its mean value $\frac{c^2}{p} \bra{x} M \ket{x}$.\\\\
        According to Chernofl-Hopffling bound (look this up in lecture notes) $O(\frac{\log \frac{1}{\eta}}{\xi^2}$ measurements will suffice to estimate the meant to any desired accuracy $\xi$ with any probability of success $1- \eta$. Similarly we can estiamte the prob $p$ in the post-selection step to any accuarcy $\xi$ by applying C-H bound to the ancilla measurment outcome. As the ancilla measuremnt outcome is a random variable with two outomces 0 and 1, so finding the mean of this variable gives p. Using these values of $\frac{c^2}{p} \bra{x} M \ket{x}$ and $p, c= \frac{1}{k}$ we get out our value $M= \bra{x} M \ket{x}$.
 \end{itemlist}
 \section{Lecture 16}
 Let's stop and think about HHL a little bit more. We have considered the ideal situation but we now want to consider how it is effected by approxiamtion errors
 \subsubsection{Summary analysis of approxiamtion erros and runtime in HHL (non-examinable)}
 A full analysis of runtime and approximation errors can be found in Physical Review Letters vol 103.150502 (2005).\\
 We assume we can make $\ket{b}$ correctly (the reason we don't want to focus on this is the issue of creating $\ket{b}$ is an orthongonal process to HHL and should be considered separately so introduces no errors). Similarly, the controlled rotation C-ROT is a final unitary (it can be efficently implemented using some universal gate set by Solovay-Kitaev theorem)\\
 We will assume that C-rotations can be done exactly\\
 Convsider the phase estimation part. PE for a unitary $U$ and $n$ qubit lines gives an estimate of $\lambda$ up to $n$ bits of precision. Denote this estimate $\lambda'$ and assume that it is the closest n-bit approximation $\lambda'$ is $\lamdba$ up to additive error $\eta = \frac{1}{2^n}$\\
                       The PE process uses Hamalitonian simulation that requires execution of controlled $U$, $U^2$, $U^4$,...,$U^{2^{n-1}}$ If $U = e^{iA}$ then $C-U = e^{i \tilde A}$ with $\tilde A = \begin{pmatrix} 0 &|& 0 \\ -& & - \\ 0 & |& A \end{pmatrix}$. $\tilde A$ remains $s$-sparse and row-computbale if $A$ was. \\\\
                       Thus we need to implement the following $e^{i \tilde A t}$ for $t = 1,2,4,..., 2^{n-1}$ which is done using hamiltonian simualtion for s-spare matrices with circuit size $O(\log N t s^2)$. So the total circuit size $O(\log N t_0 s^2)$ where $t_0 = 1 + 2 + ... + 2^{n-1} \sim 2^n = \frac{1}{\eta}$. \\\\
                       After the controlled rotation $C_{rot}$ using the $\lambda$'s produced and the corresponding measuremtn in post-selction step the final state is:
                       $$
                        \ket{\hat x'} = \frac{1}{D'} \sum_j \frac{c}{\lambda'_j} \ket{u_j}, D' = \sqrt{p'} = \sqrt{ \sum_j |\beta_j|^2 \frac{c^2}{\lamdba_j}}
                       $$
                       $\ket{\hat x}$, $D$, $p$ - will denote the expressions with $\lambda'$s replaced by true $\lambda$ values. Our requirement is that $\ket{\hat x}$ should be wihtin $\epsilon$ of $\ket{\hat x}$ so we need to choose a suitably large $t_0 = \frac{1}{\eta}$ ( number of qubit lines). To establish dependenace of $t_0$ on $\epsilons$ we need the following facts:\\
                       a) If $\lambda$ has additive error $\delta(\lamdba) = \eta$ then $\frac{1}{\lambda}$ has $\delta( \frac{1}{\lambda}) \sim \frac{1}{\lambda} \delta(\lambda)$ so $\frac{1}{\lambda}$ has relative error $\frac{\delta( \frac{1}{\lambda})}{\frac{1}{\lamdba}} = \frac{\eta}{\lamdba}$\\
                       b) If $A'$ and $B'$ approximate $A$ and $B$ to relative error $\xi$ then $\frac{A'}{B'} $ approximates $\frac{A}{B}$ to relative error $\xi$ too $\frac{A'}{B'} = \frac{A( 1+ O(\epsilon))}{B( 1+ O(\epsilon))} = \frac{A}{B}( 1+ O(\epsilon))$\\\\
                       By $(a)$ $\frac{1}{\lambda'_j}$ approximates $\frac{1}{\lambda_j}$ to relative error $O(\frac{q}{\lambda_j})= O(Kq)$. Similarly $D'$ approximates $D$ to relative error $O(Kq)$. Then by $(b)$ $A' = \frac{\beta_j c}{\lambda'_j}$, $\beta' = D'$ the ampltitudes of $\ket{\hat x'}$ approximate those of $\ket{\hat x}$ up to multiplicative error $O(Kq)$.
                       $$
                        || \ket{\hat x'} - \ket{\hat x}|| = || ( 1+ O(k\eta)\ket{\hat x} - \ket{\hat x}|| = O(kq) = \epsilon
                       $$
                        for a given $\epsilon$ we choose $kq = \epsilon, t_0 = \frac{1}{\eta}$ we gget $t_0 = \frac{k}{\epsilon}$ and the PE step (run once) has runtime $O(\log N t_0 s^2) = O(\log N \frac{k}{q} s^2)$ for $k = poly \log N$ so runtime is $O(poly \log N) \frac{1}{\epsilon})$.\\\\
                        \textbf{Further comments on erros and runtime}:\\
                        In the above analysis we assumed that the PE process (with $n$ qubits outputs the closest n-bit approximation $\lambda'$ to $\lambda$ whereas it outputs a superposition of all $\lambda_k$ = \lambda' \pm \frac{k}{2^n}$ (peaked around $k=0$) which is another source of error. TO mitigate this we use PE(b), use more qubit lines to get clsoer approxiamtion with anu high probability. \\\\
                        In the HHL paper they use a slighly improved PE which uses non-uniform superposition state at the start. It may be shown that an optimal choice of this non-uniform superposition can lead to a quadratic reduciton in the variance of the $k$-value outcomes. In other works $\delta(\lambda)$ is more concentrated on smaller vlaues.\\\\
                        In the HHL paper, the authors present an algorithm that can handle all conditioned matrices A by introducing a further qubit "flag" and filtering functions to separate well and ill conditioned parts of $A$ and processing the former. \\\\
                        In the HHL algorithm the PE step is the soruce of the worst accuracy depdnance $O(\frac{1}{\epsilon}$. This was ubsequently improved  by Berry et all arXiv 1212.1414 from $O(\frac{1}{\epsilon} \rightarrow O(\log( \frac{1}{\epsilon})$ this technique invovles rpelacing PE process by an entirely different method for implementing $A^{-1}$ . This new method is called "linear combinations of unitaries" (LCU). Implementing $A^{-1}$ as a linear combination of unitaries each of which is individually implmented. Then we represent the linear sum in terms of another unitary $V$ on a larger space with extra ancilla qubits.
                        \section{Lecture 17}
                        \subsection{Clifford computations (and classical simualtion of QC)}
                        Vague question, what is the key "quantum effect/resource/ingredient" giving quantum computing its benefits or advantage over classical computation.
                        \subsubsection{Classical simulation of quantum computation}
                        We are given classical description of the following:\\
                        1) description of a quantum circuit $C$ which is just a collection of 1 or 2 qubit gates on $n$ qubits with size $N =$ poly(n).\\
                        2) description of the input (e.g. $\ket{l_1...l_n}, l_k \in \{0,1\}$ or $\ket{\alpha_1} ... \ket{\alpha_n}$ a general product state)\\
                        3) cominated output lliines ( e.g. 1st line to read off the answer)\\\\
                        We want to by classical means only:\\
                        \textbf{Weak} simulation: sample the output distribution (using classical computer with access to calssical randomness)\\
                        \textbf{Strong} simulation: calculate the output probabilities of $p_0$ and $p_1$ (probability of getting 0 or 1 on the first line).\\\\
                        We want to do it efficently in classical poly(n) time known as "efficent classical simulation".\\
                        Note, the quantum process itself gives only a weak simulation "self-simulation" as it only gives a sample of the output distribution.\\
                        "Direct" strong classical simulation is always possible but not efficent in general. \\\\However, if we are promised that our state is a product state at every stage then we can give an efficent simulation. Each step we only involve 1 or 2 qubits in known state $\ket{\alpha_i} \ket{\alpha_j}$. Consider the 2-qubit case: $U$ acting on 2 qubits with $\ket{\psi} = \sum_{i_1..i_n} C_{i_1...i_n} \ket{l_1...l_n}$ then in general you get $U \ket{\psi} = \sum_{i_1}^n V^{k_1 k_2}_{l_1 l_2} C_{k_1 k_2 l_3...l_n} \ket{ \beta_{k_1} \beta_{k_2} l'_3... l'_n}$ (let $\tilde C_{i_1... i_n} =  V^{k_1 k_2}_{l_1 l_2} C_{k_1 k_2 l_3...l_n} $ . When $\ket{\psi}$ is a product state we have $C_{i_1 ... i_n} = a_{i_1} b_{i_2} c_{i_3} ... c_{i_n}$ so $\tilde C_{i_1 .. i_n} = ( \sum U_{i_1 i_2}^{k_1 k_2} a_{k_1} b_{k_2} ) c_{i_3} ... c_{i_n}$. so we can factorise back into the form $f_1... f_n$. Sometimes it was claimed that entanglement was the secret ingredient as it looks like if you take away entanglement then you can classically simulate the system. However, this argument shows that if we have no entanglement then we get no quantum advantage. This shows that entangelment is necessary but not sufficent for quantum advantage. "Most entangled states are too entnagled to be useful as quantum computatoinal resources" Fross, Flanimma, Eisert Phys Rev Letter 107 1905 01 '99. Universal quantum compuation with little entanglement Physical Rev Letter  110 060504 '13. So the story with entanglement is complicated. so we are going to consider something else called Clifford computations.
                                        \subsubsection{Clifford Compuation}
                                        Prelminary defintions:
                                        $n$-qubit Pauli group $\mathcal{P}_n = \{ \pm 1, \pm i P_1 \otimes... \otimes P_n\}$ with each $P_i \in \{ I, X,Y,Z\}$.\\
                                        Define a Clifford operation $C$ on $n$ qubits as such that conjugates Paulis to paulis. $\forall P \in \mathcal{P}_n C P C^{\dagger} \in \mathcal{P_n}$. So the Clifford group is the normalizer of subgroup $\mathcal{P}_n$ in $U(2^n)$.\\
                                        Clifford circuits found many applications:\\
                                        1) the theory of quantum error correction ('stabilizer codes')\\
                                        2) insights into quantum compuational power\\
                                        3) verification of quantum computations\\\\
                                        \textbf{Examples}:
                                        All Paulis are clifford\\
                       $H$, $S = \begin{pmatrix} 1 & 0 \\0 & i \end{pmatrix} $ phase ($\frac{\pi}{2}$) gate\\
                       $CX$ 2 qubit gate\\
                       $Z= S^2$ and $X = HZH$\\
                       $SWAP_{12} = CX_{12} CX_{21} CX_{12}$\\
                       Full explict characterisation of Clifford gates:\\
                       \textbf{Theorem}: $C$ on $n$ qubits is Clifford iff $C$ is a circuit of $H$, $S$, $CX$ gates\\
                       Clifford computation circuits that only involve clifford gates.\\
                       Note: Generating entanglement $\ket{0} \ket{0} \rightarrow^{H} \frac{1}{\sqrt{2}} ( \ket{0} + \ket{1}) \ket{0} \rightarrow^{CX_{12}} \frac{1}{\sqrt{2}} ( \ket{0} \ket{0} + \ket{1} \ket{1})$\\
               \textbf{Grottesman-Knull Theorem}: Let $C$ be any Clifford circuit on $n$ qubits, $N = poly(n)$ with input any product state $\ket{a_1} ... \ket{a_n}$ and output a measurment on the 1st line. Then the output can always be classically strongly efficently simulated.
               \section{Lecture 17}
               \textbf{Proof}: The idea is that instead of evolving the input state $\ket{\psi_{in}}$ to a final state $\ket{\psi_{final}}$ for measurement we will back propagate the final $Z$ measurement. \\\\
               Let's write the measurement $Z_1 = Z \otimes B \otimes ... \otimes B$ and recall that $Z = \ket{0} \bra{0} - \ket{1} \bra{1} = \Pi_0 - \Pi_1$. So we can write the difference of the probabilities as $p_0 - p_1 = \bra{\psi_{final}} Z_1 \ket{\psi_{final}} = \bra{\psi_{in}} C^{\dagger} Z_1 C \ket{\psi_{in}}$. Let $C = C_NC_{N-1} ... C_1$ with $C_i \in \{ H, S, CX\}$ so :
               $$
               p_0 - p_1 = \bra{\psi_{in}} C_1^{\dagger} ... C_{N}^{\dagger} Z_1 C_N ... C_1 \ket{\psi_{in}}
               $$
               Each conjugation gives us a Pauli, and thus we get:
               $$
               p_0 - p_1 = \bra{\psi_{in}} \tilde P_n \otimes ... \otimes \tilde P_n \ket{\psi_{in}}
               $$
               All update rules for Pauli products by H, S, CX conjugations are easy to compute ( we have only 1 and 2 gate computations so it is a constant size computation)\\\\
               Finally we make use of the fact that our input state is a product state. Therefore:
               $$
                p_0 - p_1 = \prod_{i=1}^n \bra{a_i} \tilde P_i \ket{a_i}
               $$
               Each of these terms is a product of $n$ $2\times2$ matrix calculations so is efficently computable. We just need to remember that they are normalised so $p_0 + p_1 =1$ so we get $p_0, p_1$ values via efficent classical computation. So Cliffords are easy.\\\\
               Let us extend unitary Clifford circuits to allow the inclusion of intermediate measurements (1-qubit) measurements\\\\
               Distinguish two scenarios, non-adaptive or adaptive (if we change future gates or measurements depending on the outcome of measurements). Non-adaptive: Gates cannot depend on earlier measurement outcomes. Adaptive: Gates can depend on earlier measurement outcomes.\\\\
               \textbf{Theorem} Let $C$ be any Clifford circuit with intermediate measurements with an arbitary product state input and a single line output. Then:\\
               a) If $C$ is non-adaptive then the output is classically efficiently strongly simulatable\\
               b) If $C$ is adaptive then full universal quantum computation is possible\\\\
                       Note: if input is restricted to the compuatational basis states then $(b)$ remains classically weakly efficently simulatable (so b) relies on the arbitary product state input for its universality). R. Sazsen & M van den Nest arXic 1305.6190 (2013).\\\\
                       Proof: for a) we observe that non-adaptive Clifford circuits are reducible to the full unitary case, and then use GK. To see how this reduction works consider a measurement between $C_1$ and $C_2$ on one of the qubit lines with $C_2$ independant of measurement outcome. This is equivalent to introducing an extra qubit ancilla which the measurement is swap for a CX on the ancilla, as then measuring the ancilla at any point would give the original circuit but you don't even need to measure it you can just discard it.\\\\
                       for b) we will use a Bravyi-Kitaev idea of so called magic states. Fact: Clifford gates with 1-qubit (non-Clifford) T gate with 
                       $$
                        T= \begin{pmatrix} 1 & 0 \\ 0 & e^{- \frac{\pi}{4}} \end{pmatrix} = \sqrt{s} = \text{phase } \frac{\pi}{4} \text{gates}
                       $$
                       is universal. In fact if you add anything to the Clifford group that is non clifford you regain universal quantum computation. Let $\ket{A}$ denote 1-qubit state $\ket{A} = \frac{1}{\sqrt{2}} (\ket{0} + e^{i \frac{\pi}{4}} \ket{1})$. \\We will implement a $T$ gate using $\ket{A}$ ancilla and an adaptive Clifford circuit (T-gadget).\\\\
                       If we state with any n-qubit state plus the ancilla $\ket{A}$. If we apply a $CX$ from any line to the ancilla and then measure and discard the ancilla. If we get 1 in this measurement we apply an $S$ gate to the line we applied the $CX$ from. This generates a $T$ gate ( up to overall phase). Easy to check (ES3) this gives an adaptive Clifford circuit that always implements T gate (up to overall phase that depends on $m$) on line $k$ and probabilities are $\frac{1}{2}, \frac{1}{2}$ regardless of output $\ket{\psi}$\\\\
                       Remark: To give increase in compuational power we could:\\
                       1) allow inclusion of new gates (T-gate)\\
                       2) add extra "magic" states as extra inputs\\\\
                       For Clifford circuits with product state inputs, single line outputs and intermediate measurements\\
                       a) non-adaptive -> classicaly strongly efficently simulatable\\
                       b) adaptive -> full universal quantum computing power\\\\
                       a) $C_0 M(i_1, y_1) C_1 M( i_2, y_2) C_2$...\\
                       b) $C_0 M(i_1, y_1) C_1(y_1) M(i_2, y_2) C_2(y_1, y_2)$ ...
                       \section{Lecture 17}
                       \subsection{Quantum Error Correction/Stabilizer formalism}
                       Classical computers are incredibly reliable about $10^{-9}$ accuracy. This is not the case for quantum hardware, the gate fidelities are about 99.9\% nowhere near classical computers. So if you ran a thousand gates in a row then as the errors are multiplicative the final state gets very perturbed, so we need to think about error correction. Our go to place is classical intituion, how is this done classically?\\\\
                       Consider a single classical bit $x$ that we want to store. Let's assume that it gets flipped with probability $p$. One way to improve the probability of reading the correct value is instead of storing $x$ we store $xxx$. Readout $y = y_1 y_2 y_3$. The wrong answer will be returned wih probability $3p^2(1-p) + p^3 = p^2(3-2p) \sim O(p^2)$ so for $p \in (0,\frac{1}{2})$ $p^2(3-2p) < p$. So this map $x \rightarrow x_1 x_2 x_3$ is an error correcting code.\\\\
                       Now let's talk about quantum errors. Not talking about a classical value but rather a state. We want to preserve a qubit $\ket{\psi}$. So the process of error correction can be described as:
                       $$
               \ket{\psi'} = DNE\ket{\psi} \ket{0}^{\otimes n} 
                       $$
                       with $E$ is the unitary encoding, $N$ is the noise operation, and $D$ is decoding operation. The aim is to obtain $\ket{\psi'} \approx \ket{\psi}$ for a given set of noise operations.\\\\
                       \textbf{Attempt #1}: Measure $\ket{\psi}$ in the computational basis to get 0 or 1. Then encode using classical repetition code. This is not a good ida as we destroy superposition/entanglement\\
                       \textbf{Attempt #2}: Engineer a map $\ket{\psi} \rightarrow \ket{\psi} \ket{\psi} \ket{\psi}$ this violates the no-cloning so isn't possible\\\\
                       Different approach (that works!). This is a 2-step process:\\
                       1. Encode $\ket{\psi} = \alpha \ket{0} + \beta \ket{1}$ as $\ket{E(\psi)} = \alpha \ket{000} + \beta \ket{111}$. This is NOT the same as cloning. It uses maps:\\ Encoding: $$
                      \Qcircuit @C=1em @R=.7em {
                        \lstick{\alpha \ket{0} + \beta \ket{1}} & \ctrl{1} & \ctrl{2}\\
\lstick{\ket{0}} & \targ & \qw \\
\lstick{\ket{0}} & \qw & \targ} $$           
Decoding:
$$
                      \Qcircuit @C=1em @R=.7em {
\lstick{\ket{x_1}} & \ctrl{3} & \qw & \ctrl{4} & \qw\\
\lstick{\ket{x_2}} & \qw & \ctrl{2} & \qw & \qw \\
\lstick{\ket{x_3}} & \qw & \qw & \qw & \ctrl{2}\\
\lstick{\ket{0}} & \targ & \targ & \qw & \qw \\
\lstick{\ket{0}} & \qw & \qw & \targ & \targ
}
$$
For any input superposition of the form $\alpha \ket{x_1 x_2 x_3} + \beta \ket{x_1 x_2 x_3 \oplus 111}$ the decoding circuit performs the map $( \alpha \ket{x_1 x_2 x_3} + \beta \ket{x_1 x_2 x_3 \oplus 111}) \ket{0} \ket{0} \rightarrow ( \alpha \ket{x_1 x_2 x_3} + \beta ( x_1 x_2 x_3 \oplus \ket{11}) \ket{x_1 \oplus x_2} \ket{x_2 \oplus x_3}$\\
If we measure two output qubits we learn $x_1 \oplus x_2$, $x_1 \oplus x_3$ without disturbing the original state. The result of measuring these output qubits is called the syndrome.\\
\begin{tabular}{|c|c|}
        N & Syndrome\\ \hline
        $I \otimes I \otimes I$ & 00\\
        $ I \otimes I \otimes X$ & 01\\
        $ I \otimes X \otimes I$ & 10\\
        $I \otimes X \otimes X$ & 11\\
        $ X \otimes I \otimes I$ & 11\\
        $ X \otimes I \otimes X$ & 10\\
        $ X \otimes X \otimes I$ & 01\\
        $ X \otimes X \otimes X$ & 00\\
        \end{tabular}\\
        If we assume only one bit has been flipped the syndrome tells us which one as they have all distinct syndromes. After we dectect a bit-flip error (on a single qubit) we apply the same bit-flip operation to that qubit and restore the original encode state $\alpha\ket{000} + \beta\ket{111} \rightarrow \alpha \ket{0} + \beta \ket{1}$ by reverse the original encoding circuit.\\\\
        Consider the effect of a $Z$ (or "phase") error. It maps the encoded state on a single qubit:
        $$
        \alpha \ket{000} + \beta{111} \rightarrow \alpha \ket{000} - \beta\ket{111}
        $$
        would have Syndrome measurement 00, so we cant even detect a $Z$ error. However, they can be detected using a different code. If we notice that $Z = HXH$, so $Z$ is effectively $X$ but in a different basis the $\ket{+}$ $\ket{-}$ basis. We can use a different code $\ket{\psi} = \alpha \ket{+++} + \beta\ket{---}$ so what is our new encoding circuit:
        $$
         $$
         \centerline{\Qcircuit @C=1em @R=.7em {
                              \lstick{\alpha \ket{0} + \beta \ket{1}} & \ctrl{1} & \ctrl{2} & \gate{H}\\
                        \lstick{\ket{0}} & \targ & \qw & \gate{H} \\
        \lstick{\ket{0}} & \qw & \targ & \gate{H}}} $$  
        $$
        Decoding:
$$
                      \Qcircuit @C=1em @R=.7em {
\lstick{\ket{x_1}} & \gate{H} & \ctrl{3} & \qw & \ctrl{4} & \qw\\
\lstick{\ket{x_2}} & \gate{H} & \qw & \ctrl{2} & \qw & \qw \\
\lstick{\ket{x_3}}& \gate{H}  & \qw & \qw & \qw & \ctrl{2}\\
\lstick{\ket{0}} & \qw & \targ & \targ & \qw & \qw \\
\lstick{\ket{0}} & \qw &\qw & \qw & \targ & \targ
}
$$
A code of this kind can protect against $Z$ errors. But there is a problem as this code can no longer protect against $X$ errors.:
$$
\alpha \ket{0} + \beta \ket{1} \rightarrow \frac{1}{2 \sqrt{2}} \alpha( \ket{0} + \ket{1}) ( \ket{0} + \ket{1}) )(\ket{0} + \ket{1}) + \beta ( \ket{0} - \ket{1}) (\ket{0} - \ket{1}) (\ket{0} - \ket{1})$$
$$
\alpha \ket{0} + \beta \ket{1} \rightarrow \frac{1}{2\sqrt{2}} \alpha ( \ket{000} + \ket{111}) ( \ket{000} + \ket{111}) ( \ket{000} + \ket{111}) + \beta ( \ket{000} - \ket{111}) (\ket{000} - \ket{111}) (\ket{000} - \ket{111})
$$
This is a 9-qubit code.\\\\
To decode, we first apply the decoding circuit for bit-flip error to each block. Assuming that at most one bit flip error has occured in each block the resulting state will be $\alpha \ket{+++} + \beta \ket{---}$ (with at most one phase flip error). We map this state to $\alpha \ket{0} + \beta \ket{1}$ using the decoding algorithm for the phase flip code.\\\\
So with a great amount of effort we managed to protect against only two types of errors. It turns out this approach works for all errors, this is because the matrices $\{ I, X,Y, Z\}$ with $Y = i XZ$ form a basis for the complex vector space of $2x2$ matrices, so an arbitary error acting on a single qubit can be written as a linear combination of $I$, $X$, $Z$ and $iXZ$.\\\\
\section{Lecture 21}
\textbf{Theorem: Quantum error correcting criteria}
Assume we have a code subspace with a basis $\{\ket{\psi_i}\}$. A necessary and sufficent condition for a set of errors $\{E_a\}_a$ to be correctable is:
$$
\bra{\psi_i} E_a^{\dagger} E_a \ket{\psi_i} = C_{ab} \delta_{ij}
$$
with $C_{ab}$ a hamiltonian matrix:
\subsection{Stabilizer formulism}
Every pair of elements of $\mathcal{P}$ either commute or anticommute\\
Consider a set $\mathcal{M}$ of size $k$ that consists of Paulis s.t. any two elements from $\mathcal{M}$ commute. We can respresent $\mathcal{M}$ as a block matrix with entries being individual (single qubit) Paulis with $k$ rows. Example:
$$
\begin{pmatrix} I & X & Z\\ Y & Y &Z \\ X & Z & X \end{pmatrix}
$$
with each row refered to as $\mathcal{M}^{(i)}$. We say a state $\ket{\psi}$ is stabilized by $\mathcal{M}^{(i)}$ if $\mathcal{M}^{(i)} \ket{\psi} = \ket{\psi}$. If $\mathcal{M}^{(i)}$ and $\mathcal{M}^{(j)}$ are stabilizers of $\ket{\psi}$ then so is $\mathcal{M}^{(i)}\mathcal{M}^{(j)}$.\\\\
Let's assume that elements $\mathcal{M}^{(i)} \in \mathcal{M}$ are independnat, and consider the set $S$ of all $n$-qubit states that are stabilzed by all elemenets of $\mathcal{M}$. $S$ is a subspace of $\mathbb{C}^{2^n}$. By construction, all elements of $\mathcal{M}$ commute, so the projector $\Pi_s$ onto $S$ is the product of the projectors onto all +1 eigenspaces of $\mathcal{M}_i$s. Define: 
$$
\Pi_s = \frac{1}{2^k} \prod_{i=1}^k ( I + \mathcal{M}^{(i)})
$$
Let's find the dimension of $S$:
$$
Tr \Pi_s = \frac{1}{2^k} Tr \Prod_{i=1}^k ( I - M^{(i)}) = \frac{1}{2^k} \sum_{T=\text{subsets of size k}} \prod{i \in T} \mathcal{M}^{(i)} = 2^{n-k}
$$
(check this^)\\
When $k=n$ there is a unique state stabilized by all elements of $\mathcal{M}$. We call this state a stabilizer state. \\\\For an $n$-qubit stabilizer state $\ket{\psi}$, we can define the stabilizer group as the group of $n$-fold tensor product of Palis that stabilze $\ket{\psi}$.\\\\
Stabilizer groups are all Abelian (check this).\\
Clifford circuits (CNOT, H, P) = stabilizer circuits. Now lets look at some examples:\\
\textbf{Single qubit case}:
$X$ stabilises $\ket{+}$\\
$-X$ stabilises $\ket{-}$\\
$Z$ stabilises $\ket{0}$\\
$-Z$ stabilises $\ket{1}$\\
$Y$ stabilises $\ket{i}$\\
$-Y$ stabilises $\ket{-i}$\\
\textbf{Stabilizer groups}
$\ket{0} \rightarrow \{I, Z\}$\\
$\ket{+} \rightarrow \{I, X\}$\\
$\ket{0} \otimes \ket{+} \rightarrow \{ I \otimes I, I \otimes X, Z \otimes I, Z \otimes X\}$\\\\
Consider the stabilizer group of $\frac{1}{\sqrt{2}}( \ket{00} + \ket{11}) = \ket{\phi^+}$, as $XX$ stabilises $\ket{\phi^+}$ and $-YY$ stabilises $\ket{\phi^+}$ and as $XX(-YY) = ZZ$ we have the stabilizer group is:
$$
 \{ I, XX, ZZ, -YY\}
$$
this is a unique way of specifying a state. Similar to how you need 4 amplitudes to describe a 2 qubit state and here we have 4 states, however, we will see later that there are shorter ways of describing a state using the stabiliser formulism.\\\\
When $\mathcal{M}$ has $k< n$ commuting Pauli matirces, we can regard the subspace $S$ stabilised by elements of $\mathcal{M}$ as encoding $n-k$ qubits. This subspace may have desirable error correcting properties.\\\\
Consider an element $\ket{\phi} \in S$, let $E$ be a Pauli error. If $E \in \mathcal{M}$ then $E \ket{\phi} = \ket{\phi}$, so the error has no effect on the state. If $E$ anticommutes with some $M \in \mathcal{M}$ then $ME \ket{\phi} = - EM \ket{\phi} = - E \ket{\phi}$ and by measuring $M$ we can detect that an error has occurred.\\
\textbf{Claim}: Quantum error correction criterion is satisfied if for each pair of error operators $E_a$, $E_b$:\\
1) $E_a^{\dagger} E_b \in \mathcal{M}$ or 2) there exists $M \in \mathcal{M}$ s.t. $M$ anticommutes with $E_a^{\dagger} E_b$\\\\
We can define a stabilizer code with distance $d$ if each Pauli matrix $E = E_a^{\dagger} E_b$ has weight (number of non-trivial paulis) $< d$ and satisfies conditions 1) and 2), but there exists $E$ of weight $d$ s.t. it fails to satisfy 1) and 2).\\\\
A code that corrects errors on up to $t$ qubits must have $d \geq 2 t +1$. For all $E$ of weight $<d$ have $E \leq \mathcal{M}$ or there exists $M \in \mathcal{M}$ that anticommutes with $E$.\\\\
Consider unitary matrix $U$ s.t. $U$ commutes with all elements in $\mathcal{M}$ but $U \cancel \in \mathcal{M}$. Then $U$ represents an error that we cannot correct. More precisely, such $U$ allows us to perform logical operations on data encoded in stabilizer code by non-trivially acting on the corresponding subspace. \\
Consider two unitarise $U_1$, $U_2$ that anticommute with $M \in \mathcal{M}$. THis allows us to implement logical $X$ and $Z$ operations. \\
Think about shors 9 =-qubit stabilizer code:
$$
\begin{pmatrix}
        Z& Z & 1 & 1 & 1 & 1 &1 &1 &1\\
        Z& 1 & Z & 1 & 1 & 1 &1 &1 &1\\
        1& 1 & 1 & Z & Z & 1 &1 &1 &1\\
        1& 1 & 1 & Z & 1 & Z &1 &1 &1\\
        1& 1 & 1 & 1 & 1 & 1 &Z &Z &1\\
        1& 1 & 1 & 1 & 1 & 1 &Z &1 &Z\\
        X& X & X & X & X & X &1 &1 &1\\
        X& X & X & 1 & 1 & 1 &X &X &X\\
\end{pmatrix}
$$
This code has distance 3, encodes 1 qubit $X^{\otimes n}$ and $Z^{\otimes n}$ commute with everyone in the stabilizer code but are not contained within it.\\\\
An $n$-qubit state has $2^n$ element stabilizer group $G$, how to efficently speciify $G$:\\
\textbf{Lemma}: Given nay stabilizer state its stabilizer group $G$ can always be generated by $n$ elements.
For example $\ket{\phi^+}$ with stabilizer group $G = \{ I , XX, -YY, ZZ\}$ can be described by a generating set $< XX, ZZ>$ or $<XX, -YY>$. \\
To describe a $n$-qubit stabilzer state on a cliassical computer we need $2n+1$ bits per generator meaning $O(n^2)$ classical bits.
\section{Example sheet 3}
$$
 || A || = \max_{\ket{\psi}} |A \ket{\psi}|
$$
Here $|A \ket{\psi}|$ is not $\bar{\psi} A \ket{\psi}$ but just the magnitude of the vector created.
So try $\ket{\psi} = \begin{pmatrix} x \\ y \end{pmatrix}$ with $|x|^2 + |y|^2 = 1$ so:$$
|| A || = \max_{x^2 + y^2 =1} |A \begin{pmatrix} x\\ y \end{pmatrix}|
$$\\\\
$A$ can be expressed as a linear combination of unitary matrices of the form $U_{ab} = X^{d-b} Z^{d-a}$ with $X \ket{k} = \ket{ k + 1 mod d}$ and $Z = e^{ \frac{2\pi i k}{d}} \ket{k}$, as these are linearly independant and unitary.\\\\
There is a different method for finding a linear combination of unitaries to represent a non-unitary matrix (elaborated on in the solutions). Use SVD:
$$
A = UDV
$$
with $D$ real and diagonal. It suffices to make any such $D$ as $D = \sum_i c_i U_i$.\\\\
Need to check $A$ is well conditioned when performing HHL on $A\bm x = b$ as we need to take its inverse (so need to check it has no zero eigenvalues). In example sheet this is shown by knowing $||C|| \leq \frac{1}{2}$ and $A = I - C$ so eigenvalues must lie in $[\frac{1}{2}, \frac{3}{2}]$. For HHL we also require $A$ to be row-spares and row computable, and for $\ket{b}$ to be implemented able in poly(log N) time.\\\\
\textbf{If you ever need to find out how close to states are,  you use the swap test!!!}\\\\
\textbf{Chernofl-Hoelfing bound}\\
If we westimate $Pr(0)$ as frequency $f$ of 0's seen in $k$ samples of a 0/1 distribution, then:
$$
 |f - Pr(0)| < \xi, \text{with probability} \geq 1 - \epsilon
$$
if $k \geq \frac{\log \frac{1}{\epsilon}}{2 \xi^2}$\\\\
\section{Lecture 21}
\subsection{Alternative approach to proving GK theorem}
We will discuss unitary part in lectures. The part that deals with measurements will be discussed in ES4.\\\\
We will start with the stabiliser state and update the list of generators for each Clifford gate applied to it.\\\\
Let;s start with something concrete. Starting with the list of zero states $\ket{0}^{\otimes n}$ we can write the list of generators of this state as $\bra{Z} I$,$I Z I I$,..., $I \ket{Z}$.\\\\
We will introduce the stabilizer tableau, giving two $n\times n$ matrices + extra space
$$
\ket{0}^{\otimes 4} \begin{pmatrix} 0 & 0& 0 & 0 & 1 & 0  & 0 & 0\\
0 & 0& 0 & 0 & 0 & 1  & 0 & 0\\
0 & 0& 0 & 0 & 0 & 0  & 1 & 0\\
0 & 0& 0 & 0 & 0 & 0  & 0 & 1\\
\end{pmatrix}
$$
Each row represents one generator of the stabilizer group. The first block which in this case is all zeros has 1 if $X$ or $Y$ and the second block has 1 if $Z$ or $Y$.\\\\
\textbf{Rules}:\\
1)How to apply a Hadamard gate to the $i$-th qubit: \\
We swap the $i$-th column of the $X$ matrix with the $i$-th column of the $Z$ matrix\\\\
2) Apply $S$ gate to the $i$-th qubit:\\
Bitwise $XOR$ the $i$-th column of the $X$ matrix into the $i$-th column of the $Z$ matrix\\\\
3) Apply $CNOT$ ( $i$-th qubit-control, $j$-th qubit target):\\
Bitwise $XOR$ the $i$-th column of the X matrix into the $j$-th column of the $X$ matrix, and then bitwise $XOR$ the $j$-th column of the $Z$ matrix to the $i$-th column of the $Z$ matrix.
\subsection{Hybrid Quantum-Classical computation}
Let's define a model:\\
Input: a) Classical part description of a quantum circuit, input state and measurement lines, b)  Quantum part input $\ket{0}^{\otimes n}$\\
Computation described as:
$$
 M_2C_2M_1C_1\ket{0}^{\otimes n}
$$
$M_1$ gives a classical string $x_1, ..., x_n$ that we feed into classical processing, and the output of the classical processing affect the block of gates $C_2$. This is repeated again and again for $C_3$, $M_3$ etc. etc.\\\\
$C_1$, $C_2$ ,... are just sequences of 1 and 2 qubit gates.\\
We will distinguish intermediate states after each measurement $M_n$ as $\ket{\psi_n}$ with input state as $\ket{\psi_0}$.\\
We will distinguish two flavours of this type of computation, which essentially depends on what happens with the intermediate states.\\
Type 1: The input state is carried through the computation. In other words we have $\ket{\psi_{out}} = M_N C_N ... M_1 C_1 \ket{\psi_0}$\\
Type 2: Every round (consisting of application of $C_i$ followed by M_i) we make use of $\ket{\psi_{i-1}} = \ket{0}^{\otimes n}$.\\\\
Example of hybrid quantum-classical computation of type 1) Pauli-based computation ( Bravyi-smith, smoh '16) covered this lecture, type 2) variational quantum eigensolver ( last lecture of this course).
\subsubsection{Pauli-based computation}
Start with an input state $\ket{\psi_{in}}$ on $t$ qubits\\
Apply a sequence of pairwise commuting Pauli-measurements $P_1...P_s \in \mathcal{P}_t$ sequentially to the input state\\
The choice of each $P_i$ can in general depend on previous measurement outcomes. When there is no such dependancy we say that PBC is non-adaptive.\\\\
Considering the adaptive case. There is a nice tree diagram in the notes that I can't replicate in latex. \\\\
\textbf{Theorem}: Let circuit $C$ be any generally adaptive quantum circuit on $n+t$ qubits with input state $\ket{\psi} = \ket{\sigma } \otimes \ket{\rho}$ where $\ket{\sigma}$ is a stabilizer state on $n$ qubits, $\ket{\rho}$ is any state on $t$ qubits. Suppose that the unitary steps of $C$ are all Clifford gates. Then:\\
i) $C$ may be weakly simulated by a generally adaptive PBC circuit $\tilde P_1 ... \tilde P_s$ on $t$ qubits and with $s \leq t$ steps.\\
ii) If $C$ is non-adaptive (with final measurement outputs) then the corresponding PBC circuit $\tilde P_1, ... \tilde P_S$ can be chosen to be non-adaptive \\
iii) If some $Z$ measurement in $C$ are postselcted to outcome +1, then this circuit can be weakly simulated by a PBC circui in which some of the Pauli measurements are postselected to outcome +1\\
\textbf{Lemma A}: Let $P, Q \in \mathcal{P}_n$ be anti-commuting Pauli operators. let $\ket{\psi}$ be an eigenstate of $P$ with $P\ket{\psi} = \lamdba_p \ket{\psi}$, $\lambda_p = \pm 1$. Then 1) a measurement of $Q$ on $\ket{\psi}$ yields $\lambda_a = \pm 1$ with equal probabilities.\\
2) The operator $V( \lambda_P, \lambda_Q) = \frac{\lambda_p P + \lamdba_q Q}{\sqrt{2}}$ is always a unitary Clifford operation.\\
3) $V(\lambda_p, \lambda_q)$ is the normalised projection of $\ket{\psi}$ onto the $\lamdba_Q$ eigenspace of $Q$.\\\\
Remark: Hence measurement of $Q$ on $\ket{\psi}$ is equivalent to classically chossing a uniformly random $\lamdba \in \{ \pm 1\}$ and applying clifford unitary.
\section{Lecture 23}
\textbf{Lemma B}: For any $$P = \pm A_1 \otimes ... \otimes A-n \otimes B_1 \otimes... \otimes B_t \in \mathcal{P}_{n+t}$$,  with$A_1, B_1 \in \{ X,Y,Z, I\}$ write :
$$
 \tilde P = \pm B_1 \otimes... \otimes B_t \in P_t
$$
(with the same oeveral sign as $P$).\\
If $P$ commutes with $Z_1...Z_n \in \mathcal{P}_{n+t}$, then each $A_i$ is either $Z$ or $I$. If for all $i$, each $A_i$ is either $Z$ or $I$, then for any $t$-qubit state $\ket{\psi}$ the measurement of $P$ on $\ket{0}^{\otimes n} \otiems \ket{\psi}$ and the measurement $\tilde P $ on $\ket{\psi}$ give the same output distribution, and corresponding post-measurement states: $\ket{0}^{\otimes n} \ket{\psi'}, $\ket{\psi'}$ respectively, with the same $t$-qubit state $\ket{\psi'}$.\\\\
\textbf{Proof of theorem:}\\
Let $C$ be any adaptive circuit with all unitary parts Cliffords and let's have $K$ measurements (all being $Z$ measurements)).
$$
\ket{\sigma} = \ket{0}^{\otimes n}
$$
(we can replace $\ket{0}^{\otimes n}$ by any stabilizer state by working with the coresponding set of generators)\\\\
First, consider the output state $\ket{\psi_{out}} = M_k C_k .. M_1 C_1 \ket{0}^{\otimes n} \otimes \ket{\rho}$\\
a) Starting with the rightmost Clifford gate and working to the left we will commute each gate to the end of the circuit beyond the last measurement. As a result, each $Z$ measurment (denoted as $M_i$) will become conjugated into a Pauli measurement $P_i \in \mathcal{P}_{n+t}$ which can be efficently determined.\\\\
Unitaries after $P_K$ have no effect on the measurment outomces so we delete them, so now we are left with a sequence of large Pauli measurments on $n+t$ qubits.\\\\
b) Preface this sequence of measurement in (a) with dummy $Z$ measurments for each of the first $n$ lines. This will have no effect on $\ket{0}^{\otimes n}$ becomes $Z$ measurment gives $+1$ with certainity. We create a list of measurements $Z_1, ... , Z_n, P_1, ..., P_K$\\\\
c) Define our PBC process. Our $t$ g=qubit register is initially in state $\ket{\rho}$. We are going to look at the list and we will work sucessively through $P_j$ starting from $P_1$ and three things can happen when looking at each $P_i$:\\
1) If $P_j$ is completely dependant on measurements already performed (this may be efficently determined), then we delete $P_j$ from the list and calculate the measurement outcome from previous recorded measurement results. Move to the next measurement in the list.\\
2) If $P_j$ commutes with all previous measurements, then we measure $\hat P_j$ instead as in Lemma $B$ and record its value $\lamdba_{P_j}$ and move to the next measurement in the list.\\
3) If $P_j$ anticommutes with some previous measurement $N$ (which could possibly be a dummy $Z$) with outcome $\lambda_N$. Then classically randomly choose $\lambda_P \in \{ \pm 1\}$ and record it. Then delete $P_j$ from the list and replace it by the unitary Clifford $V(\lambda_N, \lamdba_{P_j})$ using Lemma A. Move to the next element of the list.\\\\
Once we have processed all the elements in the list, we will have performed $s \leq k$ measurments on the $t$ qubit register which are independant and commuting Pauli measurements, and this process was assisted by some classical randomised efficent computation. \\
Since measurements are independant on $t$ qubits this implies $s <t$.\\\\
\textbf{Proof of ii)}: If $C$ is non-adaptive, we may representive it as a untiary circuit followed by a final measurement $Z_{i_1},..., Z_{i_n}$ on qubit lines $i_1$,..., $i_n$ then b) gives us a non-adaptive list $Z_1,...,Z_N, P_1,..., P_K$ where $P_k = U Z_{i_k} U^{\dagger}$ which are communiting and independant.\\\\
\textbf{Proof of iii)}: Proceed as if there was no postselection except c) iii). Suppose that the measurement $P_j$ in that step is postselected on outcome +1, we aren't going to randomly flip a coint to get $\pm 1$ we are going to set $\lambda_{P_j} = +1$ and replacing $P_j$ with $V(\lamdba_N, 1)$ will produce the same outcome. If a dependant measurment determined outcome is inconsitent with the imposed post-selection outcome (as in c) i)) then postselecction measurment on the original measurment has probability 0.\\\\
\textbf{Extended Grottesman-Knull theorem}:\\
Let $C$ be any adaptive Clifford circuit with input $\ket{\sigma} \otimes \ket{\rho}$, where $\ket{\sigma}$ is a stabiliser state of $n$ qubits, $\ket{\rho}$ is an arbitary state on $t$ qubits. Output is given by measurment of any number of specified qubit lines. Then:\\
u) $C$ can be weakly simulated by an adaptive circuit $C^*$ on $t$ qubits with input $\ket{\rho}$ assisted by poly(n+t)-time classical computation and $C^*$ having at most $t$ intermediate or final measurments. \\
ii)  If $C$ is non-adaptive then $C^*$ may be taken to be unitary (with $Z$ basis measurments at the end)\\
iii) If some $Z$ measurments in $C$ are post selected on $+1$ then the same holds for measurments in $C^*$.
\section{Lecture 23}
\subsection{Hybrid Quantum-Classical Compuation}
\subsubsection{The Variational Quantum Eigensolver (VQE)}
We highlighted two regimes in hybrid compuation. THe first of these is when the input state is carried though the computation (e.g. PBC). The second is that every round (consiting of unitary operation- measurment - classical processing) requires preparation of a fresh input state (VQE).\\
VQE represents a heuristic approach for solving a range of optimisation problems/quantum simulation of chemistry and material science. For example we can solve constraint satisfaction problems (MaxCut), solving electronic structure Hamiltonians. Any problem that can be cast in the form of finding the ground state energy of Hamiltonian $H$.\\\\
VQE consists of the following steps. Given: problem $P$ , $\epsilon >0$, parameterized quantum circuit $V(\bm \theta)$, $\bm \theta = ( \theta_1, ..., \theta_n)$ (so callled controls/angles), $n$ qubits.
\begin{romanlist}
\item Map the problem $P \rightarrow H_p$ such that solving $P$ would amount to finding the ground state of $H_p = \sum_i h_i$\\
\item Prepare $\ket{\psi(\bm \theta)} = U(\bm \theta) \ket{0}^{\otimes n}$\\
\item Compute the cost function (which is just energy in disguise): $C(\bm \theta) = \bra{\psi(\theta)}H_p \ket{\psi(\theta)} = \sum_i \bra{\psi(\bm \theta0} h_i \ket{\psi(\bm \theta)}$\\
\item Use classical (poly-time) optimisation algorithm $\mathcal{A}$. Take $\mathcal{A}(\bm \theta) = \bm \theta'$ (it can be one of many optimization algorithms\\
\item Run steps $2 ->4$ with $\bm \theta'$ until:
        $$
        |C(\theta') - C(\theta_{opt}) | < \epsilon
        $$
        or if $C(\theta_{opt})$ is not known then until 
        $$
        |C(\theta_{i+1}) - C(\theta_i)| < \epsilon
        $$
        where $\theta_i$ are a sequence of controls obtained in step 4.
\end{romanlist}
One potential ansatz is constructed of $k$ layers with each layer like the following (for the $i$th layer) (with consequentive layers having the $CX$s shift one down:
$$
 \centerline{ 
             \Qcircuit @C=1em @R=.7em { 
               & \gate{U_{1,i}(\theta_{1,i})} & \ctrl{1} &  \\
               & \gate{U_{2,i}(\theta_{2,i})} & \targ &  \\
               \\
               & \gate{U_{n-1,i}(\theta_{n-1,i})} & \ctrl{1} &  \\
               & \gate{U_{n,i}(\theta_{n,i})} & \targ &  \\
                               }}
$$
typically we have $U$ being arbitary rotation gates s.t. as 
$$
U_{ij} = e^{- i \frac{\theta_{ij}}{2} \sigma_z}
$$
\section{Example sheet 4}
Can come up with a lot of interesting questions in stabiliser forumalism theoreom. \\\\
Measuring a Pauli is equivalent to projection on to the $+ 1$ or $-1$ eigenspace. Therefore you can write the probability of measuring $+1$ on a Pauli measurement as:
$$
P(+1) = || \frac{1}{2}( I + Q) \ket{\psi}||^2 
$$
$$
P(-1) = || \frac{1}{2}(I - Q) \ket{\psi} ||^2
$$
Remember that a 5 qubit pauli like $XXXXX$ does not do the same as measuring each qubit in the $X$ basis, it does one measurmeent and projects the space into the half of the Hilbert space with that eigenvalue.\\\\
Remember pauli projector:
$$
P_{\ket{0}} = \frac{1}{2} ( I-Z), P_{\ket{1}} = \frac{1}{2} ( I + Z)
$$
The projector onto subgroup generated by $<XX>$ is:
$$
P_S = \frac{1}{2}( I \otimes I + X \otime X)
$$
and the space is spanned by:
$$
\{ \frac{\ket{00} + \ket{11}}{2}, \frac{\ket{10} + \ket{01}}{2}\}
$$
This course has been extended this year from 16 to 24 lectrures so the hamiltonian simulation and pauli based computation and last example sheet so watch out that this won't be in the past papers
\\\\\\\\\
Not revising non-abelian HSP, and NP
\end{document} 
