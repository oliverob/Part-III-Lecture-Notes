\documentclass[12pt, a4paper, twoside, titlepage]{article}
\usepackage{amsfonts}
\usepackage{amsmath}
\usepackage{bm}
\usepackage{cancel}
\usepackage{braket}
\begin{document}
\title{Quantum Field Theory}
\maketitle
\section{Preliminaries}
\subsection{Natural units}
$$
[c] = LT^{-1} \> \>
[\overline{h}] = L^2MT^{-1} \> \>
[G] = L^3M^{-1}T^{-2}
$$
We set $c=\overline{h}=1$ to give the natural units. All quantities expressed in natural units scale with some power of mass or energy. We use notation below to describe this:
$$
X \tilda M^{\delta}, \. [X] = \delta
$$
where $\delta$ is the scaling dimension. e.g. $[E] = +1$ and $[L] = -1$. In order to convert back from natural units we need to know what we have calculated. For example, if it is energy then we multiply by $c^2$ as $E=mc^2$ but in natural units $c=1$. Basically everything is done in units of energy/mass with $c$ and $\overline{h}$ giving the scale of choice between length, time and energy.
\subsection{Classical Fields}
A classical scalar field is one that maps from Minkowski spacetime to the "field space":
$$
\phi: \mathbb{R}^{3,1} \rightarrow \mathbb{R}
$$
The scalar part of the definition implies that the field is invariant under Lorentz transformations that are defined by keeping the metric ($\eta = {+1,-1,-1,-1}$) invariant:
$$
\Lambda^{\mu}_{\sigma}\Lambda^{\nu}_{\tau}\eta^{\sigma \tau} = \eta^{\mu \nu}
$$
the further restrection of $\deta \Lambda = +1$ removes the possibility of reflection and defines the Lorentz group $G_L = SO(3,1)$.\\
The active convention for describing a scalar field is:
$$
\phi(x) \rightarrow^{\Lambda} \phi'(x)
$$
where $\phi'(x) = \phi (\Lambda^{-1} \cdot \vec{x})$. It is clear that $\Lambda^{-1} \in SO(3,1)$ and $(\Lambda^{-1})^{\mu}_{l}\Lambda^{l}_{\nu} = \delta^{\mu}_{\nu}$.
\subsection{Spacetime derivatives}
$$
\partial_{\mu}\phi(x) = \frac{\partial \phi(x)}{\partial x^{\mu}}
$$
$$
\partial_{\mu}\phi(x) \rightarrow^{\Lambda} \partial_{\mu}\phi'(x) =  \frac{\partial \phi(\Lambda^{-1}\cdot x)}{\partial x^{\mu}} = (\Lambda^{-1})^{\nu}_{\mu} \partial_{\nu}\phi((\Lambda^{-1} \cdot x)}
$$
\textbf{4-Vector Field}
$$
V: \mathbb{R}^{3,1} \rightarrow \mathbb{R}^{3,1}
$$
$$
v^{\mu}(x) \rightarrow^{\Lambda} \Lambda^{\mu}_{\nu}v^{\nu}(\Lambda^{-1}\cdot x)
$$
\subsection{Lagrangians}
\textbf{Action}
$$
S[q] = \int_{t_i}^{t_f} dt L(q(t),\dot q(t))
$$
Principle of least action: If you vary the path with fixed endpoints then the stationary action ($\partial S = 0$) is equivalent to the Euler-Lagrange equation
\subsubsection{Lagrangian for scalar field theory}
Demand Lorentz invariant action, locality and at most two time derivatives. Langrangian is a functional as it must give a number for every field.
$$
L(t) = L[\phi, \partial_{\mu} \phi] = \int d^3x \mathfrak{L}(\phi(x),\partial_{mu}\phi(x))
$$
where $\mathfrak{L}$ is the Lagrangian density. This ensures locality as the field is EOM will only be influenced by the local field.
$$
S_{t_i,t_f}[\phi, \partial_{\mu}\phi] = \int_{t_i}^{t_f} L[\phi, \parital_{\mu}\phi] dt = \int_{t_i}^{t_f} dt \int d^3x \mathfrak{L}
$$
Take infinite time interval (nice interval to take when assuming that before and after interaction particles where very far apart)
$$
S = \int_{\mathbb{R}^{3,1}} d^4x \mathfrak{L} (\phi(x), \partial_{\mu}\phi(x))
$$
Consider L.T. $\Lambda \in SO(3,1)$ and require that $\mathfrak{L}$ is a scalar field:
$$
\mathfrak{L}(x) \rightarrow^{\Lambda} \mathfrak{L}(\Lambda^{-1}\cdot x)
$$
then
$$
S =  \int_{\mathbb{R}^{3,1}} d^4x \mathfrak{L} (x) \rightarrow^{\Lambda}  \int_{\mathbb{R}^{3,1}} d^4x \mathfrak{L} (\Lambda^{-1} \cdot x) = S'
$$
choose variables of $y^{\mu} = \Lambda^{-1}^{\mu}_{\nu} x^{\nu}$ giving:
$$
S' = \int_{\mathbb{R}^{3,1}} d^4y \mathfrak{L} (y) 
$$
so the action is Lorentz invariant if the Lagrangian density is a scalar field.
\subsubsection{General Lagrangian}
\begin{equation}
        \mathfrack{L} = \frac{1}{2}\partial_{\mu}\phi \partial^{\mu} \phi - V(\phi)
\end{equation}
general terms you can have if you want at most two time derivatives. Not quite most general as could have an arbitrary function of $\phi$ as a pre-factor to the kinetic term but is ruled out later by a dimensionality constraint. No need to separately consider $\phi \partial_{\mu}^{\mu}\phi$ but once it is inside the action this is the kinetic term by integration by parts (differs only by a surface term).\\\\
\textbf{Everytime two indicies are contracted top and bottom they use the metric}:
$$
\eta_{\mu \nu}\Lambda^{\mu}_{\sigma} = \Lambda_{\mu \sigma}
$$
$$
\partial_{\mu}\phi \partial^{\mu} \phi  = \eta_{\mu \nu}\partial^{\mu}\phi \partial^{\nu} \phi 
$$

\section{Lecture 3}
\textbf{Principle of Least Action}
Vary the field conifguration: $\phi(x) \rightarrow \phi(x) + \delta \phi (x)$ but fix the boundary condition: $\delta \phi(x) = \phi(t,x) \rightarrow 0$ for $|\bm x| \rightarrow \infinity$ and $t \rightarrow \infinity$. Consider variation of action:
$$
\partial S = \int d^4 x \left[ \frac{\partial \mathfrak{L}}{\partial \phi} \cdot \delta \phi + \frac{\partial \mathfrak{L}}{\partial (\partial_{\mu} \phi)} \delta(\partial_{\mu} \phi)\right]
$$
$$
= \int dx^4 \left[ (\frac{\partial \mathfrak{L}}{\partial \phi} - \parital_{\mu} (\frac{\partial \mathfrak{L}}{\partial{\partial_{\mu} \phi}}  )\delta \phi + \parital_{\mu} (\frac{\partial \mathfrak{L}}{\partial (\partial_{mu} \phi) } \delta \phi) \right]
$$
$$
 B = \int_{\mathbb{R}^{3,1}} dx^4 \partial_{\mu} (\frac{\partial \mathfrak{L} }{\partial \partial_{\mu} \phi} \delta \phi) =  \int_{\partial(\mathbb{R}^{3,1})} dS_{\mu} \frac{\partial \mathfrak{L} }{\partial \partial_{\mu} \phi} \delta \phi
$$
The boundary $\partial \mathbb{R}^{3,1}$ represents where $| \bm x | \rightarrow \infty$ and $t \rightarrow \pm \infty$. Therefore, $\delta \phi = 0$ and $\partial (\mathbb{R}^{3,1})$. This means we can set this integrand to zero. The above needs modifying to match notes but it is essentially the same derivation we use for Euler-Lagrange in classical mechanics.
\subsubsection{Euler-Lagrange equation}
\begin{equation}
        \frac{\partial \mathfrak{L}}{\partial \phi} - \partial_{\mu} \left( \frac{\partial \mathfrak{L}}{ \partial(\partial_{\mu} \phi) } \right) = 0 
\end{equation}
$$
\frac{\partial \mathfrak{L} }{\partial \phi} = - V'(\phi)
$$
$$
\frac{\partial \mathfrak{L}}{\partial (\partial_{\mu} \phi)} = \partial^{\mu}\phi
$$
gives general field equation of motion:
\begin{equation}
\partial_{\mu}\partial^{\mu} \phi + V'(\phi)  =0
\end{equation}
$$
\partial_{\mu}\partial^{\mu} \phi = \frac{\partial^2}{\partial t^2} - \nabla^2_{\bm x}
$$
This is in general a second order PDE which is hard to solve normally need computer. We want to consider special case that is easy to solve e.g. choose a quadratic potential to give a linear EOM:
$$
V(\phi) = \frac{1}{2} m^2 \phi^2
$$
gives the Klein-Gordan equation:
\begin{equation}
\partial_{\mu}\partial^{\mu} \phi + m^2\phi  =0
\end{equation}
This is indeed a lorentz invariant equation, which is good as the whole point of the action stuff was to generate EOM that were lorentz invariant. This is a wave equation, and we can immediately say quite a lot about it. It has wave-like solutions, e.g. trial solution:
$$
\phi = e^{i \bm x \cdot \bm p} = e^{i\omega t - i \bm k \cdot \bm x}
$$
gives dispersion relation:
$$
\omega_k = \sqrt{|k| + m^2}
$$
You can see that when we go to the quantum scale this will correspond to free massive particles as, as this is basically the wave-particle dual for a standard particle of mass m. This gives a free particle as we have a linear classical equation so it has a superposition principle so you can add together solutions so you will get non-interacting particles. \\\\
The general case is the case when we have a non-linear PDE given by a non-quadratic potential. Yoy won't have localised wave packets they will disperse. If you try to form lumps of fields then they will interact with each other and disperse. No superposition principle. may have exotict solutions like solitons. In notes from last year there is also a short section of applying this to maxwells equation of electromagneticism if his doesn't do this after covering maxwells stuff go read the ntoes from last year.
\subsection{Symmetry}
\textbf{Definition}: Variation of the field that leaves the action invariant\\
Typically has the structure of the group and if it is a continuous variation then it behaves like a manifold (this is a Lie Group). Why are they important in field theory - because they tell us about the conserved quantities in the theory. As every symmetry implies a conservation law by Noether's Theorem. Important in any theory as we will bung some particles in and have some coming out at the but the key constraint on what is allowed and what is not is the conservation laws in the theory e.g. tells us we cant have an electron go in and a positron come out as that would violate a conservation law. More generally than that, they have much richer consequences e.g. non-abelian symmetries then infact the imprint which that leaves on the observerables of that theory that is much more than conservation laws. It leaves restrictions on the allowed masses and spectrums etc. Here we are generally talking about global symmetry which is the things we can actually compute (and we will use the field as a proxy for this) change under the application of the symmetry whereas gauge symmetry the observables actually remain unchanged.\\\\ We will implemment symmetries as a variation of the field not the coordinates, below are listed some symmetries and their action on coordinates and on fields. \\\\

\textbf{Translation}: $\bm x^{\mu} \rightarrow \bm x^{\mu} + \bm c^{\mu}$ for a $\bm c \in \mathbb{R}^{3.1}$
$$
\phi(x) \rightarrow \phi'(x) = \phi(x - c)
$$
Automatic that all the actions that are written down are invariant under translation as we integrated over all space so they are translation invariant.\\\\
\textbf{Lorentz}: $\bm x^{\mu} \rightarrow \Lambda^{\mu}_{\nu}\bm x^{\nu}$
$$
\phi(x) \rightarrow \phi(\Lambda^{-1} \cdot x)
$$
If massless ($m=0$) and $\nu = 0$ then also get a \textbf{scale transformation}: $\bm x^{mu} \rightarrow \lambda x^{\mu}$ for $\lambda \in \mathbb{R}^+$
$$
\phi(x) \rightarrow \lambda^{-\Delta}\phi(\lambda^{-1} x), \Delta = [\phi]
$$
\subsection{Internal Symmetry}
Does not act on the coordinates, and its generators commute with all the generators of the global symmetry above. These give properties of particles such as electric charge, flavour and colour.\\\\
\textbf{Example}: Complex scalar field
$$
\psi: \mathbb{R}^{3,1} \rightarrow \mathbb{C}
$$
Need real lagrangian so need to contract with complex conjugate. \textbf{When finding the Euler-Lagrange equation of a complex scalar field you treat $\phi$ and $\phi^*$ as independent variables so you get two E-L equations}. Below is one possible complex field with a specific form of potential that generates an internal symmetry.
$$
\mathfrak{L} = \partial_{\mu}\psi^* \partial^{\mu} \psi - V(|\psi|^2)
$$
This has an internal symmetry as:
$$
\psi(x) \rightarrow \psi'(x) = e^{i\alpha}\psi(x)
$$
$$
\psi^*(x) \rightarrow \psi^*'(x) = e^{-i\alpha}\psi^*(x)
$$
which gives broth $\mathfrak{L}$ and $S$ are invariant. It doesn't need to be invariant in the lagrangian density to be a symmetry but this stronger case often occurs in internal symmetries unlike in the case of Lorentz transformation where $\mathfrak{L}$ was not invariant.\\\\
Continous symmetry form matrix Lie groups e.g. Lorentz group
$$
G_L = \{\Lambda \in Mat_4(\mathbb{R}), \Lambda \cdot  \eta  \cdot \Lambda^T = \eta, \det \Lambda = 1\} = SO(3,1)
$$
One can ask about how the different space-time symmetries fit together. The lorentz transformations corrospond to $SO(3,1)$ and when combined with transformations (which don't commute with lorentz transformations) you get the Poincare group. Finally, if we specialise to the case where the mass is zero then we get space invariance and this further enhances the group to give the conformal group (SO(4,2)).
\section{Lecture 4}
\subsection{Finite vs Infinitesimal transformations}
You don't have to understand the full non-linearity of the Lie Group everything can be thought of by just considering the behaviour of the group elements near the identity. Group elements sufficently near the identity can be written in the following form:
$$
g = \exp(\alpha X), \alpha \in \mathbb{R}, X \in \mathbb{L}(G)
$$
For all the examples in the course we are only considering matrix lie groups so all the elements are matrices. $\mathbb{L}(G)$ is called the Lie Algebra of the gorup. Therefore, $\exp( \alpha X) = \Sigma_{n=0}^{\infty} \frac{1}{n!} (\alpha X)^n$. We are going to consider this for $|\alpha| <<1$ to give $\exp( \alpha X) = I + \alpha X + O(\alpha^2)$, which is called an infintesimal transformation. 
\subsubsection{Example}
$$
\alpha = \partial_{\mu}\psi^* \partial^{\mu} \psi - V(|\psi|^2)
$$
can undergo finite transformation:
$$
\psi \rightarrow g \psi, g = \exp(i \alpha)
$$
so has infintesimal transformation:
$$
\psi \rightarrow \psi + \delta \psi, \delta \psi = i\alpha \psi
$$
\subsubsection{Lorentz Transformations}
$$
x^{\mu} \rightarrow x'^{\mu} = \Lambda^{\mu}_{\nu} x^{\nu}
$$
$$
\Lambda^{\mu}_{\sigma}\Lambda^{\nu}_{\tau}\eta^{\sigma \tau} = \eta^{\mu \nu}
$$
$$\deta \Lambda = +1$$
Can express lambda as:
$$
\Lambda = \exp(s \Omega), \Omega \in \mathbb{L}(SO(3,1))
$$
$$
\Lambda = \delta^{\mu}_{\nu} + s \Omega^{\mu}_{\nu} + O(s^2)
$$
Expand in linear order:
$$
(\delta^{\mu}_{\alpha} + s \Omega^{\mu}_{\alpha})\eta^{\alpha \beta}(\delta^{\nu}_{\beta} + s \Omega^{\nu}_{\beta}) = \eta^{\mu \nu}
$$
Consider $O(s)$:
$$
s\Omega^{\mu \nu} + s\Omega^{\nu \mu} = 0 
$$
so:
$$
\Omega_{\mu \nu} = - \Omega_{\nu \mu} 
$$
As this means the diagonal must be zero and it is symmetric about the diagonal, $\Omega$ only has six degrees of freedom. Now linearise the transformation of the scalar field:
$$
\phi(x) \rightarrow \phi(\Lambda^{-1} \cdot x)
$$
$$
(\Lambda^{-1})^{\mu}_{\nu} =  \delta^{\mu}_{\nu} - s \Omega^{\mu}_{\nu} + O(s^2)
$$
so to $O(s)$
$$
\phi(\Lambda^{-1} \cdot x) = \phi(x - s \Omega \cdot x + O(s^2)) = \phi(x) - s\Omega^{\mu}_{nu}x^{nu}\partial_{mu}\phi(x) + O(s^2)
$$
so the infinitesimal transformation is:
$$
        \phi(x) \rightarrow \phi(x) + \delta\phi(x)
  $$      
\begin{equation}
        \delta \phi(x) = -s \Omega^{\mu}_{\nu}x^{\nu}\partial_{\mu}\phi(x)
\end{equation}
\subsection{Noether's Theorem}
A continous symmetry implies a conserved current.
$$
\phi(x) \rightarrow \phi + \delta \phi(x)
$$
As the variation is local the transformation will just be in terms of the field and its derivatives at the point:
$$
\delta \phi(x) = X(\phi(x), \partial\phi(x))
$$
How does $\mathfrak{L}$ vary?\\
\textbf{Lorentz transformation}
$$
\phi(x) \rightarrow \phi(\Lambda^{-1}\cdot x)
$$
gives
$$
\delta \mathfrak{L} =  -s \Omega^{\mu}_{\nu}x^{\nu}\partial_{\mu}\mathfrak{L}(x) =\partial_{\mu}( -s \Omega^{\mu}_{\nu}x^{\nu}\mathfrak{L}(x))  
$$
above we used the fact that $\Omega$ is anti-symmetric to vanish the term $-s \Omega^{\mu}_{\nu}\partial_{\mu}(x^{\nu})\mathfrak{L}(x)$. So we can rewrite the variation of the lagrangian under the lorentz transformation as a total derivative.\\
\textbf{Translation}
$$
\phi(x) \rightarrow \phi(x-c)
$$
$$
\delta \phi(x) = X = - c^{\mu}\partial_{\mu}\phi(x)
$$
$$
\delta \mathfrak{L} =  - c^{\mu}\partial_{\mu}\mathfrak{L}(x)
$$
If the lagrangian transforms as a total derivative then the action will be invariant.\\
\textbf{General symmetry}
\begin{equation}
        \delta \mathfrak{L} = \partial_{\mu} F^{\mu}
\end{equation}
$$
F^{\mu} = F^{\mu} (\phi(x), \partial\phi(x))
$$
As:
$$
\parital S = \int_{\mathbb{R}^{3,1}} d^4x \delta \mathfrak{L} = \int_{\mathbb{R}^{3,1}} d^4x \partial_{\mu} F^{\mu} =  \int_{\partial (\mathbb{R}^{3,1} )} dS_{\mu} F^{\mu}
                $$
So if $\phi$, $\partial \phi$ decay fast enough then this goes to 0.
\subsubsection{Conserved current}
$$
j^{\mu} (x) = j^{\mu}(\phi(x), \partial\phi(x))
$$
is conserved if 
$$
\partial_{\mu} j^{\mu} = 0
$$
when $\phi$ obeys its E-L equations.
$$
j^{\mu}(x) = j^{\mu}(t,x) = (j^0(x,t), \bm J(x,t))
$$
$j^0$ is charge density and $\bm J$ is the current density.\\
Total charge in region $V < \mathbb{R}^3$ = $ Q(t) = \int_V dV j^0(t,x)$.\\
Rate of change of $Q(t)$ with conserved current`: $$\frac{dQ(t)}{dt} = \int_V dV \frac{\partial}{\parti`:w
al t} j^0(t,x)  = - \int_V dV \nabla \cdot \bm J = - \int_{\partial V} d \bm S \cdot \bm J $$
This means that whenever we have a conserved current we have a conserved charge, and any changes in charge are accounted for by the flux of current across the boundary.\\\\
\textbf{Proof of Noether's theorem}
$$
\partial \mathfrak{L} (x) = \frac{\partial \mathfrak{L}}{\partial \phi} \delta \phi + \frac{\partial \mathfrak{L}}{\partial(\partial_{\mu}\phi)} \delta (\partial_{\mu} \phi)
$$
$$
\partial \mathfrak{L} (x) = [\frac{\partial \mathfrak{L}}{\partial \phi} - \partial_{\mu} (\frac{\partial \mathfrak{L}}{\partial (\partial_{\mu} \phi)}) ] \delta \phi + \partial_{\mu} (\frac{\partial \mathfrak{L}}{\partial(\partial_{\mu} \phi)} \delta \phi)
$$
if $\phi$ obeys EL then
\begin{equation}
\partial \mathfrak{L} (x) = \partial_{\mu} (\frac{\partial \mathfrak{L} }{\partial(\partial_{\mu}\phi)} X(\phi, \partial \phi))
\end{equation}
As for any symmetry we have 
$$
\delta \mathfrak{L} = \partial_{\mu} F^{\mu}
$$
so define conserved current to be:
$$
j^{\mu} = \frac{\partial \mathfrak{L} }{\partial(\partial_{\mu}\phi_a)} X(\phi_a, \partial \phi_a) - F^{\mu}
$$
As both of these are equal to the variation of the lagrangian it will be conserved.
\section{Lecture 5}
Focus on spacetime translations:
$$
x^{\mu} \rightarrow x^{\mu} - \epsilon^{\mu}
$$
$$
\phi(x) \rightarrow \phi(x + \epsilon) \approx \phi(x) + \epsilon^{\mu}\partial_{\mu}\phi(x) + O(\epsilon^2)
$$
$$
\delta \phi(x) = + \epsilon^{\mu} \partial_{\mu} \phi(x)
$$
$$
\delta \mathfrak{L}(x) = + \epsilon^{\mu} \partial_{\mu} \mathfrak{L}(x)
$$
$$
\delta \phi(x)  = \epsilon^{\nu}X_{\nu}(\phi)
$$
$$
\delta \mathfrak{L} (x) = \epsilon^{\nu} \partial_{\mu} F^{\mu}_{\nu}(\phi)
$$
$$
X_{\nu} = \partial_{\nu} \phi, F^{\mu}_{\nu} = \delta^{\mu}_{\nu}\mathfrak{L}
$$
So this can give us 4 conserved currents(\textbf{energy-momentum tensor}):
\begin{equation}
        T^{\mu}_{\nu} = \frac{\partial \mathfrak{L}}{\partial (\partial_{\mu}\pphi)}\partial_{\nu} \phi - \delta ^{\mu}_{\nu} \mathfrak{L}
\end{equation}
$$
\partial_{\mu}T^{\mu}_{\nu} = 0
$$
Conserved charge that corresponds to energy comes from the zeroth component:
$$
E= \int_{\mathbb{R}^3} d^3x T^{00}
$$
similarly for translations in space:
$$
 P^i = \int_{\mathbb{R}^3} d^3x T^{0i}
$$
\subsection{Example: free scalar field/Klein Gordan}
$$
\mathfrak{L} = \frac{1}{2} (\partial_{\mu}\phi)(\partial^{\mu}\phi) - \frac{m^2}{2}\phi^2
$$
$$
T^{\mu \nu} = \partial^{\mu}\phi\partial^{\nu}\phi - \eta^{\mu \nu} \mathfrak{L}
$$
Look at conserved energy when $\mu = \nu = 0$.
$$
E = \int_{\mathbb{R}^3} d^3 T^{00} = \int_{\mathbb{R}^3} d^3 (\frac{1}{2}\dot\phi^2 + \frac{1}{2} |\nabla \phi|^2 + \frac{1}{2} m^2 \phi^2)
$$
This formula has several nice features as the energy is bounded below which is something we often want as we often have ground states. It also follows the same pattern as the Lagranagian in classical dynamics if the first term is interpretted as being the kinetic energy and the remaining as potential energy (it goes from $L= T-V $ to $E= T+V$).\\\\
The energy momentum tensor is clearly a symmetric tensor, now that is not neccessarily a general feature but in a system when you have a rotationals symmetry you can always .... didn't really make much sense have a look in the notes. Basically he was trying to say that energy doesn't really make an awful lot of sense until you couple it to gravity and then the energy momentum tensor becomes very important. \\\\
\subsection{Canonical Quantisation}
\subsubsection{Quick review of classical hamiltonian mechanics}
Consider non-relativistic particle ($m=1$) in potential $V(q)$:
$$
L(q(t),\dot q(t) ) = \frac{1}{2} \dot q^2 - V(q)
$$
Define momentum $p$ conjugate to $q$:
$$
p = \frac{\partial L}{\partial \dot q} = \dot q
$$
Define Legendre transform as
$$
H = p\dot q - L(q, \dot q)
$$
with $\dot q$ replaced with $p$ to give:
$$
H = \frac{1}{2} p^2 + V(q)
$$
Now consider system with N degrees of freedom:
$$
L = L(\{q_i(t)\}, \{\dot q_i(t)\})
$$
$$
H = \sum_{i} p_i\dot q_i - L
$$
and eliminate $\dot q$.\\\\
\textbf{Poisson Bracket}:  For $F = F(p,q)$ and $G = G(p,q)$, $\{F,G\} =  \sum_i \frac{\partial F}{\partial q_i}\frac{\partial G}{\partial p_i} - \frac{\partial F}{\partial p_i}\frac{\partial G}{\partial q_i}$
$$
\{q_i,q_j\} = \{p_i, p_j\} = 0, \{q_i, p_j\} = \delta_{ij}
$$
\textbf{State} of system at time $t=0$ is given by specifying your momentum and position ($q_i(0)$ and $p_i(0)$). So time evolution is completely governed by the hamiltonian. For any function $F(p,q)$, $$\dot F = \{F,H\}$$
In particular, choose $F=p_i$ or $q_i$ gives hamiltons equations. $$\dot q_i = \{q_i, H\} = \frac{\partial F}{\partial p_i}$$$$\dot p_i = \{p_i, H\} = -\frac{\partial F}{\partial q_i}$$
$$
\dot Q = 0 \iff \{H,Q\} = 0
$$
\section{Lecture 6}
\subsubsection{Quantization}
States are vectors in a Hilbert space. Quantization means turning classical functions into a hermitian operator. The basic rule as first written down by Dirac is:
$$
\{,\} \rightarrow \frac{1}{i\hbar}[,]
$$
replace poisson bracket with the commutator. e.g. can apply this recipe to the coordinates and their momentum
$$
q_i \rightarrow \hat q_i, p_i \rightarrow \hat p_i
$$
and 
$$
[\hat q_i, \hat q_j] = [\hat p_i, \hat p_j] = 0, [\hat q_i, \hat q_j] = i\hbar \delta_{ij}
$$
In general the problem of quantisation is finding a representation for this algebra which is in this case is the $\frac{\partial}{\partial x}$ representation of $p$.\\\\
The time evolution of the quantised system is described by the transformed Hamiltonian $\hat H( \hat p, \hat q)$. This often leads to ambiguity as the ordering of the $\hat p$ and $\hat q$ is important in $\hat H$ (most of this we will be able to brush under the carpet as we will mainly deal with free states). If we deal with the Schrödinger picture then the states are thought of as time dependent:
$$
-i\hbar \frac{\partial }{\partial t} \ket{\psi(t)} = \hat H \ket{\psi(t)}
$$
The special case is when $\ket{\psi(t)}$ is an eigenstate of the Hamiltonian where
$$
\hat H\ket{\psi} = E \ket{\psi}
$$
\textbf{Field Theory} is an infinite dimensional dynamical system. We think about this as replacing the dynamical variable $q_i(t)$ with $\phi(\bm x, t)$. So in the dynamical system we have a discrete label $i$ whereas the field has a continuous label $\bm x \in \mathbb{R}^3$. It turns out it is not valid to treat these the same but for the purposes of this course we can pretend it is possible. Though in AQFT you will find out you need to take an appropriate limit using something called a regulator. There are two types of infinite in this system: the IR (infrared) divergences associated with the infinite range of $x$ ('large distances'), the UV (ultraviolet) divergences are associated with the fact we have a continuous infinite that we can consider infinitely close together points ('short distances').\\\\
As we have a dynamical variable the next step is defining a conjugate momentum:
$$
\pi(\bm x,t) = \frac{\partial \mathfrak{L}(\bm x, t)}{\partial \dot \phi(\bm x, t)}
$$
Perform a Legendre transform to get a Hamiltonian density
$$
\mathcal{H} = \pi(\bm x,t) \dot \phi(\bm x,t) - \mathfrak{L}(\bm x,t)
$$
Just like in classical dynamics we can now eliminate the time derivative from the hamiltonian density using the definition of $\pi$ to give $\mathcal{H} = \mathcal{H}(\phi(\bm x,t), \pi(\bm x,t))$. Define the hamiltonian as:
$$
H = \int_{\mathbb{R}^3} d^3x \mathcal{H}(\bm x,t)
$$
\textbf{Example}
$$
\mathfrak{L} = \frac{1}{2} (\partial_{\mu}\phi)(\partial^{\mu}\phi - V(\phi)
$$
One problem with the hamiltonian treatment is it breaks lorentz invariance so we need to split this up into space and time components.
$$
\mathfrak{L} = \frac{1}{2} \dot \phi^2 - \frac{1}{2}|\nabla \phi| ^2 - V(\phi)
$$
Conjugate momentum
$$
\pi(\bm x, t) = \frac{\partial \mathfrak{L}}{\partial \dot \phi} = \dot \phi(\bm x, t)
$$
$$
\mathcal{H} = \pi \dot \phi - \mathfrak{L} = \frac{1}{2} \pi^2  + \frac{1}{2}|\nabla \phi| ^2 +V(\phi) = T^{00}
$$
$$
H = \int_{\mathbb{R}^3} d^3x \frac{1}{2} \pi^2  + \frac{1}{2}|\nabla \phi| ^2 +V(\phi) =  \int_{\mathbb{R}^3} d^3x T^{00} = E 
$$
Now lets quantise it. 
$$
\{\phi(\bm x, t), \pi(\bm y, t)\} = \delta^{(3)}(\bm x - \bm y)
$$
Suppress/fix the time coordinate for the moment.\\
\textbf{Canonical Quantisation}
$$
\phi(\bm x) \rightarrow \hat \phi(\bm x)
$$
$$
\pi(\bm x) \rightarrow \hat \pi(\bm x)
$$
$$
[\hat \phi(\bm x, t), \hat \pi(\bm y, t)] = i \hbar \delta^{(3)}(\bm x - \bm y)
$$
$$
H = \int_{\mathbb{R}^3} d^3x \frac{1}{2} \hat \pi^2  + \frac{1}{2}|\nabla \hat \phi| ^2 +V(\hat \phi) $$
except in the case of free field theory we aren't going to do this as describing the hilbert space is very hard so we will do it for free field theory and perturb around from it. In the above note that we normally work in natural units but we have left the $\hbar$ here to show the parallel with the classical case. The hamiltonian is very hard to understand here except in the case of free field theory as we are taking derivatives of operators and stuff all over the place.\\\\
Can solve the free field theory because of the superposition principle allowing us to effectively consider it as an infinite number of simple harmonic oscillators.
\subsection{Free Field Theory}
$$
\mathfrak{L} = \frac{1}{2} (\partial_{\mu}\phi)(\partial^{\mu}\phi) - \frac{m^2}{2}\phi^2
$$
Equation of motion:
$$
\partial_{\mu}\partial^{\mu}\phi + m^2 \phi  = 0
$$
As this is linear we can find general solution by solving with Fourier transform.
$$
\phi(\bm x, t) = \int \frac{d^3 p}{(2\pi)^3}e^{i\bm p \cdot \bm x} \tilde \phi( \bm p, t) 
$$
If we want $\phi(\bm x,t) \in \mathbb{R}$ then we need $\tilde \phi^*(\bm p,t) = \tilde \phi(-\bm p,t)$. The equation becomes:
$$
(\frac{\partial^2}{\partial t^2} + |\bm p|^2 + m^2) \tilde \phi(\bm p,t) = 0
$$
This is essentially the equation for simple harmonic motion so solutions are:
$$
\tilde \phi(\bm p, t) = A_{\bm p} e^{i \omega_{\bm p} t} + B_{\bm p} e^{-i \omega_{\bm p} t} 
$$
for $\omega_{\bm p} = \sqrt{|\bm p|^2 + m^2}$. In reality $ A_{\bm p}^* =  A_{-\bm p}$ to give a real field. Also the action becomes:
$$
S= \int dt \int d^3 x \mathfrak{L}(\bm x,t) = \frac{1}{2} \int dt \int d^3 p \tilde \phi^*(\bm p, t) (- \frac{\partial^2}{\partial t^2} - |\bm p|^2 - m^2)\tilde \phi(\bm p,t)
$$
This is what gives the sense of an infinite set of a decoupled simple harmonic oscillators, as for any fixed value of $\bm p$ this gives the action of the harmonic oscillator.
\section{Lecture 7}
\subsection{Quantum Simple harmonic oscillator}
$$
L = \frac{1}{2} \dot q ^2  - \frac{\omega^2}{2} q^2
$$
$$
H = \frac{1}{2} \dot p^2  + \frac{\omega^2}{2} q^2
$$
First off replace $q \rightarrrow \hat q$ and $p \rightarrow \hat p$ with $[\hat q, \hat p] = i\hbar$. Define creation and annihilation operators:
$$
\hat a = \sqrt{\frac{\omega}{2}} \hat q + \frac{i}{\sqrt{2\omega}} \hat p
$$
$$
\hat a^{\dagger} = \sqrt{\frac{\omega}{2}} \hat q - \frac{i}{\sqrt{2\omega}} \hat p
$$
\begin{equation}
[\hat a, \hat \a^{\dagger}] = \hbar
\end{equation}
Rule of ordering always have annihalition operators on the right
$$
\hat H = \frac{\omega}{2}(\hat a^{\dagger} \hat a +  \frac{\hbar}{2})
$$
$$
[\hat H, \hat a ] = -\omega \hbar \hat a
$$
$$
[\hat H, \hat a^{dagger} ] = \omega \hbar \hat a^{dagger}}
$$
Consider:
$$
\hat H \ket{E} = E \ket{E}
$$
Communation relations tell us
$$
\hat H(\hat a^{dagger} \ket{E} ) = (E+\hbar \omega) \hat a^{\dagger} \ket{E}
$$
$$
\hat H(\hat a \ket{E} ) = (E-\hbar \omega) \hat a \ket{E}
$$
Either spectrum unbounded below or have $a\ket{0} = 0$. Therefore:
$$
\hat H \ket{0} = \frac{\hbar \omega}{2} \ket{0}
$$
and the raising opertator gives the infinite tower of excited states
$$
\ket{n} = (\hat a^{\dagger}^n) \ket{0}
$$
$$
\hat H \ket{n} = (n+\frac{1}{2}) \hbar \omega \ket{n}
$$
Define number operator: $\hat N = \hat a^{\dagger}\hat a$
$$
\hat N\ket{n} = n\hat{n}, \hat H = \omega \hat N + \frac{1}{2} \hbar \omega
$$
\subsection{Quantisation}
$$
[\hat \phi(\bm x), \hat \pi (\bm y)] = i\delta^{(3)}(\bm x - \bm y)
$$
Dirac delta function
$$
\int_{\mathbb(R)^3} f(\bm x) \delta^{(3)}(\bm x - \bm y) = f(\bm y)
$$
$$
\delta^{(3)}(\bm x) = \int \frac{d^3p}{(2\pi)^3} e^{i \bm p \cdot \bm x}
$$
$$
\hat H = \int_{\mathbb{R}^3} d^3 x (\frac{\hat \pi^2}{2} + \frac{|\nabla \hat \phi|^2}{2} + \frac{m^2}{2}\hat \phi^2
$$
Write field in fourier space. This is a defintion that we then check produces thecorrect communator relations.
$$
\hat \phi(\bm x) = \int \frac{d^3p}{(2\pi)^3} [\hat a_{\bm p} e^{\bm p \cdot \bm x} + \hat a^{\dagger}_{\bm p} e^{-\bm p \cdot \bm x} ] \frac{1}{\sqrt{2\omega_{\bm p}}
$$
for $\omega_{\bm p} = \sqrt{|\bm p|^2 + m^2}$.
$$
\hat \pi(\bm x) = \int \frac{d^3p}{(2\pi)^3}  (-i) \sqrt{\frac{\omega_{\bm p}}{2}} [\hat a_{\bm p} e^{\bm p \cdot \bm x} - \hat a^{\dagger}_{\bm p} e^{-\bm p \cdot \bm x} ]}
$$
Make the claim that
$$
[\hat \phi(\bm x), \hat \phi(\bm y)] = [\hat \pi(\bm x), \hat \pi(\bm y)] =0, 
[\hat \phi(\bm x), \hat \pi (\bm y)] = i\delta^{(3)}(\bm x - \bm y)
$$
is equivalent to:
$$
[\hat a_{\bm p}, \hat a_{\bm q}] = [\hat a^{\dagger}_{\bm q}, \hat a^{\dagger}_{\bm p}] = 0
$$
$$
[\hat a_{\bm p}, \hat a^{\dagger}_{\bm q}]  = (2\pi)^3 \delta^{(3)}(\bm p - \bm q)
$$
Check that:
$$
[\hat \phi(\bm x), \hat \pi(\bm y)] = \frac{-i}{2}\int \frac{d^3p}{(2\pi)^3}\int \frac{d^3q}{(2\pi)^3} \sqrt{\frac{\omega_{\bm q}}{\omega_{\bm p}}[\hat a_{\bm p} e^{i\bm p \cdot \bm x} + \hat a^{\dagger}_{\bm p} e^{-i \bm p \cdot \bm x} , \hat a_{\bm q} e^{i\bm q \cdot \bm y} + \hat a^{\dagger}_{\bm q} e^{-i \bm q \cdot \bm y} ]
$$
$$
[,] = -e^{i\bm p \cdot \bm x - i \bm q \cdot \bm y}[\hat a_{\bm p} , \hat a^{\dagger}_{\bm q}] + e^{-i\bm p \cdot \bm x + i \bm q \cdot \bm y}[\hat a^{\dagger}_{\bm p} , \hat a_{\bm q}] = -(2\pi)^3 \delta^{(3)}(\bm p - \bm q) (e^{i\bm p \cdot \bm x - i \bm q \cdot y} + e^{-i\bm p \cdot \bm x + i \bm q \cdot y})
$$
first do $\int d^3 q$ which just sets $\bm p = \bm q$. Then collect up everything that is left as:
$$
[\hat \phi(\bm x), \hat \pi(\bm y)] = \frac{1}{2} \int \frac{d^3p}{(2\pi)^3} (e^{i\bm p \cdot \bm x - i \bm q \cdot y} + e^{-i\bm p \cdot \bm x + i \bm q \cdot y}) = i \delta^{(3)}(\bm x - \bm y)
$$ 
Above set is done by inspection using the integral expression for the delta function.\\\\
\section{Lecture 8}
Can subsitute these values for $\hat \phi$ and $\hat \pi$ into $H$ to give:
$$
\hat H = \frac{1}{4} \int \frac{d^3 p}{(2\pi)^3} \frac{1}{\omega_{\bm p}} (-\omega_{\bm p}^2 + |\bm p|^2 + m^2)(\hat a_{\bm p} \hat a_{\bm p} + \hat a^{\dagger}_{\bm p} \hat a^{\dagger}_{- \bm p})+ (\omega_{\bm p}^2 + |\bm p|^2 + m^2)(\hat a_{\bm p} \hat a^{\dagger}_{\bm p} + \hat a^{\dagger}_{\bm p} \hat a_{ \bm p})
$$
$$
\hat H = \frac{1}{2} \int \frac{d^3 p}{(2 \pi)^3} \omega_{\bm p} (\hat a_{\bm p}\hat a^{\dagger}_{\bm p} + \hat a^{\dagger}_{\bm p} \hat a_{\bm p})
$$
\subsection{The Vacuum}
This is an infinite set of non-interacting quantum simple harmonic oscillators. The state $\ket{0}$ has the lowest energy and is annihaliated by all of the lowering operators:
$$
\hat a_{\bm p} = 0 \forall \bm p \in \mathbb{R}^3
$$
$$
\hat H \ket{0} = E_0 \ket{0}
$$
$E_0$ is the vacuum energy. Use the same trick of reordering the operators so the annihilation operator is always to the right.
$$
\hat H = \frac{1}{2}  \int \frac{d^3 p}{(2 \pi)^3} \omega_{\bm p} (\hat a_{\bm p}\hat a^{\dagger}_{\bm p} + \hat a^{\dagger}_{\bm p} \hat a_{\bm p}) = \frac{1}{2}  \int \frac{d^3 p}{(2 \pi)^3} \omega_{\bm p} (2 \hat a^{\dagger}_{\bm p} \hat a_{\bm p} + (2\pi)^3\delta^{(3)}(0)) 
$$
This is our first sight of the infinites of quantum field theory. First write as normal ordered hamiltonian
$$
\hat H = \int \frac{d^3 p}{(2 \pi)^3}  \omega_{\bm p}  \hat a^{\dagger}_{\bm p} \hat a_{\bm p} +\int \frac{d^3 p}{2} \omega_{\bm p}  \delta^{(3)}(0) 
$$
Therefore vacuum energy is the remaining constant piece:
$$
E_0 = \frac{1}{2} \int d^3p \omega_{\bm p} \delta^{(3)}(0)
$$
As we have two types of infinity:\\
IR: Large distance (low energy)
$$
(2 \pi)^3 \delta^{(3)}(\bm p) = \int_{\mathbb{R}} d^4 x e^{i \bm x \cdot \bm p}
$$
divergence of $\bm p = 0$ comes from the infinite volume of space. So the cure for the infrared infinity is to put the theory in a large box of size L and then consider $L \rightarrow \infty$ by taking $x \in [\frac{-L}{2}, \frac{L}{2}]$.
$$
(2\pi)^3 \delta^{(3)}(\bm p) = \lim_{L\rightarrow \infty} \int^{\frac{L}{2}}_{-\frac{L}{2}} d^3 x e^{i \bm x \cdot \bm p} $$
$$
(2\pi)^3 \delta^{(3)}(0) = \lim_{L \rightarrow \inty} V_L
$$
So this infinity is just reflecting the fact that the volume of the box grows to infinty. All this tells us is we should be calculating an energy density rather than an energy as if we have constant energy density it will give us infinite energy when integrated over infinite space. So we should consider the energy density:
$$
\epsilon_{0} = \lim_{L \rightarrow \frac{E_{(L)}_0}{V_L}
$$
$$
E_0^{(L)} = \int_{-\frac{L}{2}}^{\frac{L}{2}} d^4 x \int \frac{d^3 p}{(2\pi)^3} \frac{1}{2} \omega_{\bm p} \implies \epsilon_0 \int \frac{d^3p}{(2\pi)^3} \frac{1}{2} \omega_{\bm p}
$$
with $\omega_{\bm p} = \sqrt{|\bm p|^2 + m^2}$. Infinity covers from large $|\bm p|$ with:
$$
\epsilon_0 \sim \int |\bm p|^3 d|\bm p| = \infty
$$
This is giving a UV divergences as we are adding up infinitily small distances (infinity high frequencies and momentum).\\\\
Can cure UV diveregence by introducing a UV cutoff so short distacne rather than large distance. So introduce $\lambda >> m >> L^{-1}$ with $L$ is size of box.
$$
\epsilon_0^{(\lambda)} = \int_{|\bm p|<\lambda} \frac{d^3 p}{(2\pi)^3} \frac{1}{2} \omega_{\bm p} = \frac{4 \pi}{2(2\pi)^3} \int^{\lambda}_{0} |\bm p|^2 \sqrt{|\bm p|^2 + m^2} d |\bm p| = \frac{1}{16\pi^2} \lambda^4(1+O(\frac{m^2}{\lambda^2}))
$$
Alternatively put theory on a space time lattice with lattice spacing choosen so $a << m^{-1}$, find $\epsilon^{(a)} \sim \frac{1}{a^4}$.\\\\
One viewpoint is called effective field theory:
You need to define QFT as a continuum limit (a limit in which the cutoff goes to infinity/lattice spacing goes to zero). In order to define this limit you need to identify the quantities that hold fixed in that limit. Theory need to be valid up to some maximum scale.\\\\
\textbf{Specific to vacuum energy}: This is a very special type of divergence. Just and additive constant we add to the energy. Can remove this additive constant by considering a different quantisation with ordering chosen so it is normal ordering of all products of field operators ($\hat \phi$) and momentum ($\hat \pi$) operators. 
\section{Lecture 9}
Normal ordering symbol: $:X:$ place all annihilation operators to the right. $:\hat a \hat a^{\daggger}^2 \hat a \hat a^{\dagger}: = \hat a^{\dagger}^3 \hat a^2$
$$
:\hat H: = \frac{1}{2}  \int \frac{d^3 p}{(2 \pi)^3} \omega_{\bm p} :(\hat a_{\bm p}\hat a^{\dagger}_{\bm p} + \hat a^{\dagger}_{\bm p} \hat a_{\bm p}): =   \int \frac{d^3 p}{(2 \pi)^3} \omega_{\bm p} \hat a^{\dagger}_{\bm p} \hat a_{\bm p}
$$
This implies $(\epsilon_0)_{\text{Normal}} = 0$ as we can see that as $\hat a\ket{0} = 0$ so this will annihilate the ground state.\\\\
We care about the ground state energy as gravity cares about energy and says that energy gravitates. The energy momentum tensor appears on the RHS of Einstiens equations seen below:
$$
R_{\mu \nu} - \frac{1}{2}Rg_{\mu \nu} = 8 \pi G T_{\mu \nu}
$$
We might reasonably ask that if we have some QFT coupled to gravity then:
$$
\bra{0} \hat T_{\mu \nu} \ket{0} = \epsilon_0 g_{\mu \nu}
$$
This leads to problems as if you allow the expectation to the RHS of Einsteins equations, then it contributes a cosomological constant to Einstiens equations. Measurement gave a value to the cosmological constant of $\lambda \sim (10^{-3} eV)^2$ which is very small compared to the energy scales of HEP. We might consider QFT with a UV cuttoff $\lambda$ of $\lambda \sim M_{pl} = \sqrt{\frac{\hbar c}{G}}$ then we would get $\epsilon_0 \sim \lambda^4$ which would give $\lambda \sim (10^{28} eV)^2$. So either QFT does not predict the vacuum energy or it does but it is many many orders of magnitude off from the observation.\\
\textbf{Casmir Effect}:
If we put two metal plates a distance $d$ apart, we can repeat the calcualtion of the energy density in the presence of these plates. Now we need to take into account the boundary conditions of the field that the electric field has to vanish on the plate. So in this context we get $\epsilon_0 = \epsilon_0(d)$ which therefore exerts a force on these plates and can correctly predict the force on these plates (so vacuum energy is a real thing that can be measured). \\\\
\subsection{Excited states}
Define:
$$
\ket{\bm p} = \hat a^{\dagger} \ket{0} \forall \bm p \in \mathbb{R}^3
$$
\textbf{Energy}
$$
\hat H - E_0 =  \int \frac{d^3 p}{(2 \pi)^3} \omega_{\bm p} \hat a^{\dagger}_{\bm p} \hat a_{\bm p}
$$
$$
(\hat H - E_0) \ket{\bm p} = (\hat H- E_0) \hat a^{\dagger} \ket{0} = \hat a^{\dagger} (\hat H- E_0) \ket{0}  + [(\hat H- E_0),\hat a^{\dagger}] \ket{0}
$$
$$
[(\hat H- E_0),\hat a^{\dagger}] =  \int \frac{d^3 p}{(2 \pi)^3} \omega_{\bm p} [ \hat a^{\dagger}_{\bm p} \hat a_{\bm p}, \hat a^{\dagger}_{\bm q}] =  \int \frac{d^3 p}{(2 \pi)^3} \omega_{\bm p} \hat a^{\dagger}_{\bm p} [ \hat a_{\bm p}, \hat a^{\dagger}_{\bm q}] =  \int \frac{d^3 p}{(2 \pi)^3} \omega_{\bm p} \hat a^{\dagger}_{\bm p} (2 \pi)^3 \delta^{(3)}(\bm q - \bm p)
$$
$$
(\hat H - E_0) \ket{\bm p}  = \omega_{\bm p} \hat a^{\dagger}_{\bm p} \ket{0} = \omega_{\bm p} \ket{ \bm p}
$$
$E_1 = \omega_{\bm p} = \sqrt{|\bm p|^2 + m^2}$\\
Consider classical momnetum of the field: $p^i = \int d^3x T^{0i}$ and $T^{\mu \nu} = \partial^{\mu}} \phi \partial^{\nu} \phi - \eta^{\mu \nu} \mathfrak{L}$. This is converted into a quantum operator on ES2 to give result:
$$
\hat{\bm p} = \int \frac{d^3 p}{(2\pi)^3} \bm p \hat{a^{\dagger}_{\bm p}} \hat a_{\bm p}
$$
$$
\hat{\bm p} \ket{\bm p} = \bm p \ket{\bm p}
$$
Therefore, the state $\ket{\bm p}$ are corresponds to a relativistic particle of mass m, spin 0, momentum $\bm p$ and energy $E_{\bm p} = \sqrt{|\bm p|^2 + m^2}$. Do the same process of this to angular momentum, and you get that there must be $\bm J \ket{0} = 0$ for $\ket{0}$ of a particle in its rest frame which confirms it is a spin 0 particle.
\subsection{Multiparticle states}
We can construct states correpsonding to any number ofthese particles by simply acting by a lot of operators:
$$
\ket{\bm p_1, \bm p_2,... , \bm p_n} = \hat a^{\dagger}_{\bm p_1} \hat a^{\dagger}_{\bm p_2} ... \hat a^{\dagger}_{\bm p_n}  \ket{0}
$$
This gives us energy $E- E_0 = \sum_{a=1}^n E_{\bm p_a}$ and momentum $\bm p = \sum_{a=1}^n \bm p_a$. These particles must be identical particles as:
$$
\ket{\bm p_1, \bm p_2} = \hat a^{\dagger}_{\bm p_1} \hat a^{\dagger}_{\bm p_2}  \ket{0} = \hat a^{\dagger}_{\bm p_2} \hat a^{\dagger}_{\bm p_1} \ket{0} = \ket{\bm p_2, \bm p_1} 
$$
\section{Lecture 10}
Introduce particle number operator:
$$
\hat N = \int \frac{d^3p}{(2 \pi)^3} \hat a^{\dagger}_{\bm p} \hat a_{\bm p}
$$
$$
\hat N \ket{\bm p_1, \bm p_2,... , \bm p_n} = n \ket{\bm p_1, \bm p_2,... , \bm p_n} 
$$
for free field $\hat H$ is conserved quanitity: $[\hat H, \hat N] = 0$. THis is not the case when we have interactions
\subsection{Normalisation}
We can choose without any loss of generality to take our vacuum state to have norm 1: $\bra{0}\ket{0} =1$ and
$$
\bra{p} \ket{q} = \bra{0} \hat a_{\bm p} \hat a^{\dagger}_{\bm p} \ket{0} = (2\pi)^3 \delta^{(3)}(\bm p - \bm q)
$$
We want to change this normalisation system to make it lorentz invariant (above is not lorentz covariant). Instead of considering directly the overlap of the states we will consider the completeness identity:
$$
\hat I = \int \frac{d^3p}{(2 \pi)^3} \bra{\bm p}\ket{\bm p}
$$
This will help us to renormalise the states to a lorentz invariant way as we need to change this measure by changing this completeness identity as the unit operator is by definition lorentz invariant. $d^3p$ is an integral over the spatial components so is not lorentz invariant, so we could try and reforumalte this as the below \textbf{relativistic integration measure}:
$$
\int d^4 p \delta (p_{\mu}p^{\mu} - m^2) \theta(P_0)
$$
First note the $\delta$- function and the $\theta$ function (picks out the positive root): impose the constraints 
$$
p^{\mu}p_{\mu} = m^2, p^0 > 0 \implies P_0 = + \sqrt{|\bm p|^2 + m^2} = E_{\bm p}
$$
$$
\int d\mu_p = \int d^3 p \int^{\infty}_0 \frac{d(p_0)^2}{2 p_0} \delta (p_o^2 - |\bm p|^2 - m^2) = \int \frac{d^3 p}{2 E_{\bm p}}
$$
Rewrite the completeness relation:
$$
\ket{\bm p^4} = \sqrt{2 E_{\bm p}} \ket{\bm p}
$$
$$
\hat I = \int \frac{d^3p}{(2 \pi)^3} \frac{1}{2 E_{\bm p}} \bra{\bm p^4}\ket{\bm p^4} = \int \frac{d\mu_p}{(2 \pi)^3} \bra{\bm p^4}\ket{\bm p^4} $$
For multiparticle states are renormalised to:
$$
\Pi_{a=1}^n (\sqrt{2 E_{\bm p_a}}) \times \ket{\bm p_1, \bm p_2,... , \bm p_n}
$$
\subsection{Complex Scalar Fields}
The details are only change in very small places and we will now highlight these main places:
$$
 \mathfrack{L} = \frac{1}{2}\partial_{\mu}\phi^* \partial^{\mu} \phi - m^2 \psi^* \psi$$
 can rewrite interms of real and imaginary parts and what you get is the sum of the lagrangian for two real scalar fields $\phi_1(x)$ and $\phi_2(x)$ where $\psi= \frac{\phi_1(x) + i\phi_2(x)}{\sqrt{2}}$. This makes manifest a global symmetry:
 $$
 \psi \rightarrow e^{i\alpha}\psi, \psi^* \rightarrow e^{-i\alpha}\psi^*
 $$
 This gives rise to a conserved current that you would not have if you just had one field and it is:
 $$
 j^{\mu} = i(\partial^{\mu}\psi^*) \psi - i \psi^* (\partial^{\mu} \psi)
 $$
 We can write down hamiltonian formulation:
 $$
 \pi = \frac{\mathfrak{L}}{\partial \dot \psi} = \dot \psi^*
 $$
 Try quantisation:
 $$
 \psi(\bm x, t) \rightarrow \hat \psi(\bm x,t)
 $$
 $$
 \psi^*(\bm x, t) \rightarrow \hat \psi^{\dagger}(\bm x,t)
 $$
 $$
 [\hat \psi(\bm x) , \hat \pi(\bm y)] = i \delta^{(3)}(\bm x - \bm y)
 $$
 $$
 [\hat \psi^{\dagger}(\bm x) , \hat \pi^{\dagger}(\bm y)] = i \delta^{(3)}(\bm x - \bm y)
 $$
 Now do mode expansion:
 $$
 \hat \psi(\bm x) = \int \frac{dp^3}{(2 \pi)^3} \frac{1}{\sqrt{2 E_{\bm p}}} (\hat b_{\bm p} e^{i \bm p \cdot \bm x} + \bm c^{\dagger}_{\bm p} e^{-\bm p\cdot \bm x})
 $$
 get e.g.
 $$
 [\bm b_{\bm p}, \bm b^{\dagger}_{\bm q}] = (2\pi)^3 \delta^{(3)}(\bm p - \bm q)
 $$
 $$
 \ket{\bm p, +} = \hat \bm c^{\dagger}_{\bm p}\ket{0}
 $$
 $$
 \ket{\bm p, -} = \hat \bm b^{\dagger}_{\bm p}\ket{0}
 $$

 $$
 :\hat H:  =   \int \frac{d^3 p}{(2 \pi)^3} \omega_{\bm p} ( \hat b^{\dagger}_{\bm p} \hat b_{\bm p}+ \hat c^{\dagger}_{\bm p} \hat c_{\bm p}))
 $$
 So we have a $U(1)$ conserved charge by starting from the classical charge we can get:
 $$
 \hat Q = i\int d^3 x (\hat \pi \hat \psi - \hat \psi^{\dagger} \hat \pi^{\dagger}) =\int \frac{d^3 p}{(2 \pi)^3} \omega_{\bm p} ( \hat b^{\dagger}_{\bm p} \hat b_{\bm p}- \hat c^{\dagger}_{\bm p} \hat c_{\bm p}))
 $$
 To check this is conserved we have to check $[\hat Q, \hat H] =0$ which is easy as all the things commute. If we evaluate the conserved charge:
 $$
 \hat Q  \ket{\bm p, \pm} =  \pm \ket{\bm p, \pm}
 $$
 So for the complex scalar field we have two particles that in addtion to all the quantum numbers we have a charge number which distinguish between particle and antiparticle. We have operators that conserve the $U(1)$ charge. Even in the interacting theory this symmetry is conserved.
 \subsection{Time dependance}
 So far we have worked in the schrodinger picture, so really:
 $$
 \hat \phi(\bm x) = \hat \phi_S(\bm x)
 $$
 so there is no time dependence in the field operator the time dependance all lives in the space.
 \section{Lecture 10}
 $$
 i \frac{d}{dt} \ket{\bm p(t)}_S = :H:\ket{\bm p(t)}_S = E_{\bm p} \ket{\bm p(t)}_S
 $$
 implies
 $$
 \ket{\bm p(t)}_S = e^{-i E_{\bm p}t}\ket{\bm p(0)}_S
 $$
 \textbf{Heisenburg picture}\\
 States (time independent):
 $$
 \ket{\psi}_H = e^{i\hat H t}\ket{\psi(t)}_S
 $$
 Operators (time dependent):
 $$
 \hat O_H = e^{i \hat H t} \hat O_S e^{-i\hat H t}
 $$
 Define Heinsburg picture field operator:
 $$
 \hat \phi(x) = e^{i\hat Ht} \hat \phi(\bm x)e^{-i\hat H t}
 $$
 LHS x is four vector, RHS is three vector (I think not sure)\\
 This obeys the Klein-Gordon operator. As shown in ES2 and below using mode expansion:
 $$
 \hat \phi(x) = \int \frac{d^3 p}{(2\pi)^3} \frac{1}{\sqrt{2E_{\bm p}}} ( (\hat a_{\bm p})_H e^{i \bm p \cdot \bm x} + (\hat a^{\dagger}_{\bm p})e^{-i \bm p \cdot \bm x})
 $$
 $$
 (\hat a_{\bm p})_H = e^{i \hat H t} \hat a_{\bm p} e^{-i \hat H t} \implies \frac{d}{dt} (\hat a_{\bm p})_H = i [\hat H, (\hat a_{\bm p})_H] = -i E_{\bm p}(\hat a_{\bm p})_H
 $$
 Therefore:
 $$
 (\hat a_{\bm p})_H (t) = e^{-i E_{\bm p} t} (\hat a_{\bm p})(0)
 $$
 similarly 
 $$
 (\hat a^{\dagger}_{\bm p})_H (t) = e^{+i E_{\bm p} t} (\hat a^{\dagger}_{\bm p})(0)
 $$
 so (below equation needs fixing watch the 15 min of lecture 10)
 $$
 \hat \phi(x) = \int \frac{d^3 p}{(2\pi)^3} \frac{1}{\sqrt{2E_{\bm p}}} ( (\hat a_{\bm p})_H e^{i \bm p \cdot \bm x} + (\hat a^{\dagger}_{\bm p})e^{-i \bm p \cdot \bm x})
$$
 $$
 \bm p \cdot \bm x = p_{\mu}x^{\mu} = E_{\bm p}t - \bm p \cdot \bm x
 $$
 \subsection{Interating QFT}

 $$
 \mathfrak{L} = \frac{1}{2} \partial_{\mu} \tilde \phi \partial^{\mu} \tilde \phi - V(\phi)
 $$
 implies hamiltonian
 $$
 H = \int d^3 x\frac{1}{2} \dot{ \tilde \phi}^2 + \frac{1}{2} |\nabla \tilde \phi|^2 + V (\tilde \phi)
 $$
 If we want to have a sensible theory we want it to be bounded below at some point giving the lowest energy:
 $$
 \tilde \phi(\bm x, t) = \tilde \phi_0
 $$
This must be a minimium point so:
\begin{equation}
        \frac{\partial V}{\partial \tilde \phi}|_{\tilde \phi = \tilde \phi_0} = 0
\end{equation}
\begin{equation}
\frac{\partial^2 V}{\partial \tilde \phi^2}|_{\tilde \phi = \tilde \phi_0} \geq 0
\end{equation}
WLOG take $V(\phi_0) = 0$. Now expand the field around the minimium:
$$
\phi(x) = \tilde \phi(x) - \tilde \phi_0
$$
Therefore:
$$
V(\tilde \phi) = \sum_{n=0}^{\infty} \frac{1}{n!} \frac{\partial^n V}{\partial \tilde \phi^n}|_{\tilde \phi = \tilde \phi_0} \phi^n
$$
The result is a lagrangian of the following form:
$$
\mathfrak{L} = \mathfrak{L}_0 + \mathfrak{L_{int}}
$$
with $\mathfrak{L}_0$ being the free terms and the $\mathfrak{L_{int}}$ being the interacting terms (all generated by this expansion around the minimium).
$$
\mathfrak{L}_0 = \frac{1}{2} \partial_{\mu} \phi \partial^{\mu} \phi - \frac{m^2}{2} \phi^2
$$
$$
m^2 = \frac{\partial^2 V}{\partial \tilde \phi^2}|_{\tilde \phi = \tilde \phi_0} 
$$
$$
\mathfrak{L_{int}} = - \sum_{n=3}^{\infty{ \frac{\lambda_n}{n!} \phi^n
$$
$$
\lambda_n = \frac{\partial^n V}{\partial \tilde \phi^n}|_{\tilde \phi = \tilde \phi_0}
$$
If we include general interacting terms there is no way to make progress, we need to work in a perturbation expansion $\lambda_n$ telling us how far we deviate from the free state. These $\lambda_n$ are "coupling" constants and are parameters of the theory that we put in. We can only hope to calculate it when it is close to the free field theory (when the coupling constants are "small".\\
\textbf{Dimensional analysis}\\
$$
[S] = 0, [x] = -1, [\partial_\mu] = +1
$$
$$
S = \int_{\mathbb{R}^{3,1}} d^4 x \mathfrak{L}
$$
so $[\mathfrak{L}] = 4$ which implies that:
$$
[\partial_{\mu} \phi \partial^{\mu} \phi] =4 \implies [\phi] = +1
$$
$$[\frac{\lambda_n \phi_n}{n!}] = 4 \implies [\lambda_n] = 4-n$$
so if $n\neq 4$ then $[\lambda_n] \neq 0$.\\\\
Consider process with $E \geq m$. We must be expanding in an effective dimensionless parameter, so cannot be just $\lambda, \lambda^2$ as that would have different dimensions so need to add a power of energy into the mix.
$$
\tilde \lambda_n = \lambda_n (E)^{n-4}
$$
This gives $[\tilde \lambda] = 0$ so can expand nicely. Consider different cases:\\
If $n<4$ then called \textbf{relevant coupling}. This will become weakly coupled at high energy as if $\lambda_n << (E)^{4-n)$ then $\tilde \lambda_n <<1$. If we choose $\lambda_n << (m)^{4-n}$ then the coupling can be small for all energies, so perturbation theory is good for all energy scales:\\\\
If $n=4$ called \textbf{marginal coupling}. Perturbation theory is good for $\lambda_4 <<1$.\\\\
If $n>4$ it is called \textbf{irrevelent coupling}. Correspondingly it means that compared to the relevant case the high and low energy cases are swapped around. So in general the coupling constant can be small at low energy, but if you go to high enough energy the coupling constant will be large. So perturbation theory is only good at $\lambda << (E)^{4-n}$. The key fact is that irrevelent couplings lead to non-renormalisable theories.\\\\
\textbf{Examples}:\\
$\phi^4$-theory: 
$$
\mathfrak{L} = \frac{1}{2} \partial_{\mu} \phi \partial^{\mu} \phi - \frac{m^2}{2} \phi^2 - \frac{\lambda}{4!}\phi^4
$$
As $n=4$ pertrubation theory is good for $\lambda<<1$. With the exception of adding a $\phi^3$ term this is the only renormisiable Lagrangian of a single scalar field.
\subsubsection{Scalar-Yukawa Theory}
$$
\phi : \mathbb{R}^{3,1} \rightarrow \mathbb{R}
$$
$$
\psi: \mathbb{R}^{3,1} \rightarrow \mathbb{C}
$$
$$
\mathfrak{L} = \frac{1}{2} \partial_{\mu} \phi \partial^{\mu} \phi - \frac{m^2}{2} \phi^2 + \frac{1}{2} \partial_{\mu} \psi \partial^{\mu} \psi^* - \frac{M^2}{2} \psi \psi^* - g \psi^* \psi \phi 
$$
This is will be weakly coupled if $|g| << m, M$.
\section{Lecture 11}
\subsection{Scattering}
Looking for the amplitude of the transition from an initial state to a final state:
$A_{i \rightarrow f} = \lim_{T\rightarrow \infty} [_S\bra{f, t= +\frac{T}{2}}e^{-i\hat H T}\ket{i, t= -\frac{T}{2}}_S]$. This will be constrained by the overall conservation laws like energy, momentum and charge. We are going to think about our theory perturbatively, so $\hat H$ is thought of as:
$$
\hat H = \hat H_0 + \hat H_{int}
$$
For example in $\phi^4$- theory:
$$
\hat H_0 = \frac{1}{2}\int_{\mathbb{R}^3} d^3x (\hat \pi^2 + |\nabla \hat \phi|^2 + m^2 \hat \phi^2)
$$
$$
\hat H_{int} = \frac{\lambda}{4!} \int_{\mathbb{R}^3} d^3x \hat \phi^4
$$
KEY ASSUMPTION: the particles are free in the infinite past and the infinite future (independant of thinking about perturbation theory just a consequence of locality). This implies that $\ket{i}$ and $\ket{f}$ are eigenstates of $\hat H_0$. You can justify this but we won't as it is a very involved dicusssion that can be found in textbooks.\\\\
We are going to work in the \textbf{interaction picture} as it is very convenient for perturbation theory. It is Heinsberg with respect to $\hat H_0$ but Schordinger with respect to $\hat H_{int}$:\\\\
States:
$$
\ket{\phi(t)}_{I} = e^{i\hat H_0 t} \ket{\psi(t)}_S
$$
Operators:
$$
\hat O_I(t) = e^{i\hat H_0 t} \hat O_S e^{-i \hat H_0 t}
$$
Time evolution:\\
For schordinger you get:
$$
i \frac{d}{dt}\ket{\psi(t)}_S = \hat H \ket{\psi(t)}_S \implies \ket{\psi(t)}_S = e^{-i\hat H t} \ket{\psi(0)}_S
$$
Interaction picture:
$$
i\frac{d}{dt}\ket{\psi(t)}_I = i \frac{d}{dt}(e^{i\hat H_0 t} \ket{\psi(t)}_S) = - \hat H_0 e^{i \hat H_0 t} \ket{\psi(t)}_S + e^{i\hat H_0 t} i \frac{d}{dt} \ket{\psi(t)}_S 
$$
$$
 i\frac{d}{dt}\ket{\psi(t)}_I = - \hat H_0 e^{i \hat H_0 t} \ket{\psi(t)}_S +  e^{i\hat H_0 t} \hat H \ket{\psi(t)}_S =  e^{i\hat H_0 t} \hat H_{int} \ket{\psi(t)}_S
$$
\begin{equation}
        i \frac{d}{dt} \ket{\psi(t)}_I = \hat H_I (t) \ket{\psi(t)}_I
\end{equation}
for $\hat H_I = e^{i \hat H_0 t}\hat H_{int} e^{-i \hat H_0 t}$. Compare with ODE:
$$
i \frac{d}{dt} F(t) = g(t) F(t) \implies F(t) = e^{\frac{1}{i}\int^t dt' g(t')}F(0) 
$$
The order of $H$ is important as $[H(t), H(t')] \neq 0$ in general which makes this more complex than the ODE case. As can be seen by guessing the solution: 
$$
\ket{\psi(t)} = e^{\frac{1}{i}\int^t dt' \hat H_I(t')}\ket{\psi(0)} 
$$
This DOES NOT WORK though as can be seen by subsituting it into 12. There is an easy fix though, for any opeartor valued function $\hat O(t)$ define the following time ordering symbol:
$$
T[\hat O(t_1)\hat O(t_2)] = \begin{cases}\hat O(t_1) \hat O(t_2) & t_1 \geq t_2\\
\hat O(t_2) \hat O(t_1) & t_1 \leq t_2
\end{cases}
$$
So a valid solution is actually:
$$
\ket{\psi(t)}_I = T(e^{\frac{1}{i}\int^t dt' \hat H_I(t')})\ket{\psi(0)} 
$$
Very slick two line proof in David Tongs notes, now we will check that the first couple terms match:
$$
T(e^{\frac{1}{i}\int^t dt' \hat H_I(t')}) = T(I + \frac{1}{i}\int^t dt' \hat H_I(t')} - \frac{1}{2} \int^t \int^t dt'dt'' (\hat H_I(t')\hat H_I(t'') )+ ...)
$$
$$
T(e^{\frac{1}{i}\int^t dt' \hat H_I(t')}) = I + \frac{1}{i}\int^t dt' \hat H_I(t') - \int^t  dt' \int^{t'} dt'' (\hat H_I(t')\hat H_I(t'') )+ ...
$$
Check equation 12:
$$
i \frac{d}{dt}T(e^{\frac{1}{i}\int^t dt' \hat H_I(t')})\ket{\psi(0)} = (\hat H_I(t) + \hat H_I(t) \frac{1}{i}\int^t dt' \hat H_I(t') +...) \ket{\psi(0)} 
$$
$$
i \frac{d}{dt}T(e^{\frac{1}{i}\int^t dt' \hat H_I(t')})\ket{\psi(0)} = \hat H_I(t) (I +  \frac{1}{i}\int^t dt' \hat H_I(t') +...) \ket{\psi(0)} \approx \hat H_I(t) T(e^{\frac{1}{i}\int^t dt' \hat H_I(t')})\ket{\psi(0)} 
$$
So the amplitude can be written in the interaction picture: $$A_{i \rightarrow f} = \lim_{T\rightarrow \infty} [_S\bra{f,t=+\frac{T}{2}}\ket{i,t=\frac{T}{2}}_S] = \lim_{T\rightarrow \infty}[_I\bra{f,t=\frac{T}{2}}\ket{i,t=\frac{T}{2}_I]} =  \lim_{T\rightarrow \infty}[_I\bra{f,t}$$
                $$ 
                $$
A_{i \rightarrow f} = \lim_{T\rightarrow \infty} [_S\bra{f,t=+\frac{T}{2}}\ket{i,t=\frac{T}{2}}_S]  T(e^{\frac{1}{i}\int^{\frac{T}{2}}_{-\frac{T}{2}} dt' \hat H_I(t')}) \ket{i,t=-\frac{T}{2}_I}]  
                $$
        $$
        A_{i \rightarrow f} = \lim_{T\rightarrow \infty} [_S\bra{f,t=+\frac{T}{2}}\ket{i,t=\frac{T}{2}}_S] =\frac{T}{2}S\ket{i,t=-\frac{T}{2}_I}]  
$$
 THESE LAST FEW EQUATIONS MESSED UP LOOK AT LAST FIVE MINS OF LECTURE 11\\\\
 \section{Lecture 12}
 \subsection{Scattering in $\phi^4$ theory}
$$
\hat H_0 = \frac{1}{2}\int_{\mathbb{R}^3} d^3x (\hat \pi^2 + |\nabla \hat \phi|^2 + m^2 \hat \phi^2)
$$
$$
\hat H_{int} = \frac{\lambda}{4!} \int_{\mathbb{R}^3} d^3x \hat \phi^4
$$
$$
\hat H_I(t) = e^{i\hat H_0 t} \hat H_{int} e^{-i\hat H_0 t} = \int d^3 x (e^{i\hat H_0 t} \hat \phi(x) e^{-i\hat H_0t })^4= \int d^3 \hat \phi_I^4 
$$
$$
\hat \phi(x) = \hat \phi_+ (x) + \hat \phi_- (x)
$$
$$
\hat \phi_+(x) = \int \frac{d^3 p}{(2\pi)^3} \frac{1}{\sqrt{2E_{\bm p}}} \hat a_{\bm p}e^{-ip \cdot x}
$$
$$
p^{\mu} = (E_{\bm p},\bm p0
$$
$$
p \cdot x - p_{\mu}x^{\mu} = E_{\bm p} t - \bm p \cdot \bm x
$$
$$
\hat \phi_-(x) = \int \frac{d^3 p}{(2\pi)^3} \frac{1}{\sqrt{2E_{\bm p}}} \hat a^{\dagger}_{\bm p}e^{ip \cdot x}
$$
Think about initial state as $\ket{i} = \ket{p_1, ..., p_{n_1}} = \Pi_{a=1}^{n_i} \sqrt{2 E_{\bm p_a}} \hat a^{\dagger}_{\bm p_a} \ket{0}$ and the final state as $\ket{f} = \ket{p_1, ..., p_{n_f}} = \Pi_{a=1}^{n_f} \sqrt{2 E_{\bm p_a}} \hat a^{\dagger}_{\bm p_a} \ket{0}$. So:
$$
\Lambda_{i \rightarrow f} = \bra{f} T[\exp(\frac{1}{i} \frac{\lambda}{4!} \int_{\mathbb{R}^3} d^4x \hat \phi_I^4)] \ket{i}
$$
To work this out explictly we will have to expand the exponential and work it out order by order:
$$
A_{i \rightarrow f} = \sum_{l=0}^{\infty} \lambda^l A_{i \rightarrow f}^{(l)}
$$
$$
A_{i \rightarrow f}^{(l)} = \frac{1}{l!}(\frac{1}{i(4!)})^l \int d^4 x_1 ... \int d^4 x_l
\bra{f} T[\hat \phi^4(x_1) \hat \phi^4(x_2) ... \hat \phi^4(x_l)] \ket{i}
$$
Need to convert T ordering into normal ordering. Simplest case:
$$
T_{\alpha} = T[\hat \phi(x) \hat \phi(y)]
$$
$$
T_{\alpha} = \hat \phi(x) \hat \phi(y) = (\hat \phi^+(x) + \hat \phi^-(x)) (\hat \phi^+(y) + \hat \phi^-(y)) = \hat \phi^+(x)\hat \phi^+(y) + \hat \phi^+(x)\hat \phi^-(y) + \hat \phi^-(x)\hat \phi^+(y) + \hat \phi^-(x)\hat \phi^-(y)
$$
$$
T_{\alpha} = \hat \phi^+(x)\hat \phi^+(y) + \hat \phi^-(x)\hat \phi^+(y) +[\hat \phi^+(x), \hat \phi^-(y)] + \hat \phi^-(x)\hat \phi^+(y) + \hat \phi^-(x)\hat \phi^-(y)
$$
$$
T[\hat \phi(x) \hat \phi(y)] = :\hat \phi(x) \hat \phi(y) + D(x-y)
$$
where $$
D(x-y) = [\hat \phi^+(x), \hat \phi^-(y)] = \int \frac{d^3 pd^3q}{(2\pi)^6} \frac{1}{\sqrt{2E_{\bm p}E_{\bm q}}} [\hat a_{\bm p},\hat a^{\dagger}_{\bm q}]e^{i(q\cdot y - p \cdot x} =  \int \frac{d^3 p}{(2\pi)^3} \frac{1}{2E_{\bm p}} \hat a^{\dagger}_{\bm p}e^{ip \cdot (x-y)}
$$
for $x^0 < y^0$.
\begin{equation}
        T[\hat \phi(x) \hat \phi(y)] = :\hat \phi(x) \hat \phi(y): + \Delta_F (x-y)
\end{equation}
$\Delta_F (x-y)$ is the Feynman propogator:
$$
        \Delta_F (x-y) = \begin{cases} D(x-y) & x^0 > y^0 \\
        D(y-x) & x^0 < y^0\end{cases} = \bra{0} T[\hat \phi(x) \hat \phi(y)] \ket{0}
$$
\subsection{Wicks Theroem}
$$
T(\phi_1 \phi_2... \phi_n) = :\phi_1 ... \phi_n : + : \text{ all possible contractions }: 
$$
\subsection{Integral representation of $\Delta_F$}
$$
I_F(x-y) = \int_{C_F} \frac{d^4 p}{(2\pi)^4}\frac{i}{p^2-m^2} e^{-ip \cdot (x-y)}
$$
Split up over $p^0$ and $\bm p$:
$$
I_F(x-y) = \int \frac{d^3 p}{(2\pi)^3} i e^{-i\bm p \cdot (\bm x-\bm y)}
        \int_{C_F} dp_0 I(p_0)$$
        $$
I(p_0) = \frac{e^{-ip_0 \cdot (x-y)^0 }}{(p_0)^2- |\bm p|^2 -m^2}   $$
This has poles that lie in the real axis so you cant straight fowardly you need to specify the contour.
$$
\frac{1}{p_0^2 - E_{\bm p}^2}= \frac{1}{2E_{\bm p}}( \frac{1}{p^0 - E_{\bm p}} - \frac{1}{p^0 + E_{\bm p}})
$$
So the poles are at $p_0 = \pm E_{\bm p}$ wiht residues:
$$
Res_{p^0 = \pm E_{\bm p}} (I (p_0)) = \pm \frac{1}{2E_{\bm p}} e^{\bm i E_{\bm p} (x^0 - y^0)}
$$
\section{Lecture 13}
The contour is defined as going in to the lower half plane for $-E_{\bm  p}$ and into the upper half plane for $E_{\bm p}$. Now want to evalutate this integral:
$$
\int \frac{d^3p}{(2\pi)^4} i e^{i \bm p (\bm x - \bm y)} \int_{C_F} dp_0 I(p_0)
$$
Set $p_0 = Re(p_0) + i Im(p_0)$ clearly the behaviour of $I(p_0)$ as $p_0 \rightarrow \infty$ will be dominated by the real exponential part:
$$
I(p_0) \sim e^{Im(p_0) (x^0 - y^0)
$$
this decays rapidly as $p^0$ in the the LHP for $x^0 > y^0$ and in the UHP for $x^0 < y^0$. Now apply Jordan's Lemma and close in the upper or lower plane. For $x^0 > y^0$ we need to close the contour in the LHP and use cauchy's theorem to only pick up the residue from the positive pole (bearing in mind cauchy's theorem needs a counterclockwise contour and we have a clockwise one so pick up an extra minus sign):
$$
\Delta_F(x-y) = \int \frac{d^3p}{(2\pi)^4} i e^{i \bm p(\bm x- \bm y)} \times - 2 \pi i Res_{p_0 = E_{\bm p}} (I(p_0)) = \int \frac{d^3p}{(2\pi)^4} i e^{i \bm p(\bm x- \bm y)} \frac{1}{2E_{\bm p}} e^{-i E_{\bm p}(x^0 - y^0)} $$
$$
\Delta_F(x-y) = \int \frac{d^3 p}{(2 \pi)^3} \frac{1}{2 E_{\bm p}} e^{-i p (x-y)} = D(x-y)
$$
for $x^0 < y^0$ close the contour in the UHP and pick up pole $p= - E_{\bm p}$ and reproduces the correct time ordered feyman propogator.
\subsubsection{"i$\epsilon$ prescription}
Consider a contour that goes along the real axis and alter the integral a little bit to push the poles above on the left and down on the right. So poles now at $p^0 = E_{\bm p} - i \epsilon$ and $p^0 = - E_{\bm p} + i \epsilon$:
$$
\Delta_F(x-y) = \lim_{\epsilon\rightarrow 0^+} ( \int \frac{d^4 p}{(2\pi)^4} \frac{i e^{-i p(x-y)}}{ p^2 - m^2 + i \epsilon})
$$
This is a more convenient way of automatically making it time ordered.
\section{Lecture 14}
$$
\mathfrak{L} = (\partial_{\mu}\psi^* \partial^{\mu} \psi) - M^2 \psi^* \psi + \frac{1}{2} \partial_{\mu} \phi \partial ^{\mu} \phi - \frac{m^2}{2} \phi^2 - g \phi\psi^* \psi
$$
We have $[g] = +1$ and $g << m, M$. Mode expansions:
$$
\hat \phi(x) \int \frac{d^3 p}{(2\pi)^3 } \frac{1}{\sqrt{2 \epsilon_{\bm p}}} (\hat a_{\bm p} e^{- i p x} + \hat a^{\dagger}_{\bm p} e^{i p x})
$$
$$
\hat \psi(x) \int \frac{d^3 p}{(2\pi)^3 } \frac{1}{\sqrt{2 E_{\bm p}}} (\hat b_{\bm p} e^{- i p x} + \hat c^{\dagger}_{\bm p} e^{i p x})
$$
$$
\hat \psi^{\dagger}(x) \int \frac{d^3 p}{(2\pi)^3 } \frac{1}{\sqrt{2 E_{\bm p}}} (\hat b^{\dagger}_{\bm p} e^{- i p x} + \hat c_{\bm p} e^{i p x})
$$
with $\epsilon = \sqrt{|\bm p|^2 + m^2}$ and $E = \sqrt{|\bm p|^2 + M^2}$. We also have $\hat a_{\bm p} \ket{0} = \hat b_{\bm p} \ket{0} = \hat c_{\bm p} \ket{0} = 0$ and the standard communation relations apply like $[\hat a_{\bm p}, \hat a^{\dagger}_{\bm p}] = (2\pi)^3 \delta^{(3)}(\bm p - \bm q)$. We will have $\hat a^{\dagger}$ creates a $\phi$ particle, $\hat b^{\dagger}$ creates a $\psi$ particle and $\hat c^{\dagger}$ creates a anti-$\psi$ particle. Going to try and consider:
$$
\psi+ \psi \rightarrow \psi + \psi
$$
Going to have $p_1 = (E_{\bm p_1}, \bm p_1)$ and $p_2$ going in, and $p'_1$ and $p'_2$ going out. We can treat these initial and final states as eigenstates of the hamiltonian so the appropriate states to choose are:
$$
\ket{i} = \sqrt{2 E_{\bm p_1}} \sqrt{2 E_{\bm p_2}} \hat b_{\bm p_1}^{\dagger} \hat b_{\bm p_2}^{\dagger} \ket{0}, \ket{f} = \sqrt{2 E_{\bm p'_1}} \sqrt{2 E_{\bm p'_2}} \hat b_{\bm p'_1}^{\dagger} \hat b_{\bm p'_2}^{\dagger} \ket{0}
$$
$$
A_{i \rightarrow f} = \bra{f} \hat S \ket{i}, \hat S = T[ \exp(\frac{g}{i} \int d^4 x \hat \phi \hat \psi^{\dagger} \hat \psi)]
$$
Only way to make progress is to expand the exponentials in powers of the coupling constant. Have to think what is the first order to which scattering can occur. Restrict ourselves to the case the $\{p_1, p_2\} \neq \{p'_1, p'_2\}$ so will certainly get zero to the first order in the expansion as they correspond to linearly independent states in our hilbert space as they have different momenta. In order to get a non-zero answer we need to bring down terms with the same operators as in the $\ket{i}$ and $\ket{f}$ states as only contributions will come from comunators. So need to bring down at least two creation operators ($\hat b^2$) and two annihaltion operators ($\hat b^{2\dagger}$). So we need to bring down at least two powers of the interaction hamiltonian. 
$$
A^{(2)}_{i \rightarrow f} = \frac{1}{2} (\frac{g}{i})^2 \int d^4 x_1 \int d^4 x_2 M(x_1, x_2)
$$
$$
M(x_1, x_2) = \bra{f} T[ \hat \phi(x_1) \hat \psi^{\dagger}(x_1) \hat \psi(x_1)\hat \phi(x_2) \hat \psi^{\dagger}(x_2) \hat \psi(x_2)] \ket{i}
$$
Using wicks theorem we can express this as a sum of all possible contractions plus the normal order. We need an explict power of $\hat b^2$ and $\hat b^{2\dagger}$ so if we contract any of the $\psi$ fields they will give zero when stuck back into $\ket{f}$ and $\ket{i}$. However, we are fine to contract the $\phi$ fields so:
$$
M(x_1, x_2) = \Delta_F^{(\phi)} (x_1-x_2) \bra{f} : \psi^{\dagger}(x_1) \hat \psi(x_1) \psi^{\dagger}(x_2) \hat \psi(x_2) : \ket{i}
$$
Immediate simplification comes from the fact that only the terms with the $\hat b$ in give non zero terms so $\psi^{dagger}$ only consists of creation terms and $\psi$ only consists of annihilation operators. so 
$$
M(x_1, x_2) = \Delta_F^{(\phi)} (x_1-x_2) \bra{f} \psi^{\dagger}(x_1) \psi^{\dagger}(x_2) \hat \psi(x_1) \hat \psi(x_2)  \ket{i} = \Delta_F^{(\phi)} (x_1-x_2)N 
$$
By inserting a complete set of free particle eigenstates $I = \sum_{\phi} \ket{\phi} \bra{\phi} = \ket{0} \bra{0} + \sum_{n \geq 1} \ket{\text{n particle}}\bra{\text{n particle}}$. So can write $N$ as:
$$
N = N_i N_f, N_i = \bra{0} \hat \psi(x_1) \hat \psi(x_2) \ket{i} = \sqrt{2 E_{\bm p_1}} \sqrt{2 E_{\bm p_2}} \int \frac{d^3 p}{(2 \pi)^3} \int \frac{d^3 q}{(2\pi)^3} \frac{1}{\sqrt{2 E_{\bm p}} \sqrt{2 E_{\bm q}}} e^{-ipx_1 - i q x_2} \bra{0} \hat{b_{\bm p}} \hat{\bm b_{\bm q}} \hat{b_{\bm p_1}^{\dagger}} \hat{ b_{\bm p_2}^{\dagger}} \ket{0} = \sqrt{2 E_{\bm p_1}} \sqrt{2 E_{\bm p_2}} \int \frac{d^3 p}{(2 \pi)^3} \int \frac{d^3 q}{(2\pi)^3} \frac{1}{\sqrt{2 E_{\bm p}} \sqrt{2 E_{\bm q}}} e^{-ipx_1 - i q x_2} A$$
Use commutator functions:
$$
A = (2\pi)^6 (\delta^{(3)} (\bm p_1 - \bm p) \delta^{3}(\bm p_2 - \bm q) + \delta^{(3)}(\bm p_2 - \bm p) \delta^{(3)}(\bm p_1 - \bm q))
$$
So:
$$
N_i = (e^{-i p_1x_1 - i p_2x_2} + e^{- i p_2 x_1 - i p_1 x_2})$$
Can do same calculation to get answer for $N_f$:
$$
N_f = \bra{f} \hat \psi^{\dagger}(x_1) \hat \psi^{\dagger}(x_2) \ket{0} = (e^{i p_1' x_1 + i p_2' x_2} + e^{i p'_2 x_1 + i p'_1 x_2})
$$
So
$$
A^{(2)}_{i \rightarrow f} = \frac{1}{2} (\frac{g}{i})^2 \int d^4 x_1 \int d^4 x_2 N_f N_i \Delta^{(\phi)}_F(x_1-x_2) =  \frac{1}{2} (\frac{g}{i})^2 \int d^4 x_1 \int d^4 x_2 N_f N_i \int \frac{dk^4}{(2\pi)^4} \frac{i e^{ik(x_1-x_2)}}{k^2 - m^2 + i \epsilon}
$$
$$
A^{(2)}_{i \rightarrow f} = \frac{1}{2} (\frac{g}{i})^2 \int \int \int \frac{d^4 x_1 d^4 x_2 dk^4}{(2\pi)^4} \frac{i }{k^2 - m^2 + i \epsilon} (e^{ix_1( k + p'_1 - p_1)} e^{ix_2(k-p'_2+p_2)} +e^{ix_1(k + p'_2 - p_1)}e^{ix_2(k-p'_1 + p_2)} + x_1 \iff x_2)
$$
Last term above is the same as the first with $x_1$ and $x_2$ exchanged which is idnetical as they are symmetric in the metric. First evaluate by integral over $x_1$ and $x_2$ as the integrand depends purely exponential so gives two set of four dimensional delta functions and then can do integral over $d^4 k$. The result from doing this is:
$$
A_{i \rightarrow f}^{(2)} = i(-ig)^2 (2 \pi)^4 \delta^{(4)}(p_1 + p_2 - p_1' - p_2')\times (\frac{1}{(p_1 - p_1')^2 - m^2 + i \epsilon} + \frac{1}{(p_1 - p'_2)^2 - m^2 + i \epsilon})
$$
Above section has some incorrect signs that are corrected on the moodle notes
\section{Lecture 15}
Energy momentum conservation:
$$
\bm p_1 + \bm p_2 = \bm p'_1 + \bm p'_2
$$
$$
E_{\bm p_1} + E_{\bm p_2} = E_{\bm p'_1} + E_{\bm p'_2} =
$$
In COM frame:
$$
\bm p_1 + \bm p_2 = \bm p'_1 + \bm p'_2 =0 \implies |\bm p_1| = |\bm p_2|, |\bm p'_1| = |\bm p'_2|
$$
$$
E_{\bm p_1} = E_{\bm p_2} = E_{\bm p'_1} = E_{\bm p'_2}
$$
In COM frame: $(p- p')^{\mu} = (0, \bm p_1 - \bm p'_1)$ so $(p- p_1)^2 \leq 0$. As the deminoar $(p_1 - p'_1)^2 -m^2 < 0$ and similarily $(p_1-p'_2)^2-m^2< 0$. Therefore it is safe to assume that $\epsilon=0$ as the other part of the denominator is never zero.
\subsection{Feynmann diagrams}
Contributions to $\bra{f} \hat S - I \ket{i}$ can be written in a diagramatic language of feynamn diagrams. Now need to move to paper!!
\subsection{Fermions}
So far we have only covered scalar fields that transform under the lorentz transformation as:
$$
\phi(x) \rightarrow \phi(\Lambda^{-1}x)
$$
We have shown that $\hat \phi$ creates a particle of spin 0. We want to make a particle of half-spin.\\\\
Non-relativistic description of a spin: $\hat{\bm S} = (\hat S_z, \hat S_y, \hat S_z), \hat{S}^2, \hat{S}_z$ paricles of spin s = $0, \frac{1}{2}, \frac{3}{2}$ haave $2 s +1$ spin states:
$$
\hat{S}^2 \ket{s, s_z} = s(s+1)\hbar^2 \ket{s,s_z}
$$
$$
\hat S_z \ket{s,s_z} = s_z \hbar \ket{s,s_z}
$$
so get $s_z$ at interval spacings with
$$
-s \leq s_z \leq s
$$
Need to be able to describe these independant states so this motivates the fact we need multiple components to the field. So need fields that transform in non-trival representations of the Lorentz group.
\section{Lecture 16}
Consider the lorentz group: $G_L = \{ \Lambda \in Mot_4(\mathbb{R})\}$ where $\Lambda \eta \Lamda^T = \eta, det \Lamda = +1$.\\
Consider a representation $D$ of $G_L$ of dimension $N$ which is a way of representing the group elements in terms of matrices:
$$
D: G \rightarrow Mat_N(\mathbb{C})
$$
(with $det D \neq 0$ I think)\\
This map is a smooth homorophism and so preserves the group multiplication, so:
$$
D(\Lambda_1)D(\Lambda_2) = D(\Lambda_1 \Lambda_2) \forall \Lambda_1, \Lambda_2 \in G 
$$
Matrices are linearly maps that act on vectors and in this case they act on the representation space. So the representation acts on vectors in the representation space $V = \mathbb{C}^N$. This forumalises the variaty of ways the same group can act on vectors. Now we want to think about what it means for a field to transform under the lorentz group. So we will write an explicit form of the matrices.
$$
D(\Lamdba_{AB})
$$
A field is a function from spacetime into the field space: $\Psi: \mathbb{R}^{3,1} \rightarrow \mathbb{C}^N$. such a field transforms in representation D of $G_L$ if:
$$
\psi_A(x) \rightarrow^{\Lambda \in G_L} \psi'_A(x) = \sum_{B=1}^N D(\Lambda)_{AB}\psi_B(\Lambda^{-1} \cdot x)
$$
The field is now a vector and the lorentz group acts on these elements under different representation of the lorentz group.
\subsubsection{Representations of the Lorentz group}
\textbf{Scalar field}: $\phi(x) \rightarrow \phi(\Lambda^{-1} \cdot x)$, $N=1$ and $D(\Lambda) =1 \forall \Lambda \in G_L$. This is the trival representation.\\
\textbf{4-vector field}: $A^{\mu}(x) \rightarrow \Lambda^{\mu}_{\nu}A^{\nu}(\Lambda^{-1} \cdot x)$, this is the fundamental representation which is where you simply use element itself as its representation (only exists for matrix Lie groups). $D(\Lambda) = \Lambda)$, $N=4$.\\
\subsection{Spinor Representation}
Going to spend the rest of the lecture deriving this. It is generally hard to construct representations, so we find representations of the Lie Algebra and then utilise the relationship between them. In physics this means finding a representation of the infinitesimal transformations.\\\\
\textbf{Lie Algebra} $\mathbb{L}(G_L)$: Correspond to the tangent vectors at the identity on the Lie group and is useful as there exists an exponential map from the Lie Algebra to the Lie Group:
$$
Exp: \mathbb{L}(G_L) \rightarrow G_L
$$
A representation of the Lie Algebra. As $\mathbb{L}(G_L)$ is a linear map, so want matrices:
$$
R: \mathbb{L}(G_L) \rightarrow Mat_N(\mathbb{C})
$$
As alegbra of a Lie Algebra is the Lie bracket so $R$ must preserve the Lie Bracket so:
$$
R([X,Y]) = [R(X),R(Y)] \forall X,Y \in \mathbb{L}(G_L)
$$
So it is a representation that preserves all the important structure of the Lie Algebra.\\\\
Given a representation $R$ of $\mathbb{L}(G_L)$, we can find a representation $D$ of $G_L$:
$$
\Lambda = Exp(X), X \in \mathbb{L}(G_L), D(\Lambda) = Exp(R(X))
$$
There is a slight wrinkle here which is the fact that several Lie Groups have the same Lie Algebra like SU(2) and SO(3), so this can't map back to give you all of the Lie Group. So you don't get a represenation of the original group but rather the universal cover $\tilde G_L$ of $G_L$. For example with $SO(3)$ has a larger group spin(3,1) which is a double cover and is exactly a 2:1 map $\rho$:
$$
\rho : \tilde G_L \rightarrow G_L
$$
Key examples of a double cover is that $SO(3)$ is double covered by $SU(2)$ written: $SO(3) \leftarrow^{2:1} SU(2)$.\\\\
Consider proper Lorentz transformations with $G_L = SO(3,1)$ so $\Lambda = Exp(\omega)$. For example: $$\Lambda^{\mu}_{nu}  = \delta^{\mu}_{nu} + \omega^{\mu}_{\nu} + \frac{1}{2} \omega^{\mu}_l \omega^l_{\nu} + ...$$
So as $\Lambda \in G_L \implies \Lambda \eta \Lambda^T = \eta$ we get constraint on $\omega$ of $\omega^{\mu \nu} + \omega^{\mu \nu} = 0$. This can be shown to be neccessary and sufficent. So the Lie Alegbra of the Lorentz group is $\mathbb{L}(G_L) = \{\omega \in Mat_4(\mathbb{R}), \omega + \omega^T = 0\}$. This has 6 independent variables/generators. The natrual basis is matrices where we put a 1 in the $(\mu, \nu)$, -1 in $(\nu, \mu)$ position and zeros everywhere else above the leading diagonal. We can write this very explictly in terms of the metric tensor:
$$
(M^{\rho \sigma})^{\mu \nu} = \eta^{\rho \mu} \eta^{\sigma \nu} - \eta^{\sigma \mu} \eta^{\rho \nu}
$$
Above one set of indicies labels the set of generators $\rho, \sigma$ and then another set of incidies $\mu, \nu$ which give you the matrix entries in the generators. E.g.
$$
                (M^{01})^{\nu \mu} = \begin{pmatrix} 0 &1 & 0 & 0\\
                -1 & 0 & 0 &0\\ 
        0 & 0 & 0 & 0 \\
0& 0 & 0 & 0\end{pmatrix}
$$
This is the generator of infinitesmial lorentz transformation corresponding to a boost along the x-axis. So a general element of the Lie Algebra lorentz group will look like:
$$
\omega^{\mu}_{\nu} =  \frac{1}{2} \Omega_{\rho \sigma}(M^{\rho \sigma})^{\mu}_{\nu}
$$
with $\Omega_{\rho \sigma} = - \Omega_{\sigma \rho}$. We have 6 indepedent constanst that appear here that are a way of parametising the elements of the lorentz group. We need to check the brakcets and as everything is linear it is enough to just check the generators:
\begin{equation}
[M^{\rho \sigma}, M^{\tau \nu}] = \eta^{\sigma \tau}M^{\rho \nu} - \eta^{\rho \tau} M^{\sigma \nu} + \eta^{\rho \nu}M^{\sigma \tau} - \eta^{\sigma \nu} \eta^{\rho \tau} 
                \end{equation}
                Finally can write $G_L$ in terms of generators:
                $$
                \Lambda = Exp( \frac{1}{2} \Omega_{\rho \sigma}M^{\rho \sigma})
                $$
                Clifford Algebra which is four objects ($\gamma^0, \gamma^1, \gamma^2, \gamma^3$) that obey certain relations. The defining relation is $\{\gamma^{\mu}, \gamma^{\nu} \} = \gamm^{\mu} \gamma^{\nu} + \gamma^{\nu}\gamma^{\nu} = 2 \eta^{\nu \mu} \mathbb{I}$. Dirac invented this algebra as a trick. If you can find a representation of this algebra you can then manipulate them into the spinor algebra.\\\\
                Consider the pauli matrices they obey the nice anticommuniator: $\{\sigma^i, \sigma^j \} = 2 \delta^{ij} \mathbb{I}_2$.  Consider the simple representation of the clifford algebra as:
                $$
                        \gamma^0 = \begin{pmatrix} 0 & \mathbb{I}_2\\
                                \mathbb{I}_2 & 0 \end{pmatrix}, \gamm^i = \begin{pmatrix} 0 & \sigma^i \\
                        - \sigma^i & 0 \end{pmatrix}
                $$
                The trick is to define:
                $$
                S^{\rho \sigma} = \frac{1}{4}[ \gamma^{\rho}, \gamma^{\sigma}] = \frac{1}{2} \gamma^{\rho} \gamma^{\sigma} - \frac{1}{4} \{ \gamma^{\rho}, \gamma^{\sigma} \} = \frac{1}{2} \gamma^{\rho} \gamma^{\sigma} - \frac{1}{2} \eta^{\rho \sigma} \mathbb{I}_2
                $$
                Check that $[ S^{\rho \sigma},S^{\mu \nu}] =\eta^{\sigma \mu}S^{\rho \nu} - \eta^{\rho \mu} S^{\sigma \nu} + \eta^{\rho \nu}S^{\sigma \mu} - \eta^{\sigma \nu} S^{\rho \mu}$ 
                \section{Lecture 17}
                Each $\Lambda$ is mapped to $S[\Lambda] \in Mat_4 (\mathbb{C})$ to give the spinor represenation:
                \begin{equation}
                        S[\Lambda] = \exp(\frac{1}{2} \Omega_{\rho \sigma} S^{\rho \sigma})
                \end{equation}
                \begin{equation}
                        S^{\mu \nu} = \frac{1}{4} [\gamma^{\mu}, \gamma^{\nu}]
                \end{equation}
                with $\gamma$ defined by:
                \begin{equation}
                        \{ \gamma^{\mu}, \gamma^{\nu} \} = 2 \eta^{\mu \nu} \bm I_4
                \end{equation}
                This representation is a representation of the covering group (which is larger than the lorentz group) of the lorentz group not the lorentz group itself. e.g. when we rotate about $2\pi$ we get a minus sign rather than identity. We can focus on rotations in space by setting our parameters to pick out just the spatial components ($\Omega_{00} = \Omega_{0i} = \Omega_{i0} = 0$)
                $$
                \Omega_{ij} = - \epsilon_{ijk} \phi^k
                $$
                This is a rotation about the vector $\phi$ by an angle specified by the magnitude of this vector. Exercise: for $\phi^1= \phi^2=0$ show that $\Lambda = \exp (- \phi^3 M^{12})$ acts on 4-vectors as a rotation through an angle $\phi^3$ about the z axis.
                $$
                                S^{ij} = \frac{1}{4} [ \gamma^{i} , \gamma^{j}] =^{i\neq j \text{ as antisymmetric}} \frac{1}{2} \gamma^i \gamma^j =\frac{1}{2} \begin{pmatrix} 0 & \sigma^i \\ - \sigma^i & 0 \end{pmatrix} \begin{pmatrix} 0 & \sigma^j \\ - \sigma^j & 0 \end{pmatrix}
                $$
                $$
                                S^{ij} = \frac{1}{2} \begin{pmatrix} - \sigma^i\sigma^j & 0 \\0& - \sigma^i\sigma^j  \end{pmatrix}  = -\frac{i}{2} \epsilon^{ijk} \begin{pmatrix} \sigma^k & 0 \\ 0 & \sigma^k \end{pmatrix}                  $$
                                $$
                                S[\Lambda] = \exp (-\frac{1}{2}  \epsilon_{ijk} \phi^k S^{ij})= \begin{pmatrix} e^{\frac{i\bm \phi \cdot \bm \sigma}{2}} & 0 \\ 0 & e^{\frac{i \bm \phi \cdot \bm \sigma}{2}} \end{pmatrix}
                                $$
                                Specialise futher to the case $\phi^1, \phi^2 = 0$ then
                                $$
                                S[\phi^3] = S[\Lambda]|_{\phi_1 = \phi_2 = 0} = \begin{pmatrix} e^{i \frac{\phi^3}{2}} & 0 & 0 & 0 \\
                                        0 & e^{-i \frac{\phi^3}{2}} & 0 & 0\\
                                        0 & 0 & e^{i \frac{\phi^3}{2}} & 0 \\
                                0 & 0 & 0 & e^{-i \frac{\phi^3}{2}} \end{pmatrix}
                                $$
                                $$
                                \Lambda[ \phi^3 = 2\pi] = \bm I_4
                                $$
                                but 
                                $$
                                S[2\pi] = - \bm I_4
                                $$
                                So clearly a double cover. This is also clear that it is a good starting point for talking about fermions as they are antisymmetric under exchange.
                                \subsection{Spinor Fields}
We want to have fields that transform in representations of the Lorentz group. Will easily give us singlets for the scalar fields but also lead to particle fields. A field is defined as:
$$
\psi: \mathbb{R}^{3,1} \rightarrow \mathbb{C}^4
$$
$$
x \rightarrow \psi^{\alpha}(x) \in \mathbb{C}
$$
where $\alpha = \{1,2,3,4\}$ are the spinor indices. No significance to writing a spinor index up or down. Transformation property under a lorentz transformation ($\Lambda \in G_L$):
$$
\psi^{\alpha}(x) \rightarrow \psi^{\alpha}'(x) = S[\Lambda]^{\alpha}_{\beta}\psi^{\beta}(\Lambda^{-1} \cdot x)
$$
$$
\psi(x) \rightarrow^{\Lambda} S[\Lambda] \cdot \psi(\Lambda^{-1} \cdot x)
$$
Key property of this is that under a $2\pi$ rotation that leaves $x$ invariant wil change the field:
$$
\psi(x) \rightarrow -\psi(x)
$$
The spacetime derivative transforms as (follows from before using chain rule):
$$
\partial_{\mu} \psi(x) \rightarrow^{\Lambda} (\Lambda^{-1})^{\nu}_{\mu}S[\Lambda]_{\beta}^{\alpha}\partial_{\nu}\psi^{\beta} (\Lambda^{-1} \cdot x) 
$$
Conjugate field:
$$
\psi^{\dagger}_{\alpha} (x) \rightarrow \psi^{\dagger} \cdot S[\Lambda]^{\dagger}
$$
Note: $S$ is not unitary:
$$
S[\Lambda]^{\dagger} \neq S[\Lambda]^{-1} \forall \Lambda \in G_L
$$
As the lorentz group is a non-compact lorentz group as the boosts are not bounded so the manifold is a hyperbolid and there is a theorem that says any simple non-compact lie group has no unitary representation (missed what he actually said here 36 minute lecture 16/11).\\\\
\textbf{Conjugation} \\
Deduce from definition of clifford algebra that:
$$
(\gamma^0)^2 = \bm I^4
$$
So eigenvalues of $\gamma^0$ are real and it is hermitian ($
(\gamma^0)^{\dagger} = \gamma^0
$)\\
Can also deduce that for $i \in \{1,2,3\}$:
$$
(\gamma^i)^2 = - \bm I_4
$$
So $\gamma^i$ is anti-hermitian ($(\gamma^i)^{\dagger}  = - \gamma^i$).
$$
\gamma^0 \gamma^i \gamma^0 = \gamma^0 \{\gamma^i, \gamma^0\} - \gamma^0 \gamma^0 \gamm^i = (\gamma^i)^{\dagger}, \gamma^0 \gamma^0 \gamma^0 = (\gamma^0)^{\dagger}
$$
So
$$
(\gamma^{\mu})^{\dagger} = \gamma^0 \gamma^{\mu} \gamma^0
$$
Now we can look at the conjugate properties of the spinor generators of the representation:
$$
(S^{\mu \nu})^{\dagger} = \frac{1}{4} ( [\gamma^{\mu}, \gamma^{\nu}])^{\dagger} = \frac{1}{4} ( [ (\gamma^{\nu})^{\dagger}, (\gamma^{\mu})^{\dagger}]) = \frac{1}{4} [ \gamma^0 \gamma^{\nu} \gamma^0, \gamma^0 \gamma^{\mu} \gamma^0] = \frac{1}{3} \gamma^0 [\gamma^{\nu}, \gamma^{\mu}] \gamma^0 = - \gamma^0 S^{\mu \nu} \gamma^0
$$
Now lets consider:
$$
(S[\Lambda])^{\dagger} = [\exp( \frac{1}{2} \Omega_{\rho \sigma} S^{\rho \sigma})]^{\dagger} = \exp(\frac{1}{2} \Omega_{\rho \sigma} (S^{\rho \sigma})^{\dagger}) = \exp( - \frac{1}{2} \Omega_{\rho \sigma} \gamma^0 S^{\rho \sigma} \gamma^0) = \gamma^0 \exp(- \frac{1}{2} \Omega_{\rho \sigma} S^{\rho \sigma}) \gamma^0
$$
Using the fact the $\gamma^0)^2 = 1$ on the last equality. As the inverse of $e^P $ is $e^{-P}$:
\begin{equation}
        (S[\Lambda])^{\dagger} = \gamma^0 (S[\Lambda])^{-1} \gamma^0
\end{equation}
This will allow us to construct real scalar quantities out of these fields.
\section{Example class 2}
In the Heisenberg picture for the free scalar field:
$$
\dot \phi(x) = i [H, \phi(x)] = \pi(x), \dot \pi(x) = i [H, \pi(x)] = \nabla^2 \phi(x) - m^2 \phi(x)
$$
The reason the signs flip when we go into the Heinsberg picutre is because $e^{i\bm p \cdot \bmx} e^{-i E t} = e^{- i p_{\mu} x^{\mu}}$ as $p_{\mu} x^{\mu} = p_0 x^0 - \bm p \cdot \bm x$
Tips for question 4:
$$
\int d^3 x\> x^j e^{i(q - p) \cdot x} = -i(2\pi)^3 \frac{\partial}{\partial q^j} \delta^{(3)}(q-p)
$$
$$
L=q^k \frac{partial }{\partial q^j} - q^j \frac{\partial}{\partial q^k}
$$
If $f(q) = f(|q|)$ (is spherically symmetric, then $Lf = 0$.
\textbf{Feymann diagram notation}\\
Let $\Gamma$ be a Feynmann diagram, and let $v(\Gamma)$ be the final value of the diagram (the sum of all the terms it represents in the perturbation series. So $\bra{0}S \ket{0} = \sum_{\Gamma} v(\Gamma)$. $|\Gamma|$ is the value of $\Gamma$ after applying Feynmann rules: write down $-i\lambda$ at each vertex, write down a propogator at each (internal) edge, multiply and integrate. $v(\Gamma) = \frac{1}{|Aut(\Gamma)|} |\Gamma|$ with $|Aut(\Gamma)|$ being the symmetry factor of $\Gamma$, as $Aut(\Gamma)$ is a map of the vertices to themselves that preserves the graph structure. Count how many ways you can connect the half edges (draw the vertices and half edges without connecting them and then count the number of ways you could connect them up and still generate the same diagram)\\
Every feynmann diagram only corresponds to one term of the expansion you get addition from having multiple possible feynmann diagram. However, it is actually representing a collection of multiple term e.g $\int \phi^{\cdot}(x)\phi^*(x) \phi^{\cdot} \phi^*$ and $\int \phi^{*}(x)\phi^{\cdot}(x) \phi^{\cdot} \phi^*$ are represented by the same diagram. Number of half edges is equal to the number of fields in the perturbation. In general for a disjoint union of connected diagrams:
$$
\Gamma = \Gamma^{n_1}_1 \Gamma^{n_2}_2... \Gamma^{n_k}_k \implies v(\Gamma) = \frac{1}{n_1! ... n_k!} v(\Gamma_1)^{n_1} ... v(\Gamma_k)^{n_k}
$$
\section{Lecture 18}
Define Dirac Adjoint:
$$
\bar \psi(x) = \psi^{\dagger}(x) \gamma^0
$$
Under a Lorentz Transformation $\Lambda \in G_L$:
$$
\bar \psi(x) \rightarrow^{\Lamdba} \bar \psi'(x) = \psi^{\dagger}(\Lambda^{-1} \cdot x) S^{\dagger}(\Lambda) \gamma^0 =  \psi^{\dagger}(\Lambda^{-1} \cdot x) \gamma^0 S(\Lambda)^{-1} \gamma^0 \gamma^0 = \bar \psi(\Lambda^{-1} \cdot x) S(\Lambda)^{-1}
$$
Define $\Sigma(x) = \bar \psi(x) \psi(x)$. We claim $\Sigma (x)$ is a real scalar field.\\
Proof it is real: $\Sigma^*(x) = (\psi^{\dagger}(x) \gamma^0 \psi(x))^{\dagger} = \psi^{\dagger}(x) (\gamma^0)^{\dagger} \psi(x) = \Sigma(x)$\\
Proof it is scalar: $\Sigma(x) \rightarrow \Sigma'(x) = \bar \psi(\Lambda^{-1} \cdot x) S(\Lambda)^{-1} S(\Lambda) \psi(\Lambda^{-1} \cdot x) = \bar \psi(\Lambda^{-1} \cdot x) \psi (\Lambda^{-1} \cdot x) = \Sigma (\Lambda^{-1} \cdot x)$\\\\
Can also form a vector field: $V^{\mu}= \bar \psi(x) \gamma^{\mu} \psi(x)$ transforms as
$$
V^{\mu} (x) \rightarrow \Lambda^{\mu}_{\nu} V^{\nu} (\Lambda^{-1} \cdot x) = \Lambda^{\mu}_{\nu \bar \psi(\Lambda^{-1} \cdot x) \gamma^{\nu} \psi(\Lambda^{-1} \cdot x)
$$
Proof of above: 
$$
V^{\mu} \rightarrow \bar \psi(\Lambda^{-1} \cdot x) \cdot S(\Lambda)^{-1} \cdot \gamma^{\mu} \cdot S(\Lambda) \cdot \psi(\Lambda^{-1} \cdot x)
$$
Hence this is true if $S[\Lambda]^{-1} \gamma^{\mu} S[\Lambda] = \Lambda^{\mu}_{\nu} \gamma^{\nu}$. Two spinor acting on gamma matrix is equivalent to one lorentz transfomration. Exercise is to prove this at the lienarised level by expanding out the exponential expressions for these terms and it becomes equivalent to the identity: $[\gamma^{\mu}, S^{\rho \sigma}] = \eta^{\rho \mu} \gamma^{\sigma] - \eta^{\sigma \mu} \gamma^{\rho}$. It is a lot more work to prove the full identity.
        \subsection{Dirac Action}
        Action for spinor fields, needs to be lorentz invariant and real. We want the $\mathfrak{L}$ to be a real (at least up to surface terms) scalar field. 
        $$
        \mathfrak{L}(x) = \mathfrak{L_1} + \mathfrak{L_2} = \bar \psi(x) i \gamma^{\mu} \partial _{\mu} \psi(x) - m \bar \psi(x) \psi(x)
        $$
        We know that $\mathfrak{L_2} = - m \Sigma(x)$ so is a real scalar field.
        $$
        \mathfrac{L_1} \rightarrow^{\Lambda} \bar \psi(\Lambda^{-1} \cdot x) S(\Lambda)^{-1} i \gamma^{\mu} (\Lambda^{-1})^{\nu}_{\mu} S[\Lambda] \partial _{\nu} \psi(\Lambda^{-1} \cdot x) = \bar \psi( \Lambda^{-1} \cdot x) i \Lambda^{\mu}_{\rho} \gamma^{\rho} (\Lambda^{-1})^{\nu}_{\mu} \partial _{\nu} \psi(\Lambda^{-1} \cdot x)
        $$
        So
        $$
        \mathfrac{L_1} \rightarrow^{\Lambda} i \bar \psi(\Lambda^{-1} \cdot x) \gamma \partial_{\mu} \psi(\Lambda^{-1} \cdot x)
        $$
        So $\mathfrac{L_1}$ is a scalar field. now check it is real:
        $$
        \mathfrac{L_1} = i \psi^{\dagger}(x) \gamma^0 i \gamma^{\mu} \parital_{\mu} \psi(x)
        $$
        $$
        \mathfrac{L_1}^* = -i \parital_{\mu} \psi^{\dagger}(x) \gamma^{\mu})^{\dagger}  \gamma^0 \psi(x) = - i \partial_{\mu} \psi^{\dagger}(x) \gamma^0 \gamma ^{\mu} \gamma^0 \gamma^0 \psi(x) = - i \partial_{mu} \bar \psi(x) \gamma^{\mu} \psi(x) = i \bar \psi(x) \gamma^{\mu} \partial_{\mu} \psi(x) - i \partial_{\mu} (\bar \psi(x) \gamma^{\mu} \psi(x) = \mathfrac{L_1} + \text{ surface term}
        $$
        \textbf{Dirac action}
        \begin{equation}
                S = \int d^4 x (\bar \psi(x) \cdot (i \gamma^{\mu}\partial_{\mu} - m) \cdot \phi(x))
        \end{equation}
        Having got this action we can derive the field equation (EOM). For these purposes it is convient to think about the two independant fields are $\bar \psi$ and $\psi$ rather than real and imaginary parts.
        \begin{equation}
                (i \gamma_{\mu}\partial_{\mu} -  m) \cdot \psi(x) = 0
\end{equation}
Vary the action with respect to $\psi(x)$ gives $i(\partial_{\mu} \bar \psi) \cdot \gamma^{\mu} + m \bar \psi  = 0$. \textbf{Notation}  on 4 vector $\cancel{V} = \gamma^{\mu} V_{\mu}$ so dirac equation becomes $(i \cancel{\partial} - m )\cdot \psi = 0$.\\\\
Consider rotations ($\bm \phi$): $\Omega_{jk} = - \epsilon_{ijk} \phi^k$ or boosts $\Omega_{i0} = - \Oemga_{0i} = \chi_i$. We can look at the spinor representation of these and they are all block diagonal:
$$
                                        S(\bm \phi) = \begin{pmatrix} e^{i \frac{\bm \phi \cdot \bm \sigma}{2}} & 0 \\0&e^{i \frac{\bm \phi \cdot \bm \sigma}{2}} \end{pmatrix}
$$
$$
                                        S(\bm \chi) = \begin{pmatrix} e^{ \frac{\bm \chi \cdot \bm \sigma}{2}} & 0 \\0&e^{- \frac{\bm \chi \cdot \bm \sigma}{2}} \end{pmatrix}
$$
This reflects something non-trival about the spinor representaion as every lorentz transformation is representated by a block diagonal matrix. So the spinor representation is a reducible representation.
\section{Lecture 20}
The rules for constructing the field theory are to basically find representaitons of the lorentz groups that introduce fields that transform in these reperesentations that produce particles that also live in the corresponding representaiion of the lorentz group. So we are talking about the representation theory of the lorentz group $G_L = SO(3,1)$. We actually we are considering representations of its cover $spin(3,1)$ we normally do this be exponenting a Lie algebra represenation which is much simplier as it is linear and there is no distinct between the covering group and the group. Once we get to classifying the representations of the Lie Algebra there is a useful trick of passing to the complexification of the Lie Algebra, as if you project back from these onto the real lie algebra they will give you representations. There are many different Lie Algebras that have the same complixification. Therefore:
$$
\mathbb{L}(spin(3,1)) = \mathbb{L}(SO(3,1)), \mathbb{L}_{\mathbb{C}} (SO(3,1)) = \mathbb{L}_{\mathbb{C}}(SO(4)) = \mathbb{L}_{\mathbb{C}}(SU(2)) \oplus \mathbb{L}_{\mathbb{C}}(SU(2))
$$
We know the finite dimensional representations of $\mathbb{L}_{\mathbb{C}}(SU(2))$ which are claisificed by $j=0, \frac{1}{2}, ...$ and $-j \leq j_z \leq j$. So the representaiton of spin $R_j$ has dimesnion $2j +1$. So you can classify the representations of the lorentz group you can classify by combining these so they are labled by two spins $(j_+, j_-)$ so the dimension $R_{(j_+, j_-)} = (2j_+ +1, 2j_- +1)$. So you can see these dimensions will grow very rapidly with spin.
$$
\begin{tabular}{|c|c|c|c|}
        & j_+ & j_- & D\\
        scalar & 0 & 0 & 1\\
        Left handed Weyl spinors & $\frac{1}{2}$ & 0 & 2\\
        Right handed Weyl spinors & 0 & $\frac{1}{2}$ & 2\\
 4 vector & $\frac{1}{2}$ & $\frac{1}{2}$ & 4
\end{tabular}
$$
The Dirac repinos term is a combination of left and right hadned Weyl spinors = $(\frac{1}{2}, 0 ) \oplus (0, \frac{1}{2})$.
\subsection{Solutions of Dirac Equation}
$$(i \gamma^{\mu} \partial_{\mu} - m \bm I_4 ) \phi(x) = 0$$
act with $(-i \gamma^{nu}\partial_{\nu} - m \bm I_4)$:
$$
(\gamma^{\mu} \gamma^{\nu} \partial_{\nu} \partial_{\mu} + m^2 \bm I_4) \phi(x) = 0
$$
As $\frac{1}{2} \{ \gamma^{\mu}, \gamma^{\nu} \} = \eta^{\mu \nu} \bm I_4
$$
(\partial_{\mu}\partial^{\mu} + m^2 ) \phi(x) = 0
$$
So each of the ocmpoentsn of $\phi$ always obeys the Klien Gordane equation, and we know these have wave like solutions with on shell solution:
$$
\phi \sim e^{\pm i p \cdot x}
$$
So $p^{\mu}p_{\mu} = m^2$, $p^0 = E_{\bm p} = \sqrt{|\bm p|^2 + m^2}$ so we will search for the positive frequency solutions using the antzats
$$
\psi_{\alpha}(x) = u_{\alpha}(p)e^{-ip\cdot x}
$$
pluggin into Dirac:
$$
(p_{\mu} \gamma^{\mu} - m \bm I_4) u(p) = 0}
$$
Use the chiral representation for $\gamma$ then:
$$
                                        \begin{pmatrix} - m \bm I_2 & p_{\mu}\sigma^{\mu}\\
                                        p_{\mu} \bar \sigma^{\mu} ^ - m\bm I_2 \end{pmatrix} u = 0
$$
with $\sigma^{\mu} = (I_2, \sigma^i)$ and $\bar \sigma^{\mu} = (I_2, - \sigma^i)$. Show that the general solutino is:
$$
u(p)= \begin{pmatrix} \sqrt{p_{\mu} \sigma^{\mu}} \xi \\ 
\sqrt{p_{\mu} \bar \sigma^{\mu} \xi\end{pmatrix}
$$
where $\xi$ is a n arbitary 2 componetn complex vector. As you can always diagonlise a hermitian matrix so the sqrt of $H$ is defined as $\sqrt{H} = U^{-1} \sqrt{\Lambda} U$ for $U$ s.t. $H = U^{-1} \Lambda U$ and $\sqrt{\Lambda} = diag\{ \sqrt{\lamdba_1}}, \sqrt{\lambda_2}\}$.\\\\
Choose basis vectors $\xi^s$ with $s=1,2$ (spin component) and each on is a 2 component complex vector with 
$$
(\xi^r)^{\dagger} \xi^s = \delta^{rs}
$$
E.g. Could choose $\xi^1 = \begin{pmatrix} 1 \\ 0 \end{pmatrix}$ $x^2 = \begin{pmartix} 0 \\ 1 \end{pmatrix}.
So the basis of solutions are of the form:
$$
\psi^{(r)}(x) = u^{(r)} (p) e^{-i p \cdot x}
$$
with $u^{(r)} = \begin{pmatrix} \sqrt{p_{\mu} \sigma^{\mu}} \xi^r \\ 
\sqrt{p_{\mu} \bar \sigma^{\mu}} \xi^r \end{pmatrix}$
Similarly we have "negative" frequency solutions:
$$
\psi^{(r)}(x) = v^{(r)} (p) e^{i p\cdot x}
$$
with basis $\eta^s$ giving:
$$
v^{(s)} = \begin{pmatrix} \sqrt{p_{\mu} \sigma^{\mu}} \eta^s \\ 
-\sqrt{p_{\mu} \bar \sigma^{\mu} \eta^s\end{pmatrix}$
$$
This gives a total of four linearlly independant solutions. \\\\
We are going to need quite a few identities in order to do practical calculations, series of identities that are proved on example sheet 3 (useful to be able to derive these):
$$
u^{r \dagger} (p) u^s(p) = 2p_0 \delta^{rs}
$$
$$
\bar u^r(p) u^s(p) = 2m \delta^{rs}
$$
$$
v^{r \dagger} (p) v^s(p) = 2p_0 \delta^{rs}
$$
$$
\bar v^r(p) v^s(p) = -2m \delta^{rs}
$$
$$
\bar u^s(p) v^r(p) = 0
$$
$$
u^{s \dagger}(p) v^r(p') - 0
$$
for $p' = (p_0, - \bm p)$. Instead contracting the spin indicies
$$
\sum_{s=1}^2 u_{\alpha}^s(p) \bar u_{\beta}^s(p) = (\cancel{p} +m )_{\alpha \beta}
$$
$$
\sum_{s=1}^2 v_{\alpha}^s(p) \bar v_{\beta}^s(p) = (\cancel{p} -m )_{\alpha \beta}
$$
\subsection{Quantization of spinor field}
$$
\mathfrac{L} = i \bar \psi(x) \gamma^{\mu} \parital_{\mu} \psi(x) = m\bar \phi(x) \psi(x)
$$
Symmetry of dirac action:
$$
\psi(x) \rightarrow e^{- i \alpha} \psi(x)
$$
$$
\bar \psi(x) \rightarrow e^{i \alpha} \bar \psi(x)
$$
Conserved current:
$$
j^{\mu}_{\nu} = \bar \psi \gamma^{\mu} \psi = V^{\mu}
$$
$$
\partial_{\mu} j^{\mu}_{nu}
$$
Conserved charge:
$$
Q= \int d^3 x j^0_{\nu} = \int d^3 x \bar \psi \gamma^0 \psi = \int \psi^{\dagger} \psi
$$
\section{Lecture 21}
\subsection{Hamiltonian Formulation}
Conjugate momentum of $\psi(x)$:
$$
\pi(x) = \frac{\partial \mathfrak{L}}{\parital \dot \psi(x)} = i \bar \psi(x) \gamma^0 = i \gamma^{\dagger}(x)
$$
Hamilitonian:
$$
H = \int d^3 x \mathrm{H}
$$
$$
\mathrm{H}(x) =  \pi(x) \dot \psi(x) - \mathfrak{L}(x) = \bar \psi(x) (- i \gamma^i \partial_i + m) \psi(x)
$$
Expand the field in a complete set of solutnos of the dirac equation, these are lablled by $p_{\mu}$, $s=1,2$ (the momentum is on shell which means $p^{\mu}p_{\mu} = m^2$.
$$
\psi(x) = u^s (p) e^{-ip \cdot x}, v^s(p) e^{i p\cdot x}
$$
$$
\psi_{\alpha}(x) = \sum_{s=1}^2 \int \frac{d^3 p}{(2\pi)^3} \frac{1}{\sqrt{2 E_{\bm p}}} (b^s_p u_{\alpha}^s(p) e^{-i p\cdot x} + c^{s\dagger}_p v^s_{\alpha}(p)e^{i p\cdot x})
$$
\subsubsection{Spin-Statistics Theorem}
Correlates in a relativistic field theory how we quantise a field with its spin. IN particular it says that if we have particles with half integer spin we must quantise these particles as ferminons. If we don't do this then the energy of our system will be unbounded below. \\\\
Qu
antization in the schrodinger picture (at fixed time): $\psi(x) = \psi(\bm x, t) \rightarrow \psi(\hat x)$ and $\psi^{\dagger}(x) = \psi^{\dagger}(\bm x, t) \rightarrow \hat \psi^{\dagger}(\hat x)$. Impose cannonical anti-commutator relations:
$$
\{ \hat \psi_{\alpha}(\bm x) , \hat \psi_{\beta}^{\dagger}(\bm y) \} = \delta_{\alpha \beta} \delta^{(3)}(\bm x- \bm y)
$$
$$
\{\hat \psi_{\alpha}(\bm x), \hat \psi_{\beta}(\bm y) \} = \{\hat \psi^{\dagger}_{\alpha}(\bm x), \hat \psi^{\dagger}_{\beta}(\bm y) \} = 0,
$$
Mode expansion:
$$
\hat \psi_{\alpha}(\bm x) = \sum_{s=1}^2 \int \frac{d^3p}{(2 \pi)^3} \frac{1}{\sqrt{2 E_{\bm p}}} (\hat b_{\bm p} u^s_{\alpha} (p) e^{i \bm p \cdot \bm x} + \hat c^{s \dagger}_{\bm p} v^s_{\alpha}(p) e^{- i \bm p \cdot x})
$$
This implies that all the anticommuators vansih except:
$$
\{ b^s_{\bm p}, b^{r \dagger}_{\bm q} \} = \{ c^s_{\bm p}, c^{r \dagger}_{\bm q} \} = (2 \pi)^3 \delta^{rs} \delta^{(3)}(\bm p - \bm q)
$$
If we consider this in the context of the Ferminonic Simple Harmonic Oscillator then:\\\\
We introduce $\hat b$ and $\hat b^{\dagger}$ but rather than posing there commuator relation we posit a anticommuator:
$$
\{\hat b, \hat b^{\dagger} \} = 1, \{ \hat b, \hat b\} = \{ \hat b^{\dagger}, b^{\dagger} \} = 0
$$
This implies that $\hat b^2 = \hat b^{\dagger 2} = 0$ so they are nilpotent operators. We can try and define a hamiltonian for our fermions:
$$
H = \frac{1}{2} \omega (\hat b^{\dagger} \hat b - \hat b \hat b^{\dagger} ) = \omega( \hat N_F - \frac{1}{2}) 
$$
with $N_F =  \hat b^{\dagger} \hat b$. With respect to this operator $\hat b$ and $\hat b^{\dagger}$ are raising and lower operators. Can check that:
$$
[\hat H_F, \hat b] = - \omega \hat b, [\hat H_F, \hat b^{\dagger}] = \omega \hat b^{\dagger}
$$
Ground state $\ket{0}$ has $\hat H_F \ket{0} = -\frac{1}{2} \omega \ket{0}$. Then there is also  a $\ket{1} = \hat{b^{\dagger}} \ket{0} = \frac{1}{2} \omega{1}$ but no more states as $(\hat b^{\dagger})^2 \ket{0} = 0$. This means you can only fill a fermionic system once so this imposes the pauli exclusion principle.\\\\
Starting with our hamiltonian for a free field after normal ordering we get:
$$
H = \int \frac{d^3 p}{(2 \pi)^3} E_{\bm p} \sum_{s=1}^2 (\hat b^{s \dagger}_{\bm p} \hat b^s_{\bm p} - \hat c^{s}_{\bm p} c^{s \dagger}_{\bm p}) =  \int \frac{d^3 p}{(2 \pi)^3} E_{\bm p} (\sum_{s=1}^2 (\hat b^{s \dagger}_{\bm p} \hat b^s_{\bm p} + \hat c^{s \dagger}_{\bm p} c^{s}_{\bm p} )- (2\pi)^3 \delta^{(3)}(0)) 
$$
So we can clearly see we have a hamiltonian that is just made up of number operators for all the possible momentum and particles so the field is again just an infinite number of harmonic oscillators. If we had used commuators instead of anticommuators we would have got a minus sign between the bs and cs, which would mean we could have an arbitarily negative energy by occupying the c states. So have to use fermion quantisation description rather than the bosonic (anticommuator means they are antisymmetric under exchange whereas commuator means they are symmetric under exchage).
$$
\hat Q = \int \frac{d^3 p}{(2\pi)^3} \sum_{s=1}^2 (\hat b^{s \dagger}_{\bm p} \hat b^s_{\bm p} - \hat c^{s \dagger}_{\bm p} \hat c^s_{\bm p})
$$
So $b$ particles and $c$ particles have opposite charge.\\\\
\textbf{Particle Spectrum}:
$$
\hat b^{r}_{\bm p}\ket{0} = \hat c^r_{\bm p} \ket{0} = 0, r= 1,2 \forall \bm p \in \mathbb{R}^3
$$
1-particle state:
$$
\hat b^{s \dagger}_{\bm p} = \ket{p,s}
$$
As $b$ have positive charge $Q=+1$ and energy $E_{\bm p} = \sqrt{|\bm p|^2 + m^2}$ but we also have the spin index which can take two values (as appropriate for a spin 1/2 particle). These states are assocaited with the two different spinor indices which are then assocaited with our arbitary choose of $\xi$ basis so we could pick zbasis? .... (more to be done to show that these represent the spin of these particles need to act on it with angular momentum etc.).\\\\
1-anti particle state:
$$
\hat c^{s \dagger}_{\bm p} = \bar{\ket{p,s}}
$$
so $Q=-1$ and $E = E_{\bm p}$. The fact we only get four states rather than eight states is because the hermitian conjugate is also the canonical conjugate of the field (lol not sure what this means).\\\\
Multi-particle states:
$$
\ket{p,r ; q,s} = \hat b^{r \dagger}_{\bm p} \hat b^{ s \dagger}_{\bm q} \ket{0} = -\ket{q,s; p,r}
$$
simply because $\{b_{\bm p}^{r \dagger}, b_{\bm q}^{s \dagger}\} = 0$
\section{Lecture 22}
\subsection{Interacting Fermions}
$$
\mathfrak{L} = \mathfrak{L}_0 + \mathfrak{L}_{int}
$$
$$
\mathfrak{L}_0 = \bar \psi (i \gamma^{\mu}\partial_{\mu} - m) \psi
$$
Dimensional analysis: $[\mathfrak{L}] = 4$, $[\partial ] = +1 \implies [\psi] = [\bar \psi] = \frac{3}{2}$. Consider the simplest possible self-coupling that is consistent with lorentz invariance (neeeds to be powers of $\psi \bar \psi$:
$$
\mathfrak{L}_{int} = - g_F(\psi \bar \psi)^2
$$
as $[g_F] = -2$ so this is an irrelevent coupling (leads to non-renormalisable field theory). So we need to consider fermions coupling to other fields e.g. Yukawa theory or coupling to bosons:
\subsection{Yukawa theory}
Real scalar field $\phi(x)$:
$$
\mathfrak{L} = \mathfrak{L}_0 + \frac{1}{2} (\partial_{\mu} \phi)(\partial ^{\mu} \phi) - \frac{\mu^2}{2} \phi^2 - \lambda \phi \psi \bar \psi
$$
$[\lambda] = 0$ so this is a marginal interaction.
\subsubsection{Calculate scattering amplitudes}
Dyson's formula which will give rise to Time ordered products of the field operators, and then we will need to contract them with a version of wicks theorem. In doing so you get an answer in terms of a set of Feynman rules. So the rest of the course we will fill in these theories for Yukawa theory and QED (which uses gauge fields).\\\\
\textbf{Feynmann Propogator}\\
$$
S_F(x-y)_{\alpha \beta} = \bra{0} T[\hat \psi_{\alpha}(x) \hat {\bar{\psi}}_{\beta}(y)] \ket{0}
$$
Like in the scalar case we are thinking about the Heinsberg picture, so the fields above are the Heisenberg picture field operators. We need to worry about the order that fields appear in the product in all cases as they have a vanishing anticommutator. So:
$$
                                                T[\hat \psi_{\alpha}(x) \hat {\bar{\psi}}_{\beta}(y)]  = \begin{cases} \hat \psi_{\alpha}(x) \hat {\bar{\psi}}_{\beta}(y) & x^0 > y^0\\
                                                -\hat {\bar{\psi}}_{\beta}(y) \hat \psi_{\alpha}(x)  & x^0 < y^0 \end{cases}
$$
In the case where $x^0 = y^0$ they still anticommute so in order for the expression to be unambigious it needs this minus sign.  Lets start with the first ordering above ($x^0 > y^0$) call it $M_+$. Consider the mode expansion:
$$
\hat \psi_{\alpha}(x) = \sum_{s=1}^2 \int \frac{d^3 p}{(2 \pi)^3} \frac{1}{\sqrt{2 E_{\bm p}}} ( \hat b^s_{\bm p}^2 u^s_{\alpha}(p) e^{-i p\cdot x} + \hat c^{s \dagger}_{\bm p} v^s_{\alpha} (p) e^{i p \cdot x})
$$
$$
\hat {\bar{\psi}}_{\beta}(x) = (\hat \psi^{\dagger} (x) \gamma^0)_{\alpha}
$$
So
$$
\hat {\bar{\psi}}_{\beta}(x) = \sum_{s=1}^2 \int \frac{d^3 p}{(2 \pi)^3} \frac{1}{\sqrt{2 E_{\bm p}}} ( \hat b^{s \dagger}_{\bm p}^2 \bar{u}^s_{\alpha}(p) e^{i p\cdot x} + \hat c^{s}_{\bm p} \bar{v}^s_{\alpha} (p) e^{-i p \cdot x})
$$
So
$$
M_+ = \ket{0} \hat \psi_{\alpha}(x) \hat {\bar{\psi}}_{\beta}(y) \bra{0} = \sum_{r,s=1}^2 \int \frac{d^3p}{(2\pi)^3} \int \frac{d^3 q}{(2 \pi)^3} \frac{1}{\sqrt{2 E_{\bm p} 2 E_{\bm q}}} (u^s_{\alpha} (p) \bar{u}^r_{\beta} (q) e^{-i(p\cdot x - q \cdot y)} \bra{0} \hat b^r_{\bm p} \hat b^{s \dagger}_{\bm q} \ket{0}
$$
We can replace $ \hat b^r_{\bm p} \hat b^{s \dagger}_{\bm q}$ with $\{ \hat b^r_{\bm p} ,\hat b^{s \dagger}_{\bm q} \} = (2\pi)^3 \delta^{rs} \delta^{(3)}(\bm p - \bm q)$ so contract out the p and q:
$$
M_+ = \int \frac{d^3 p}{(2\pi)^3} \frac{1}{2 E_{\bm p}} e^{-i p \cdot(x-y)} \sum_{s=1}^2 u^s_{\alpha}(p) \bra{u}_{\beta}^s (p) 
$$
and as $\sum_{s=1}^2 u^s_{\alpha}(p) \bra{u}_{\beta}^s (p) = (\cancel{p} + m\bm I_4)_{\alpha \beta}$:
$$
M_+ = \int \frac{d^3 p}{(2\pi)^3} \frac{1}{2 E_{\bm p}} e^{-i p \cdot(x-y)} (\cancel{p} + m\bm I_4)_{\alpha \beta} 
$$
Similarly:
$$
M_- = \int \frac{d^3 p}{(2\pi)^3} \frac{1}{2 E_{\bm p}} e^{i p \cdot(x-y)} (\cancel{p} - m\bm I_4)_{\alpha \beta} 
$$
and we have:
$$
                                                        S_F(x-y)_{\alpha \beta} = \begin{cases} M_+ & x^0 > y^0 \\
                                                        -M_- & x^0 < y^0 \end{cases}
$$
Use the same technique as for the scalar case to turn this into a contour integral that picsk up the minus signs by depending on the plane you close it in:
$$
S_F(x-y)_{\alpha \beta}  = \int \frac{d^4p}{(2\pi)^4} i \frac{(\cancel{p}+ m)_{\alpha \beta}}{ p^2 - m^2 + i \epsilon} e^{- i p\cdot (x-y)
$$
\textbf{Wicks theorem for fermions}\\
Now the individual creation operators and annihilation operators do not communte with themselves so you also have to keep track of the ordering of the sets of the same operators:
$$
T[\hat \psi_{\alpha}\hat {\bar{\psi}}_{\beta}(y) ] = : \hat \psi_{\alpha}\hat {\bar{\psi}}_{\beta}(y) : + \hat \psi^{\cdot}_{\alpha}\hat {\bar{\psi^{\cdot}}}_{\beta}(y) 
$$
The operators still anticommute within the normal ordering! Take the vaccum expecation value:
$$
\hat \psi^{\cdot}_{\alpha}\hat {\bar{\psi^{\cdot}}}_{\beta}(y) = \bra{0} T[ \hat \psi_{\alpha}(x) \bar{\psi}_{\beta}(y)] \ket{0} = S_F (x-y)_{\alpha \beta}
$$
The contraction between $\psi \psi$ and $\bar \psi \bar \psi$ would vannish.The fermonic operators anticommute inside normal ordering:
$$
: \hat \psi_1 \hat \psi_2: = - : \hat \psi_2 \psi_1 :
$$
\textbf{Wicks Theorem}\\
Let $\hat \pi$ be an arbitary product $\hat \psi$, $\hat \bar \psi$ with $\hat \psi(x_1) = \hat \psi_1$ and let 
$$
\hat \pi = \hat \psi_1 \hat \psi_2 \hat \bar \psi_3 \psi_4
$$
For any such product, the time ordered product is normal order of the sum over all the contractions:
$$
T[\hat \pi] = : G[\hat p\i]:
$$
\subsection{Gauge Fields}
4-vector fields assocaited with the electromagentic fields, as the observed fields $E$ and $B$ can be expressed in terms of the 4-vector potential $A^{\mu}$ which is a 4-vector field meaning it tranforms as:
$$
A^{\mu}(x) \rightarrow \Lambda^{\mu}_{\nu} A^{\nu}(\Lamdba^{-1} \cdot x)
$$
This field is not in of its self a physical thing there are transformations you can perform on A that level the physcially observable things $E$ and $B$ unchanged which is what makes this a guage field. There is the observeable field strength tensor which contains $\bm E$ and $\bm B$:
$$
F^{\mu \nu} = \partial^{\mu} A^{\nu} - \partial^{\nu} A^{\mu}
$$
This field strength tensor is invariant under gauge transformations of $A^{\mu}$:
$$
A^{\mu} \rightarrow A^{\mu} + \partial^{\mu} \lambda, \lamdbda: \mathbb{R}^{3,1} \rightarrow \mathbb{R}
$$
We need to write an action for this field theory with Lagrangian desnity (which is the only thing it could be to produce linear EOM whilst being lorentz invaraint and guage invariant):
$$
\mathfrak{L} = -\frac{1}{4} F_{\mu \nu} F^{\mu \nu}
$$
Conjugate momenta:
$$
                                                                \pi_{\mu} = \frac{\partial \mathfrak{L}}{\partial \dot A_{\mu}} = \begin{cases} 0 & \mu  = 0\\ - F^{oi} = E^i & \mu = \{1,2,3\}\end{cases}
$$
This wont work with our normal canonical commuation relations as the zeroth term is 0. This is because we need to treat the gauge transformation properly\\
Configuratoins of $A^{\mu}$ that are related by a gauge transformation are physically indisguisable.\\
The configuration space is all configuratoins of the field as some fixed time, and it is covered by sets of gauge orbits. A is foliated by gauge orbits and we need to fix a gauge. \\\\
\textbf{Lorentz gauge}
$$
\partial_{\mu} A^{\mu} = 0
$$
This condition is lorentz invariant, and also doesn't completely fixed the gauge. Now think about dynamics with this gauge, by modifying the Lagrangian:
$$
\mathfrak{L}_{\epsilon} = -\frac{1}{4} F_{\mu \nu} F^{\mu \nu} - \frac{1}{2\epsilon} (\partial_{\mu} A^{\mu})^2
$$
This reduces to Maxwell when the lorentz gauge is imposed and now we will have conjugate momentum for all $\mu$. We can further speicify to the Feynmann gauge with $\epsilon = 1$ but you don't need to. It is a good excercise to leave $\epsilon$ unset and check that it makes no difference to the final equations.
$$
                                                                \partial_{\mu} (\frac{\partial \mathfrak{L}}{\partial(\partial_{mu}A^{\mu}} = - \partial_{\mu} F^{\mu \nu} - \partial_{\mu} (\eta^{\mu \nu} (\partial_{\rho}A^{\rho}) =  - \partial_{\mu}\partial^{\mu} A^{\nu} + \partial_{\mu} \partial^{\nu} A^{\mu} - \partial^{\nu} \partial_{\mu} A^{\mu} = - \parital_{\nu}\partial^{\nu} A^{\mu}$$
                                                                So the canconial momenta is:
                                                                $$
\pi^0 = -(\partial_{\mu} A^{\mu}), \pi^i = \partial^i A^0 - \dot A^i
                                                                $$
Strategy (Guple-Bleuler procedure) is to first quantitise and then constrain to the lorentz gauge. Consider in the schrodinger picture:
$$
A^{\mu} \rightarrow \hat A^{\mu}, \phi^{\mu} \rightarrow \hat \pi^{\mu}
$$
with Non-vanishing commuators:
$$
[\hat A^{\mu} (\bm x), \hat \pi^{\mu}(\bm y)] = i \eta^{\mu \nu} \delta^{(3)}(\bm x - \bm y)
$$
We have a spin 1 field so we expect to be quantitsed in a bosonic way hence why we are using commuators. \\
Consider mode expansion of field theory:
$$
\parital_{\nu}\partial^{\nu} A^{\mu} = 0
$$
So this has solutions of the form $ \epsilon^{\mu} e^{\pm i p \cdot x}$ wit. These are labled by on shell 4-momentum ($p_{\mu}p^{\mu} = 0, p^0 \geq 0 \implies p^{\mu} = (E_{\bm p}, \bm p)$ and the $\epsilon$ (the polarisation 4vector).\\\\
Chose an orthonormal basis $\epsilon^{(\lambda)}_{\mu} (p)$ ($\lamdba in \{0,1,2,3\}$) with $\eta^{\mu \nu} \epsilon_{\mu}^{(\lambda)}(p) \epsilon_{\nu}^{(\rho)}(p) = \eta^{\lambda \rho}$ this  tells us that $\epsilon^{(0)}$ is timelike and $\epilson^{(i)}$ is spacelike. can show completeness relation: $\eta_{\lambda \rho} \epsilon_{\mu}^{(\lambda)}(p) \epsilon_{\nu}^{(\rho)}(p) = \eta_{\mu \nu}. $ \\\\
It is useful to choose $\epsilon^{(1)}, \epsilon^{(2)}$ to be transverse ($p^{\mu} \epsilon_{\mu}^{(i)} =0$) but $p^{\mu} \epsilon_{\mu}^{(3)} \neq 0$. Now we can mode expand te field operator:
$$
\hat A_{\mu}(x) = \int \frac{d^3 p}{(2\pi)^3} \frac{1}{\sqrt{2 |\bm p|}} \sum^3_{\lambda = 0} \epsilon^{(\lambda)}_{\mu} (p) (\hat a^{\lamdba}_{\bm p} e^{i \bm p \cdot \bm x} + \hat a^{\lamdba \dagger}_{\bm p} e^{-i \bm p \cdot \bm x}
$$
$$
\hat \pi_{\mu}(x) = i \int \frac{d^3 p}{(2\pi)^3} \sqrt{\frac{|\bm p|}{2}} \sum^3_{\lambda = 0} \epsilon^{(\lambda)}_{\mu} (p) (\hat a^{\lamdba}_{\bm p} e^{i \bm p \cdot \bm x} - \hat a^{\lamdba \dagger}_{\bm p} e^{-i \bm p \cdot \bm x}
$$
Now should check that the field commutation relations imply the oscillator communation rleaitons:
$$
[a^{\lamdba}_{\bm p}, a^{\rho \dagger}_{\bm q} ] = - \eta^{\lambda \rho}(2 \pi )^3 \delta^{(3)} (\bm p - \bm q)
$$
This creates a hilbert space with a non-positive norm so is not positive semidefinitie (this is because we aren't talking about a physical space, when we restrict otthe physcial sybspace this will dissappear). We can calculate the hamiltonian:
$$
\hat H = \int \frac{d^3 p}{(2\pi)^3} |\bm p| (\sum _{\i=1}^3 \hat a^{i \dagger}_{\bm p} \hat a^i_{\bm p} - \hat a^{0 \dagger}_{\bm p} \hat a^0_{\bm p}
$$
So the vaccum states are defined such that:
$$
\hat a^{\lamdba}_{\bm p} \ket{0} = 0
$$
The "Big" "Hilbert" space:
$$
\ket{\bm p_1, ..., \bm p_n, \lambda_1,..., \lambda_n} = \hat a^{\dagger \lambda_1}_{\bm p_1}...\hat a^{\dagger \lambda_n}_{\bm p_n} \ket{0}
$$\section{Lecture 24}
Heisnberg field operator
$$
A_{\mu} (x) = \int \frac{d^3 p}{(2\pi)^3} \frac{1}{\sqrt{2|\bm p|}} \sum_{\lamda=0}^3 \epsilon^{(\lambda)}_{\mu} (p) (\hat a_{\bm p}^{\lambda} e^{-i p \cdot x} + \hat a_{\bm p}^{\lambda \dagger} e^{i p \cdot x})
$$
$$
[\hat a_{\bm p}^{\lambda}, \hat a^{\mu}_{\bm q}] = - \eta^{\lambda \mu} (2\pi)^3 \delta^{(3)}(\bm p - \bm q)
$$
We can get particles which we call photons by acting in different componetnts on the vaccum state as above. We know this is wrong as there should only be two degrees of freedom corresponding to a photon as the polarisation is aligned trasverse in 2d whereas this gives us 4 degreees of freedom. There is also a mathematical problema s the big "Hilbert space" spanned by this states contain negative norm states and 0 norm states so it is not really a hilbert space.
$$
\bra{\bm p, 0} \ket{\bm q, 0} = \bra{0} \hat a_{\bm p}^0 \hat a^{0 \dagger}_{\bm q} \ket{0} = -(2 \pi)^3 \delta^{(3)}(\bm p - \bm q)
$$
These problems go away when we impose the lorentz gauge constraint which classically is:
$$
\partial_{\mu} A^{\mu} = 0 $$
To impose this quantumly we use the Gupta-Bleuler strategy. We define a physical subspace s.t.
$$
\bra{\Psi'} \partial_{\mu} \hat A(x) \ket{\Psi} = 0 \forall \ket{\Psi'}, \ket{\Psi} \in \mathcal{H}_p
$$
$$
\partial_{\mu}\hat A^{\mu} = \int \frac{d^3 p}{(2\pi)^3} \frac{1}{\sqrt{2|\bm p|}} \sum_{\lamda=0}^3 (-i p^{\mu})\epsilon^{(\lambda)}_{\mu} (p) (\hat a_{\bm p}^{\lambda} e^{-i p \cdot x} - \hat a_{\bm p}^{\lambda \dagger} e^{i p \cdot x})
$$
This won't constrain transverse polarisations for which $p^{\mu} \epsilon_{\mu}$ as it will vanish always. Choose a canonical lorentz frame with $p^{\mu} = (p,0,0,p)$, and orthonormal basis $\epsilon_{\mu}^{(\lambda)}(p) = \delta^{\lambda}_{\mu}$. The point is that the GB condition then allows only photons with polarisation obeying $p^{\mu}\epsilon_{\mu} = 0$ so we have transverse photons: $\ket{\psi} = \ket{\bm p, \lamdba = 1,2}$ and also certain linearly combinations of unphysical photons: $\ket{\phi} = \ket{\bm p, 0} - \ket{\bm p, 3}$ ("time like photon and longitudal photon"). So we haven't coompletly removed the extra degrees of freedom but this is good enough, as these states have zero norm and the expectation of the Hamiltonian is also zero. This means you can get rid of them by taking a quotient, by identifying states which differ by a state with zero norm $\ket{\psi} \sim \ket{\psi'} + \ket{\phi}$. The upshot after all of this is we can restrict our attention to the two transverse operations.\\\\
\subsubsection{Photon propogator}
$$
\Delta_{\mu \nu}(x - y) = \bra{0} T[\hat A_{\mu}(x) \hat A_{\nu}(y)] \ket{0}
$$
Evaluate using the mode expansion in exactly the same way and we will get an integral representation of the form:
$$
\int \frac{d^4 p}{(2 \pi)^4} \frac{-i \eta_{\mu \nu}}{p^2 + i\epsilon} e^{-i p\cdot (x-y)}
$$
\subsubsection{Coupling to matter}
Maxwell equation with source
$$
\partial_{\mu} F^{\mu \nu} =J^{\nu}
$$
$$
J^{\nu} = (\rho(\bm x,t), \bm J(\bm x,t))
$$
The charge conservation equation means we must have a conserved current:
$$
\partial_{\mu} J^{\mu} = 0
$$
Comes from the action:
$$
S = \int d^4 x \mathfrak{L}
$$
$$
\mathfrak{L} = - \frac{1}{4} F_{\mu \nu} F^{\mu \nu} - A_{\mu} J^{\mu}
$$
$$
F_{\mu \nu} = \partial_{\mu} A_{\nu} - \partial_{\nu} A_{\mu}
$$
$F$ is invariant under the lorentz gauge transformation, and so is $A_{\mu}J^{\mu}$ as
$$
A_{\mu} \rightarrow A_{\mu} + \partial_{\mu} \lambda \implies A_{\mu}J^{\mu} \rightarrow A_{\mu}J^{\mu} + \partial_{\mu} \lambda J^{\mu} = A_{\mu} J^{\mu}
$$
Source is provided by the spinor field so write down the dirac lagragian:
$$
\mathfrak{L}_F = \bar \psi(x) (i \gamma^{\mu}\partial_{\mu} - m) \psi(x)
$$
This has a U(1) symmetry that gives rise toa conserved current:
$$
\psi (x) \rightarrow e^{-i\alpha} \psi(x), \bar \psi(x) \rightarrow e^{i\alpha} \bar \psi (x) \implies \partial_{\mu} J^{\mu}_{\nu} = 0, J^{\mu}_{\nu} = \bar \psi \gamma^{\mu} \psi 
$$
\subsubsection{Quantum Electrodynamics}
$$
\mathfrak{L} = - \frac{1}{4} F_{\mu \nu} F^{\mu \nu} + \bar \psi(i \gamma^{\mu} \partial_{\mu} - m) \psi - e( \bar \psi \gamma^{\mu} \psi) A_{\mu}
$$
Check that this is gauge invariant:\\
It is obviously a lorentz scalar as this was already demonstrated for the dirac lagrangian and the FF term. We have shown that $\bar \psi \gamma^{\mu} \psi$ is a lorentz vector so contracts to a lorentz scalar with $A_{\mu}$. Now we can keep the symmetry if we turn it into a local symmetry:
$$
\psi(x) \rightarrow e^{- e \alpha(x)} \psi(x), \bar \psi(x) \rightarrow e^{i e \alpha(x)} \bar \psi
$$
Want to check this is equivalent to $A^{\mu} \rightarrow A^{\mu} + \partial^{\mu} \alpha (x)$.\\\\
Define a covariant derivative:
$$
D_{\mu} \psi(x) = \partial_{\mu} \psi(x) + i eA_{\mu} (x) \psi(x)
$$
$$
D_{\mu} \psi(x) \rightarrow \partial_{\mu} (e^{-i e \alpha(x)} \psi(x)) + ie(A_{\mu}(x) + \partial_{\mu}\alpha(x)) e^{-ie \alpha(x)} \psi(x) 
$$
$$
D_{\mu} \psi(x) \rightarrow  e^{-ie\alpha(x)} (\partial_{\mu} \psi(x) - ie \partial_{\mu} \alpha(x) \psi(x) + ie A_{\mu} \psi(x) + i e \partial_{\mu} \alpha(x) \psi(x) = e^{- ie \alpha(x)} D_{\mu} \psi(x)
$$
Now we write
$$
\mathfrak{L} = - \frac{1}{4} F_{\mu \nu} F^{\mu \nu} + \bar \psi( i \gamma^{\mu} D_{\nu} - m ) \psi
$$
$$
\mathfrak{L}_{int} = -e \bar \psi \gamma^{\mu} \psi A_{\mu}
$$
Dimensional analaysis: $[\mathfrak{L}] =4$ $[\psi]=[\bar \psi] = frac{3}{2}$ and $[A^{\mu}] =1$ so $[e] = 0$. So $e$ is a marginal coupling. The true expansion parameter $\alpha = \frac{e^2}{4\pi}$. If we identify $e$ with the charge of the electron and restoring units by dividing through by $\hbar c$ and this comes out to $\alpha \approx \frac{1}{137}$. It is the fact that this number being small that allows pertubration theory ot work as each term differes form the next by powers of this number.
\section{Examples Class 3}
To show that $Tr \gamma^{\mu}=0$ we observe that $\{\gamma^{\mu},\gamma^{\nu}\} = 0$ for $\mu \neq \nu$ so $\gamma^{\nu} \gamma^{\mu} (\gamma^{\nu})^{-1} + \gamma^{\nu} = 0$ and then take trace of both sides. Also neat trick:
$$
\gamma^{\nu} \gamma^{\mu} = \frac{1}{2} \{ \gamma^{\nu}, \gamma^{\mu}\} + \frac{1}{2} [ \gamma^{\nu} , \gamma^{\mu}]
$$
Q5:
$$
(p \cdot \sigma) (p \cdot \bar \sigma) = p^2 \neq 0 
$$
$$
\bar \psi \prop \bar u b^{\dagger} + \bar v c 
$$
I forgot the bars on the u and v\\\\
Fermonic sign in feynmann rules: if we consider $\bra{0} T(\psi_1 \psi_2 \psi_3 \psi_3 \ket{0} = \bra{0} T(\psi^,_1 \psi^._2 \psi^,_3 \psi^._4) \ket{0} + \bra{0} T(\psi^._1 \psi^._2 \psi^,_3 \psi^,_4) \ket{0} +  \bra{0} T(\psi^._1 \psi^,_2 \psi^,_3 \psi^._4) \ket{0}  $ to make sure that the contractions are between ajacent operators we need to count how many times we need to swap fermonic operators and this gives us the parity. No way of doing this form feynmann diagrams (as it depends on the order of the external operators the ones defining the final and initial state).\\\\
When looking for relative signs just need to think about how many times you need to swap positions to go from one contraction to another\\\\
The reason you cant contract out any more terms as you will either get a loop or one particle will pass throughout without interacting. Also in feynmann rules look at the back of q8 to see the rules for a fermionic paying particular attention to the u and v associated with ingoign and outgoing lines.
\end{document}
